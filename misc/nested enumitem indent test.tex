\documentclass{article}

%% Formatting
\usepackage[letterpaper,margin=1.5in]{geometry} % page setup
\usepackage[shortlabels]{enumitem}  % customizeable enumerators
\usepackage[USenglish]{babel}   % ensure correct hyphenation
\usepackage[T1]{fontenc}        % validate output font
\usepackage[utf8]{inputenc}     % validate input characters
\usepackage{siunitx}    % SI units
\usepackage{graphicx}   % include graphics
\usepackage{booktabs}   % tables
\usepackage{float}      % [H] option for floats
\usepackage{parskip}    % remove paragraph indentations

%% Content
\usepackage{amsmath}    % math
\usepackage{physics}    % physics

%% Metadata
\newcommand{\Title}     {Problem Set 6}
\newcommand{\DueDate}   {March 5, 2018}
\newcommand{\Course}    {PHY 200B}

% Problem Set 6
% Physics 200B
% Due at start of class, Monday March 5, 2018
% Zangwill 9.5
% Zangwill 9.17 Add part c: Explain the importance of this result for actual measurements. Bear in mind that experimentalists often want to extract intrinsic quantities such as resistivity.
% Zangwill 9.23 and 9.24
% Zangwill 10.6 Add part c: Repeat the derivation a third way, using only Ampe`re’s law and symmetry.
% Zangwill 10.8 Add part d: Give at least two examples of situations that use planar coils. You are welcome to use any internet resources you like on this one.
% Zangwill 10.20 and 10.24

\begin{document}
{\huge\textbf{\Title}}

Due \DueDate \hfill \Course

\hrulefill

\begin{enumerate}[align=parleft]%,labelsep=26pt]
    \item This is an item in the first list, with some subitems.
    \begin{enumerate}[(a), align=parleft,labelsep=26pt]
        \item Show that the electrostatic potential produced at point $C$ by the current injected at point $A$ is
        \[
            \varphi_{AC} = -\frac{I}{\pi d\sigma}\ln(a+b).
        \]

        \item Prove that
        \[
            \exp\pqty{-\pi d\sigma R_{AB,CD}} + \exp\pqty{-\pi d\sigma R_{BC,DA}} = 1.
        \]
    \end{enumerate}
    \item This is the second item in the first list.
\end{enumerate}
\end{document}