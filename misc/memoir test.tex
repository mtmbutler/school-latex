\documentclass[12pt]{book}

\usepackage[utf8]{inputenc} % If utf8 encoding
\usepackage[T1]{fontenc}    %
\usepackage[USenglish]{babel} % English please
\usepackage[final,babel]{microtype} % Fewer badboxes

\usepackage[top=2.1in, total={4.5in,7.2in}]{geometry} % Margins

\usepackage{libertine} % Libertine font

\usepackage{emptypage} % Suppresses page numbers on empty pages
\usepackage{titletoc} % http://ctan.org/pkg/titletoc
\titlecontents{chapter} % <section-type>
  [0pt]% <left>
  {\addvspace{1em}}% <above-code>
  {\bfseries\chaptername\ \thecontentslabel\quad}% <numbered-entry-format>
  {}% <numberless-entry-format>
  {\bfseries\hfill\contentspage}% <filler-page-format>

\renewcommand{\baselinestretch}{1.05}

\author{PJ Moser}
\title{A Long Walk Home}
\date{}

\begin{document}
\frontmatter
\maketitle
\tableofcontents

\chapter*{Prologue} \chaptermark{Prologue}
\addcontentsline{toc}{part}{Prologue}
\noindent\textit{Fadeout---Lights dim on an old man with grey hair and grey beard sitting alone at his desk, trying to understand the most difficult relationship there is---Father and children---you're bonded forever---There's no place to hide---or run---you have chosen each other forever and you're supposed to love one another---How do we do that? How? Somebody help me understand. Drops the pen, stares off into space---sun peaks through the curtain---another day---of---what?}

\noindent End of a two-act play written by Carver

\noindent Corte Madera, California

\noindent Three a.m., Jan 2, 1991\\

My family and I were visiting Northern California from Missouri over Thanksgiving break in 2012. My dad had asked me a year earlier to be the executor of his will. He'd already asked several of my older siblings---there were eight of us kids---but they'd turned him down for one reason or another. He'd explained those reasons to me, or rather explained them away, and assured me that being an executor was really a simple matter. At the time of our conversation I'd just gotten back from a surf session in Bodega Bay, so I was relaxed and in a good mood.

``Everything you need to know is in a green notebook I keep in my office,'' he'd told me. ``Just call my lawyer. He'll walk you through everything.''

I didn't know exactly what an executor did, but I couldn't think of a reason not to do it for my dad, especially since he stored my surf gear in his barn. It was convenient to have a board and wetsuit waiting for me whenever I visited California. Surfboards are expensive to travel with when you fly, and they're easily damaged.

So I'd told him, ``Okay.'' My understanding at the time was that I'd be in charge of a legal document once he died. I think we both thought that. It didn't sound overly complicated or involved. Certainly not anything that changed your life.

In 2012 he was 82 years old, physically strong and active. He'd called himself Carver since the early `80s. Living in Bolinas at the time---a groovy town on the Marin coast filled with artists and people who liked to live off the grid---he'd decided he was a driftwood carver. So that's what he'd started calling himself.

We'd arranged this visit over Thanksgiving so he could show me ``the green notebook.'' This was the first place I'd go after he died. Our schedules hadn't matched up since I'd agreed to be the executor. I did live out of state, but we weren't close in the other sense as well. So we were finally meeting at his home in Sebastopol to go over the details.

His wife was lying on the couch when I walked into their house, swaddled in blankets. She was buried so deep I couldn't even see her.

``Let's talk in my office,'' my dad said quietly. ``Maya's sleeping.''

I didn't question why Maya was sleeping on a couch in the middle of the afternoon. I knew my dad had become her full-time caregiver several years back---she suffered from dementia---but I wasn't a part of their daily life. None of us kids were beyond the occasional get- together that you planned well in advance. My dad wasn't someone you just dropped in on. I'd agreed to be his executor because my responsibilities would begin after he died. That's when I'd get involved.

I checked my dad out from behind as I followed him into his office. He moved easily, spry for his age. I was 49 in 2012, taller than him by that time, though he'd always towered in my mind as the epitome of strength, a sequoia in the grove of manhood. He'd like that image of himself. I grew up in nearby Occidental climbing redwoods, my brothers and I pushing each other to the tops of hundred-foot trees where we'd cling like squirrels and gaze toward the true giant on our property, an enormous redwood my dad had dubbed ``The Prophet'' after a book of poetry by Kahil Gibran. We'd wonder how to go about reaching its limbs and what the world might look like if we ever managed to scramble up there. A frightening thought to us.

Our dad was the Role Model of Will and Determination. Picking himself up by the bootstraps was never enough for him: he had to do cartwheels. Let me illustrate: he celebrated his 64th birthday by running a marathon; when his knee buckled at the seventeen-mile mark, he limped and lunged the last nine miles to the finish line where he collapsed and passed out. He celebrated his 70th birthday by walking seventy miles. To make it more of a challenge---you see what we're dealing with here---he decided it had to be done in less than twenty-four hours. He made it in just over twenty, hiking all day and through the night by himself across the hilly backcountry and narrow farm roads of Sonoma County.

Now, he did agree that family and friends could join him on certain legs during the home stretch, but we were alerted to the rules: he wouldn't be talking to anybody; he had to keep his pace of at least three miles an hour if he was going to complete the trek in less than one full rotation of the earth. He'd set a goal and By God he'd complete it if he had to ignore you or any pain that flared up along the way. His or yours.

People walked with him. They rooted him on from the roadside with signs and cheers. They waited for him at the finish line with drinks, food, and cameras to celebrate his accomplishment and commemorate the moment. Fanfare was important to my dad. He thrived on public acknowledgement of his accomplishments. I sometimes wonder if that's the reason why I've grown into such a quiet person. My slice to his hook.

I wasn't there for my dad's walk-a-thon. I saw pictures. One image showed him happily exhausted, fairly collapsed into a chair with his feet propped up on a cooler. Later he retraced a few of those 140,000-odd steps with a writer and a photographer who were doing a profile of him for an area magazine. He told them the story behind his hike, a story none of us kids had known while we were growing up because it'd happened before we were born: how he did it for the men of his platoon back in Korea, his fellow Marines, thirty-six of whom were killed in one night. He explained for the article, titled ``A Long Walk Home'': \textit{My best friend died in my arms. I didn't have the spiritual insight or the life experience to appreciate or understand what had happened to me. So I put it away. I covered it all up. I didn't even tell my wife. I had a lot of demons.} This was Charlie Company, the 7th Regiment, at the Chosin Reservoir in December of 1950. The article went on to say that my dad was the last one of the platoon still alive.

\textit{They don't tell in those books the worst things,} my dad revealed. \textit{Frozen bodies piled liked cordwood around the foxholes. They broke off arms and legs to stack them up.}

My dad had an epiphany the day after the attack: the war was wrong. He argued with a lieutenant. \textit{I threw my rifle down in the snow, and said I wasn't going to fight anymore. They sent me to a military hospital in Japan and subdued me. Put me in a straightjacket. I was just totally crazy.}

He walked to release the trauma. He walked to shed the shame he'd felt for fifty years of being a coward who refused to fight. He walked because his best friend, Charlie Barnett, and the other men of his platoon weren't around to stride country lanes or breathe in the fresh scent of redwoods.

\textit{After I finished the walk, I felt like there was a lot of closure. I've done everything I can do. Now I'll take the rest of it to my grave.}

We kids never heard the story because it isn't true. He was in Quantico, Virginia during the Korean War. A Marine, yes, but working as an office clerk. As far as I can tell, he pulled the details of his lie from the books and magazines that don't tell you the worst things.

\begin{center}$*$\end{center}

I sat at the desk in his office as he took out the green notebook and began flipping through the dozen colored tabs. ``It's all here,'' he said, turning the pages. ``Everything you'll need to know.'' At the front of the notebook he tapped a lawyer's card slipped into a clear plastic pocket. ``Just call him when I die.'' He lifted his eyes to mine. ``That's all you gotta do.''

Maya cried out from the living room, a wail that rose steadily to fill the small house. My dad lifted his voice: ``It's OKAY, honey. It's OKAY.''

He waited a minute, listening. He nodded at me. ``She's fine. I tell her it's okay, and she quiets down. She just needs to hear my voice.'' I nodded back. If you say so.

My dad rested his palm on the notebook. ``It'll be easy,'' he told me. ``The lawy---''

Maya wailed again, louder this time, and longer.

``You're FINE, HONEY,'' he yelled over her cries. ``I'M HERE.''

Her wailing continued randomly for another minute, then subsided. My dad waved it off.

This happened all the time; he had it under control. I wanted to feel calmed by his words, but there was the yelling. And Maya's voice had put me on edge: it was a kind of haunted-house shriek. My dad went for a reassuring tone: ``The lawyer will take care of everything.''

I think back to that visit and know I missed an urgency in his voice or manner: the desperation of a man near the end of his rope who refused to believe he could not bring his wife back to health. He had always been able to move mountains by sheer force of will, but Maya's dementia broke him as it broke her. She'd be hospitalized four days after I left---near death on that couch, lying in her own waste---and my dad's mind, relieved of that overwhelming responsibility, would start its own drift from reality. It must have already begun. But his manic personality, coupled with a well-honed poker face, made him hard to read. Still, the desperation was there, it must have been, and I missed it. Our meeting itself was a sign that something had changed in his life, or was about to change, and he wanted to get his ducks in a row.

He held the green notebook like a man who'd organized his life and already determined how it would end. The dozen colored tabs covered the entire spectrum of possibilities for him, each color blending smoothly into the next: yellow to blue to green, and the shades in between. ``PLAN YOUR WORK---WORK YOUR PLAN.'' That was the first line, written in red ink, of a journal I later found stashed on his bookshelf in that very office. The green notebook was his plan once he died. Here's what I should have asked him: What's the plan between now and then?

Instead I said, ``Sounds good.''

And just like that I was in charge. My answer put me on a long road beside him, impelled me to care for a man whom I considered more like an uncle for most of my adult life because of how infrequently I saw him. I discovered the most amazing thing on our journey: the dementia that slowly robbed him of his memory, his reason, and his ability to care for himself also stripped him of the most abrasive parts of his personality. Dementia gave me the surprising gift of a relationship with my dad, one that I had denied for so long and yet deeply desired. 

\mainmatter
\part{}
\chapter{}

The phone call came late at night. It was April, 2013. A woman on the other end introduced herself as Mercedes. She and her husband were longtime friends of my dad and Maya. I later found out that Mercedes had visited Maya once a week during her last years at home to give my dad a much-needed break. Mercedes' voice was low and calm. She got right to the point.

``Your dad needs help,'' she said. ``Isn't there family here who can help him?''

I stared at the wall of my bedroom, a small knot of resistance tightening in my gut. Mercedes had gotten my number from my dad's tenant---I'll call her Kay---a woman about my age who lived in Maya's art studio on the back of the property. I'll have more to say about Kay shortly since she became a big part of my dad's life, but basically she'd moved in the year before to do a work-for-rent exchange: she helped my dad with Maya and did housecleaning. When my dad's dementia got worse, she took on the role of his primary caregiver. I hadn't met Kay yet, but we'd talked on the phone and corresponded through email.

Mercedes was friends with Kay, so she knew my dad had children who lived in the area. Or maybe she'd heard my dad talk about us. Apparently we were topics of conversation from time to time, my dad boasting about our achievements and what a great family we were. She'd already called my two sisters who lived in Sonoma County, asking for their help. They rebuffed her. It must have seemed outrageous to Mercedes: Eight kids? And not one was willing to step in and help my dad in a crisis?

``It's complicated,'' I said. How could I put our myriad relationships into a nutshell for her? It was impossible, even if I had the time. I appreciated what Mercedes and Kay were doing for my dad, looking out for him and Maya, so I gave her my version of how things stood. I didn't want her to think I was uncaring or unreasonable. I took a breath and let it out slowly. ``My dad left us for long periods of time,'' I said in a modulated voice. ``Sometimes there were years when he didn't keep in contact. We didn't know where he was, who he was with, or what he was doing.'' I already felt like I'd said too much. These were personal feelings, deep resentments that had just started to surface for me and that I didn't fully acknowledge or understand. I didn't even know this woman.

Mercedes didn't want to hear it. ``Whatever your issues are, you need to get over it. Because he needs help. I'm begging you to come out.''

I heard the strain and urgency in her voice---it was the only reason I continued to listen to her. I imagined telling my siblings they needed to get over it. Let's just say I'd rather stand in the center of a bullring waving a red flag than confront my siblings over why they should be stepping in and helping out my dad. You think superfund sites are toxic? Try digging in the family compound. They don't make hazmat suits strong enough to survive that job.

But Mercedes was basically telling me to grab a shovel.

I should explain: my siblings are not the problem. We get along exceptionally well. I love being part of a big family, and our collective friends usually marvel at how we truly enjoy each other's company. We don't necessarily see each other a lot, but we stay in touch and visit when we can. At so many points in my life, my siblings have been my best friends. One reason we get along so well is because we weren't raised to tell one another what to do, or how to think and act. We don't pull guilt trips. As long as what you're doing isn't causing problems for me, we are live and let live. It would have been ludicrous for me to take my dad's side over any one of my sibling's. Not only would it go against our general west-coast upbringing (you do your thing, I'll do mine), but I'd have zero credibility because my dad was a chronic bullshitter who only really thought about himself. Taking his side would be like urging your best friends to throw their life savings into junk bonds.

That's a harsh assessment of my dad. It's accurate enough, for my part. He had many admirable qualities as well, but he never seemed to recognize them in a way that gave him a strong sense of who he was as a person. It was like his inner compass had no true north, so he was constantly jagging in different directions looking for the right path to take. Or maybe his true north was so awful---he had stories of his own abuse as a child---that he'd rather follow any direction but that one. In any case he lied and reinvented himself to try and capture identities that he so desperately wanted to believe himself to be. A Korean war hero, for one.

As for my dad's dementia---this was the looming crisis Mercedes had called about---I was still in a hands-off mode. Kay had actually called me earlier that day to lodge her ongoing concerns about his memory problems: the man who used to work as a professional handyman, who could practically build or fix anything, had driven down to the local hardware store to ask someone how to put the registration sticker on his license plate. Kay had been helping him apply for Medi-Cal---Maya's skilled nursing facility cost over \$8,000 a month---but my dad couldn't gather the information for the application. He couldn't remember his Social Security number so they had to drive to the Social Security office for a new card. Once he got there, he couldn't recall his father's name; they couldn't issue him a card without that information. He racked his brain for seven minutes before finally coming up with William. On the way home he couldn't figure out how to use his credit card to pump gas; he'd forgotten his pin number. The Medi-Cal application required financial information that he'd forgotten. He had to drive to the VA for income verification. He had to go to Wells Fargo for bank statements. He couldn't figure out how to write a cover letter, where to put the forms, or how to address the envelope.

He gave up, overwhelmed.

Kay helped him finish the forms and mail everything in. The next day my dad asked her when they were going to finish the forms. She reminded him they'd done it the day before.

Maya's daughter, visiting from Los Angeles earlier that month, had arrived to investigate cheaper healthcare options for her mom. They considered moving Maya back home with my dad and hiring a full-time caregiver along with Hospice; this would cut their costs in half. Or there was a Board and Care facility in nearby Forestville where Maya could live---basically assisted living---which was also about half the price of the skilled nursing facility.

At first my dad agreed to the option of Maya coming home because it would save them money. Within twenty-four hours he flipped 180 degrees. He told me over the phone that there was no way she could come home. He could not take care of her. He'd said in a distressed voice, ``I need money, man!''

I'd asked my dad if he wanted me to fly out---this was a week before Mercedes called--- but his response was lukewarm. Probably my offer had been, too. I teach at a university and still had a month left in the spring semester: classes to prep, exams to give, papers to grade. Besides, my dad had seemed more concerned over the phone with getting money than in having me out there. It seemed to me that Maya's bills were a long-term problem, not something that flying out or writing him a quick check would fix. I also told myself that I didn't need to be there: too many chefs spoiled the soup. Maya's daughter and my dad were working this out. Kay and Mercedes had begun searching the county for Board and Care options. My dad's friends knew his and Maya's situation much better than I did. They didn't need me.

But here was Mercedes on the phone, urging me to get on a plane.

She didn't know me. She might be a friend to my dad and Maya, but she had no idea who this man really was or what he was capable of. She probably thought she knew, but she didn't. Get over it? More like go around it. Stay as far away from it as possible. That was my solution. None of this was her business anyway.

But she wasn't asking for herself, she was asking for him. If I understood nothing else from our conversation, I understood that. I knew he needed help. I didn't know if I was the right person to help him, or if he would even accept help from me. He'd never been one to depend on other people. Mercedes had climbed onto a perilous branch by calling me, putting herself in a vulnerable position between family members. Two of my my sisters had already basically told her to butt out. Mercedes was brave to keep trying, a testament to her deep friendship for Maya and my dad. I didn't want to respond to her courage with indifference. At the very least, I told myself, I could do something to ease her mind.

``Okay,'' I said. ``I'll fly out.''

\begin{center}$*$\end{center}

I pulled into my dad's driveway and parked outside his barn. I'd flown into San Francisco and rented a car, made the hour-and-a-half drive north to Sebastopol. It was late May. I'd waited until after final exams were over. I planned to stay for five days. My basic goal was to gauge my dad's living situation: to mark any significant changes or decline in his mental faculties, and to determine what his needs might be.

I always loved the drive up to Sonoma County. My wife Linda and I had lived in Missouri for fifteen years---we're both academics---but we'd grown up in the area, so this was home for us. California was in a severe drought at the time, but redwoods are evergreen, and though many of them were unusually brown around the edges, just the presence of those massive trees felt soothing to me.

I had good memories of coming to this property. My dad and Maya had bought the place in 1998, the year after Linda and I moved to Missouri. Whenever I visited the area, once a year or so, I'd swing by my dad's house, grab my surfboard and wetsuit, and head out to the coast for an early-morning session. My dad came with me once or twice, but for the most part I surfed by myself. So my memories related more to being back home in the redwoods and going surfing than seeing my dad. He'd be congenial enough and give me his usual bear hug, ask how the family was doing. He'd stow my gear in the barn once the wetsuit and board had dried off. He'd do all that for me, which I appreciated.

I loved his property: a quiet acre filled with trees, half an hour drive from the beach. I loved leaving the coast after a cold session in the waves and steadily heating up as I drove inland, the sun thawing my bones through the windshield. I'd arrive in Sebastopol completely relaxed, my mind and body as calm as the large redwood looming outside my dad's front door.

My interactions with my dad were minimal. He had an oversized personality. He came at you, if you know what I mean. Like a desperate salesman at the used car lot. My reaction to him was usually to grow quiet in a neutral way. He loved confrontation, so if he smelled resistance on you, he'd double his efforts. My strategy was to give as little fuel as possible to his flame, keep his burners on low.

I grabbed my bag from the backseat of the rental car and headed up the brick pathway toward the house. His Prius was parked in the carport. He'd had the car for several years now. It was the first popular hybrid that environmentally-conscious people were supposed to drive. They were all over Sonoma County. ``Hey, if you're not part of the solution, you're part of the problem,'' he'd told me when I first remarked on the car.

My slogan dad. If you didn't know him, you'd be impressed with his conviction.

\begin{center}$*$\end{center}

I knocked on the front door, which was mostly glass panels so you could see inside the house. The dining and living room areas were connected, one large L-shaped space. When no one answered, I stepped inside. My dad never locked his doors.

``Hello?'' I called out.

My dad came out of his office, which doubled as his bedroom. He wore his normal outfit: faded blue jeans, a tucked-in flannel shirt with a t-shirt underneath, thick grey socks, house shoes, and a brown leather hat with Maui scrolled across the bill from his time in the Islands during the mid-`80s. It was the kind of cap you might buy on vacation because you thought you'd look cool in it---what the groovy captain of a tropical cruise boat might wear---but then stuck up in your closet a month after you got home. It worked for my dad, though. Definitely a conversation piece that got him talking about his time living on the beach, carving driftwood for tourists, and surfing every day. An idyllic couple of years for him. In the outline for an autobiography I later found in one of his journals he called this period on Maui ``in most ways the happiest time of my life.'' I'm sure the hat held many good memories for him, but he wore it religiously every day, everywhere he went, so it was faded and sweat-stained.

We hugged.

He looked at my bag. ``Kay says you're staying here.''

``I'll be here tonight and tomorrow night,'' I told him. ``I'm having dinner at Kathy's tonight, but I'll be back.'' Kathleen was my older sister. She lived in nearby Santa Rosa.

``All right. Looking forward to it, man.''

``Me too.''

He stood a moment and waited for me to say something, which I didn't. ``Well,'' he said, pointing to his office with his thumb, ``I'm gonna finish watching my show.''

``Okay, dad. I'll drop my stuff upstairs.''

``Make yourself at home.'' He turned back to his office and added, ``glad you're here.'' He disappeared, closing the door behind him. ``Me too.''

\begin{center}$*$\end{center}

I must have been upstairs in my dad and Maya's room before that day, but I couldn't remember when. They'd owned the place for fifteen years by that point, but I'd never spent the night there. I'd never been invited to.

It was an old farmhouse they'd bought when they were almost seventy years old. They'd stripped it down to the studs and rebuilt it with the help of a contractor friend and my older brother Steve, who is also a contractor. This was to be the dream home of two soul mates who'd met in their silver years. Their artists' retreat. My dad converted the old garage---I called it a barn because that's what it looked like---into his wood-carving workshop. He built a straw-bale house on the back of the property for Maya as her art studio. They filled their home with canvases and carvings: her large abstract watercolors on the walls and his finely polished driftwood birds soaring from nooks and tabletops. Maya's artistic touch permeated the house in a soft whitewash that covered the knotty planks of the ceilings, bathing the space in warmth and light.

Kay was the one who'd suggested I stay with my dad. She thought it'd give me a better sense of his daily routine. I'd planned to stay with my mother-in-law Tam in Petaluma, a twenty- minute drive from Sebastopol. Whenever my family visits Sonoma County---Linda and our two sons, Miles and Ryan---we normally stay with Tam. But I'd agreed to spend the first two nights of the trip with my dad, a Wednesday and Thursday, to give Kay a much-needed break. The remaining three nights I'd be over in Petaluma.

The master bedroom upstairs was beautiful: a converted attic with open-beamed ceilings and large triangular windows on the far wall that looked over a meadow and redwood trees. The space was bright and open---art on the walls, books on the shelves, a cozy reading nook under one of the windows complete with stuffed chair, a soft blanket, and brightly-colored pillows. Maya's touch was everywhere. It was probably why my dad preferred to sleep in a chair downstairs in his office. The upstairs room had a stillness to it, everything perfectly in place, that told you it hadn't been used as a bedroom for a long time. Maya's clothes still hung in a walk-in closet. Her shoes were neatly arranged on the floor in there, her hats placed on the shelves above like she might come home any moment and need them for an outing to the beach or their favorite restaurant in town. They'd be exactly where she left them.

I imagined my dad began sleeping in his office to be closer to Maya, who'd ended up on a couch in the living room. At some point in his caregiving he could no longer push her up the flight of stairs to get her bathed and into bed. Certainly not for lack of trying, he later told me. She'd become dead weight in his arms. He didn't have the strength to get her upstairs anymore. This would've been after hauling her out to his workshop and propping her on a NordicTrack that he'd installed to keep her muscles toned. He'd always been a great believer in exercise, and I'm sure he'd read somewhere that it helped specifically with dementia patients. I'm sure that's true. But common sense might tell you that if you have to prop someone on an exercise machine, maybe they don't belong there.

But my dad had lost his common sense by then. This is probably fairly common in itself for someone in his situation. He'd been solely dedicated to Maya's care for a number of years, to convincing himself that he could make her well again. He just didn't know when to slow down or stop, or to ask someone for help. Was it his huge ego that wouldn't allow him to admit defeat? Was it his deep love for Maya, wanting so badly for her to be healthy again? Was it his own dementia taking root? I don't know. But reports from my sister Kathleen conjured up nightmarish scenarios of Maya's limbs swinging and jerking like a marionette on that machine.

If my dad slept downstairs when Maya had her night wails he could shout out from his office that he was there---she was okay, everything was fine. If she got out the front door in the dark, stepped through the woods in her nightgown over to the neighbor's house frightened because there was a strange man in her house, he could realize it sooner and go track her down--- tell her she was okay, everything was fine---and bring her home. Kay had reported that my dad spent most of his days now in that downstairs office: watching his favorite TV show, Two And A Half Men; eating his meals; taking naps in his chair. Kay and Mercedes had organized social gatherings for him, inviting his friends over for lunch and dinner. But after five months of doing this, they were burned out. Now it was time for his family to step in.

I dropped my bag on the floor upstairs. Maya's daughter had left clean sheets and pillowcases folded on the mattress after her stay the month before. I made up the bed and unpacked a few things. My oldest sister Terry had agreed to drive down from Oregon and attend a meeting I'd set up with the Alzheimer's Association in Santa Rosa for the next day. We were also consulting with an attorney to update my dad's will. My youngest sister Cindy, who also lived in the area, said she'd come to the meetings, too. I'd emailed all my siblings before the trip and gotten their two cents worth of advice, all of it encouraging and grateful that I was stepping in. I'd offered the job to any one of them who wanted it---assuming Power of Attorney to make financial and medical decisions for my dad---but nobody took me up on it. I hadn't expected them to, but I'd wanted to make the offer just the same. One last attempt, maybe, to get out of whatever I was about to get myself into.

That night I'd have dinner with Kathleen and her spouse, Cynthia. Kathleen is a public health nurse, and Cynthia used to have her own business in San Francisco working with the dying. They wouldn't be able to make the meetings that week, but I wanted to see them and get their insights into my dad's situation.

So the family was stepping in.

Just cautiously.

Beyond meeting with the Alzheimer's Association and lawyer the next day, my first priority on that trip was to get a handle on my dad's finances. He'd told me on the phone that he could only afford one more month of care for Maya at the skilled nursing facility. If that was true, their lives were about to change quickly. She had a care meeting on Friday, which my dad would attend. It'd be a good chance for me to see how he reacted in that situation, if he could possibly be responsible for Maya's care. She'd just started on Hospice, so moving her back home was a real possibility.

I'd know more about my dad's financial situation once I got into his office the next day.

\begin{center}$*$\end{center}

``Morning, dad.''

He was awake. I'd startled him by opening his door. ``Morning,'' he said. He yawned and stretched. He'd slept in his clothes, his feet propped on a footstool. His chair was one of those modern streamlined recliners that swivel on a wooden base: dark green leather all around, cushioned head pillow. It was comfortable enough for him to spend most of his days and nights in.

``Did I wake you?''

``No, I was just resting.''

``You have breakfast yet?''

He shook his head. ``I'm not really hungry.''

A portable room heater was running on high. It must have been eighty degrees in that office, which was a small space. Apparently he didn't sleep with blankets over him. He dressed in layers and left the heater on all night. Maybe all day, too. ``I'm going to make some for myself. Wanna join me?''

``Sure.''

I wasn't quite aware of this at the time---he definitely seemed different from his normal self: calmer; certainly more forgetful, but more agreeable too. I learned that my dad had to be ``cued'' or prompted to do routine acts: to eat, to go to bed, to take a shower. Probably Kay had told me to check with him at meals, but I hadn't yet thought of his behavior as a condition. I knew I'd woken him up and that he was probably hungry. To my mind he was covering the truth just out of habit. My dad didn't trust people. He often had a look in those sharp blue eyes of his, if you knew to watch for it, that was like a hawk perched on a branch: always looking for his advantage.

But that look was absent now, replaced by a genuineness I wasn't used to. I didn't connect it right away to his failing mental abilities. I thought maybe he was tired, or worn down by everything that'd happened with Maya. As I say, he was tricky to read. Normally he kept his cards so close to his vest you'd mistake them for a shirt pattern. But as long as he was being agreeable, I didn't mind playing along. My job that week was simply to engage with him and see how I might be able to help.

I made us breakfast. He liked bran cereal sprinkled with fruit and nuts; he took it with sweetened coconut milk, a cup of tea, and a one-a-day vitamin. His diet in general was healthy, part of a routine he and Maya had started years back. He believed diet and exercise were keys to a long, healthy life.

I hadn't stopped to shop for food the night before, so I ate cereal with him. After washing the dishes, I met him back in his office. He sat in his recliner. I settled into the chair next to him at the desk. ``You mind if I look over your papers?''

``Go right ahead.''

His response was so pleasant. It was disarming. I would've expected him to say something like Knock yourself out. He liked those punchy expressions.

Where to start? The lawyer had requested information on his bank accounts, government benefits, real property deeds, basically all current asset information: wills, tax statements, mortgages. The works. I knew about the green notebook. I knew my dad had called me the month before in a panic and said I need money, man! I knew Maya's care was costing them \$8,000 a month. Other than that, I was pretty much in the dark about their income and debts, or what they held in the bank. At least he allowed me to look over his finances, which was another example of this ``new dad from outer space.'' That's about how it felt to me. If a meteorite had slammed into his meadow and popped out an alien, I wouldn't have been more surprised.

I surveyed his desk. At least he was neat and orderly in a military way. Everything always had a place for my dad, and everything in that place would be carefully arranged and labeled. I recognized bills from the local utility company along with other opened envelopes wedged into sections of a steel mesh desktop organizer. Probably his most recent payments. I opened a drawer filled with files, all labeled in my dad's old-fashioned handwriting. The first file had become something of a catch-all, crammed with papers. I glanced through them: Medicare, Medi-Cal, VA, phone company, mortgage company, Social Security, bank statements, car insurance, cable, garbage collection, more utility bills. A separate filing cabinet beside the desk was also stuffed with folders and papers.

My dad watched me as I pulled out random forms and glanced through them, looking for dates and whether any payments were due. He seemed anxious. I set the papers down on the desk next to a small calendar where he marked his appointments. He'd gone for a physical at the VA two weeks before. Lauren Hibdon, of the Santa Rosa Alzheimer's Association, had recommended he get one before our initial meeting later that day.

``So you had a good physical at the VA?''

My dad looked at me.

I held up his calendar and pointed to Friday, May 10th. ``Your physical at the VA. It went well?''

He put on the reading glasses he kept in his shirt pocket and leaned forward. He studied the date. ``Oh, yeah. They sure know what they're doing.''

It was an answer I would come to recognize. When my dad wasn't sure what you were talking about, he'd pretend to understand by giving a stock answer vague enough to seem reasonable. He had two or three of them up his sleeve. Oftentimes his answer was preceded by a throat clearing, which bought him time. ``I got this little cold,'' he'd explain. He'd cough lightly once or twice, then come up with variations of ``That's what they say'' or ``She's a real pro.'' Sometimes he'd nod and affirm, ``I heard that. I've never tried it myself.'' This was especially true if you got him on the phone. It was harder for him to read the situation if he couldn't see you. He was so good at covering and so affable that people who didn't know him, even healthcare workers, wouldn't know he had dementia. I had to make a point of reminding dental hygienists or doctors' aides that his answers might not be correct about when his last appointment was, or if he'd ever had surgery, or whether he was allergic to something like iodine in a contrast dye. I'd assumed at first they'd know these things, that it would be on a chart they'd consulted before examining him or giving him an MRI.

Once my dad had actually been diagnosed with dementia at his physical, the Santa Rosa VA quickly organized a team for his care: primary doctor, psychiatrist, dentist, nutritionist, and social worker. He'd been a patient at the San Francisco VA for years. It's where he got his pacemaker checked every six months and his teeth worked on regularly (he had a dozen implants in his jaw by this time). He was still driving himself to these appointments, a three-hour round trip across the Golden Gate Bridge and through the Presidio and Richmond districts. My younger sister Pam, who lived in Oregon, had driven with him earlier that year. She mentioned that he'd had trouble finding his car after their lunch. That was another item on my checklist that week: organize transportation for my dad's appointments so he didn't have to drive.

Now with his VA team in place, many of those appointments could be handled in Santa Rosa, which was only a twenty-minute drive from his house. I liked the idea of other people overseeing his care. My dad had Kay, who lived on the property. He had the doctors and healthcare staff at the Santa Rosa VA. We'd consult with a lawyer that week about his financial affairs. The Alzheimer's Association would also be able to offer him services. I could organize all of that, put the pieces in place, and then all of those people would care for him.

This was one of my early delusions. Those people and agencies were all great resources, and there would be more resources for my dad as his condition steadily declined. Here's what I still had to learn sitting in his office going over his calendar: it was my job to oversee his care. If I didn't follow up on an appointment, if I didn't call back about a bill, if I didn't arrange a plumber or electrician to complete a home repair, then it generally didn't get done. I relied on a lot of people to help me do these things. I couldn't have cared for my dad without them. But I was ultimately responsible for anything that happened to him.

I remember the day that realization hit me. It was a bell-ringer. I was standing in my kitchen in Missouri talking with Linda. We were discussing some problem with my dad. I suddenly felt like a single parent watching over a child. With our two sons, Linda had always been there with me. We always shared the load as parents, making decisions together. As college professors with flexible work schedules, we had an ideal life to take turns caring for Miles and Ryan. But in any situation I knew I could defer to her judgement, especially when the boys were younger. She was their mother and would always have their best interests at heart. I could depend on that, which let me off the hook for assuming complete responsibility. I have no doubt that she bore most of the emotional weight of parenting through those years, of feeling I am responsible for these lives.

That's how I suddenly felt with my dad. I had Linda and my siblings to support me, along with my dad's friends and his caregivers. But he was my charge, and I felt that weight.

The question was, how much would I give of myself to ensure his well-being? He was not my child. I would not give my life for his as I would for either one of my sons. I had a job, a family, my own interests to pursue. I lived 2,000 miles away. How much time should I give this man who never had much time for me?

I turned the page on my dad's calendar. ``You have a lot of appointments coming up. Should we write them in?''

``All right.''

I held the page open and penciled in the dates Kay had sent me from the Santa Rosa VA. My dad leaned in close, and we repeated them together: June 10---MRI at the San Francisco VA; June 11---phone appointment with Dr. Smith; June 18---audiology.

``What's that?'' my dad asked.

``For your ears,'' I said. ``They're going to check your hearing.''

``My hearing's fine.''

``They're just going to check.''

He nodded. ``I guess they know what they're doing.''

``They do, dad. You're lucky to have them.''

``Oh, yeah. They're real pros.''

We went over his appointments for July---nutritionist, optometrist, dentist---until he got tired of the details and said he wanted to nap. I shuffled papers quietly while he stretched out in his chair and fell sleep. He couldn't sustain mental activity for very long.

I eventually found what I'd been looking for: a stack of recent bank statements. My dad's finances were fine. His panic over paying bills was a symptom of his dementia, not of dire financial straits. I discovered later that he'd been so worried about paying Maya's medical bills that he'd taken out a \$15,000 loan that he didn't need.

Other surprises awaited me in those folders: Maya's canceled health insurance, unpaid property taxes (the past two years), overdue car insurance bills. The property taxes were especially alarming because my dad had taken out a reverse mortgage years before: a condition of him continuing to live in that house was paying his property taxes on time. If he didn't stay current, the bank could foreclose on the loan and kick him out. My dad hadn't been able to figure out how to pay them, or that they even needed to be paid. I was able to clear some of the bills over the phone, then set up automatic payments off his credit card. Others would be more difficult because I didn't have power of attorney yet, and my name wasn't on any of his financial accounts.

But one thing at a time.

I closed my dad's door to let him sleep. I was due to meet my sisters and Kay in Santa Rosa at the Alzheimer's Association. After that meeting, we'd see the lawyer. I was hopeful that I'd get the answers I needed to arrange my dad's care and his finances. Once I'd set all that up, I figured my job was basically done.

\begin{center}$*$\end{center}

I was happy to meet Lauren Hibdon. She'd already been a big help. She was the one who'd recommended via email that my dad get a physical. She'd also given us the name of the attorney we'd meet later that afternoon. Lauren offered a great gift to me and my family that others continued all the way through my dad's decline: professional advice. I learned that I didn't have to bear this weight and stress on my own two shoulders. I could share the load.

Asking for help wasn't easy for me to do. Our depression-era parents had raised us to be self-sufficient and independent. We were all hard workers, following their example. In our family, if you were looking for a hand, you were told to check the end of your arm. We didn't expect other people to ``do'' for us. This was as natural to who we were as earthquakes and fire season are to California. Just part of growing up in Moser territory.

But I'd seen where that attitude, as formative as it was to my own sense of self, had gotten my dad in life. In particular with Maya's dementia, he hadn't asked for help---or he hadn't done it enough---and she suffered needlessly for that decision. I know he loved her. He tried to help her, and she trusted him completely. They'd cordoned themselves off willingly from family for their own reasons, to live out their dreams together, so they bear responsibility. But when Maya's mind was going, my dad could have made better decisions to give her comfort if only he'd been willing to ask for help.

I understood that my dad was an extreme case. Still, I was wary of following his poor example. And not just with Maya's dementia. With his life in general. He made a lot of selfish decisions. And yet he seemed to expect the rest of us to play along with whatever whim struck him at the time. Like pretending he was a war resister before it became acceptable, even admirable, to be one.

Or that he was Native American: a Cherokee Indian born on a reservation in North Carolina. The older he got, the more Indian blood he claimed until he went so far as to make handwritten corrections on his official birth certificate. After the name of the doctor who delivered him in Washington D.C.---S. L. Battles---he penned: ``NOT DELIVERED BY DR. BATTLES. DELIVERED BY MID-WIFE ON CHEROKEE RESERVATION.'' At the bottom of the certificate, he wrote: ``This Certificate was forged---Born in Jackson County, North Carolina @ 11:30 pm on 14 February, 1930.'' These statements were not a dementia-induced reverie made toward the end of his life. These were decades-old lies he told to Maya and their circle of close friends in Sonoma County so that he would appear . . . what? More interesting to them? More authentically counterculture? Somebody other than who he truly was?

I don't know. He had so many admirable qualities. Why weren't they enough for him? Why did he have to invent a life for himself based on lies? There was a reason why my mom, toward the end of her life, ordered a copy of his military records and left them for us kids to find. Because his lies affected her. I'm sure she felt that he was trying to erase the past---her past, too

---and she wanted to set the record straight. To her great credit, she wasn't one to badmouth other people, even my dad. So she simply left the evidence in her papers. That's how I know he worked as an office clerk in Virginia during the Korean War.

Lauren invited us into a small conference room at the Alzheimer's Association where we all gathered around a large oval table: Kay; Lauren and a younger associate; and me and my two sisters, Terry and Cindy.

A word about Terry. There's no easy way to say this. I've probably already waited too long. But my dad molested her when she was a girl. He visited her sporadically at night over three years and fondled her, from ten to thirteen years old. This began in 1966 after the family moved from West Los Angeles to Thousand Oaks, a valley suburb about thirty-five miles from downtown LA where my dad worked as a salesman. There were seven of us kids by then, and my parents had needed a bigger house. Terry and Kathleen (two years younger than Terry) ended up in a room together off the kitchen, on the far side of the house, which enabled my dad to visit them at night without detection.

The news came out to the family in the early `90s after Kathleen confronted my dad. He had also tried to molest her as she approached puberty, unbuttoning her shirt one time, but she'd stared him down, and he never tried it again. As Terry had endured my dad's nighttime visits, Kathleen had lived with the fear, guilt, and shame of being a witness: afraid that one night he'd come for her, shamed by feelings that she couldn't protect Terry from harm, and yet glad that it wasn't her in that bed, under those hands.

Terry and Kathleen never spoke about it as kids. This was the late `60s. We were raised in the Catholic Church. It never occurred to my sisters to talk to one another about it, or to tell my mom. Or to tell anyone. They didn't have the vocabulary to express what they were experiencing. They simply understood that they needed to endure what was happening to them.

By 2013 Terry and Kathleen had both worked through the trauma in their own ways, and worked it out separately with my dad. Neither ever trusted him again. Terry had made rules for my dad's visits with her own daughter, and she'd also warned the other mothers in my family. But my sisters managed to forgive him. Their stories are not mine to tell in full, but it's important to understand the emotional barriers they had to overcome to find compassion for my dad in the end. I had a less tortuous path to travel, experiencing neglect rather than physical or sexual abuse, but I met them at the same destination. I doubt I would have felt such compassion for my dad, or even started on my journey, if dementia had not altered his personality so drastically.

I'll also say a word about Kay. None of us kids knew her, and we weren't clear on her relationship with my dad. That is to say, how intimate it was. She lived in Maya's art studio at the back of the property. Maya had been in declining mental and physical health for a number of years. My dad liked women, let's put it that way. And they liked him. Our collective musings had nothing to do with Kay, since we didn't know her, and everything to do with what we did know about my dad. He'd been in a number of relationships since leaving my mom in the late `70s, and probably a few before he left her. He kept pictures of these women in an old brown attach\'e case I later found in his office closet. He also wrote about them in the journals he kept on his bookshelf. I'll have more to say about the journals later. But wherever my dad was in life, a woman wasn't far away.

Kay looked to be in her late forties, tall and attractive. She called herself a ``west-county hippie.'' Sonoma County was filled with them while I was growing up there in the `70s, though there hadn't been a distinction between east and west at that time. The subsequent growth of the wine and tourism industry, along with various tech booms in Silicon Valley south of San Francisco, had driven up the cost of living in the area---apparently in the east, closest to Napa--- which had given the west-siders a newfound sense of identity.

Kay farmed seaweed and gave massages. She'd known my dad and Maya for many years. She'd been one of my dad's cheerleaders during his seventy-mile walk in 2000. I learned that she was part of a group of self-dubbed ``mermaids'' who surfed and boogie-boarded the local beaches around Bodega Bay. My dad loved the ocean and rivers---a surfer, a diver, an avid kayaker in his later years---and I could see him being attracted to Kay and wanting to help her out with a place to live and part-time work. He loved to play the role of Sage, of Guide. He had a lifetime's knowledge about the ocean, and I could picture him happily mansplaining currents and rip tides to the mermaids: where to paddle out, the best place to catch waves, how to handle a wipeout. They may have known this information already and simply humored my dad through goodwill. He was friendly and funny with people, women especially.

At any rate Kay and my dad seemed very close, and her emails and phone calls showed that she cared a great deal about him. She knew nothing about his past, or his lies. As yet it wasn't my place to point them out to her.

Lauren was great in our hour-long meeting. After we went around the table and introduced ourselves, she allowed us to voice our concerns about my dad. Kay had the most detailed day-to-day information about his condition, so she provided a quick run-down of his memory problems, which she'd started to catalogue back in February of 2013: missed appointments, inability to remember basic names and dates, his repeated denials and frustrations.

Lauren listened and modeled behavior that I would later adopt with my dad: patience and emotional support. Above all she provided resources that gave us information we could use to make better decisions for him: like getting a physical, because only a doctor can diagnose dementia. She recommended the Sebastopol Senior Center and local support groups. My dad had been isolated for a long time caring for Maya, so more social activity would keep him stimulated. Her informational sheets on dementia were my first guidebook to this foreign territory I'd just entered.

I learned dementia wasn't a disease in itself but a term that covered a number of different symptoms involving memory loss and cognitive decline. Alzheimer's, she said, was the most common form of dementia. I learned the difference between ``senior moments''---normal gaps in my memory when I forget somebody's name or a fact I used to know---and the kind of memory loss that impeded my dad's quality of life: how to pay a bill, or when to take daily medication. I learned to reaffirm my dad's words, whatever they were, rather than contradict him. I learned to agree with his whims even as I diverted them so that we weren't standing there and arguing with one another. Or at least when we did argue, I could quickly realize my mistake and try not to do it again.

It was all helpful advice and a relief to know that many other people dealt with dementia on a daily basis and were willing to help us. Once we'd gotten the information and wrapped up the meeting, only one questioned remained: Would my dad accept any of this support?

He lived by himself. He still drove himself around in the Prius. He'd valued his freedom his entire life. Even in his multiple marriages and relationships, he'd always been a loner. And if he didn't like what was happening at any given moment, he'd be on the first bus out of town.

That could still happen---maybe was even more likely to happen given his current state of mind. Trying to tell my dad what to do would be like opening the cage of a feral cat and saying Here, kitty, kitty.

You'd better be prepared for claws and fangs.

\begin{center}$*$\end{center}

The next day we met at Fircrest Convalescent Hospital in Sebastopol for Maya's healthcare evaluation. There must have been fifteen of us crowded into a small room: Maya's doctor, her nurses, and other Fircrest staff; a Hospice nurse; a local minister; my two sisters and I along with my dad; Maya's daughter up from Los Angeles; and Maya's son listening in from the state of Maine via conference call. Maya was there as well, bundled into a wheelchair.

``Hi Maya,'' I'd said to her when I walked into her room before the meeting. I leaned down so she could see my face more clearly.

``You remember Pat?'' my dad asked.

She looked at my face. Hers was expressionless. She checked my features slowly. ``No,'' she said clearly.

``That's okay,'' I said. ``I'm Carver's son. You mind if I give you a kiss?''

``Well, no.''

I leaned in and kissed her cheek. She was 84 years old. She had been diagnosed with

Lewy Body Dementia. But even until the end of her life she would have moments of clarity and humor, love and recognition, especially with my dad. They'd been together for over twenty years by then, officially married for the past thirteen years. His visits pulled her from a veritable limbo of spoon feedings, brief changings, and sleeping in her wheelchair or bed. She was still fairly cognizant at this time, and her daughter wanted her to be part of her own evaluation. I was there to support my dad and make sure he understood what was happening with Maya's care. This would normally be a situation where he'd assert himself, make sure everyone in the room understood who was in charge. I sat next to him as we settled into chairs that were pushed all the way against the walls so we would all fit in the room. We introduced ourselves in turn, and the doctor and nurses started their reports. Most of them knew one another, so they conversed easily. They laughed about all of us being crammed in there together.

So far my dad had been fairly quiet, which worried me.

I was also distracted from the conversation. I'd golfed that morning with my oldest brother Mike who'd found out the month before that the cancer he'd been battling the past few years had spread to his lungs. Fourth stage melanoma. He was still active and looked healthy enough, but the cancer was spreading. Back in January he and and his wife Kathy had driven up from their home in Mill Valley and done an initial recon mission into my dad's office. They'd pulled financial papers and helped him figure out the basics of Medicare. This had been just over a month after Maya was admitted to the hospital. But Mike's cancer prevented him from getting too involved. He was actually listed on my dad and Maya's will as their First Trustee. He'd be legally in charge if my dad and Maya couldn't make decisions for themselves, which was currently the case.

Typical for my dad, he hadn't told Mike about naming him as Trustee. My meeting with the attorney the day before was to start the process of transferring legal responsibility over to my name. Even if Mike had wanted the job, which he didn't, he couldn't have served. He'd only been given six months to live.

``No one told ME about this,'' my dad complained loudly.

I lifted my head. A few people looked in his direction. Others shifted their gaze to the floor.

``She's been here FIVE MONTHS and this is the first I'm hearing about it.'' His voice bounced off the walls. Luckily they'd kept the door open so some of the sound escaped into the hallway.

One of the nurses said quietly that the information was in Maya's chart. Another nodded her head in support.

``Well no one told ME!''

The room fell silent. Five seconds, ten seconds. I was supposed to be my dad's advocate here. Now would be the moment for me to say something on his behalf: follow up on his complaint, find out why he hadn't been kept informed. I decided the best thing I could do for him was to let him speak his mind, even if he was out of his mind. I doubted he'd been able to follow the conversation very well; it concerned a switch in Maya's medications. He was frustrated, that was all.

I dropped my head and waited, wondering if my dad was done. The healthcare staff moved on with their reports. My dad settled back in his chair, his shoulders slightly sagging again, his gaze falling to his hands which he flexed compulsively a couple of times then folded in his lap.

He'd been agreeable on my visit so far, but I recognized him. I knew the man inside. I imagined the future as his dementia grew worse and he became more unhinged, more combative. His resistance to others had always been extreme. And he was so physically strong and intense.

What had I gotten myself into?

At least one good thing came out of the meeting: Maya had been taken off Hospice. She'd be staying at Fircrest a while longer. My dad would not have to care for her at home.

\begin{center}$*$\end{center}

My last morning of the trip I sat down with Terry and Kay to work out a plan for my dad's care. Kay was agreeable to taking on more responsibility, so we started mapping out what that meant exactly: making him meals, taking him to doctor appointments, driving him on errands, going on walks with him. My dad would pay her twenty dollars an hour, and she would keep a daily record of her activities and send them to me once a month. This would be my way of tracking his progress---or decline, I might say.

Kathleen and Cindy also offered their general support. That meant occasional visits with my dad. Or if I needed to call them in an emergency, they'd be a resource and local contact.

So things seemed more or less stabilized, which had been the goal of my trip.

My dad had Kay. She was now more than a concerned friend and renter: she was his primary caregiver. To my mind she was the best person to make decisions for him. She lived on the property and had daily contact with him. He trusted her and would want to please her. She was also skilled at gently working him for his own good. He'd already grown to depend on her a great deal. He knew he had memory problems and needed help in some areas of his life. That was a positive sign. He and Kay would work out his day-to-day life between themselves.

Once we wrapped up the details of her new responsibilities, I felt good about leaving.

I was happy to deal with the financial and legal matters from Missouri. I would write a letter to the Department of Motor Vehicles and request that my dad be retested. He shouldn't be driving. His doctor at the VA was required to file a report to the DMV after his dementia diagnosis. But that was the VA. Who knew if that letter would ever get sent. Mercedes also agreed to write a letter to the DMV: they might pay more attention if they received two separate requests. In the meantime, Kay said she'd try to convince my dad to let her drive him to his appointments.

I'd only be gone from California for two months. Linda and I had a sabbatical year coming up, and we'd decided to spend it in Davis, a university town two hours east of Sebastopol, near Sacramento. It's where we'd both gone to graduate school. My dad's decline wasn't the primary reason why we chose Davis. We knew the town already, so the transition to living there would be easy. The schools would be good for Ryan, entering sixth grade, and my wife and I would be able to pursue research projects at the University. I was also looking forward to spending more time with my brother Mike.

But my dad did factor into the decision. It'd be easier to take care of problems if I lived closer. My general plan was to visit him once or twice a month during that year: drive from the Central Valley out to the coast, check on him, and get in a surf session or two.

I received my first monthly report from Kay in June. She'd logged almost twenty hours, mostly arranging his medical appointments. All in all very manageable. She'd also sent an email about his resistance to going shopping with her or changing his diet. He frequented three restaurants in the area, always ordering either hot dogs, spaghetti, or BLT's, and he stubbornly refused to eat anything else.

He also insisted on driving himself. When Kay suggested maybe that wasn't such a good idea, he told her: ``I will fight you tooth and nail on this one.''


\chapter{}

``Alcatraz,'' I said to my dad. He followed my line of sight from Highway 101 to the center of the San Francisco Bay. Often the entire Bay was fogged in on summer days, but that morning, July 15th, was bright and clear. We were rumbling southbound from Santa Rosa on the San Francisco VA shuttle, headed to an optometry appointment to get my dad's eyes checked out.

``I lived there,'' he told me.

I cocked my head at him. ``Really?''

``With my Native American brothers and sisters.''

My first instinct was to turn away and roll my eyes, like so far back in my head I could see the brake lights on northbound traffic. I'd picked my dad up at ten that morning in Sebastopol to catch the noon shuttle. My family and I had arrived in California the week before, ready for our sabbatical year. My dad's driving topped my list of priorities. I hoped that by taking a day and going with him on the shuttle, he'd realize how convenient it was to have somebody chauffeur you to the City and bring you back. That way he'd want to take the shuttle instead of driving there himself. My general plan of attack was first to get him off the highways. Later we'd see what we could do about paring down his local driving.

He never took the shuttle again. Turned out that it wasn't that convenient. It left Santa Rosa three times a day during the week: 8 a.m., noon, and 4 p.m. Noon was the most reasonable time to get him up and fed and over to Santa Rosa ready to travel. Since it was an hour and a half trip, all his appointments had to be after 2 p.m. just to be on the safe side (traffic into the City could be monstrous). Appointments stopped at 5 p.m. The northbound shuttle didn't leave San

Francisco until 6 p.m. At the very least any appointment would be a nine-hour trek. We met veterans on that shuttle who traveled regularly from Eureka to San Francisco: they'd already been on the shuttle for five hours by the time we climbed aboard in Santa Rosa.

The way my dad's dementia progressed, the shuttle made little sense for him anyway. He couldn't be expected to make the trip by himself. He might not remember where to check in once he got there. He might not remember to bring his ID card. Or to pack a lunch for himself. Or to bring money for the cafeteria, which he might not recall they had. What if he missed the shuttle coming back? What would happen to him then?

Much of my experience with my dad's dementia was like being on the Hogwarts Stairways: I had to constantly adapt to changes in direction and destination. Sometimes my dad's needs changed slowly, and I could plan for them; other times they shifted with the wind, and I had to scramble to keep up. I learned that the best plan was to expect changes to the plan. I also eventually realized that where we were going wasn't always the most important part. It was traveling there together.

I didn't turn away from my dad and roll my eyes. It's terrible to say, but I egged him on. ``You lived on Alcatraz?''

``We went out there, took over the whole island. It was a big protest.''

This was a famous incident from the late `60s. A group of Native Americans had boated out there and claimed Alcatraz after the island had been decommissioned as a prison. As far as I knew, my dad was still living in Southern California with my mom and eight of us kids at that time. Because of my dad's stories, I grew up believing that I was one sixteenth Cherokee---he'd told us that he was an eighth---but I'd never heard a story about him living on Alcatraz. As I'd gotten older, and my dad's Indian blood thickened to fifty percent (he described himself in a short-lived blog from 2006 as ``Metis---one half Cherokee''), I eventually discarded the whole ancestry.

I don't know what imp came over me on that shuttle. I wasn't trying to be mean. I wanted to see how far he'd take it. Was is dementia? Did he really believe what he was saying? Surrounding himself with Native American accessories and spouting their beliefs was one thing. I'd gotten used to him calling elder Native Americans Grandfather. But taking over Alcatraz? Really? Why didn't he just tell me he'd spent time with Martin Luther King Jr. in a Birmingham jail? Lent the man his pen and scraps of paper to write on?

I turned and faced him. ``So you actually lived on Alcatraz.'' He nodded. ``Oh, yeah.''

``For how long?''

``A few months.''

He didn't meet my eyes. His gaze drifted out the window. His voice faltered at the end too, like maybe even he couldn't believe he'd just said that.

I didn't press him. I was a little sorry not to hear more. It would have distracted me from the jarring rattle of the wheelchair lift opposite us. We'd definitely grabbed the freshman-first- day seats on the shuttle. Every time we hit a bump or a dip on the highway (we must've been pushing seventy in the fast lane) the whole steel contraption slammed and shuddered like a broken elevator landing in the basement.

We arrived in one piece. The shuttle stopped outside Building 200, the main entrance and check-in point for most veterans. The Medical Center is quite the campus. It sits on a beautiful rise at the end of Clement Street, nestled between the Golden Gate Bridge to the north and Golden Gate Park to the south. I knew the area mostly because of its proximity to Ocean Beach, a nearby stretch of coast famous for its treacherous waves. The Medical Center at Fort Miley had a full hospital and a Community Living Center where veterans with dementia lived full-time. It wouldn't be long before I'd put my dad on their year-long waiting list.

I got him settled in the waiting room. I'd shared my lunch with him on the shuttle, so he was fed and relaxed. I kept an eye on him as I waited in line to check in, but he'd been coming to the Medical Center for a number of years so he was comfortable there. The place was a bee-hive of activity, but more like bees that had been smoked: nobody was buzzing too quickly. My general experience with the VA was once you got in to see the doctors and other healthcare workers, they were caring and very professional. The frustrating part was simply cutting through the clusterfuck of forms and procedures and approvals and referrals and message-machine call backs clarifying mind-bending cul-de-sacs of outdated directions to actually get into the same room as one of those professionals. The care was entirely free for my dad, which was fantastic. Neither of us minded the waiting. But the lack of clarity and coordination in the VA system was enough to drive you to dementia, if you weren't already there.

We didn't wait long. A young optometrist examined my dad's vision as I sat in a chair at the open end of a three-walled module. The doctor was polite, very good with my dad. The Medical Center had a long connection to the University of California, San Francisco Medical School, so you saw a lot of med students walking the hallways in their colored scrubs. For his part my dad was glib, making jokes and answering the doctor's questions with ``yes sir'' and ``no sir.''

I watched my dad as if I were attending a play. I knew the actor well, had seen his virtuoso performances on many stages over the years. So I assumed I already understood his range: all his blocks and turns and practiced flairs.

But I was truly mesmerized.

I was so struck by my dad's desire to please this young doctor. He was so sage, as the French say. Like a schoolboy on his best behavior. I wouldn't have been surprised to hear him repeat a line straight out of Leave It To Beaver (``I'lll take the garbage out right away, Mrs. Cleaver. Yessiree.'')

My dad had always been charming, but as far as I could tell this wasn't an act. And I had my spyglass out. I would spend a lot of time with my dad during that sabbatical year. I was in the beginning stages of understanding that dementia had transformed his personality. This awful condition had stripped him of a bear-hug ego which could be, frankly, unbearable. He was docile with the optometrist, grateful for every little service rendered him. He was Mr. Yes Sir; No Sir; Thank You Sir, May I Have Another. It was hard to believe this was the same man who loved confrontation, whose self-centeredness and deep insecurities burned so much oxygen in a room that it was hard to breathe around him. It left the eight of us kids in various states of deprivation, sucking for air and emotional connection. Thank God for my mom, whose love and deep sense of family saved us from suffocating.

I had no answer for this transformation. I wasn't ready to believe it was complete, but I also couldn't see it as an extended ruse. He truly seemed a new person, one that I actually didn't mind being around.

We got him fit for two pairs of new glasses. He'd get them in the mail in a couple of weeks. My dad couldn't remember the last time he'd had his eyes checked. I felt good about what we'd accomplished that day. I thought: Well, if he insists on driving, which he can legally do since he has a license, maybe an up-to-date prescription will help prevent an accident. This was my denial: thinking his eyesight might be the problem rather than the progression of his dementia. Eyesight was the easier fix.

I later heard someone describe dementia as having a swiss-cheese brain: the holes were random. My dad might not remember to eat, or to take a shower, or how to put a registration sticker on his license plate. But he knew how to start his car and drive himself over to a Board and Care facility in Forestville where Maya had recently been moved. The Hogwarts Stairways again: you couldn't plan for which connections were going to work. You had to go through them one at a time and figure it out. Sometimes by the time you figured it out, they'd already changed direction again.

I was relieved that Maya had made the transition from the skilled nursing facility. The Board and Care was more homey and about half the price. The downside was that Maya now lived twenty minutes away from my dad. Twice a week he would drive himself through the backroads to spend time with her.

\begin{center}$*$\end{center}

The driving situation hit crisis mode at the end of the year.

I'd gotten settled in Davis with Linda and Ryan. My oldest son Miles had just started college in Arizona, so we were all living in the West. I hadn't visited my dad as much as I thought I would during the fall months. I had good excuses. I developed an odd vertigo condition, which I was seeing a physical therapist for. It hit me randomly and made me nervous about driving on highways. It was frightening to see everything outside the windshield suddenly somersault when you're driving sixty miles an hour.

I was also spending long hours every day on my research project, which involved sifting through and cataloguing hundreds of newspaper articles pulled from online databases. The work could be tedious and mentally exhausting.

The two situations were probably related: vertigo caused by eye strain and fatigue from staring at a computer screen ten to twelve hours a day. That is to say, I didn't develop vertigo to get out of visiting my dad.

At least I don't think so.

It's true, I saw my visits more as a duty. I was in Trustee mode rather than Loving Son mode. I worked with his bank to organize his and Maya's financial affairs and followed up on bills or any house repairs that Kay mentioned to me. My primary responsibilities, as I saw them, were legal and administrative. I could do those things from Davis. I'd hired Kay to live on-site and manage my dad's day-to-day needs.

And that was going well. The two of them had developed a nice routine. She made him lunch and dinner every day. My dad had been getting out of his office and busying himself in the yard. He still insisted on driving, but so far there hadn't been any repercussions. Kay was averaging fifteen hours of work every month, pretty light overall. My dad was in the stage of mild dementia. He had memory issues, but he was physically strong and healthy. He needed cueing, but he could dress himself, make his own breakfast, and manage his hygiene.

He did have a fall at Thanksgiving. He'd had dinner at Kathleen's house and later got sick in his bathroom and ended up bumping his head on the toilet role dispenser. Kay found him the next morning on the bathroom floor and got him back into his room.

But that'd been an unusual situation: the food not sitting right in his stomach. Normally he was so healthy. I talked to him about getting a medical alert device that he could wear on his wrist or around his neck. That way he could contact Kay if something like that ever happened again. But even if my dad agreed to wear the device---he ended up refusing, so that stairway went nowhere---the real problem was that once he pressed the alert button, the company automatically called an ambulance. That seemed like overkill. At the same time I didn't want him spending any more nights on the bathroom floor. We eventually bought a baby monitor so Kay could listen in on him from the art studio, which was about a hundred feet from the house.

My dad's situation was generally stable, so I felt better about seeing him once every couple of months during this period. But his driving stayed on my mind. It was an obvious concern even though he insisted he was fine whenever anyone brought up the topic. If Kay pushed him on not driving, he got pissy and obstinate. We'd already written letters to the DMV to have his license revoked. So far I hadn't heard anything back. Until I did, I was left to rely on Kay to monitor him as best she could.

But I still worried about it, which bothered me. I didn't want to have to worry about him or his driving. I wasn't sure what to do. He wouldn't respond well to ultimatums. If I went in strong and took his car away, I wanted to have something in its place. How would he see Maya? Or get to his appointments? Or lunch at his favorite eateries in town and at the beach? Kay wasn't always around. Even so, her old car wasn't too reliable. My dad was still fit and strong. I could easily imagine him getting upset and marching the mile or so into town, or the eight miles over to Forestville, just to prove that he could do it and give us the finger. He'd walked those narrow, twisty roads many times before. I didn't want him wandering around Sonoma County and winding up in ditch someplace.

I could leave the car in the driveway and lie to him: tell him it didn't run and needed repairs. Put him off indefinitely. The problem was he'd drive it into town and get it fixed. He'd already done that before. One time he couldn't figure out how to turn his heater off. He thought it was broken so he took it into the dealer. Two hundred and fifty bucks later he returned home with a heater that he would forget how to work again. He did the same thing with yard tools: a weed eater, a generator, the water pump filter. He used to do all those repairs and replacements himself. When he could no longer remember how to run them, he assumed the damned things must be broken.

But he would also have moments of keen self-awareness. He'd admit to Kay, ``My brain doesn't work anymore. I muddle on through. Without you I wouldn't be able to do much of anything.''

When Kay sent me reports like that, my heart went out to him. He sounded so vulnerable.

And he was. A couple of days before I came for a visit in October, he drove himself to an oral surgery appointment at the San Francisco VA. On the way back, City road crews detoured him off his normal route, probably only a block or two. But he couldn't find his way back to the Golden Gate Bridge. It took him several hours of driving around to get his bearings. He didn't get home until late that night: exhausted, famished, upset. Kay had called the California Highway Patrol when he wasn't home by 7:30 p.m. Luckily the only thing hurt was his pride.

After that experience he finally agreed to let Kay drive him to his next appointment. It turned out that if was he able to follow the same route he always took, he could manage. But any glitch along the way that forced him to problem-solve would completely derail him. He simply couldn't put two and two together.

\begin{center}$*$\end{center}

The point of no return on my dad's driving came when he wasn't even behind the wheel.

My older brother Steve stayed with my dad at the end of December. Several of us kids were taking turns with him to give Kay a break over the holidays. Steve and his daughter Jackie, who was Ryan's age, had driven up from Southern California to spend a couple of days at my dad's place. On this particular day, December 29th, they drove him out to Bodega Head, a bulk of land that forms the northern edge of Bodega Harbor and overlooks the town of Bodega Bay itself. There's a marine laboratory out there and lots of hiking trails for visitors, most of them safe if you stay back from the edges.

It was one of those beautiful winter days in Northern California: clear sky, no wind, sunny weather. The coast there is typical of Sonoma County beaches---lots of rocks to climb on and wildlife to check out: seals and dolphins and even whales during migration season. The abundant sea life also draws predators. Bodega Bay is the northern limit of the Red Triangle, an area infamous for its high density of white sharks. I've never seen one in the water myself, but it's in the back of every surfer's mind out there.

The three of them hiked down to a sandy beach just north of the parking lot where my dad posed for a few pictures that Steve later sent me. My dad looks good in the photographs. He's wearing blue jeans and his brown house shoes; a green and blue windbreaker that he must have borrowed from Steve. In one picture he's staring off in the distance toward the ocean, a place he loved to visit even when he could no longer push through the waves on his surfboard.

In the background of another picture I see my brother's white pick-up truck parked on the bluff overlooking the beach. It's where my dad asked if he could rest after they'd poked around the tide pools and watched Jackie climb rocks. My dad was feeling tired after their little adventure. He'd managed to get down to the beach and back up to the parking lot all by himself ---no small feat for an eighty-three-year-old man with dementia. But he'd also refused any help. He'd wanted to do it all by himself. Steve and Jackie still wanted to look around some more. One of our surfer friends had told Steve about a spot on the south end of Bodega Head. A small swell was running, and Steve thought there might be waves breaking.

``Do you mind staying in the truck?'' Steve asked my dad.

``Okay,'' my dad said. ``I'll stay in the truck.''

So Steve and Jackie left. They went to find the surf spot. They also checked out the area where the county utility company had planned, back in the 1960s, to build a nuclear power plant just west of the San Andreas earthquake fault (that idea didn't get very far). My brother later told me they were gone twenty minutes.

In organizing family to stay with my dad over the New Year, I hadn't told Steve not to leave my dad by himself. It didn't occur to me that he might do that. In my brother's defense, he hadn't spent enough time around my dad to know that this might cause a problem. How would he know? It's one of those small gaps in communication that leads to life changes.

As Steve and Jackie walked back to the parking lot, my brother could see his truck. My dad wasn't in it. Steve started scoping the area---the parking lot, the beach below. Maybe my dad had walked back down to the water? Steve's brain immediately started clicking, like when your kid suddenly disappears in a public place: What's the next step? Who should he ask? He looked for people on the trails. Maybe one of them had seen an old man wandering around?

My dad had been tired, and some of those trails skirted sheer drop offs into the ocean. The day was warm and sunny, but water temperature in December averaged in the low 50s.

How long could he last in the waves?

Steve approached his truck, eyeing the cab just in case he'd missed something. He passed a black two-door coupe on the way. There was my dad sitting inside the car, waving at him from the passenger seat.

Steve opened the door: ``Dad? How you doin'?''

``Good. I was waiting for you to come back.''

``This isn't my car.''

``Yes it is.'' My dad smiled. ``You know, I found a thousand dollars in a wallet under the armrest.''

Steve stared at the console between the seats. He checked outside the car, hoping the owner wasn't in sight. What would have happened if some guy had caught my dad sitting inside his car, going through his wallet? Or had my dad imagined finding the cash? Steve wasn't sticking around to find out. He checked my dad to make sure he didn't have any large bills stashed inside his pockets, then he said, ``Let's go, dad.''

``Okay.''

Once they'd gotten back into my brother's truck and sped out of there, my dad told him that he'd gotten tired of waiting and went to look for them. When he couldn't find them, he'd wandered back to the parking lot. As soon as my brother made my dad realize that he'd climbed into the wrong vehicle---the two couldn't have been more different: a black coupe and a white truck?---my dad started to laugh.

``Oh, Kay's gonna love that story. She'll laugh. Wait until I tell her.''

Well, when my brother told me, something inside snapped. Right then I decided that my dad wasn't driving any more. He was getting in other people's cars and going through their wallets? What was next? Entering an off-ramp and driving the wrong direction on the highway?

I don't know why that particular incident hit me so hard. Yes, it could have ended badly. But looking back, his detour in the City two months prior was more harrowing for him. And that could have ended much worse. I'd been hearing stories of his memory lapses and cognitive decline for the past year. I'd been monitoring him from far and near. I'd let him drive me out to the coast that fall to see how he did behind the wheel. But I hadn't been ready to force the issue and take his keys until Steve told me that story. My dad had been laughing about what he'd done. I finally realized he was out of his mind.

Literally.

\begin{center}$*$\end{center}

It was time to make a plan. I spoke with Kathleen. She talked to my dad about not driving during a Sunday visit, one week after the incident at Bodega Head. She and Cynthia had gone to his house to watch the 49ers play the Packers in the playoffs. My dad was a 49ers fan, and watching TV was a low-stress activity they could all do together. Kathleen told him she was concerned that he might take a wrong turn and get lost. He totally balked. He started to spin some story to cover himself. She wanted to take his car keys right then, but she didn't know if I or Kay might need the car to take him someplace. She later emailed me: I do feel strongly he should not drive, even in Sebastopol, even sometimes. To be truthful, my worst fear is not him getting lost. It's that he will have a memory slip and mistake the gas pedal for the brake and plow though a group of children in a cross-walk.

I completely agreed. At that point I thought such a scenario was entirely possible. I made an appointment with my dad's doctor at the Santa Rosa VA to see if she could help. Her earliest opening was January 14th. I planned to take my dad and have the doctor examine him and tell him that he shouldn't be driving. He tended to respect authority figures, so I hoped she'd have more luck convincing him than we did. Kathleen said that she wanted to come, too. She was familiar with the DMV revoking a license for medical reasons because those forms were processed through her office at the Department of Health Services. Once his doctor signed the forms, she could alert the DMV and get his license suspended.

I called my dad. He didn't want to go the doctor's appointment. ``I've got plenty of work to do around the house,'' he told me. ``I forget things sometimes, but it's not a problem.''

Three days later I called him again. It was Sunday, January 12th.

``I've already taken care of that driving problem,'' he assured me.

``I know,'' I said. ``I still want you to see the doctor. The appointment's on Tuesday.''

``I don't know about any appointment. I don't need to go, I don't understand what this is about.''

``We're worried about your driving, dad. We want you to see your doctor.'' ``It's all taken care of.''

Since my dad wasn't going to the appointment---I couldn't kidnap him---and Kay couldn't make it either, I asked Kay to make a list of my dad's recent memory issues. At least the doctor could look that over and hopefully agree to sign the form. Here's what Kay jotted down:

Carver doesn't always know what day it is in relation to his calendar. Every day since Dec. 23rd Carver has been expecting Cindy for dinner. X-mas eve he left the porch light on and waited all night expecting her. Every day I let him know that she is coming on Monday the 30th for dinner. We look at the calendar and go over the dates. It shows Sunday the 29th Steve and Jackie are coming, and Monday the 30th Cindy is coming for dinner @ 6:30, and Tuesday Terry is coming. It's all written down. Then I count down from the present and try to spell out clearly where we are today in relation to the upcoming dates. I slowly go over the dates while we look at the calendar till he thinks he's got it, but he doesn't get it. Then he'll come to me a short while later having erased the visit with Cindy on the 30th and moved it to the 26th, convinced that she is coming over today. I told him changing the time won't make it so. Cindy is not coming tonight for dinner. The next day it was the same thing. He thought Cindy was coming for dinner and once again I took the time to explain what day it was in relation to the 30th. This scenario continued until Steve and Jackie arrived on Sunday.

He is unable to operate the CD player. The remote is pretty straight forward. I show him daily, sometimes several times a day, which button to push to control it, but it doesn't stick.

He falls apart in high stress situations and tends to make things more complicated.

It has been a general rule that I do my own laundry and dishes. If my dishes are in the dish rack, I request that he let them dry and I will collect them asap. Even when I remind him, within an hour he will have put away all my dishes that don't belong in his kitchen. Then I have to search them out and hopefully retrieve them. I know now that he always throws away my tupperware.

While I was doing laundry, unbeknownst to me, he put all his dirty laundry in with mine. I found him putting the wash into the drier but it hadn't gone through its spin cycle.

I refuse to drive with Carver. He's confused and overreacts, applying his brakes arbitrarily and causing near accidents.

Kathleen and I went to the appointment on January 14th, Kay's list in hand. The doctor listened to us and signed the form. Later that day Kathleen sent it to the Santa Rosa DMV. At that point my dad's license would be suspended. The DMV would send him a letter and give him a chance to appeal, but any appeal would have to be signed by a doctor.

Believe it or not, the administrative work was the easy part.

\begin{center}$*$\end{center}

The next day I got a call from my dad.

``I want my car keys,'' he said.

I'd asked Kay to grab them from his office so that my dad wouldn't drive. When he couldn't find them, he'd walked out to the art studio and knocked on Kay's door. She offered to drive him anywhere he wanted to go. ``No,'' he told her. ``I want to do it myself, damn it.''

She referred him to me, closed the door, and called Kathleen.

I explained the situation to my dad over the phone: ``I visited the doctor yesterday, dad. You remember? She said you can't drive.''

``I already took care of that problem. Kay said I needed to talk to you. I want my keys.''

I could hear he was upset. I calmly explained to him about the form and what that meant. He kept insisting he could drive. I put him off: ``If you want, we can set up an appointment with your doctor and get it all straightened out. I'm happy to call her.''

``Okay,'' he said. But he didn't sound like he was willing to wait.

That evening Kathleen and Cynthia drove over to my dad's house with a steering wheel lock for the Prius. They knocked on Kay's door first. There was no answer. They went inside the main house to confront my dad. They'd worked out their good-cop, bad-cop routine. Kathleen had anticipated just this scenario. She'd emailed me the week before: My experience with taking away the car in the elderly population is it is a

F-I-G-H-T! So I'm willing to be the bad guy here so you can still have access to Dad and meet his other multiple needs---and he can blame me.

They sat down with my dad in the dining room. ``We talked to you about this,'' Kathleen began. ``The doctor said you can't drive.'' She wanted to be gentle with him, but firm. For her it was a question of other people's safety, and she wasn't going to compromise.

Cynthia chimed in, working on my dad's Native American sensibility. She stroked his arm, and his ego: ``You're our elder, Carver. It's time for us to take care of you. This is the cycle of life. Now it's our turn.'' Cynthia has a lot of discipline when it comes to saying the right thing for the right reasons. I'm sure she kept a straight face the whole time.

And it worked. He looked at them both. ``Yeah, you're right,'' he said. ``It's time for that.'' Kathleen nodded, relieved. ``So you agree not to drive.''

My dad shook his head. ``Nope, I never said that.''

``If you drive,'' she warned, ``I'm going to call the police.''

``Then I'll see you in court!''

Kathleen didn't want to rile him up more than necessary. She told him they'd have to agree to disagree. She and Cynthia went out to the Prius and put on the wheel lock in case my dad tried to hot-wire the car---she didn't put it past him---or smooth-talk a neighbor into getting him the keys when Kay wasn't around. They knocked on Kay's door again to give her the keys to the wheel lock. This time Kay answered. She'd been doing yoga before, de-stressing from having to cope with my dad's anger. Kathleen offered to take the Prius right then, but Kay needed it to run errands for my dad.

The next morning Kathleen called the police. She wanted them to be aware of the situation: the doctor's visit, the forms, the DMV. She gave them my dad's driver's license number, which I'd given her. Early on I'd made a copy of all of my dad's ID cards in case he ever lost any of them. When the police looked him up on their computer, they discovered that his license had already been suspended.

Our letters to the DMV had actually worked! We didn't know about it because my dad had gotten rid of the letter that the DMV had mailed to him informing him of the suspension. He probably didn't understand what the letter meant and tossed it into the recycling bin. Or else he had understood and simply threw the letter away.

The policeman on the phone offered to have a patrolman meet Kathleen and Cynthia at my dad's house. The officer would talk with my dad and get him to hand over his driver's license. It my dad refused, the officer would take it from him.

Oh, boy.

\begin{center}$*$\end{center}

When they arrived at my dad's house that afternoon, Kathleen asked Cynthia to stay in the car. That way my dad wouldn't associate Cynthia with losing his license. Cynthia had offered to drive him places once he no longer had his license, and Kathleen wanted to keep that relationship in tact.

Kathleen didn't call my dad ahead of time. She figured she'd surprise him.

The officer arrived on a motorcycle. He was about thirty, big and strong, the kind of man my dad would have admired and respected. As they walked up the brick pathway to the house, the officer adjusted his gun, getting ready for whatever he might have to do. He glanced at Kathleen. ``I hate doing this,'' he said under his breath.

Kathleen understood exactly what he meant: pumping yourself up to deal with people, coming off as the hard-ass. She'd always had to put on armor when dealing with my dad. You just never knew what he was going to do, how he was going to react.

They knocked on the door. No answer. Kathleen stepped inside and called out. My dad emerged from his office. He saw the officer and said, ``Oh hello, sir.'' ``You mind if I come in?'' the officer asked.

``Not at all.''

They sat down in the dining room. ``Mr. Moser, can I take a look at your license?''

My dad retrieved it from his office and handed it over. The officer looked him straight in the eye. ``I'll be taking this license because you're not allowed to drive.''

``Yep,'' my dad said. ``That's all taken care of. I'm selling the car.''

``Okay,'' said the officer. ``You realize if you're caught driving, you will go to jail.'' ``Yes sir,'' my dad said. ``Nothing to see here. I'm going to save a lot of money on insurance.''

Once the officer had the license in hand, he stood to leave. My sister intended to leave with him. She wasn't hanging around for any fall-out that might land on her. As they went out the door, my dad lifted his voice, ``A pleasure meeting you.''

Kathleen was so mad. She filled me in later over the phone. ``This is what he does to you,'' she said. ``He gets everybody wound up, then acts like it's no problem. My stomach was in knots getting ready for a fight. He looked like the nicest old guy standing there. He wasn't fumbling around. He found his license and brought it out right away.'' She let out an exasperated breath. ``He made me look like one of those daughters who's out to get her parents. I'll see you in court! doesn't come out in front of the police officer. Instead it's Yep, we've already talked about it. He wasn't acting like that twelve hours ago, let me tell you.''

Three days later she was back at my dad's place to watch the 49ers lose to the Seahawks by a couple of field goals in the NFC championship. They'd already arranged the visit before the whole police scenario. Kathleen decided to go and act like everything was fine. She and Cynthia brought dinner. He welcomed them. But she could tell he hadn't forgotten.

It was so frustrating for her. He forced her to be the bad guy and tell him not to drive when he already knew he shouldn't be driving. From long practice with him--- starting when she was a girl, staring him down as he unbuttoned her shirt---she had to gear up and prepare for these emotional battles: to know her leverage, what she would take from him, and where her bottom line was. Because she might have to go there fast. That was the Phil Moser treatment, always looking for his advantage. And if he couldn't find one . . . well, he was off to the next opportunity. No remorse. No looking backward. Go with the numbers he'd once told me, giving me sales advice: some buy, some don't. Who cares? Move down to the next door. Kathleen didn't have people like that in her life anymore. She avoided them at all costs. So dredging up those martial arts drained her physically and emotionally.

The three of them sat in close quarters in my dad's office, watching the football game. During a commercial break, my dad said, ``Kay wanted to know what that piece of cardboard was the officer took away.''

My dad had developed a strategy of pretending that other people wanted to know things that he'd forgotten. That way he wouldn't have to admit that he didn't know himself.

``That was your driver's license, dad.''

Kathleen could see the form the officer had given my dad sitting on his desk. She'd noticed that Kay's car was gone when they'd pulled into the driveway, and the wheel lock was off the Prius. She wasn't sure what that meant. Kay had mentioned to me on the phone that my dad had been busy cleaning out the Prius the day after the police showed up. He'd wanted Kay to drive him to the Toyota dealer right away so he could sell it. In fact he'd offered to sell her the car for one dollar. That police officer must have really scared him.

A commercial for my dad's auto insurance company came on the TV during a time-out. ``Oh, yes,'' my dad said. ``We'll be parting ways because I'm not driving.''

\begin{center}$*$\end{center}

I ended up selling the car to Kay, but for a fair price. Not only did she need a car, but I thought keeping the Prius around might help my dad's overall transition. He'd still see the car in his driveway and be able to ride in it, he just wouldn't drive it himself. He seemed to accept that finally. He never tried to drive again and didn't mention missing it, at least not to me. My dad always had a strong will. Once he decided something in his mind, there was no looking back. You didn't walk seventy miles in twenty hours by checking your rearview mirror.

My dad got the last laugh about his driving, anyway. It was late March by the time Kay and I finalized the car sale. There'd been no big hurry as long as my dad wasn't driving. It turned out that my dad had been doing more than cleaning out his car the day after he got his license revoked. When I called his insurance company to cancel his policy, they told me he'd already done that back in January. Kay had been driving him around for two months without insurance.




\chapter{}

``Come in, Carver.''

``Hello there,'' my dad said.

We filed into Kathy Vincent's office---me, my dad, Kathleen---and sat in front of her desk. Kathy was my dad's social worker at the Santa Rosa VA. We were there so she could assess his needs and what benefits he might be eligible for. It'd been two weeks since the police had taken away my dad's license. Now that we had his driving situation under control, we turned to improving his healthcare needs. Kathy became our guide to the multiple resources available through the VA, if my dad qualified.

Kathy looked to be in her late forties. She gave a good first impression: calm, competent, caring. She had a no-frills look of someone who worked a lot under difficult circumstances.

We introduced ourselves and chatted a bit. My dad, sitting between me and my sister, was friendly. He wore jeans, a jean jacket buttoned up over two shirts, and his Maui hat. He came to this meeting because we'd told him that we were looking for ways to get Kay more money for the care she provided him. He and Kay had grown very close, so I figured he would want to help her out. It wasn't a total lie. Kay was interested in getting certified as a healthcare worker. If she did, then the VA might pay her salary, depending on the agency she contracted through. That never ended up happening, but it was critical for Kathy to see my dad in person and interview him. So he had to be at that meeting.

I summarized our concerns for Kathy. My dad's needs were increasing dramatically. Kay had averaged fifteen hours of care a month from the previous May to November. In December, her work had doubled to thirty hours. That month (January), she was on track to log forty-five hours. And the type of care she provided was also changing: she now had to monitor him taking a shower---giving him soap, shampoo, and hand mitts to scrub himself, then laying out clean clothes for him afterward. She'd begun to notice fecal smears around the toilet seat. ``Basically my dad needs more help,'' I told Kathy. ``A professional who can come to the house and give Kay a break.''

Kathy listened, then turned her attention to my dad. ``Is that right, Carver. You need more help around the house?''

``I do all right.''

``Can you tell me why you're here today?''

``I'd like to get Kay some help. She's a hard worker.''

``I'll bet she is.''

My dad sounded entirely competent. I was worried that he'd put on a good performance for Kathy, and we'd get nowhere. Then she said, ``Can you tell me what year it is, Carver?'' My dad smiled. ``Well . . . .'' He glanced down at his hands. After ten seconds or so, he looked up and said, ``I guess it's slipped my mind.''

``What month is it, Carver? Can you tell me that?''

He shifted in his chair. He nodded. ``December, I think.'' He looked to me for confirmation. I had to glance away. My eyes landed on a calendar pinned to the wall behind him. I reminded myself: This test is for his own good.

``It's January,'' Kathy said.

My dad gave a small laugh. He flexed his eyebrows. ``Pretty close.''

``What day is it, Carver?''

My dad straightened in his chair. ``Oh, it's a great day.''

``Can you tell me the date today?''

My dad's easy smile faded. His eyes dropped to the desk. We sat quietly and waited. If there'd been an old clock on the wall, we would have heard the steady tick, tick, tick. As a salesman, as a man who'd always wanted to impress people and win them over, so much of his life had depended on a quick mind and glib tongue. In April of 1969 he received a certificate for completing the Dale Carnegie course on Effective Speaking and Human Relations. Around that same time he'd been named Salesman of the Year at Premier Fastener Company in Los Angeles. His journals are filled with rules and strategies that salesmen memorize to answer every question or respond to any objection a customer might raise about buying a product. He was a master salesman because he was so good at reading people and telling them exactly what they wanted to hear.

He sat next to me now in complete silence: unable to give the right answer, unwilling to admit he didn't know it, incapable even of covering himself with a funny remark. His eyes betrayed a mind desperately searching for an answer that he knew he should know. But he couldn't for the life of him grasp it, or even figure out why it was out of his reach.

I suddenly felt sad for him, sitting there, struggling. I wanted the test to end.

Kathy had gotten her answer. She told him it was the 30th.

I don't remember my dad's reaction. He probably smiled, perhaps made some small remark as we continued to chat with Kathy. She had put him through the ringer. But he recovered by the time we got back out to the parking lot. ``I knew the answers to her questions,'' he told us. ``But I held back because I wanted to get Kay more help.''

He still had enough of the ``old Phil Moser'' in him---this is what he called himself in his journals---to put some spin on his poor performance.

After he said that, I didn't feel so bad for him.

We were able to put his name on various lists that he now qualified for: part-time in-home care (normally six-months to a year wait), adult dare-care facilities (he might be able to visit places within a few weeks), and residential stay in San Francisco for times when Kay or others needed an extended break. When my dad declined further and needed to live in a skilled nursing facility, the VA would take care of all of his expenses. It was a good start and led to the best decision we made for my dad's quality of life: to get him into adult day care---get him out of his room, out of his house, and into a social group that kept him physically and mentally active.

The most surprising thing of all was how much he enjoyed it.

\begin{center}$*$\end{center}

My dad didn't start adult day care until mid-May. When he reached the top of the VA's list in early March, they sent me the names of care providers in the area. I visited websites from the VA list and asked Kathleen and Cynthia if they had any recommendations. Cynthia said she'd heard good things about Primrose, in Santa Rosa. We weren't confident my dad would actually go. Back in December I'd suggested to him that we visit the Sebastopol Senior Center to see if they might have fun activities for him. He knew the place well. He told me he used to drop Maya off there when he needed a break from taking care of her. He'd never stayed himself. The Senior Center was a small, local place that offered a calendar of activities---gentle exercises, board and card games, informational talks, men and women's support groups. It was all too tame and sedentary for my dad. ``Not my crowd'' was how he put it after we left.

What was his crowd?

Artists. Seekers. Healers. Shamans.

My dad had spent his later adult life, since his break with my mom in the late `70s, striving wildly for inner peace. If that sounds like a contradiction, then you understand the central obstacle my dad faced in life: himself. His path took him through peyote vision quests, LSD trips, solo and group Ecstasy binges. He launched himself into intense running routines to achieve ``runner's high.'' He wanted to run sixty miles by his 60th birthday in 1990, ``1 mile for each year of my life,'' he wrote in his journal. He craved the artist's life, a parallel path to the drugs and exercise that helped him forge a new identity, at least in name. Here is what he wrote on December 1st, 1983. He'd been living in Bolinas with his second wife Kerry (though it turns out they never officially married). By then he'd already given himself the name Carver:

This Odyssey that I begin is scary because I'm going to be a ``stranger in a strange land'' (thank you Robert Heinlein). My ways of being (basically distrustful and self-serving) are going to have to be modified. I choose not to go into a new situation of strangers and basically good simple people and put on aires, be a know-it-all, try to impress them with my knowledge, dazzle them with my brilliance (or is that my bullshit) same thing maybe. No, this time around I'd like to listen, observe, restrain my comments \& judgements and maybe learn a tad.

I might find that I like myself better this way. It's certainly worth the effort (\& It will be an effort)! That sort of behavior has to be learned---I'm just not programmed to keep a low profile. I need to think about this and come up with some practical ideas to make this way of being a reality. Silence is thunder maybe I can evoke the Cherokee in me to help me to just ``be'' \& not worry about being anything . . . These are some of the ways of being that Carver has---he has merely to use them. Ole Phil Moser didn't speak this language, but Carver can show him how. Nice thing about taking on a new identity. You can take the good stuff from the old one, add in a healthy sprinkling of new ways of being and come up with someone who you really enjoy being with. Being your own best friend and all that.

Part of my dad taking on a new identity and being his own best friend was breaking up with Kerry, who'd been supporting him. They'd agreed to a one-year contract for him to get financially independent with his art. He'd been carving for tourists down at Pier 39 in San Francisco, but apparently that wasn't cutting it (ha ha). Here's my dad again four days later, on December 5th (``Anne'' is a reference to my mom):

So---my recipe for living comes down to some very fundamental stuff, simple, yet very hard for me to practice. Total Honesty in the relationship, and in the rest of my life. I don't owe anyone an explanation about what I do or say, ever! Not Kerry, or my children or my friends or Anne. I was not put on this Earth to fulfill anyone's expectations, or to impress anyone with my accomplishments. I really only have to make myself happy and give others justice, by my standards, not theirs.

My dad made four more entries over three pages, then there's a four-year gap in the journal. By late summer 1988 he was back in Northern California after having lived on Maui for two years: more carving, more women, more running, more mind-altering drugs. My dad was still searching, still excoriating himself:

New Moon---New Focus

And how it was---was the spirits looked down and picked the most fucked-up mother who couldn't tell his ass from 3rd base, wouldn't know real from peanut butter, couldn't put 2 words together without power-tripping or trying to control someone and decided (?) to intrust to this fool, this scam-artist this trickster, this moral chameleon, this quoter of quotes and liar of liars and this terrified, macho half man, dim witted (spiritually, that is) bafoon---a smidgen of Sacred Wisdom because he had the the audacity in his one moment of 58 1\/2 years, the clarity of vision to admit he didn't quite know it all (most of it though, just ask him) They happened to be in a playful mood, or a charitable one and decided that maybe this asshole could be of some use to the planet, after all---(you can see where his self esteem was by the way he chose to describe himself, huh?) Really loves himself a whole lot, huh? Whew---Anyway. They took a shot at giving him a reality pill and damn---miricle miracle of miracles---It took---shot gun became a clear clean, dead center bulls-eye---(I guess he was ready this Time) He's off and running (mostly his mouth, so far) but he (I) am laying out a modicum of love and even a molecule of wisdom here \& there (mostly there) God do you think this entity is becoming real? Anything is possible, what? Be nice wouldn't it to empower oneself with honesty, reality love, etc, etc and on into---all the way into------------- consciousness---------------------.

Big Step for the Carver

My dad was always fully aware of who and what he was. He sought tirelessly to shake the skin of Phil Moser---the scam artist, the trickster, the liar, the half-man, the buffoon, the asshole, the moral chameleon (his code word, perhaps, for molester). Maybe the closest he ever got to achieving his goal was under the grip of dementia as it slowly crushed his memory. He had to lose himself in order to find a kind of peace. From mind-altering sweat lodges and marathon runs to circle time at Primrose for sing-alongs and trivia games? You can see why we'd be skeptical about his chances of liking adult day care.

And yet, miracle of miracles (to quote my dad), he loved it.

\begin{center}$*$\end{center}

One of the reasons my dad tried out Primrose was because I'd told him it was free: the VA was paying for everything. And it would help Kay out, too. He liked both of those ideas. The Day Club staff sealed the deal for him: enthusiastic, attractive women who encouraged him at every turn. He spent over four hours there on April 2nd and couldn't wait to go back.

``Carver had a terrific time,'' Kay wrote in an email to me the next day. ``He was in his element laughing, cracking jokes, chatting up the ladies. We had so much fun, it was like a party with ever changing activities to stimulate the mind. We played dominoes, did a half hour of chair exercises, and had lunch. He said he'd like to come back, it wasn't what he expected at all. We were both pleasantly surprised at how lively and fun it was.''

My dad waited six weeks for an opening at Primrose. In the meantime his care hours doubled from January: Kay was now logging over ninety hours a month. Morning check-in routines that used to take five and ten minutes now took an hour. My dad had to be cued to wash and brush his teeth. He'd stare at his hands in the bathroom and ask Kay, ``Okay, what am I supposed to do?'' He spilled drinks and plates of food now when he carried them back to his office. He fought her over taking showers and eating. One morning during her normal check-in she found him in the kitchen cleaning up white liquid all over the floor. My dad had gotten his own breakfast and put his box of coconut milk flat on the refrigerator shelf, and it had leaked out. She found three more boxes of coconut milk in there, all of them opened. He'd get up in the middle of the night and make phone calls to friends or try and pay bills that I'd already paid.

One night in particular she found him ``in a state of chaos.'' He'd turned on every light inside the house. She wrote:

The dirty laundry had been once again taken out of the hamper and laid out around the laundry room. The washer lid was up, every cupboard in the kitchen was open. Carver had gone through the trash in the kitchen; taken the empty coconut juice boxes and placed them in the drawer. He had then opened up two new coconut juice containers and had them on the counter. He came in with papers in hand. I asked, ``Whatcha doing?'' He said that he was taking care of business.

To prevent my dad from pulling dirty laundry out of the basket and putting it away in drawers,

Kay put a sign on the basket. But this only confused him, and he obsessed over it. He'd go on various organizing rampages. He'd take dirty dishes out of the sink and put them away on the shelves. One night he dismantled the stove top because he couldn't figure out how it worked. He peeled five bananas that were sitting on the counter then proceeded to ``clean'' the refrigerator and freezer: he threw away all of Kay's food. Eventually we installed a folding wooden gate across the kitchen entryway so that my dad couldn't take care of any more business.

As always, my dad was aware of these changes. Kay wrote: ``Carver makes a joke that I see everything, that I see all his mistakes. It must be so difficult for him knowing that I am witness to his mental decline. He can't hide it from me . . . I stopped, looked him in the eye and reassured him that all was okay. This intimate interaction and sharing helped clear the air and may have something to do with him isolating himself in his room.''

My dad was fortunate to have Kay. She was kind and understanding in a way that I could not be, at least not yet. There was too much emotional baggage between him and us kids that prevented that kind of trust and intimacy. There was so much that Kay and others in her circle of friends didn't know about him, and at that point in time I didn't think it my responsibility to enlighten them. It's like the day I took my dad for a haircut in Sebastopol and the stylist greeted him with a hearty Hey, Carver! and gave me a knowing nod---a war hero. My dad had spread his lies among close friends and acquaintances for years. These people didn't know me. Was it my responsibility to set them straight? To go around bursting bubbles? What would be the point in that? People love a hero, and they hate a whistle-blower. I just let it go.

My dad was also fortunate to have Primrose Day Club, which got him out of his room and gave him something to look forward to three days a week. He loved their exercise routines, which stimulated him. They danced, they sang, they played games. Because my dad was in good physical shape and still communicated well with jokes and charm, he could be what he'd always desired in some ways: a bit of a star, the center of attention. One day when Kay went to pick him up---this was after only two weeks at Primrose---she had to pull him away from a microphone: he'd been crooning oldies on a Karaoke machine.

It was probably inevitable with my dad, but within a month or two he developed a romance with another Day Club resident. But I'm getting ahead of myself.

\begin{center}$*$\end{center}

I should pause a moment and talk more about my dad's journals since I'm quoting from them. It was late April when I discovered two spiral notebooks on the bookshelves in his office. I'd gone to his place to cut the grass on his property, review his mail, and spend the night. I'd leave the next morning after a surf session in Bodega Bay.

My family situation in Davis at this time was in flux: Linda had progressed through several interviews for a teaching position at Santa Rosa Junior College. We'd always kept an eye out for jobs in Sonoma County, and when several openings came up at area schools, we'd decided to apply. She and I had actually met at SRJC, in a Shakespeare class, so the school held a special place for us. If she got hired, we'd stay for another year and see if we could make the move permanent. We even considered buying my dad's house and moving in so that I could care for him. This was the general situation when I pulled the notebooks from the top of his bookshelves. I'd been doing a general inventory of what he had in his office---books, papers, innumerable odds and ends. I flipped one of the notebooks open and discovered that some of the entries dated back to the late 1960s.

I sat down and started to read.

One section in particular caught my eye because my dad had typed it up and inserted the pages into a notebook: a ten-day motorcycle trip he'd taken through Northern California and southern Oregon in September of 1969. The writer was a different person from the groovy, New- Age con man (a term some of my siblings used for him) that I'd known as an adult: he was thoughtful, articulate, tolerant of the Flower Children ``blowing pot'' in the Yosemite area. He'd ``flopped'' with them for a couple days by invitation because all of the campsites were filled. At the time my dad was ``straight''---a crew-cutted salesman working in downtown Los Angeles to support his wife and eight kids in suburban Thousand Oaks. He didn't agree with the Flower Children's choice to drop out of society, but he admired their honesty about who they were and what they believed in (free love and pot) or didn't believe in (Vietnam and the establishment). He admitted they could be ``saying something with their lives that we might listen to.'' I could see the seeds of my dad's spiritual odyssey ten years down the road being planted in that compound.

But I thought, Who is this rational person? What happened to that guy? This wasn't the hippie-dippy dad I'd known, and often mocked, for the past thirty-five years.

And he wrote about his trip. That was what really got me. Somehow he understood that it was important for him to record his experience on paper. Because I love to write, I suddenly felt connected to him after decades of distance. I was discovering him all over again through words that were forty-five years old. Is this where my love of writing comes from, I asked myself, from this man who dropped out of high school? From someone who has felt more like an uncle to me for most of my life? I'm only finding out about this now, by chance, when my dad can barely sign his name?

I knew the late 1960s was the period when my dad had molested Terry. I didn't find out until I began writing this memoir, and asked her about it, that he'd suddenly stopped visiting her room right before this trip, the summer before she began eighth grade. But I read his seven-page travelogue with that man in mind. Why was he going on a solo trip at this particular moment? Was he running from something? Had those abusive hands left traces here among his typed words or corrected pen scrawls?

I looked carefully. I wanted to know what demons drove him. I wanted to understand as best I could the source of his unhappiness with himself and others.

His goal had been to visit various state parks in Northern California and Southern Oregon for potential family vacation spots. ``That was the `excuse' for the trip anyway,'' he wrote, ``but to those fortunate enough to have travelled light and alone to God's wonderland of Mountain Sea, Forest and steam Stream, the peace and oneness with the universe that comes over you in that environment is the real reason for any retreat from the noise, the confusion and the pressures of the `civilized' world.''

His motorcycle escape seemed ordinary enough: to get away from the pressures of work and family, to enjoy nature on his own time. To find ``peace.'' He'd titled his chronicle, as he'd called it, ``Easy Rider \#2'' after the classic counterculture movie that came out a couple of months before his trip. The movie must have inspired him to hit the road on what he had at hand: not a Harley-Davidson but a Honda 175.

I didn't find evidence of a monster in those pages. He mentioned us kids a few times, comments about how some of the eleven parks he'd visited might or might not suit us. He seemed genuinely relieved by the end of his trip to be cruising back to his family in Thousand Oaks: ``I arrived home with the worlds shortest 10 day beard and a heart full of joy at seeing Anne and the kids. I've heard and used the expression, `There's no place like home,' I believe I understand what it means, now.''

What I did find was evidence of a writer. He captured this scene his first night of the trip. After setting up camp at Pfeiffer Big Sur State Park in Monterey, he backtracked down Highway One to an eatery called Nepenthe for an ``Ambrosiaburger'':

Got into an interesting conversation with an older hippie about Zen and the effects on world peace, quite interesting and very thought provoking. All was not peace and love, however, and as evidenced by this conversation I overheard between two beautiful people; She; (about 16, dirty, unkempt, in need of a hair brush and a bath). He: (About 20, dirty, unkempt, in need of a hair brush and a bath,) they seemed to have a lot in common. She asked if they could go down to Monterey and see a drive-in movie: He looked thunderstruck and XXX said ``A movie?'', baby that's square, that's a stone drag, who needs it, and she said ``Well, hell, I'm tired of just going back to the pad, blowing pot and making love'' and he replied, ``thats all there is baby, that's where it's at'' and then silence and she looked wistfully into the fire and said ``Yeh''. One of the flower children was wilting---.

He had a good ear for dialogue. His mind slipped easily into the figurative at the end. It was so strange for me to hear my dad's voice on the square side of hippie language. His travelogue captured him at the tail end of a persona that he would cease to be within several years. We moved up to Occidental in 1974, which began my dad's trek away from his family and the square lifestyle, for lack of a better term, and toward what he called his ``Quest.'' He later wrote in his journal: ``I needed to define myself other than father, husband---etc I was sick of `The Waltons' image. There was a longing in me that nothing I could do seem to fill. I was in a spiritual vacuum because I had left the Church \& had nothing else that I could turn to for sustenance. I decided I had to be on my own for awhile \& try to work this out.''

My dad's journals describing his drift away from the family pulled me closer to him. I discovered connections and commonalities that I'd never imagined between us. The more time I spent with him in his world, past and present, the more I began to open up to him.

\begin{center}$*$\end{center}

This was the period when we were looking toward the end of our sabbatical in July and leaving California. Linda did not end up getting a job offer, so we were headed back to Missouri. I was worried about how long my dad could stay in his house. His living situation seemed untenable. Kay's time jumped to over a hundred and thirty hours in June. My dad's behavior was growing more erratic, which put a lot of stress on Kay. She didn't have a regular day off. Since she lived on the property, she was always on call. She'd agreed to this arrangement, but now that my dad was getting up at odd hours during the night, in addition to his shenanigans during the day, she was feeling overwhelmed. All I could see when I looked ahead a few months was a potential train wreck. I'd been my dad's juggler for the past year: trying to maintain support for Kay, who had the hardest job of anyone; organizing my dad's legal, financial, and medical affairs; and keeping my seven siblings and Maya's two children informed. It was challenging enough to keep everything balanced and moving while I was living in California. What would happen once I went back to Missouri?

The whole thing could come crashing down.

I'd already gotten a taste of what that might look like. In the first week of May I'd driven down to Southern California for the 50th anniversary celebration of The Endless Summer. I write about surf history, and this was a chance to see some old friends, meet Bruce Brown and the stars of the film, and generally enjoy the kind of surf-tribe gathering I usually missed out on by living in Missouri. I brought my youngest son, Ryan, and we spent a couple of days in Huntington Beach, where the gala took place. Afterward we drove up to Malibu and connected with Kathy Kohner Zuckerman (aka Gidget) where she worked at Duke's restaurant. We stayed with my brother Chris and his family in the San Fernando Valley, then headed down to the San Diego area where my brother Steve lives. It was great to spend time with family and get some surfing in. I only had a couple more months in California, so I was savoring every chance I had to travel along the coast and enjoy the beaches that were so formative to my own identity. I suppose writing about surfing has been my way of staying connected to an activity that I love but can't enjoy normally more than once or twice a year.

While I was in Southern California my dad had gone to a dental appointment at the San Francisco VA. I'd learned back in January that the VA had a special car service that my dad now qualified for because of his dementia. A driver picked him up at his house, drove him straight to the San Francisco VA, and then chauffeured him home. Miracle of miracles! No nine-hour appointment days. No bone-rattling rides on the commuter shuttle. I had to get his VA doctor to contact the transportation company and confirm that my dad qualified for the service. Once she'd done that, I contacted the shuttle company several days before my dad had an appointments to set up a time for them to arrive at his house. So I'd set all of that up before my trip to Southern California.

On the way back from his appointment, my dad got confused and told the driver to drop him off at the Santa Rosa VA. He sat inside their waiting room until they closed at 5pm. A kind security guard found him wandering in the parking lot---he must have been there for a few hours ---and drove him home to Sebastopol. He arrived late at night disheveled and traumatized. He'd lost his jean jacket and his favorite Maui hat back in San Francisco. All the next day he rehashed the entire experience for Kay, obsessing over his jacket and hat. This was the week before he started at Primrose. It was one of those situations where we had another breakdown in communication, which happened more than you'd like with the VA.

Why would the driver drop my dad off at a different place than where he'd picked him up? It never occurred to me that that might happen. Why didn't the driver know that my dad had dementia? I'd organized the service, so I felt terrible. I should have done more to make sure everybody was on the same page. From that point on, somebody always rode with my dad to his appointments.

\begin{center}$*$\end{center}

We left our sabbatical rental in mid-June and stayed with my mother-in-law in Petaluma.

Since I was so close to my dad's place, and my time in California was growing short, I stayed with him every week so Kay didn't get burned out. I worked with Kathy Vincent at the Santa Rosa VA to get my dad approved for more Day Club time: five days a week rather than three days. This started in June and allowed Kay to have every day off during the week after she dropped my dad off at Primrose.

The last time I stayed with my dad that summer was over three nights in mid-July. I wanted to give Kay one last break before I left and also get a sense for how long my dad might be able to live in his house. He wasn't a flight risk---he'd never wandered outside at night---but he still had occasional organizing rampages in places like the bathroom (cleaning drawers, spreading toiletries around the counter) where it didn't make sense to block him with a gate. He might need to use the toilet during the night. We'd started him on a small dose of Donepezil that month, a drug we hoped would slow his cognitive decline. After six months on the VA wait list for weekend help, my dad's number finally came up: sixteen hours of in-home care so that Kay could also have a break on the weekends.

The extra help came at the perfect time. The first night at my dad's place, I woke at two a.m. to noises coming from the downstairs bathroom. I listened intently for a few minutes, wondering if everything was okay. I heard the faucet turn on and keep running. I decided to pop down and investigate.

``Dad?'' I said. ``What's going on?''

He was in a semi-panic: ``Trying to clean up this big mess.''

The bathroom floor was covered in water, which had spread to the hallway carpet. His thick grey socks were soaked. The toilet had overflowed on him. He kept flushing it, and it kept overflowing. He couldn't figure out what the hell was wrong. He was trying to clean up the water with wads of toilet paper.

I leaned in and turned off the faucet. ``It's okay, dad. Let's get you cleaned up.''

``It's a big mess!''

``I know.''

I got him settled in some dry clothes and back into his chair for the night. It was early

Monday morning. We'd be up in another five hours to get him ready for his day at Primrose.

After breakfast, he asked me to pick out a shirt for him because he couldn't decide which one to wear. I grabbed a long-sleeved denim shirt that I knew he liked from his closet and helped him with the buttons. Once we'd gotten him dressed and his teeth brushed, we went to the front door so he could put his shoes on. The house had white carpet inside, so people normally left their shoes by the door.

I went upstairs to grab my car keys and wallet. When I got back downstairs, I couldn't find my shoes or my reading glasses. I thought I'd left the shoes by the door and the glasses on the dining room table. I checked the kitchen. Nothing. I checked the office. No glasses. I ran upstairs to grab my shoes---I must have forgotten to take them off the night before. But they weren't there.

I went back downstairs. My dad was sitting at the table waiting patiently for me. I hadn't found my shoes because he was wearing them. He'd also tucked my glasses into the pocket of his denim shirt.

``Dad,'' I groaned. ``What are you doing?'' I leaned down and helped him off with the shoes. ``You're going to drive me crazy.''

What could I do but laugh? It was one of the most endearing things I'd ever seen him do.

That afternoon, while my dad was at Primrose, I had his house inspected. The week before I'd had it appraised. Even with all of the extra help for Kay, I didn't think that my dad was going to be able to live in his house much longer. The writing, as they say, was not only on the wall, it was in the kitchen, and the laundry room, and the bathroom, and the office. He was steadily losing his grip on reality, and I knew it wasn't safe for him to live in that house by himself. Even though Kay slept only a short distance away, and we had a baby monitor hooked up downstairs, I had a sense of dread at the thought of the simplest daily tasks throwing my dad into states of panic, or him freaking out at night and possibly falling and hurting himself. And his condition was only going to get worse.

At the same time I wanted my dad to have the best quality of life possible, for as long as possible. I was doing everything I could to allow him to stay in his house. Even with all of my misgivings, I wasn't ready to make the decision to move him out permanently. The idea was overwhelming to me: to be, in a sense, a father to my dad, and to make decisions about his life for him. I never imagined being in that position. I'd done much of that already by organizing his finances and paying his bills. But the decision to take him out of his home seemed much more personal and invasive.

How would I know the right moment to make that decision? If I moved him out too soon, I might force him into a living situation where he'd be unhappy. I didn't want him acting out or to hasten his decline. But if I waited too long, he might hurt himself somehow and wind up in the hospital.

The week before my stay I'd visited three skilled nursing facilities with my sisters and Cynthia. Two of the facilities were in neighboring Petaluma, and the third in Santa Rosa. If and when my dad qualified, the VA would pay for all his costs. This was a tremendous benefit and eased my worries about my dad's dwindling bank account. My first choice was a facility in Petaluma that was near a public park and had a few redwood trees on the property. That drew my eye right away---I knew my dad would like that---and the place looked open and friendly. He wouldn't get the personalized care he received from Kay. He wouldn't be in his own home, in familiar surroundings. But he'd be safer and looked after by professionals.

After the visits I felt better about where and how my dad might end up. I started the paperwork with the VA so that my dad could be admitted when the time came. My understanding was that all I had to do was give the order to the VA, and they'd start the process of placing my dad through one of their contracted facilities. There were long wait lists, but apparently a room could be had in an emergency situation. I didn't understand their system of placement, but the VA seemed confident that my dad could get a bed if he really needed one.

Monday evening, after my dad and I got back from Primrose, I put on a Kate Wolf CD in the living room. My dad sat and listened while I wrote on my laptop. Kate Wolf had actually sung at my sister Terry's wedding back in 1978. My dad had been a longtime fan of her music, and she'd become one of Maya's favorites as well.

I got lost in my work. I looked up from the screen to find my dad in tears. I'd left the music on too long! I got up quickly and turned off the CD. Kay had told me the week before that she'd found him crying to Hawaiian music: a song that my dad and Maya used to enjoy together. He'd been grieving for her and their separation.

I put my hand on my dad's shoulder. ``It's okay, dad.''

He blew his nose and wiped his eyes. ``I'm sorry,'' he said.

``You don't have to apologize.'' I rubbed his shoulder. ``Hey, do you want to see some pictures of one of your hikes?''

He finished wiping his eyes and nodded. ``Okay.''

We sat at the table, and I put a slideshow on my laptop. In July of 2007 he'd gone on a weeklong hike in the Sierras with two of his friends. One of the men, an artist and photographer, had created a souvenir book that he'd scanned into a .pdf file. I'd found the file on my dad's old laptop and transferred it to mine.

The photographs and watercolors soon distracted my dad and put him in a good mood.

For my part I marveled at my dad's fitness: seventy-seven years old and taking off for a week of hiking around the mountain trails of Yosemite National Park. This was one of the areas he'd visited and raved about during his motorcycle trip in 1969. He had loved to go on extended hikes through California's expansive state parks, oftentimes by himself for a week or ten days at a time, carrying nothing but a backpack and a walking stick. When I was younger I used to dream of going with him on his annual treks, just the two of us enjoying nature and one another's company.

But we never made that trip.

During that sabbatical year, my dad and I had spent more time together, just the two of us, than we had in our entire lives: going to doctor appointments, eating together in the dining room, watching TV in his office when I came to spend the night. We went shopping together and ate dinner at his favorite restaurants in Sebastopol. I organized his bills, made his meals, did his laundry, even helped him take showers. I finally had a relationship with my dad, it just wasn't the one I'd imagined.

That relationship, I realized, would never happen. He could no longer go on long hikes in the Sierra Nevadas. He couldn't remember the trails he used to climb, the campsites he frequented, or the names of peaks and valleys he'd crossed and recrossed. He knew who I was because I'd been spending so much time with him, but he'd already begun to forget the names of his children and grandchildren. We didn't have the relationship I'd hoped for: I had to help him get dressed, remind him to wash his hands after he went to the bathroom, and get him settled in bed after the toilet overflowed on him in the middle of the night. It wasn't the relationship any child wants with a parent, but it was the one I had.

My dad struggled so much of his life for inner peace and acceptance of who he was. He walked ridiculous lengths and ran marathons to pound guilt, shame, and fear from his soul. I imagined him crossing finish lines in desperate hope of release and redemption, only to find none. So he began more races, clocked more time on the road, ran farther afield. I didn't want to follow in his footsteps, grasping for relationships that remained always out of reach, misty mirages on the horizon.

Whether my dad deserved my time and attention, I decided, was beside the point. Our relationship was not governed by his own New-Agey sense of karma: what comes around, goes around. My dad was sitting there beside me. Not the one I wished for, but the one I had. Damaged, yes. Even broken in many ways. But absolutely vulnerable. He was in need, and I could help him. I realized that I wanted to help him. Maybe we were both in need.

\begin{center}$*$\end{center}

The next morning I took him to see Maya. He normally saw her every Tuesday. I've neglected to mention that my sister-in-law Cynthia had been driving him once a week for most of that year to visit Maya. Once my dad could no longer drive, Cynthia had offered to help out. Thank God for loving in-laws who step in with fresh legs to do good deeds! Cynthia didn't have the baggage with my dad that we kids carried around, so she was able to give him the kind of care that made a real difference in the quality of his life and in Maya's life. When Maya was sleepy or withdrawn during their visits, Cynthia showed my dad how to soothe her: to hold Maya's hand or give her gentle massages on her shoulders, neck, and feet. My dad loved this. He loved being useful. They'd wheel Maya outside when the weather was nice, which perked her up. One time my dad walked into the room where Maya was sitting, and she said, ``You're here! I love you!'' This brought tears to my dad's eyes.

Another day my dad asked Maya, ``Would you like a big kiss or a little one?'' ``A big one,'' she said.

So my dad gave her one and asked, ``How was that?''

She smiled: ``That'll work.''

They rallied one another in amazing ways. She was most often in a wheelchair or her bed, and my dad's appearance snapped her out of the Lewy Body limbo that preoccupied her days. She came alive for him: talking, laughing, joking. And he suddenly became her caregiver once again: attentive, encouraging, loving. At times he sounded so lucid, you never would have guessed he was battling dementia himself.

Cynthia's presence also helped my dad sustain the emotional weight of visiting Maya. Since my dad was attentive, he often attracted the eye of the other residents. One day Cynthia reported that they met an elderly woman sobbing by the exit. She told them, I hate the person who put me in this place! They did their best to comfort her, but the scene rattled my dad. He felt guilty at being unable to care for Maya anymore.

I don't recall much about my own visit with Maya that particular day in July. One horrific memory blocks out the rest of it. My dad and I were sitting in the living room area with her. She was in her wheelchair, and my dad and I sat in easy chairs around her. I'd grabbed a stack of photographs before leaving the house so my dad could show them to her. She was an artist, so she'd always responded well to visuals: pictures of her own art and the art of others from her books. I'd found the photographs in my dad's brown attach\'e case tucked away in his office closet, shots of him and Maya in various places. I hadn't looked through them closely, just grabbed a handful.

My dad sat next to her and started flipping through the photographs. Maya's head drooped down and her eyes were half-closed, but she focused on each one as he held it up. I sat across from them. I couldn't see the pictures myself. I just heard my dad's short description of each one. He went into detail to help Maya remember things: the two of them at a restaurant the night before their wedding, him wearing his thunderbird bolo tie and her a pendant necklace with a pretty stone; a picture of them kissing in a garden with pink and red roses flowering behind them; another of them sitting together on their back porch at the house in Sebastopol, she in bare feet and him in his slippers.

My dad held up the next photograph. ``Uh, this is . . . me and my lady in Hawaii.'' He checked the back of the picture. ``Yep, that's what it says.'' He paused. ``I don't know how this---``

``---Dad, I'll take that!'' I jumped out of my chair and grabbed the stack from his hand as gently as I could. There were several photographs of my dad with old girlfriends. I should have looked through the pictures first! The last thing Maya needed to see was my dad with other women, especially since they lived apart.

I felt awful. These visits were supposed to cheer her up, not make her feel bad. She hadn't reacted strongly in any way. But still. She'd seen the picture.

Ugh! Stupid me.

\begin{center}$*$\end{center}

Around this time I'd considered the possibility of my dad moving to the Board and Care in Forestville and living with Maya. Perhaps the two could even room together. That would seem to help them both, since they rallied each other.

One of the obstacles was financial: the cost would be over \$10,000 a month. What would happen if they both lived another ten or fifteen years? My dad was still healthy, and Maya's condition had more or less stabilized under professional care. What would happen to them when they ran out of money? Where would they live then? If my dad moved into a skilled nursing facility, at least the VA would pay for his care. That way Maya would be able to stay in Forestville into the foreseeable future. Keeping them apart made a lot of financial sense.

Perhaps the larger obstacle was my dad's quality of life. Although he enjoyed visiting Maya once a week, having them live together didn't seem like the best idea for my dad. He was much higher functioning than she was, physically and mentally. I didn't want to put him in an environment where he was around residents who weren't capable of interacting with him very well. I preferred the care he received at Primrose: it was more interactive, and he absolutely loved the Day Club staff. And they loved him. They had quickly become his new family. He spent every day there and couldn't have been more agreeable and charming with them and the other day-clubbers, as they were called.

There was also my dad's ``romance'' with another Day Club resident. I'll call her Wendy. As romances go it was pretty tame. Wendy was higher functioning than my dad. I could never actually tell if she had cognitive impairment at all. She was quite open about being ``crazy'' for my dad. She sat next to him when they were doing activities at the Day Club. She held his hand. I'm sure he liked the attention. I'm equally sure that he wasn't the instigator. I'd be the first to say if he was, but I doubted he had the mental ability to be proactive in that way. He went along with a pleasant situation, which became ``kissing friendly'' after I left for Missouri.

I wondered if my dad's earlier tears for Maya included guilt at his budding relationship?

Cynthia reported that, after one of their visits to Forestville, my dad had told Maya ``not to believe talk about him having a girlfriend'' at Primrose. I couldn't imagine that anybody where Maya lived would know about such a thing. Cynthia never would have told her. But my dad was worried enough about it to bring it up.

I heard from Kay a couple days before I left town that my dad's in-home care agency had ``fired'' us. We'd only been their client for ten days. Kay discovered that the agency's first caregiver had taken my dad shopping and accepted a ``gift'' from him of several bags of groceries, which he'd paid for with his credit card. Kay called to complain. The agency replaced the caregiver but accepted no liability. The second caregiver came the following Saturday but had car trouble and couldn't make it on Sunday; the third arrived on Sunday without breakfast or lunch, so Kay fed her. After multiple calls from Kay back to the agency about the initial theft--- Kay had threatened to file a police report---the agency terminated the service.

So my dad had no weekend care except for Kay. I'd already filled out the paperwork for him to live in a skilled nursing facility, but we had no confirmation from the facility that they had an open bed or even from the VA that my dad qualified for the service.

Transportation was an ongoing issue. Primrose had its own shuttle that picked up adults for the Day Club and afterward dropped them off at home. A terrific service. Unfortunately my dad lived outside their pick-up route. We initially set up a drop-off point halfway between my dad's house and Primrose: Kay drove my dad to the Sebastopol Humane Society and waited in the parking lot for half an hour because the shuttle was twenty minutes late---you had to be flexible when you loaded adults with dementia. That didn't end up saving Kay any time: Primrose was only a twenty-minute drive from my dad's house. So she was still taking him every morning and bringing him back every afternoon.

Kay assured me before I left that she was fully capable and willing of managing my dad's care, his transportation, and any repair issues that came up with the house.

I had my doubts. As my dad's cognitive abilities declined, his care needs grew exponentially. Kay did the work of a caregiver, but she was not professionally trained outside of massage. I didn't feel that my dad was entirely safe in his home, but as yet I had no back-up plan in place. Even if I had decided to move my dad in with Maya, they had no open beds there.

That was how things stood at the end of my sabbatical year. It was time for me and my family to leave California. I'd have to continue the juggling act from Missouri and hope that no harm came to my dad while I weighed the pros and cons of his living situation, which seemed to be getting more precarious every day. As I'd done with my dad's driving, I'd simply have to get as much advice as possible and simply trust that I'd make the best decision for him when the time came.

The way it worked out, I wasn't able to separate what was best for my dad from what was manageable for me and my family. 


\chapter{}

I dreamed about my mom several days before my next trip to California. It was early October---the time of her death, back in 2004. I'd been stressing about flying out to see my dad: what his condition would be, the meetings I'd arranged with the lawyer, the accountant, and the Santa Rosa VA. Stressing, too, about trying to fit all my visits with family and friends into a tight window of time. Stressing about canceling classes and leaving early for Fall Break: all the grading I hadn't done yet, and how busy I'd be once I got back.

I dreamed my mom dropped me off in a rental car by my old elementary school in Occidental, which was named Harmony. She left the car by the side of the road in case I needed it. I don't know where she went from there. I think she simply walked into town. She used to own a gift shop in Occidental called The Grape Vine. It's how she paid the bills once my dad left. He never sent child support for the three of us kids still living at home. The official term for this, I believe, is deadbeat dad.

I'm always glad when I get a chance to see my mom in my dreams. It's always unexpected, and I miss talking with her. She was the first person any of us kids shared news with because she loved to hear what we were doing. It didn't matter how trivial the information. She'd be enthusiastic and encouraging. She once told me that her goal in life was to meet a new person every day. Have you ever heard such a thing? She loved people. She loved to talk with them, to laugh with them. It's what made her such a good saleswoman and how she was able to support three kids at home by herself. Perhaps she thought back to her own mother, Beryl Conway, who raised three young daughters during the Depression after her husband died due to injuries related to his job as a coal miner. This was back in Scranton, Pennsylvania. Gram Conway ironed people's clothes and scrubbed their houses for twenty-five cents an hour, so my mom understood the value of hard work and long days. She didn't need the Dale Carnegie rules that my dad copied so carefully into his journals to be successful (``Become sincerely interested in other people and they will be interested in you''). She didn't have to become interested in people to her advantage. Her advantage was simply being interested in them.

I woke from my dream feeling relaxed. My mom seemed to be telling me that I was doing the right thing by flying out and spending time with my dad. One of her great talents was bringing people together, especially family. We had so many fun gatherings over the years at our house in Occidental: the eight of us kids plus spouses and grandkids and neighbors and any newfound-friend we wanted to bring home. These are some of my best memories growing up. I wish in so many ways this book could be more about my mom. She is the well I continually draw from for strength and stability. Her core sense of family must have been one of the reasons my dad was so strongly attracted to her. One of the reasons he needed her, because his own family was so toxic.

Despite the shabby way my dad treated my mom at the end of their marriage, she would have approved of my trip to see him. She would have wanted that for me perhaps even more than for him.

\begin{center}$*$\end{center}

I wasn't sure my dad recognized me when I walked through his front door. It'd been over two months since I'd last seen him. He greeted me but didn't call me by name, which he usually did. We didn't talk much. Kay reported that he was usually tired after spending the day at Primrose. I took him upstairs to bed right after I'd popped out to the art studio to let Kay know that I'd arrived. I'd stay for five nights this time, all of them with my dad. Beyond giving Kay a vacation, I was there to see if my dad was safe living in his house.

There'd been a big change in his sleeping situation since I'd left California. He started sleeping upstairs in his and Maya's bedroom. During my last stay with him the previous July, I'd noticed that his feet were swollen. We got him in to see a podiatrist. The doctor told my dad the swelling was caused by sleeping in a chair with his feet propped up. My dad began sleeping upstairs after that. This was in mid-August, and the move upstairs had been highly successful. ``Like sleeping on a cloud!'' my dad told Kay happily after the first night.

His feet were fine now, but I worried about him going up and down those stairs. Kay assured me that he stayed in bed all night until she came in at 8:30 every morning and escorted him down. But could we always count on him to stay in bed? He had dementia, after all. It'd only take one misstep in the night to put him in the hospital with a broken hip or cracked skull. My sister Cindy had shopped for a new bed so my dad could sleep downstairs in the living room, but Kay preferred to keep him upstairs. She felt he was more contained up there.

I slept on the bed in the living room, exactly in the space where Maya had been lying the day I visited my dad to see the green notebook. My trip this time was somewhat related: I didn't want to make him leave his house, but his situation since August had grown so precarious that I didn't see any other choice.

But I wanted to be sure. The only way to do that was to spend time with him.

Kay had logged 109 hours back in July. In August, the month I left for Missouri, her care jumped to 163 hours. Her monthly time sheets now included additional hours for weekend caregivers who covered for her during the week, and friends like Mercedes who arrived in the morning to get my dad ready for Primrose or to pick him up and bring him home. The VA had approved my dad for full-time Day Care (five days a week at Primrose, which included three meals a day) and sixteen hours of home-care on the weekends. And yet Kay's hours kept increasing.

Part of me understood. His condition was declining, and he needed more care. I felt lucky that my dad had Kay. They'd developed a strong bond, as I've said, which allowed him to live at home. If Kay had not been willing to care for him, we would've had to move him out the year before. She had the hardest job of anyone on a day-to-day basis and did what none of us kids could or would do.

Yet part of me started to buckle under the strain. I worried about his finances. Kay's increasing hours and the loose network of care she'd cobbled together for him now costed my dad as much as a Board and Care facility. I arranged time off for her, coordinating family stays and replacement caregivers. I acted as a landlord, contacting repair people about a leaky faucet or replacing the water heater. In all of these situations, distance compounded my stress. A simple request from Kay to have a night shift covered might involve a dozen phone calls or emails to three or four different people. I could only keep so many plates in the air at a time.

I also felt that the more my dad's care spread among different people, the greater the chances of miscommunication and something going wrong. It was really hard to ensure that everyone was on the same page when I lived two thousand miles away.

In fact it was impossible.

The only way I could see to ensure consistent and professional care for my dad, and to keep myself sane, was to move him out of his house.

\begin{center}$*$\end{center}

That first morning I heard my dad moving around upstairs. It was about six a.m. I'd been awake for awhile because of the time difference: Missouri is two hours ahead of California. I'd been hunkering in the blankets, enjoying the feeling of waking up in the redwoods, when I heard the ceiling start to creak. My dad's bed was directly over mine. I heard him go into the bathroom. I waited to see if he'd try to come down the stairs, but he climbed back into bed and stayed there until I went up and got him at 8:30. It was was pretty adorable: he was wide awake, sitting under the covers, just waiting for me.

I made him breakfast. Kay had sent me his morning routine a couple of days beforehand, so I followed the menu on her list: scrambled eggs, bacon, juice, toast, echinacea tea, and some apple slices.

My dad's butthole also made the list.

Let me explain. Kay had emailed a care plan for my dad that covered two and a half pages of numbered and lettered points for his morning and evening routines: making his meals, getting him dressed, monitoring his hygiene, and so forth. It was all very detailed and helpful. I remembered reviewing it with Linda back in Missouri the night before I left. Toward the end of the list, Kay had written directions for helping my dad after his shower: ``Place lotion into his hands and instruct and mimic rubbing onto body parts; his butt cheeks get real dry too; make sure he doesn't put it on his butt-hole.''

``Really?'' I'd asked Linda. ``I have to read about my dad's butthole?''

It wasn't a big deal, but still. How was I supposed to make sure of that? This was a far cry from being executor of his will. There hadn't been any tabs in his green notebook labeled Lotion or Buttholes.

Kay popped into the house after breakfast. She seemed anxious. It was Tuesday, so my dad and I were hanging out until 11 a.m. when Cynthia would arrive and the three of us would go visit Maya.

Kay snapped at me for being late with her check. I apologized. I told her I'd brought it with me to give to her in person. Sometimes things got lost in my dad's mail. She also insisted that my dad take a shower. I'd already helped him get dressed for the day, and we were both sitting quietly. I was working on my laptop---I'd started to keep a journal about my visits---and my dad was listening to music.

``He's fine,'' I told her. ``He can take a shower tomorrow.''

``No,'' she said. ``He has to take one today.''

She was pacing, tense. I didn't think my dad needed a shower. He hadn't smelled ripe when I'd helped him get dressed. But she felt strongly about it. I didn't want to upset any routine they'd established, so I deferred to her.

She got my dad out of the chair, sent him to his office to get undressed.

``I'm happy to help him shower,'' I said. After all, I was there to give her a break.

She took charge. She turned on the water, which took awhile to heat up. Once my dad was in the shower, she handed him a washcloth, soap, and shampoo. She instructed him at each step: wet the washcloth, rub soap on it, scrub himself all over. Body first, butt last. Once he finished and got out of the shower, she monitored him as he dried himself off and put on underwear. She put lotion in his hand and helped him rub it over his skin.

Once they were done in the bathroom, she took him to the office and put him in fresh clothes.

The scene made me notice more changes in my dad. He needed more help with everything, especially hygiene. Earlier that month Kay had found him cleaning the toilet upstairs after he'd flushed it. He was dipping his hands in the toilet bowl water and running his fingers around the rim, probably trying to clean off poop he'd smeared there accidentally after wiping himself. She'd called it ``a new and disconcerting development.'' After that she bought sterile wipes. He now used them whenever he went number two. She also had to monitor him on the toilet, making sure he wiped himself properly.

I also noticed changes in Kay. I wondered if she was anxious because she was leaving for the week. She didn't seem to want to let go of taking care of my dad. She'd been putting so much time into him, every month more and more hours, and he'd grown so dependent on her. She'd become the closest person to him in his world. I got the feeling she didn't entirely trust me to take care of him. She probably would have felt that way toward anyone coming into the house.

Her house, in a manner of speaking. The art studio where she lived had no bathroom or kitchen, so she'd always used the main house for showering and cooking. She'd lived on the property for over two years by that point. My dad had often let her use the living room for massage appointments. She cleaned the house and took care of the property, running the mower and the weed whacker on occasion. Anyone staying in the house was essentially coming into her living space, too.

She monitored and gave orientations to the weekend caregivers. We'd had five of them from two agencies in the space of a month. I had to intervene with one of the caregivers during my visit that week. Kay didn't want the woman to wear shoes in the house (they dirtied the carpets) or to allow my dad to sleep during the day while the caregiver was there (otherwise he wouldn't sleep until late at night). The caregiver and Kay developed trust issues: my dad fell in the shower (according to the caregiver) but Kay thought he'd fallen by the sink because the caregiver hadn't monitored him closely enough in the bathroom. The caregiver had also paid for things out of pocket---lunches and sweat pants for my dad---and wanted to be reimbursed but couldn't produce accurate receipts. Kay also felt that the caregiver didn't respect her, treating her as an equal rather than her supervisor.

Kay had complaints about rodents in the art studio (voles and mice got into the hay that formed the walls of the structure) and wanted me to arrange an exterminator to come out. She asked me about getting the inside of the studio sealed because the hay was giving her breathing problems. I asked her if she wanted to move into the main house. This would solve many of her issues with the art studio and ease my mind about my dad living alone. But she had a dog and her own space in the art studio, so living upstairs didn't work for her.

The stress was mounting for both of us, each expecting the other to take on more responsibility. Neither of us was trained for what we were doing, and I could feel the friction more now than on any other visit. Our relationship was congenial, but strained.

The previous summer I'd also had to drop a bomb on her. She wanted to let the caregivers at Primrose know about my dad's PTSD from his time in Korea. I finally had to tell her the truth.

I was happy to let my dad's friends admire whatever intricate webs he'd spun, but I couldn't continue a lie when it came to people making healthcare decisions for him.

So I told Kay: ``My dad didn't fight at the Chosin Reservoir and lose his entire platoon. He's never even been to Korea.''

As long as I was dropping bombs, I decided to empty my dad's stockpile. ``He isn't Native American, either,'' I said. ``He wasn't born on a Cherokee Reservation. He's a chronic liar. You've been duped by a con man.''

I didn't actually say that last sentence. I didn't need to. Her jaw dropped about six inches.

She didn't believe me at first. She looked at me wide-eyed and stunned. She only came around after she started to poll my siblings. Yep, that's right. It's all basically bullshit. That was the gist of their responses.

I was sorry to strip her of these illusions. She was my dad's caregiver. He depended on her so much. We all did. To her credit, she continued to care for him. But I think something broke in their relationship, understandably so. He'd betrayed her trust. She'd cared for Maya and him for over two years, but they'd known each other much longer. I took no pleasure in basically saying to her, Welcome to the world of his children. If she and Mercedes and my dad's other friends had long wondered why his eight children were so stand-offish, why we hadn't jumped right in to shore him up in his time of need---well, there you go.

And there were, of course, darker corners that she hadn't glimpsed.

I did not expect her to understand, or approve of, my decision to move my dad out of his house. She would be losing her job, for one. She'd apparently worked off the books for much of her life. Her care of Maya and my dad was mostly barter: for rent and eventually my dad's Prius.

She was looking to reestablish herself ``on the books'' with taxes, retirement, and Social Security. She'd asked me to hire her officially as an Independent Contractor. She was nearing fifty, or already there, and starting to plan her financial future. Caring for my dad full-time would allow her to make progress toward that goal.

She would also be losing her home. Vacancy rates were extremely low in Sonoma County, and rents extremely high. All of that wine and tech money had made the area desirable and drove up the cost of living. Once my dad moved out, I'd be obligated by the terms of his reverse mortgage to start the process of selling his property. Kay would have to find another place to live.

So there was a lot at stake for her. And for my dad. Kay cared for him better than any other healthcare worker I might find. But she also had her own future to consider. It would be unfair to expect her to make an objective decision about what was best for my dad's quality of life because that decision also involved her quality of life. She had too much to lose.

I felt how close she'd gotten to my dad that first morning of my stay. Too close, I'd later think. Too protective. Because she was also protecting herself. I doubted she even knew how much she'd grown to depend on him.

For me it was another sign---a nebulous collection of feelings and disturbances gathering momentum inside of me---that my dad needed to go.

And soon.

\begin{center}$*$\end{center}

A man and woman were dancing to a violin player in the activity room when we walked into the Board and Care to visit Maya.

How sweet, I thought. She's giving him a full body-hug.

I realized she wasn't hugging him. ``She'' was a healthcare worker moving a resident from an easy chair into a wheelchair. I'd caught them at the apex of the transfer, her arms wrapped around his waist to keep him from falling. When the violin player finished his scratchy tune, two of the residents slowly clapped.

``My last song,'' the violin player announced. He looked to be in his sixties. ``A waltz. I'll be back at two.''

We passed through the dining room where a dozen residents were getting ready to eat lunch at cafeteria-style tables. I heard Spanish coming from the workers in the kitchen. At the table nearest me, a woman dressed in sweats stared blankly at the man across from her, whose head had dropped to the table. They were both strapped into wheelchairs.

Off the activity room we entered a living room where Maya sat in her wheelchair along with another resident, also in a wheelchair. Maya wore a thin purple nightgown and a pair of socks. Her legs were emaciated and her head drooped down. Once we got settled around her, she roused and responded to Cynthia, who'd grabbed an art book off a coffee table and started to show Maya paintings by El Greco and Joseph Wright of Derby. Cynthia slipped a rose behind Maya's right ear---she'd plucked it from her garden at home---and asked Maya questions about the books: ``Who's this Maya, do you know? What do you think of this painting?'' Maya eyed the pages. She described in halting words a rainbow in one of Wright's landscapes.

The other woman in the room was sleeping in her wheelchair. Like all the residents, she was dressed for comfort: baggy sweat pants, a light blue sweat shirt. She had a small clip on her shirt that connected to the back of the wheelchair. If she got up, the clip would snap off her shirt and sound an alarm. The woman shook like she was having a bad dream; she startled awake a moment, then settled back down, her mouth dropping into a frown. Her hands were clasped together deep in her lap, and they trembled.

Cynthia turned more pages in the art book. She paused to read a few verses under one of the paintings. ``Those are by Victor Hugo,'' she told Maya.

My dad sat on Maya's other side, holding her hand. ``He was a world-famous artist.'' Maya wrinkled her nose. ``He was?'' she asked slowly.

``You betcha.''

Poor Maya. Dementia hadn't crumbled her mind enough? She had to sit there and listen to my dad make stuff up about art, a subject she'd known so intimately.

I looked out the window to distract myself. A woman walked around the expansive front porch despite the October chill. I had to lean forward to make sure of what I was seeing: she carried a large stuffed orangutan. She set the orangutan down in a wicker chair and organized several plants on a nearby table. She then opened a side gate off the porch, went through it, closed the gate behind her, and made a loop around the yard. The entire property was fenced. She came back to the orangutan and picked it up. The whole time she was talking to herself---at least I saw her lips moving---like she was trying to figure out a problem and the answer was right on the tip of her tongue.

I turned back to watch my dad. I shouldn't have judged him so harshly with his art comment. He'd just been trying to keep up with the conversation. He really was good with Maya. He gave her kisses. He wiped her nose and mouth carefully---both dripped as she ate lunch from a tray that Cynthia had placed across her wheelchair. The living room was an odd space. The Board and Care was definitely more homey than the skilled nursing facility she'd started out in: it housed thirty-four residents and sat on an acre and a half of land that was nicely landscaped with trees and flowers. It didn't have an institutional feeling at all. The owners were very personable. They believed in the vitality of music, so songs were piped over the speakers into every room all day long. Musicians, like the violin player, came through several times a week for special performances.

But the space felt like something out of an existential play. It gave the impression of a cozy living room. There was a fireplace, writing desk, books on the shelves, tall purple candles standing in glass holders, an old-fashioned clock on the mantel. But everything was completely static. No one read the books. I'm not even sure they were real. No flame touched those wicks, no wood burned on the grate. There were actually two clocks in the room, neither of which worked. One was stuck at 12:15, the other at 12:48.

At one point my dad told us the time, and Cynthia had to tell him the clocks were broken. ``Like my dad,'' I couldn't resist joking to Maya. ``It's only right twice a day.''

Maya had kept her sense of humor. A few months back, Cynthia had asked her if she thought my dad was good-looking. Since my dad often called Maya his ``good-lookin' wife,'' Cynthia wanted to know if Maya felt the same way. ``Not really,'' Maya had answered. When my dad asked Cynthia to repeat what Maya had said, Cynthia quickly covered. ``She said, Really!''

My dad smiled, but Maya corrected her: ``I said not really.''

Maya hadn't missed a beat. Such intimate ribbing was the nature of their love, and it was heartening to see that interaction especially in Maya, whose physical state was so attenuated, her awareness so blunted, that you might think the woman she'd been had disappeared completely.

But she hadn't. She was there. My dad was one of the few people who brought her out again with his jokes and charm.

The living room seemed perfectly suited---unintentionally, one would hope---to Maya's state and that of many of the other residents. The Board and Care specialized in dementia patients, so it wasn't a surprise to find most of the residents oblivious to their surroundings. But I wondered: How many of us will end up in a holding pattern like this somewhere between life and death? In a room where time itself might as well be broken. I doubted Maya ever imagined this ending for herself. And yet here she'd been for almost a year and a half, not including the six months at the skilled nursing facility. I knew my dad had never imagined he'd end up in such a state: dependent on everyone around him, losing his grip on reality, struggling to complete the most basic human functions. My dad and Maya had planned carefully for their deaths but not at all for the end of their lives. Few of us do. We put it off until decisions have to be made for us. We're left to be wheeled around, or to dance a sad limbo with healthcare workers dipping us into chairs.

I wondered if I was seeing my dad's future in Maya: parked in a wheelchair, his mind and body tattered, unable to feed himself or communicate anything save his most basic needs. And who would listen to him? Who among us would give our precious time to stop by and visit? More than once I'd thought about the ending of Wallace Stegner's Angle of Repose, a fat novel about the West that had garnered the Pulitzer in 1971. The narrator is lying in bed after having woken from a dream, and he's listening to an eighteen-wheeler pulling hard up the highway in the distance. To riff on Stegner's words: ``In this not-quite-quiet darkness, while the diesel breaks its heart more and more faintly on the mountain grade, I lie wondering if I am man enough to be a bigger man than my father.''

Would I change my life to make my dad's life better? Could I overcome all of the resentment that I still carried so he'd know that someone cared about him?

Cynthia took Maya's lunch tray to the kitchen and came back with her dessert: a cup of tiramisu topped with cherries.

The orangutan lady returned to the porch. She set the stuffed animal down on the wicker chair again and fluffed the cushions around it. She talked to the orangutan. Or to the chair. Or perhaps to the plants. It was hard to tell. Maybe just to herself. I was fascinated by her. Especially when Dion's ``The Wanderer'' came over the speakers. She left the porch, then came back. She went through the gate but didn't close it all the way. She waited there like she'd forgotten something, or was making sure she hadn't forgotten something. She came back to the orangutan, grabbed it, said something to herself, then leaned down and spoke to the plants, then she went through the gate.

After the hour visit, we were all ready to go.

\begin{center}$*$\end{center}

I started writing about my trip that night. I'd put my dad to bed around eight, which was early for him. He'd been extra tired that day for some reason and wanted to go upstairs. I had time on my hands, so I sat at his dining room table and started to write about my mom---the dream I'd had about her and the rental car. I didn't want to write about my dad. Honestly, I didn't think he deserved a book. But somehow I knew the story was about him, at least indirectly. The story was actually about me and our relationship. For the first time it hit me that maybe I was there---maybe everything I'd done up to that point---was not about helping him, but helping myself. Am I somehow fulfilling a relationship that I always wanted with him? I wrote that night. I became acutely aware of how much he and I shared. Yes, his corny sense of humor. But also his need to write. At each epiphany in his later adult life he took the time to write down his thoughts. Here's a brief list:

December 17th, 1970: ``a day of insight \& decision Who am I?''

December 1st, 1983: ``A purification begins. My whole life is taking a new turn, old ways are being questioned, discarded, revealed.''

Summer of 1988: ``New Moon---New Focus.'' October 1st, 1990: ``A nother be-ginning.'' June, 1998: ``The Beginning.''

July, 2001: ``Here we `grow' again!''

The events are marked variously: meeting an inspirational person, leaving a woman, surviving cancer, or discovering a new philosophy about life that will finally set him on the right path to self-understanding and acceptance. He'd mentioned to me years earlier that he intended to write his autobiography. This was back in 1998 (``The Beginning'') after he underwent successful treatment for prostate cancer. His journals include a seven-page outline of that autobiography. As I sat at his dining room table and wrote, I was acutely aware that I might be finishing a book that he'd started. This made me wary because I felt that my dad wanted to use his autobiography to perpetuate his lies. He seemed aware of this tendency himself in the paragraphs leading up to his outline for the autobiography. He wrote: ``Maybe I should get it all down \& let the kids see what I have to say about my early life \& how that has impacted my personality \& the decisions I have made---Is this a good idea? Maybe I'm just looking for sympathy or a rationale for why I've made the mistakes I've made.''

I didn't want to aid and abet mistruth. His outline began, for example, by noting that his birth certificate was wrong and that he'd been born ``under the sign of the raven,'' which he'd called his ``power bird.'' The first chapter has this preamble on how peyote had transformed his vision of himself:

When I go under, I experience myself with all doors of perception open and I really believe that the quantum leaps in consciousness that have occurred could not have happened in other way for me---I've stuffed so much shit into my phs self that it took a major hit to blow off the crap. Thanks Great Spirit that this was put into my path as a means of growth. Some may get this from meditation or religion or whatever, for me it was the sacred mushrooms as a starter in the Native American Church at Heartwood when I was serving my apprenticeship for the Sacred Sweat Lodge \& continuing with the other psche physco-active medicine--- Eureka (I HAVE FOUND IT)

I believe my dad was able to find healing during his years at Heartwood. This would have been in the mid-`80s when he lived near Garberville after spending two years on Maui. He wanted to recreate himself so badly that he started to rewrite his life. I didn't want any part of his fictions, but I suddenly felt compelled to record my experiences and see where my new relationship with him was headed.

Upstairs, my dad sneezed. I stopped writing to listen as I might do with one of my sons. It felt so strange to be in his house, watching over him. Bizarre to imagine that I'd started this journey thinking my responsibilities began and ended with a green notebook. Now I had his life in my hands. We'd become part of the same story, the one I'd begun to write, or rewrite, without a clue as to where it would lead us. One thing was certain: I could no longer pretend to be hands- off, or that I cared for him simply out of a sense of duty. Whatever happened to him, I was all in. From that night forward we were literally on the same page. The blocks I'd stacked so carefully between us over the years would have to come down if I was to understand why I needed this relationship.

\begin{center}$*$\end{center}

My dad's phone started ringing around ten that night, but I didn't want to get out of bed.

It was cold inside the house. I figured if anybody I knew wanted to get a hold of me, they'd use my cell. Once the ringing stopped, I heard noises upstairs. I opened my eyes. Were they normal bathroom sounds? I recalled my dad's toilet-plugging incident from the previous summer. I got up and went over to the door that sectioned off the upstairs. I opened it, leaned in, and gave a listen.

It didn't sound like anything was wrong.

Five minutes later I was still awake. Something was definitely wrong, I could feel it. ``Fuck it,'' I said. I went upstairs and checked it out.

My dad was sitting on the edge of his bed, his hands stretched toward a small space heater.

``Everything okay, dad?''

``I was cold.''

I looked toward the bathroom.

``I cleaned up that mess in there,'' he said.

I peeked inside but couldn't see anything. When my dad had built the bathroom, he'd put the light switch, for whatever reason, against the far wall. I didn't want to step in and get my socks all wet if water covered the floor.

``What mess?'' I asked.

``I'll show you.'' He walked into the bathroom and picked up the wicker basket that served as a trash bin. He pulled out a pair of underwear and the sharp stench of excrement hit me.

``All right, dad. Let's get you cleaned up.'' I turned on the light and put liquid soap in his hands so he could wash them. ``Do you have any poop in your sweats?''

``They're fine.''

``Why don't you lower them so I can check.''

He was right, they were fine. With his sweats lowered, I leaned over to check---yes, his butthole---but he wasn't really in the right position for a close inspection. I wasn't sure how to proceed. He wasn't a baby. I couldn't flop him onto his back and spread his legs. Should I ask him to bend over and grab his ankles?

There were no plastic gloves lying around like they used at Primrose. My dad had thick leather gloves in the barn, lots of pairs, but that seemed overkill. I craned this way and that, but didn't really . . . um, dive right in, so to speak. I didn't get a view beyond an interior curve of his right butt cheek. Honestly, I wasn't prepared to check him closer than that. I did give a good sniff in that general direction. He'd wiped himself, and it all looked good from where I stood.

I got him settled back in bed. I wondered, as I climbed back in bed myself, if I should have checked him closer. I worried about that until I fell asleep and woke up sometime later to more noises coming from his room above me.

I hit the stairs again.

I wondered if he'd eaten something bad that had given him the runs. Or food poisoning. My mind was racing. What would I do then? Who would I call?

But he was sound asleep when I peeked in, snoring lightly.

False alarm. I was so relieved

The next morning after his shower I had him sit in his office chair and prop his feet up so

I could dry off his toes. He couldn't reach down there very well anymore. I laid out his clothes and left him to get dressed while I cleaned up the kitchen from breakfast. When I came out he was sitting on the toilet with the door open. He'd lost any embarrassment about it.

I remembered what Kay had written on her list: my dad could not be trusted to wipe himself properly.

Where was I supposed to stand? I wasn't comfortable staring at him while he pooped. I decided to busy myself. I put on shoes and took out the trash, including his soiled underwear from the night before. I wondered briefly if I should just wash them. Nah, I thought. Just toss the whole mess.

I came back in the house. He was wiping himself. I stood in the hallway outside the bathroom and watched. I didn't want him using the toilet bowl water to clean the rim or anything. He wiped in an interesting way: he leaned forward and stuck his right hand between his legs, then reached toward the top of his crack and wiped down toward his genitals. It seemed a bit convoluted to me. I would have lifted the right butt cheek myself and used the side option. But I let it go.

He brought the toilet paper back through his legs and inspected it. There was definitely poop on there. He folded the sheets in half so there was a clean side and then followed the same process as before and wiped himself again.

When he finished the second wipe he held the poopy paper over the trash basket beside the toilet. ``I guess that goes there,'' he said.

``Uh, you can put that in the toilet,'' I said. But I was too late. He'd already dropped it in the trash basket. ``So it doesn't stink up the bathroom,'' I said to finish the thought.

Oh well, information for next time.

He nodded at me. ``That makes sense.''

``Wipe again,'' I told him. ``Just to make sure.''

He did. It came back clean. He dabbed the tip of his penis with the toilet paper and dropped it in the trash basket.

``All right, dad.'' I moved into the bathroom next to him. ``Let's wash your hands.'' He pulled up his pants, and I flushed the toilet for him.

\begin{center}$*$\end{center}

I organized a small gathering of my siblings and their spouses for dinner at my dad's house on Wednesday. We listened to the Cardinals-Giants game for the National League pennant on the radio. I had a friendly rivalry with Kathleen, who was a diehard Giants fan. The Cards were down two games to one.

My dad had been napping when everyone arrived, getting his rest after a full day of activities at Primrose. I woke him up halfway through the game so he could listen with us and enjoy everyone's company. He couldn't follow the plays very well. He told us several times, ``I'm a 49ers fan.'' He also said funny things like, ``I didn't know the 49ers had a baseball team.''

We laughed, which encouraged my dad to join our banter. I was especially happy to see Mike. It'd been a year and a half since his cancer had spread to his lungs, and he'd decided, after initial surgeries and chemotherapy, to let the disease run its course. He felt good. He still played golf and was generally active. He'd chosen to maintain his quality of life, even if it was shorter, rather than try and extend it through extreme treatments which left him weakened and vulnerable to sickness because of the side effects. I admired his decision. It was the best one for him, and I reminded myself to enjoy the time we had left together.

I'd always admired Mike. All of us kids had. He was the oldest (ten years my senior), the smartest, the best looking. He had a Superman aura about him: six-foot-four, blue eyes, black hair. He'd been a basketball star in high school. When he joined the Air Force at 18, they'd sent him to the Defense Language Institute in Monterey to learn Hungarian because he'd scored so high on his aptitude test. He went on to an amazing career in advertising in New York and San Francisco, winning multiple Clio awards for his work with Apple Computers, Reebok, Intel, and California Cooler. He'd retired wealthy in his mid-40s and pursued a second career in fine art.

Mike had been my dad's first child, and they'd had an embattled relationship their entire lives. My dad had sent Mike off to the Maryknoll Seminary when he was fourteen. This was during my dad's devout Catholic phase; he wanted Mike to become a priest. The father-son irony can get no thicker: the son directed toward a life of celibacy to compensate for the father molesting his daughter (Terry was eleven at the time). What does the Bible say in the Second Commandment---from Shakespeare's well-known version in The Merchant of Venice: ``the sins of the father are to be laid upon the children.''

Mike would not see my dad by himself. He always wanted to have someone else with him. He needed that buffer, even as he was dying. I wanted Mike to experience this ``new'' dad courtesy of dementia: the kindler, gentler version of him. There was no possibility of mending their relationship. There had been too much trauma along the way. But I thought it could end on a note of conciliation. This was me channeling my mom.

The gathering that evening almost didn't happen. In the late afternoon I'd suddenly smelled gas in the house. I tracked it down to a valve under the kitchen stove. With half an hour before the dinner was supposed to start, I managed to get the utility company out there and patch it temporarily. I'd have to call a plumber later on and get the whole valve replaced.

The party was saved. And though the 49ers' baseball team scored three runs in the bottom of the sixth and beat the Cardinals six to four, I enjoyed watching my dad interact with my siblings. He laughed with them and joked. I liked seeing how engaged he was, how happy, and I wondered what his life would look like if he weren't able to live in his home.

As the evening ended and people rose to leave, my dad stood close by my side. He asked me, ``You're hanging around for awhile, aren't you?''

I realized he was nervous that I might leave too, especially since Kay wasn't around. He'd asked me a number of times that week: ``So, do you hear anything from Kay?'' I would remind him that she was staying at her boyfriend's house for a few days. He'd nod and say, ``Oh, right.''

My dad hadn't been on his own in such a long time now. I thought about him living in a nursing home. Would they take care of him? Would he know anyone? Will he feel lonely?

And then I thought about the gas leak in his house. And him going up and down those stairs.

\begin{center}$*$\end{center}

I spent Thursday afternoon in my dad's office going through his mail while he was at

Primrose. Most of his and Maya's letters already came to me in Missouri, but every now and then a random bill slipped through the cracks. I found nothing pressing. I flipped through a holiday catalogue sent to my dad from the Southwest Indian Foundation. All manner of gifts filled the pages: clothing, jewelry, peace pipes, mugs, moccasins, candles, silverware, books, dream catchers, herbal recipes, Navajo teas. There was information on Inter-Tribal gatherings in Gallup and opportunities to buy food baskets or wood stoves for needy families at Christmas. The brochure had a photograph of one of the families---a smiling Navajo mom and her four kids. You could follow the Foundation on Facebook.

I compared the stuff in the catalogue with items hanging from a large black and white cowhide that my dad had tacked to his office wall years before: an ornamental peace pipe, a medicine bag, lots of feathers, a plate-sized wooden engraving of a buffalo nickel; above and below the Indian profile my dad had inscribed IN US WE TRUST.

At the center of his pelt hung a drum the size of a large tambourine. At the center of the drum was a raven, my dad's animal totem, along with its name in Cherokee: GO-LA-NV. You could tell he'd made the drum himself. The art work and lettering on the stretched deer skin were a bit wonky.

Ravens and crows popped up in his journals quite a bit. In March of 1989 my dad was living in Mendocino County and attending Heartwood, the healing center in Garberville. He underwent hypnosis with the help of a shaman and peyote. His experience was recorded on sheets of yellow legal paper that he'd tucked into his journal. The writing wasn't my dad's careful script. I figured the journals were a record of what he'd said during the sessions, perhaps with prompts by the shaman. My dad referred to a ``Phil-Crow'' who seemed to represent him as a boy:

Phil-Crow called forth

He looks on Carver as his father. Looking for love \& nurturing. Looking for an elder who can teach him the sacred ways. Phil-Crow will create his own father.

Phil-Crow speak to father.

Why did you abandon me? And later in the same session:

Little Phil Crow has been abused all of his life! young adult Phil Crow was only looking for love \& understanding. He became a fierce warrior. He turned on that. He healed

[next step in the healing?]

TO FIND OUT WHO I AM ANGRY AT! To address the demon inside.

During his second session five days later he said:

If you hear the call of the sacred raven, this will help. It is a signal, an omen reminding me to heed my interior truth in this matter.

He reminds me that I am still a warrior but a peaceful one. True love comes thro gentleness. Blend with\/love my fellow

The session ended with reminders ``not to forget my children. Reach out \& try to heal the wounds . . . Letters to write\/calls to make w\/in the next week.'' I didn't remember if I got a call or letter from my dad around that time. I was living in Oakland, finishing up my last semester at UC Berkeley. I remembered that he didn't show up for my graduation. Then again, I may not have invited him. A quick Google search on raven and crow symbolism among Native Americans revealed ``messenger,'' ``transformation,'' and ``healing.''

I wanted to roll my eyes at what my dad called his Native American Church. I wanted to howl when I read his philosophy about why running was so important to him: ``Primal vibrations of my Native ancestors demand a return to the moving meditation of a trained body, a trained mind and an open spirit that listens to the wisdoms of the Grandfathers.''

There was only so much I could take about this invented ancestry, this Native American Church that he'd adopted as an adult and become its most enthusiastic disciple.

And yet I saw clearly in his words a desperate desire for self-understanding and self- acceptance. He was striving so hard to overcome anger and come to terms with an abusive childhood. I understood why he chose the raven as his animal totem: above all he wanted to be healed from some primal hurt. As an adult I've always thought that my dad must have had such a fucked-up childhood that he adopted an idealized version of the Cherokee so he could feel like he belonged to something: a people, a belief system, a way of life in nature that gave him larger purpose and identity.

It was sad that he wasn't able to find those things in his own family---with my mom and us kids.

\begin{center}$*$\end{center}

The following year at Christmas, in 1990, he received no holiday cards from any of us. It wasn't a conspiracy. We didn't get together and plan it. It simply worked out that way. He'd ``split the blanket'' with his girlfriend of five years in October of that year, moved out of Redway and was living in a group home in Marin County. He was attending the College of Marin, studying sculpture and mixed media. He was also taking ecstasy alone and in groups. It wouldn't be long before he met Maya, who had a house in San Rafael. The lack of Christmas cards pissed him off so much that he took to his journal. In lieu of a ``hate letter'' which someone advised him to write ``to purge the anger,'' he penned a two-act play. The first act introduces a composite character of us eight kids. Here's an excerpt (the ellipses are mine):

Kids: Hey Christmas coming up, gotta write some cards to friends and bro's and sis's . . . Should write a card to Dad. Where the hell is he anyway? He split up with J---. God---they were together 5 years! Wonder what happened? Oh, well, not my business. Never understood him very well anyway, He just disappeared for months, even years sometimes. Weird---Why in the hell he ever divorced Mom is something I'll never understand. They seemed so happy---never heard `em fight, or anything. What the hell went wrong? I mean she's so loving and understanding, and the rock when you need her---

Well, can't worry about that now what's done's done---He's probably doing his ``thing'' whatever the hell that is. Indian stuff or somethin'. Well, If I know Dad, he'll bounce back and be at the next family gathering . . . but hell, what do I say to him? It's not like we've been close or anything. He gets kinda preachy about the ecology and the world and I don't really want to hear all that. I'm just trying to have a good time with my bro's and sis' He lives in another world anyway. He tries to mix with us, but hell, he just looks uncomfortable most of the time. Feels guilty about the divorce, probably. Yeah, that's it. well, let him ---28 years with Mom and he decides suddenly to ``Go find himself'' Bullshit. He just wanted to screw around. Broke Mom's heart---Let him stew in his own juice for awhile, serves him right.

There's much clarity about himself in how he imagines that we see him. He knew exactly how he came off to us. We didn't buy into his newfound philosophies. We didn't believe they'd changed him in any essential way. So in that sense we represented a possible threat to any story he was spinning to other people. Though none of us, I imagine, ever would have blown his cover. I think of what Terry told me over the phone when I'd asked her about my dad molesting her: ``It never occurred to me to tell anyone about it.''

I felt the same way about my dad and his friends, even Maya. Why would I tell them that my dad was a liar, a cheat, a con man? He was their problem, not mine. He'd created an image of himself in their world: the war-resisting hero, the nature-communing m\'etis, the wood-carving artist. They loved and admired him for these qualities.

I also see a lot of guilt surfacing in his (our) monologue over the divorce with my mom. He traces wild arcs in his journals, from guilt over his past actions to deciding that the answer to all of his problems is simply to make himself happy. Here it is in a nutshell from his ``day of insight \& decision'' on December 17th, 1970 when he asked ``Who Am I?'':

Fill in all the blanks--\& you still haven't touched it---because today it came thru the fog \& the crap \& the lazy messed up intellectual bullshit that I –am---me [underlined three times]---really! accept it, revel in it, live it---learn to love it, because it's all you've got this side of the pit of nothingness to hang your hat on Charlie---you are you---not what you think you are, not what you want to be, aspire to be, dream to be---But ARE [three underlines]. \& you God-damn well better like it, because it a'nit going to change much,---you've been all this that---a 1000 times, \& nothing worked for very long, did it? The reality never quite lived up to the dream---at least not for very long---Why? Because it was unreal---not you [four underlines]. You know who you are--- accept that \& maybe---just maybe you'll find a small measure of happiness in this disaster we call life---(It might not turn out to be such a disaster).

Peace be to Me [four underlines]--- A Happy Phil to Me, always---

His self-centeredness clogs the journals. My issue with him growing up was more basic and selfish than him divorcing my mom: he never seemed that interested in what I was doing. I see from his journals that he was too absorbed in his own path to enlightenment. For the most part he saw the paths of others close to him not as positive connections that could help him achieve his goals but as detours and disruptions. Sadly, most of his goals centered on trying to connect with other people, especially his kids. My dad had so many tools in his wood shop, hundreds of them: heavy drills, saws, pry bars, routers, gouges, and knives; he had cases and cabinets of razor-sharp chisels and the most delicate sanders to smoothen the roughest of driftwood. And yet he lacked the most fundamental tool of all: how to talk with his kids.

In Act Two of his play he runs through each of us eight kids in turn. I was struck, the first time I read it, that he'd so accurately described his future office back in 1990: ``Small room, desk, books, wall full of Indian artifacts, children \& grandchildren pictures, neat, orderly bachelor pad. Guitar in corner, sculptures, etc.'' He was still a decade away from outfitting the room where I sat going through his mail, but he described it down to a tee. Here's what he wrote about me:

Patrick James---the Court Jester, the quick wit the facile brain---The ``sleeper'' knows exactly what's going on and plays ``let's pretend.'' Used to be the only one who wrote me consistantly in Hawaii and I loved it. Since I'm stateside I feel like a stranger---What did I do? We have so much in common---Does he know that I'm literate? That I passed the Mensa Test and was invited to join? That I've read most of the classic's and can wax eloquent on Philosophy, Religion and the Tao? I doubt that he knows or cares---Where did we slip a cog? Why did he write a derogatory story about me and never have the courage to show it to me? Fear? or loathing? what is our unfinished business? I don't know Pat---He's getting married and I had to hear it on the grapevine---why couldn't he tell me? I don't understand---Did he think I was not interested? He's wrong. I don't speak French, but I understand most english (small ``e''). why don't we feel comfortable with each other? I'd like to---No shared experience---no ``Hey Dad Linda \& I are going to be in the area and would like to have dinner?'' Why---Do they have dinner with Ann? Does he know that I care? How do I let him know? Maybe next Christmas he'll answer---maybe I had the wrong address---Maybe I have the wrong attitude ---Maybe.

I was stunned the first time I read it. It was a shock to see myself as my dad saw me: laid out like a patient on an operating table and sliced open to my core. Writing is so powerful. He resents a short story I wrote about him leaving my mom. I'd forgotten about that story (I guess I did hold the divorce against him after all). At the time I was considering graduate school in creative writing rather than in French, so I was working on writing samples. He must have heard about the story from my sister-in-law Francis (Mike's first wife). My dad had been doing handyman work for them in Mill Valley around this time.

His brief paragraph about me was the most intimate glimpse I'd ever had of my dad: an honest expression of his attempt to understand me and our relationship. He'd been wrestling with the same issues that I was as I wrote about him in my journal: trying to figure out the relationship. Apparently neither of us had the right tools to bridge that gap. We had still communicated with one another. I had invited him to my wedding the following year, in August of 1991. I'd walked down the aisle with him on one side and my mom on the other. He had a great time at our wedding reception, leaping through hula hoops (we'd brought them for the kids to play with, but my dad had to prove how agile he was).

We had communicated, but we'd never really talked. We wrote about each other, directing our feelings and frustrations onto the page at different times in our lives. I suppose I was still doing that. Whatever his portrayal of me, at least it was honest. He wasn't couching his feelings in layers of New Age philosophy or spinning them through cycles of Sacred Grandfather Lore. I felt a real connection with him.

Too bad a quarter century had passed since he'd written it.

His words also reminded me of times when we had connected. I'd forgotten that I'd written him regularly when he lived on Maui. Surfing had been a tangible link between us since I'd been a teenager. My dad was actually the first one to put me on a surfboard at Salmon Creek beach. He hadn't stayed with me; he went off to surf with others. But he'd given me that start. My first memory ever was of my dad. We were in the bathroom in Thousand Oaks, and he's showing me how to fold a piece of toilet paper to wipe my bottom. So I was probably around three years old. I've racked my brain for an earlier memory---believe me---but that's the one I come up with.

As I wrote about my dad's portrayal of me, I wanted to know more about father-son relationships in general. I'm an academic, so I did some research. I found an excerpt online from Frank Pittman's Man Enough: Fathers, Sons, and the Search for Masculinity. Pittman writes about ``Father Hunger''---our search as men for approval from our dads. We want them to be interested in what we're doing and above all to approve. It was true. I was fortunate that my mom had filled so much of that void for me growing up, but I saw a pattern in my life with male authority figures: searching for approval from those I admired---coaches, teachers, bosses---but also keeping them at a distance for fear of getting too close. It wasn't too hard to trace that back to my largely-absent dad and his tendency to pop in and out of my life.

The amazing part in my dad's play was that he wanted my approval. He was trying to impress me: ``Does he know that I'm literate? That I passed the Mensa Test and was invited to join? That I've read most of the classic's and can wax eloquent on Philosophy, Religion and the Tao? I doubt that he knows or cares.''

Evidently the Father Hunger cut both ways, for some men: fathers needing approval from their sons. Since my dad never knew his own dad, perhaps his Father Hunger was so deeply entrenched that it easily fixated upon his sons as we grew to adulthood. I was fulfilling a bizarre variation of William Wordsworth's, ``The Child is father of the Man.'' My dad and I were spinning to the square dance of fatherhood, our steps hopelessly out of sync.

And what happened when a father had dementia and could no longer give approval to the son? What then became of the Hunger? Where did it go? Upon whom did it fix?

\begin{center}$*$\end{center}

Thursday night I convinced my dad to sleep in the bed downstairs by fibbing to him. I'd taken apart his desk while he was at Primrose and moved the bed in there from the living room. Then I went upstairs and pulled all the blankets and sheets off his bed. When he got home later that evening I told him that I was still doing laundry. ``If you could sleep downstairs tonight,'' I asked him, ``that'd be a big favor to me.''

It worked. But the results were mixed. He didn't have to go up and down the stairs, so it was better for safety. But he roamed more downstairs. I'd startled awake at three or four in the morning on Friday to the sound of the toilet flushing. I listened hard for a minute, then heard his office door close. I figured everything was okay.

At seven a.m. on Friday I heard him moving around again. I'd told him to stay in his room when he woke up: he could only leave to go to the bathroom. He didn't follow those directions. I discovered him brushing his teeth, telling me he was ``getting ready to go.''

The kitchen light was also on, so he must have reached through the barrier gate and flipped the switch.

We ate breakfast together, and then I got him into the shower. I followed Kay's instructions for giving him cues on how to scrub himself. I handed him a washcloth. ``Wash your face first, your bum last,'' I told him.

He chuckled and said, ``That's interesting.''

I waited until he was done. He turned off the water and stepped out of the shower. He stood facing the bathroom mirror, looking down at nothing in particular. He seemed to know there was something else he had to do, but it didn't come to him. So he looked and he waited. Water dripped off his body. He gave a light shiver.

I'd grown more comfortable looking at his body: helping him get dressed and undressed, monitoring him on the toilet, making sure his feet weren't swollen, checking his skin for unusual-looking moles (Mike's melanoma had me on alert). One of the things I saw as he stood there naked, completely unself-conscious, was the man he used to be. The exercise guru. The long-distance runner. He'd always kept himself in such amazing shape. He'd celebrated his 50th birthday by surfing naked in the frigid Northern Californian waters of Goat Rock (his birthday was in February). I've already mentioned the marathon on his 64th birthday, and the walk-a-thon on his 70th. He'd also been a boxer and practitioner of martial arts, Aikido and Taekwondo.

His once hearty chest had now sunken into a soft belly. Everything sagged on him: his shoulders, his biceps, his buttocks. His skin hung off his bones, pale and wrinkled and splotched. His stance before the mirror was wide. Not like a boxer anymore. More like a drunk who'd stopped in the street a moment to get his bearings. Kay had noted in her reports that he'd had minor falls the past few months.

I also saw myself in his body. My possible future-self thirty years down the road. This was another block I had to get over in our relationship: our physical similarities. I resembled him more than any of my three brothers: same height, same hairline, similar build, though he would have been the stronger of us two. He liked to give bear hugs, to show off his strength by lifting people clear off their feet. He was still surprisingly strong, but age and dementia had compromised his balance. Sometimes he held my hand on the sidewalk before we stepped off a curb. I gazed at his body like Ebenezer Scrooge contemplating visions drummed up by the Ghost of Christmas Yet To Come.

Will this be my fate? Will I too lose my mind?

He gave another shiver. The bathroom had steamed up from his shower, and there was also a heat lamp, but I had to remind myself how quickly he got cold. I stepped forward, grabbed a towel, and wrapped it around his shoulders.

That was his cue. He nodded, snapped out of his reverie, and began to dry himself off.

I drove him to Primrose. We were early because he'd been anxious to go. We greeted several of the day-clubbers and staff who were sitting around the kitchen table drinking coffee. The Day Club had its own building separate from where the permanent residents at Primrose lived, just across the compound. Besides a few offices in back of the Day Club building, there was a main activity room, a kitchen, a dining room, and several bathrooms. No more than fifteen day-clubbers were there at any one time.

I chatted with Laurie, the Director of the Day Club, and Yolanda, one of the staff. A woman sitting at the end of the table---I'll call her Naomi---pointed at me and said, ``Handsome man.'' She waved me over for a hug. I bent down to hug her and she kissed my neck. It felt like she was giving me a hickey. I started to pull away but she hung onto me. She grabbed my hand, kissed it, held my fingers tightly.

Naomi's memory was farther gone than most of the others. She'd been going to the Day Club for five years. She was always dressed nicely, not the sweats that predominated. Someone at home spent time on her hair and putting jewelry on her. She was still aware enough to tell me that my dad ``has a girlfriend.''

This was a woman I'll call Wendy, also sitting at the table. I'd watched her as we came into the room. Her whole face lit up when she saw my dad. She wasn't shy about expressing her feelings, either. ``I'm just crazy about him,'' she told me.

``I'm just crazy,'' my dad said.

Everyone laughed. My dad liked the attention a laugh got him, but dementia had mellowed him. He didn't preen now after a wisecrack. He didn't throw out his chest and look around the room like we were playing King of the Hill and daring someone to knock him off the top.

As more day-clubbers arrived, we moved across the room to a larger table so the group could play dominos. Naomi sat on my dad's right side. Wendy slid in on his left. Wendy grabbed his hand and patted it. She and my dad kissed quickly on the lips. Wendy caught herself: ``I kissed him,'' she said under her breath to Yolanda, ``and then I thought, `Uh oh, his son is standing right there!'''

The table grew quiet a moment.

``You're fine,'' I said. ``If he's happy, I'm happy.''

Wendy smiled. Yolanda put her palm up to me, which I completed with a quiet high-five.

``All right!'' Yolanda said.

Yolanda meant it was all pretty harmless. I might have felt differently if Maya had been my mom. Or if my dad didn't have dementia. I didn't see him as the instigator of the relationship. He was in heaven, though, sitting between two women who fussed over him. He was smiling, cracking jokes, the center of their attention. Right where he wanted to be.

Graciela, another staff member, came into the room and hugged all of the residents. ``I like to start off the day with hugs,'' she said. ``It's good energy.'' She rubbed my shoulder as she passed by way of greeting.

I didn't have any tiles, but the group encouraged me to play. I teamed with a woman staring at the tiles in front of her. She wasn't sure if she had the right color or the right number. I pulled out a white tile with eleven black dots on one end. I asked her if it matched any she saw on the table. She checked a moment, then located one.

We took turns around the table. Yolanda helped the woman next to her: she showed her where a tile matched on the table, then she passed the tile down to Wendy who put them together.

``Thank you, my love,'' Yolanda said.

Naomi needed help from Yolanda too, but she managed to match a tile with a single dot. My dad could mostly match his tiles, but he had Wendy to help him when he got confused. She was sharp and matched them quickly.

We played several rounds, everyone engaging their brains, or trying to, which was the point of the game. When someone got rid of all their tiles and won, we cheered and clapped. As we finished the game and cleaned up the tiles to put them back in the box, my dad held one in front of him and considered it a moment. It was blank on one side. He said, ``It's just like my brain.''

Normally I'd say he was going for a joke, but he'd said it so quietly.

\begin{center}$*$\end{center}

One of the chores I completed that week was changing my dad's POLST form--- Physician's Orders for Life-Sustaining Treatment. I'd completed one for him the previous April when he'd started at Primrose. Since I had Durable Power of Attorney for his health care, it was my responsibility to decide, in consultation with his doctor, the medical care he received in an end-of-life situation. I was guided by my dad's Advance Health Care Directive in the green notebook that indicated a ``Choice Not To Prolong Life'' under the following circumstances:

I do not want my life to be prolonged if (1) I have an incurable and irreversible condition that will result in my death within a relatively short time, (2) I become unconscious and, to a reasonable degree of medical certainty, I will not regain consciousness, or (3) the likely risks and burdens of treatment would outweigh the expected benefits.

You can see the dilemma for people with dementia: it's incurable and irreversible, but you can live with the condition for years. So the first category didn't apply to my dad. I'd have to base my decision on category three: would the benefits of reviving him outweigh ``the likely risks and burdens''?

The first time I filled out the POLST, I didn't think twice about it: I checked the box ``Attempt Resuscitation\/CPR.'' This decision required me to check the next box down, ``Full Treatment,'' which included the use of tubes in his throat (for air or food) and any machines to get his lungs breathing and his heart pumping.

Why wouldn't I want my dad resuscitated?

Kathleen and Cynthia had persuasive arguments in favor of Do Not Resuscitate orders in certain situations. I had to consider my dad's case: he was eighty-four, he had progressive dementia, and he could no longer take care of himself. What would be his quality of life if he had to be revived and fed by machines? What would I be ``saving'' him for? A hospital bed? The corner of some room? I thought of Maya in her wheelchair. Honestly, I couldn't imagine anybody choosing that end of life for themselves. I certainly wouldn't want it.

Kathleen had worked in health care her whole life. She'd seen the treatments families had approved for the elderly because, as a society, we're so intent on keeping people alive at all costs. Oftentimes, in her view, at the expense of the patient. Both she and Cynthia had Do Not Resuscitate orders for themselves. Kathleen wasn't telling me what to do with the POLST. I'd asked her about it, and she gave me her honest opinion.

So, what was I going to decide for my dad?

I'd assumed that resuscitation meant bringing him back to full health. But my dad was not in full health. He would never be in full health again. He would steadily decline until he died of dementia or a condition brought on by dementia. What ``benefits'' could he expect if he were to die and be brought back to life?

I had to think about it. This man who'd roamed so far and wide would not want to be confined or living as an invalid in a decrepit state of life. Since he had not prepared for this eventuality, his fate rested with me.

I changed my dad's POLST: Do Not Resuscitate\/DNR (Allow Natural Death); Limited Additional Interventions; No artificial means of nutrition, including feeding tubes. I asked his doctor to sign the form and then gave a copy to Primrose.

Cynthia left me with one last thing to consider with regards to my dad. She'd spent many years taking care of the dying. She told me that one of the gifts people give us in their last stages of life is the compassion we discover in ourselves, a deep feeling of empathy for another human being that perhaps we didn't know we had. 


\chapter{}

I called the VA in early November and gave ``the order'' to place my dad in a skilled nursing facility. The process could take a month or two, but the phone call set the wheels in motion.

After months of wrestling with the pros and cons, I was relieved to have finally made the decision to move my dad out of his house. Beyond his decreasing ability to care for himself, and my increasing frustration at managing his care from a distance, one idea guided me: my dad would have to leave his house and Kay sooner or later, so it was best to make this break sooner.

Kathleen pointed out that my dad would ``imprint'' on the staff better if he moved in while he was still mentally aware, if declining. This would give my dad and the people around him a chance to bond. The better the staff knew him, the better they'd take care of him when he no longer recognized them. So a move sooner would increase my dad's long-term quality of life.

Sooner was also better from a legal stance. I'd retained an attorney back in August to help me better understand my role as Trustee, which included selling my dad's house. I'd been feeling overwhelmed, and needed advice and guidance. It was one of the best decisions I ever made and another example of how professionals can lighten the load of managing care. I found the attorney through the Santa Rosa Bar Association and met with him during my visit in October.

He told me that I personally was carrying a lot of risk because my dad had not established a formal agreement with Kay. The attorney thought she'd be considered an employee by the courts. If she got hurt on the job, or grew disgruntled and filed a claim, I'd be responsible. There might not be any problems at all with Kay, he said, everything would probably be fine. But liability was ``baked into the situation'' because I'd inherited her as my dad's caregiver.

The attorney was the one who'd told me that it was best to make the break sooner, before an emergency happened: either a medical one (like a fall) or a legal one (like a lawsuit).

He'd gotten my attention. I appreciated his objective view on the situation. I myself struggled with the idea of taking my dad out of his house. I'd emailed all of my siblings to get their input. My dad had a good quality of life in general with Kay and at Primrose. He had a beautiful piece of property that he'd shaped with Maya into an artist's haven, filled with their love and aspirations.

Kay cared deeply for my dad, but she was not a health care professional. My dad loved his home, but how much of it did he really benefit from these days? He didn't get out on his property much, or work in his wood carving studio. He used the bathroom and the bedroom. He ate at the table. Did he notice Maya's art on the walls anymore? Or his finely crafted wood sculptures?

There was no best time to move him out. After my meeting with the attorney, I realized that I wasn't willing to put my family in Missouri at risk so that my dad could stay at home. The attorney's advice was good: make the choice before the choice gets made for you.

So I made the call to the VA. I knew the hardest part lay ahead. Taking my dad out of his house would be the biggest challenge yet. Because once he left, he was never going back.

\begin{center}$*$\end{center}

I was finally ready to move my dad out of his house. But the VA had other plans. From half a dozen conversations over the next month with my dad's social worker and others at the

San Francisco VA, I learned that my dad did not meet their criteria for placement in a skilled nursing facility. Basically, he was too healthy. He had memory issues, but he could still do a lot of things for himself. He had no medical condition that required the regular presence of a nurse.

Now what was I going to do?

I was supremely frustrated. I'd done a lot of work to coordinate the move. It had taken a lot of emotional effort to weigh all of the options and finally make the decision. Now I was being blocked. At the same time, Kathy Vincent and others helped me understand that it wasn't best to put my dad in a place where he didn't belong. That might actually make him decline faster.

I grudgingly accepted their wisdom. My dad's living situation hadn't changed---I still thought he needed to move out---but his options had narrowed. The skilled nursing facility was off the table for now. If he moved into a Board and Care, the costs for him and Maya would strain their finances.

But the cost for keeping him at home kept rising, too.

The VA didn't contract with Board and Care facilities, but I asked Kathy Vincent: Could the VA continue to pay for my dad's Day Club if he lived at Primrose? If that was possible, and we deducted that amount from his total rate, then Primrose would be less expensive than what we were paying Kay. My dad could have twenty-four-hour care for less money and live in a place that was safer and where he'd be surrounded by people he already knew.

Kathy Vincent worked her magic in the VA system. I contacted Primrose, and we arranged the VA supplement as part of the total cost of the Board and Care. After a week or so of reviewing their occupancy status, Primrose set a date for my dad to move in: December 16th.

I bought plane tickets to California and arrived on December 13th.

\begin{center}$*$\end{center}

I didn't tell my dad he was going to be staying at Primrose until the day before the move.

It was a hard decision whether or not to include him in the process. Because he had largely lost the ability to keep track of time, he'd get anxious and fixate on upcoming events if he knew about them in advance. I'd asked Dan O'Brien, the nurse at Primrose, if I should tell my dad beforehand that the move was permanent. ``I wouldn't,'' Dan said, ``unless he starts asking.''

Our story was that Kay would be spending time with her family during the holidays. She'd asked for a vacation between Christmas and New Year's, so that part was generally true. Once we got my dad settled at Primrose, we'd simply extend the story: tell him Kay needed more time so he'd have to stay a little longer. Eventually Primrose would become his new normal, his new home. We might never have to tell him the move was permanent. It just would be.

At least that was the plan.

I picked him up from the Day Club at Primrose on December 15th. On the drive back to his place, I said in an offhand way, ``I heard you're going to be staying at Primrose tomorrow night because Kay is visiting her family.''

My dad gave a nod. ``That's what I understand.''

This was one of those situations where it was helpful to pretend that he already knew what I was about to tell him. It seemed a good strategy. Better to have him go along with something he wasn't sure about than to fall apart fixating on something he couldn't understand. Again, it was a hard call: to hoodwink or not to hoodwink? The pro was a calmer, more manageable dad in the short-term; the con would be dealing with him if he felt duped. Since my dad was mostly con himself, that might not take long. Then again he had dementia. He'd been mostly agreeable with other big transitions to date: losing his license, joining the Day Club. So who knew how he'd react?

I tried to pump him up for his stay at Primrose. ``It'll be like going to camp,'' I told him. ``An adventure.'' I looked over at him. ``You've always loved an adventure. It's one thing you've passed on to all your kids.''

It was true, and he seemed pleased by the comment.

I'd spent several hours that morning preparing for the move while he was at the Day Club. Primrose had given me a thirteen-page pamphlet---``For A Successful Move In''---with general guidelines on what to bring, what to expect, and how to make the transition as smooth as possible. I started with the Clothing and Toiletry Checklist. All his clothing had to have his name clearly marked on it: shoes, socks, underwear, shirts, jackets, pants---everything.

They recommended using a laundry pen rather than a permanent marker, which eventually washed out. I remember standing in the dining room with all of my dad's clothes stacked on the table, ready to be marked. I paused over the name permanent. Suddenly I felt overwhelmed. And rushed. I had to get everything on that list and decide what he'd take from his life here: furniture, carvings, photographs, books, his stereo, items for his ``memory box'' that hung on the wall outside his bedroom and helped him recognize where to go. Then I had get it all into boxes and the back of the car before I picked him up at the Day Club. And I hadn't even told him he was moving yet. There was no section in that pamphlet on How To Lie To Your Loved One For His Benefit And Yours.

I picked up the permanent marker I'd bought that morning (I couldn't find a laundry pen) and told myself: Focus on one thing at a time. You don't have to deal with the whole house and the rest of dad's life right this minute. Just get through the list.

I picked up a grey sock (ten pairs recommended) and pressed the name CARVER into the fabric on the sole. I put the sock down, grabbed another, and wrote his name again. Every item of clothing, every time I spelled out his name in big black letters, the word kept coming back: permanent. This was it. After that night he was never coming back. Eventually I would have to clean out the house, the barn, the art studio, the property---the whole kit \& caboodle as my mom would've said.

Jesus, I hoped I was making the right decision.

\begin{center}$*$\end{center}

The next morning my dad surprised me with a dream he'd had the night before: men breaking into his house and throwing him in the back of a car. We were standing in the dining room about to sit down for breakfast. He grew animated and started acting out the scene between himself and the intruders.

``Why are you taking me?'' my dad said. ``Where are we going?''

He shifted into a tough-guy voice: ``We'll tell you once we get there. Now shut up and get in the back of the car!''

He looked at me: ``And then this morning you start throwing sheets off the bed. But I told myself, `No one's going to take you anyplace you don't want to go'.''

I wanted to grab his shoulder, reassure him that he'd be fine: no bad men were coming to hurt him. On the other hand I didn't want to fuss. The calmer I stayed, the more convincing our story would be. ``I just needed to do the wash,'' I told him. ``Kay usually does it when you're at Primrose.'' It was all perfectly normal.

But he knew something was up. He'd sniffed us out.

The month before, when I thought he'd be moving into a skilled nursing facility in Petaluma, my siblings and I agreed that he should be included in the process. Kathleen, Cynthia, and my youngest sister Cindy took him on a tour of the facility. Cindy had called him the night before to let him know they'd pick him up in the morning. Kay had reported on her monthly time sheet that my dad was ``very upset'' by the news. He had needed to talk about it for a couple of hours with her before going to bed that night. When the three of them arrived the next morning, Cynthia had asked Kay how my dad was doing. Kay told her that ``he'd been upset and crying.''

Cynthia later wrote me about their visit to the skilled nursing facility. From her perspective, it went ``very well.''

But according to Kay, after the visit my dad was still upset. The skilled nursing facility had a bed open, and Cindy had encouraged my dad to accept it. Kay wrote: ``He was up again at 10:45pm crying. He didn't want to lose me and not be able to see Maya. I reassured him as best I could. Despite the fact that a room was available now at the facility Carver was adamant that he was not ready.''

This was one of the reasons why we hadn't told my dad about his move to Primrose: I didn't want to upset him. It was also a situation where distance compounded the problem. It was hard for me to know exactly how my dad was doing at any given time. Kay was reassuring him. But could she also be stirring him up? Unconsciously seeding resistance in him? She had much to lose if he moved out. I had no real way of knowing, and it made me a little crazy. I understood my suspicions were unfounded, but they fueled my frustration and worked their way into my decision-making process.

Kay did not agree that we should move my dad out, but she went along with our story for his benefit. To add insult to injury, I'd asked her not to visit my dad for awhile, until he adjusted to his new life at Primrose. I understood this was a harsh request. They'd grown so close since she'd become his primary caregiver. She'd taken care of all the daily details so he could have the best quality of life possible. But I ultimately had to do what I thought was best for him. It would have been nice if the situation had ended in a way that better reflected their close relationship.

If I'd lived closer to my dad, he might have been able to stay longer in his house and continue with Kay as his caregiver. I could have spent more time with him and given her more regular breaks. I could have gone with him to doctor appointments and picked him up at Primrose to surprise him. I could have taken him to the beach. I could have prevented misunderstandings between Kay and weekend caregivers or members of my own family because I would have been there and understood the daily changes my dad was going through.

But that simply wasn't the situation.

\begin{center}$*$\end{center}

With my dad fed and dressed on move-in day, we climbed into the car for his last drop- off. I'd managed to get all of his stuff in the trunk of the rental car: three boxes and two travel bags of clothing.

I had butterflies in my stomach on the twenty-minute drive, so he must've had them, too. I'd dropped him off at Primrose dozens of times, but this would be the last time because he'd be living there from now on, though he didn't know it yet.

Or maybe he did.

I remember driving slowly out of Sebastopol because of the rain. I'd arrived in California to the biggest storm to hit in six years. A boon for the drought. Areas were flooded off Highway 12 as we motored out of town and headed toward Santa Rosa.

I made small talk with him as we normally did. He rarely initiated conversation (one of the reasons why his dream scene that morning had been so surprising). Dementia is such an interior condition. I imagined his thoughts sparking through his mind like meteorites across the night sky: distant and random and silent.

``A lot of rain,'' I said, pointing out the flooded areas. ``Biggest storm in six years.'' ``That's what they say.''

``They're looking forward to seeing you at the Day Club.''

``They know what they're doing.''

``They care a lot about you.''

``They sure do.''

I was glad someone knew what they were doing. Dementia notwithstanding, I never gave up trying to have a real conversation with my dad. I always hoped to connect with him, to have an honest word pass between us. I looked for that constantly amid my fibs and his practiced responses.

I walked my dad over to the Day Club building and left him in good hands. Laurie, Yolanda, Graciela---they all knew my dad was moving in that day so they were extra enthusiastic. They encouraged him, told him how great it was going to be. They'd already become his new family. He saw them more often than anyone except Kay. I was so grateful and relieved to have them there and know that my dad would continue to see them most every day.

I delivered his clothing and boxes to his new room. When I went back to the reception area, the young woman said, ``Okay, have a nice day,'' and she turned back to her computer.

I had expected to meet with someone to go over my dad's inventory of things so that I could set up his room: put his clothes in a dresser, hang up his shirts and jackets, make his bed with the blankets I'd brought, put his pair of slippers under the desk and his books and photographs on top of it. I'd brought items for his memory box so that he'd recognize where to go when he finished at the Day Club.

But I was on their time. The person in charge was making her rounds, taking care of the residents. She had to supervise the inventory before I could set up my dad's room. She'd get to it when she had the time. It felt like dropping your kid off at summer camp, when the teenage counselor smiles and says, ``Okay, you can go. We got it from here.''

You nod. But do you know he wakes up in the middle of the night? you want to say. Will you make sure he eats? Will you guard him with your life? Will you protect him from bears, rattlesnakes, poison oak and that large boy over there who probably likes to throw rocks at people's heads?

But you say nothing. You hold it all inside. You smile and tell yourself, my child will be fine, and then you slouch away.

My dad was fine. I'd left him making apricot cookies with the other day-clubbers. I was the big crybaby, not him.

What was I supposed to do now that I'd been excused? I had expected to stay a couple of hours. I was finished before I was ready to be finished.

I drove back to my dad's place after rushing out like a madman, trying to make sure I had everything he needed. The house was so still. I had a couple of hours to kill before meeting Linda for lunch at her sister's house in San Rafael. School was out, so the whole family had come with me on this trip. I put in a last load of laundry, settled in at the dining room table, and started to write about rain and driving my dad away from his home.

The washing machine stopped. I put the clothes in the dryer. Thirty minutes later I pulled them out warm and clean, the ones my dad had slept in: his grey socks, a pair of white Hanes underwear, blue sweat pants, a flannel shirt, and his white handkerchief. I hung his shirt in the closet, now largely empty, and folded the rest of his clothes and put them in his dresser in the office. None of them were marked with his name.

I knew I'd be back the following summer to deal with the house and property. I wouldn't have time to get the place ready to sell before then. But I didn't want to leave for some reason. I was stalling. It felt so final to me.

I shook out a couple of rugs on the deck. I swiped away ants in the kitchen that the rain had driven out of the ground. They'd be back, too.

Then I locked all the doors and left.

\begin{center}$*$\end{center}

The next day Linda and I visited my dad at Primrose to see how his first night had gone. He seemed content though a little confused to see me. It was just after lunch, so he might've thought I was there to take him home. I asked him to show us his room. My sister Cindy had helped him set it up the day before after he'd finished at the Day Club. The Primrose staff had recommended as little fanfare as possible, keep the drama to a minimum, so I'd stayed away.

Cindy had also helped him put together his memory box outside his door, which my dad showed to me and Linda. His name was written across the top of the clear plastic box: Carver Moser. He had a picture of himself in there along with photographs of his carvings, a toy Volkswagen bus with a surfboard on top, and a surfing-Santa figurine standing in the corner. They'd included his artistic statement, written years beforehand:

As a wood sculptor, my chosen medium is driftwood which I gather from the coastal beaches of California and Hawaii. Seasoned by the sea, the sun and the wind, each individual piece suggests a form which I am inspired to work to completion by following the natural flow of lines and grain inherent in the wood.

As part of this process, I enjoy the additional stimulation of carving outdoors for the public, demonstrating my methods and sharing my vision directly with people who wish to make my sculpture part of their lives. I have carved for the public at Pier 39, in the Wood Carvers' Gallery, and at various hotels and resorts on the island of Maui.

I took a couple of pictures for my siblings: one of my dad standing by the box, another of him sitting on his bed. He looks calm in this second picture. He's gazing at the camera, his hands politely folded in his lap. No trauma, no real emotion at all on his face. He looks rather stoic, actually, with the air of an old salt: the white beard, the jean jacket, the old leather Maui hat perched on his head.

Before we left he pointed to a tree outside his window. ``That's the redwood from my home,'' he said.

I smiled. As honest and reassuring a word as I could've wished for.

I was immensely relieved to have him settled at Primrose. Kathleen and Cynthia lived ten minutes away. If I ever needed someone to check on him, they would do it. And I trusted their judgement.

An additional silver lining to his new home: it was easier for my siblings to visit him on this neutral territory because he had people taking care of his daily needs. That left us to focus on whatever relationship we could establish in the time that remained to him.

I wasn't sure when I'd see him again. Barring emergencies, most likely I'd be back the following summer. I knew he was in good hands. My final thoughts drifted toward the rapid pace of his decline in the past few months, and his growing confusion.

Would he even remember me the next time we met? 

\part{}
\chapter{}

I did not see my dad for six months. He'd settled into Primrose remarkably well. Dan O'Brien, the head nurse, had initially prepared for the move by ordering Lorazepam, a drug used to calm residents with ``transfer trauma.'' But my dad never needed it. He continued to be a model resident, in Dan's words. He walked over to the Day Club every morning during the week for his meals and activities, then returned after dinner. He shared a room with another resident. This had been a concern for us kids. My dad had a loner personality and needed to have his own space as a refuge when he got stressed. He could also be prickly around other men, especially Alpha-male types. Again, Primrose had planned ahead and picked a roommate who'd be compatible with my dad. It was fortunate the staff knew him so well. It helped make the transition seamless. My sisters had visited him in the early days of the move. If he asked about Kay or going home, we stuck to our line: Kay is staying with her boyfriend for awhile.

It seemed to work.

This was all a big relief for me. Knowing that he was safe and taken care of by professionals took an immense weight off my mind. The staff had built a close relationship with him since his move, essentially becoming his new family because he saw them every day and depended on them so heavily. My dad still functioned well. He recognized the workers, joked with them, appreciated their care, so they naturally gravitated toward him. Those relationships carried him through the move and allowed me to ``get back to my life.''

My life. I lived far from him. I didn't speak to him on the phone at all during those six months. Neither of us were ``phone'' people, but I could have made an effort simply to hear his voice, to ask him how he was doing. I figured No news is good news, as my mom used to say. He was fine where he was. They were taking good care of him. Besides, I didn't know if he would recognize my voice or know what to say to me. And what would I say to him? I'm not much for small talk. I continued to organize his affairs: monitor his bills and expenses, check in with Dan on a regular basis, be the clearinghouse of information for family and friends.

Did I understand yet that my life had permanently changed? That there was no distinction anymore between my life and his life? These things take time to sink in.

Because my dad was at Primrose, I didn't feel the need to visit him during Spring Break as I normally would have: to check on him, to give Kay a break, to take care of the property. I'd rented my dad's house to a friend of Mercedes who needed a place to stay, and she looked after the property. Linda and I spent Spring Break in Boston so I could do research at Harvard. My phone actually rang when I was in the bowels of Widener library. A plumber was calling me about a sink repair in my dad's kitchen. I'd originally gotten a cell phone (my first) because I wanted to be available for my dad. I should clarify: not necessarily to speak to him, but to arrange matters for him. In case of a medical emergency, for example. Or if a plumber needed to know what kind of garbage disposal to install. That was how I had understood my need to be available for my dad.

The plumber's call was a reminder that my life had changed. I carried my dad with me everywhere I went. I'd also chosen to be in Boston over Spring Break rather than with my dad because I knew I'd be going back to California for the entire summer. And possibly longer. I'd requested a leave of absence from work for the fall semester so I could be on site to get my dad's house ready to sell. I didn't know how long that process would take. I didn't want to have to leave for Missouri and start teaching my classes with the work unfinished back in California.

I ended up spending four months living in my dad's house, two of them on my own after Linda and Ryan left for Missouri in early August to start school. This was the strange period of our inverse relationship, my dad and I: the more he lost from his life---his house, his property, his worldly goods, his memories, even his wife Maya---the more connection I found with him. It wasn't that my connection required these losses, but actually living in his house, working land that he'd worked, waking to the rhythms of his days placed me inside his skin as never before. The longer I moved in his space, clearing what he'd spent the last two decades of his life building, the more I discovered my dad in me.

\begin{center}$*$\end{center}

``Hey, dad.''

He was napping mid-morning in his room at Primrose. It was Sunday, June 14th. There was no Day Club on the weekends.

He woke. ``Hello there,'' he said. He rubbed his eyes, a little disoriented. Or maybe just sleepy. He looked good, as he had back in December. I'd get a better sense of him once we'd spent more time together and I saw him move around. But he appeared his healthy self.

I didn't tell him who I was. I was curious if he'd remember me. ``I'm looking for your electric razor,'' I said. The staff had told me he needed a new head for it, and I wanted to make sure I bought the right brand. I checked the bathroom first. ``I'm back from Missouri for the summer,'' I said. I wondered if the information would jog his memory.

More awake now, he said: ``Yeah, I heard you were living there.''

It was a good noncommittal response. His cover reflex still worked. The razor wasn't in the bathroom. Perhaps the staff kept it elsewhere. I went over to the dresser by his bed. ``You mind if I check the drawers?''

``Look wherever you want.'' 

He was still congenial, but my identity hadn't clicked for him.

No luck finding his razor in the dresser either. I'd go ask at the front desk. I stood by his bed and let him take me in. Cynthia had alerted me the night before not to wear purple. ``Otherwise,'' she said, ``he'll think you're the help.''

Well, I wasn't wearing purple. ``You want to go back to sleep?'' ``Yes,'' my dad said.

``Okay, I'll see you later then.''

He settled back down, closing his eyes.

I shut his door and told myself it was dumb of me to try and test him, to see if he'd recognize me. He will appreciate my visits, I reminded myself. My lesson for the day: My dad doesn't know who I am unless I tell him.

After buying a razor head and an extra-large bottle of one-a-day vitamins, I drove back to Sebastopol. I'd gone there first thing that morning to start clearing the house, but a gang of fleas had popcorned onto my legs and driven me out. The renter had owned a cat, and evidently some of the biters had abandoned ship before the feline and her owner took their leave at the end of May.

The tiny beasts had been festering in the hot house for two weeks with nothing to chaw on until I walked into their lair.

Ugh! Blasted bloodsuckers foiled my plans.

I'd wanted to organize the house before my siblings arrived over the next few days. We were all gathering to do a big house clearing. But now I couldn't start until the exterminator arrived and sprayed the place. Who knew how long that would take?

It wasn't my only disheartening moment. Even before I'd arrived that morning, the strangest feeling had come over me as I pulled off Highway 116 and curved around my dad's corner lot. It was the first time I'd gone to his house without him living on the property. The whole place looked dry and brittle. California had an ongoing drought, true enough, but it was more than that: weeds grew high in the yard, gopher holes bombed the meadow, pine needles swamped the porch, cobwebs swung from the rafters. My dad had always taken such good care of the place, everything neat and orderly. I'd so looked forward to spending the summer on this property.

But the place looked abandoned.

And then the fleas started chomping. I've made a huge mistake, I thought. Linda and Ryan are arriving in two days. Where will we live with a house full of fleas? How will I ever get this place in shape to sell?

I hadn't known what to do. So I'd left. Gone to see my dad at Primrose.

Now I was back. There was nothing to do but start work on the outside. I opened the barn, found a pair of my dad's thick leather gloves, grabbed a ladder, and started pulling shit out of the rafters.

It was the kind of physical labor I'd grown up with in Occidental. I actually enjoyed it: clearing brush, chopping trees, cutting and stacking firewood. I didn't get much chance to do that kind of work in Missouri. I remember when I was thirteen or fourteen, my dad had told me at the beginning of summer: ``Go out and find a job. Don't come back until you get one.'' I'd walked up our dirt road and started knocking on neighbors' doors, asking if they needed any weeding done for \$1.25 an hour. That was about my only skill at the time that didn't involve climbing trees, playing sports, or collecting baseball cards. I spent the next several summers swinging a weed cutter in people's yards and doing odd landscaping and painting jobs. Some people have their salad days. I have my weed days.

Up on the ladder I pulled down whatever came to hand---wood beams, steel pipes, long- handled tools---and hauled them outside. I dropped them into loose piles to sort through later: wood, metal, stone, glass, furniture, garbage, recycling. I wanted everything in view so my siblings and I could decide what to do with it all: keep it, donate it, recycle it, or dump it.

It was the kind of work my dad had enjoyed, too: taking on big projects, moving things around, finding order in chaos. He loved strenuous activity. It was one thing I'd gained from him. My dad was more supercharged in his efforts than I, fueled by the high-octane of personal challenge. But we shared the pleasure of physical labor.

And the solitude. This was the part I did not want to admit: how much I enjoyed being by myself, working by myself. My dad had desperately needed that in his life. At fifty-two, my age standing in that barn, he'd left his family, adopted a new name, and was racing down the road to self-discovery. I did not have his demons, but I felt a deep pleasure in having time and space to myself, of leaving work and family cares behind. I found myself conflicted, fielding long-held resentments against my dad for taking off on his own and yet feeling that same pull myself toward solitude.

\begin{center}$*$\end{center}

I found his old pipes in a dusty corner of the workshop, wrapped in a canvas bag. His workshop was at the far end of the barn, connected by a door. There was another door on the outside that led directly into the workshop. This was where my dad had carved his driftwood. Workbenches filled three sides of the small room, every one of them holding drawers and shelves stuffed with tools. You name the tool, he had several versions of it. I don't think he ever threw a tool away, and he'd made quite a few of them himself: drills, saws, grinders, vices. Hardware that gripped, pried, cut, pounded, and gouged. But delicate tools, too, for stripping, sanding, polishing. The whole place was covered in a fine sawdust. Tools hung on the walls, stared down at me from shelves, peeked out from bins and buckets and basins. There were ropes and cables, fasteners and fittings.

I might as well have been browsing the kiosks at my local mall---I had no idea what the hell half that stuff was for. Tools and I are distant acquaintances. I regard them as I do electricity: I understand the usefulness, but I try to keep my distance for fear I'll hurt myself or fuck something up. A shovel or screwdriver is about as complicated as I want to get. Anything with moving parts alarms me. Thank god my two older brothers, Steve and Chris, and my brother-in- law Jerry, would be coming down to help sort through them. I felt like an archeologist on a hominid dig: holding up artifacts, turning them over, scratching my head.

But the pipes I knew. My dad probably hadn't smoked a pipe in thirty-five years (except maybe a peace pipe). And I found a pouch of his old tobacco in there! Captain Black. The tobacco was still damp, and the fermented smell of cherry spun me back to when I was a kid and my dad had five or so pipes lined up in a rack along with his packets of colored pipe cleaners. All the pipes were finely carved---perhaps that's why he'd kept them. I'd wondered as a boy what the difference among them was, why he might choose to smoke one rather than another. It was an amazing find, and I set them aside to show my siblings when they arrived.

I felt better after four hours of hauling stuff out of the barn. The job seemed doable to me now. I could see the piles I'd made and know that my family and all my siblings were coming in the next few days to help me sort through it.

As I'd pulled the guts out of the barn piece by piece, I kept wondering what all the stuff we collected in our lives amounted to. I took a walk behind the art studio. Underneath a small grove of oak trees sat a pile of driftwood my dad had collected for future projects, so much of it acquired through great effort, ingenuity, and patience. He once told me that after large storms had swelled the nearby Russian River to flood-warning levels, he'd kayak along the banks and look into the trees for pieces that'd gotten caught in the branches from the storm surge.

Driftwood in the trees! Who would've thought to look there? I remember admiring his imagination.

I wondered if the pile under the oaks held my dad's infamous UPOD---his Ultimate Piece Of Driftwood. He'd discovered it one day back in 2006 buried in a sand bank of the Russian River. He'd dug it out of the silt, tried towing it against a southwest wind and incoming tide; it was longer than his nine-foot kayak and weighed a few hundred pounds. He finally had to beach it and go back two weeks later to resume the struggle. That was part of the fun for my dad: not just finding a big fish, but landing the motherfucker with a gaff hook. He'd gotten the beast back to Sebastopol only after tearing a rotator cuff hauling it into the back of a rental truck. According to a blog he wrote on the topic while convalescing, he'd been searching for the UPOD for ``some 50 odd years.''

I could believe it. He'd visited Patrick's Point State Park north of Eureka during his Easy Rider motorcycle trip in 1969. ``Went out of my mind on Agate Beach North of the Park,'' he'd written in his journal, ``because there were miles of driftwood piled high on the beach, and I couldn't carry nary a stick! I'll have to make a special trip just to get a load.''

There was something about driftwood in particular that called to my dad throughout his life. Without stretching too far, I can imagine him seeing much of himself in the wood: the rough beauty, the distant journeys, the random touches on terra firma. ``Seasoned by the sea, the sun and the wind,'' he'd written in his artistic statement in the early 1980s, ``each individual piece suggests a form which I am inspired to work to completion by following the natural flow of lines and grain inherent in the wood.''

So much driftwood from the edges of the Pacific and its tributaries, so carefully gathered for years and years---maybe part of that UPOD---now sat and rotted in the back of his yard. His decomposing visions. What exactly had he tried to collect here? What hope or mania drove this hoarding?

It was my job that summer to clean everything up, lend some order to his whims. Hopefully I'd find some meaning in it all and learn a lesson for myself. What driftwood was I hoarding? What whims came in between me and the people I loved? How could I make sure my sons weren't sifting through piles of stuff at the end of my life looking for answers?

I found a carving of his that day. I'd pulled it down from a shelf in his workshop. It was only half-finished. You could see the chisel marks on it, the form coming out of the wood. I couldn't tell what it would be. Another bird? Or something abstract? I liked this work-in- progress more than some of his glossier pieces. I wasn't sure why. It was like reading his description of me in the journal: a raw and honest draft, a peek at his cuts before sanding and staining smoothed them over. He didn't often let people in so close. I held the rough wood in my hand, turned it over, tried to solve it like a puzzle: this was the work my dad had valued over many human relations, where he put so much of his time and energy---effort that embodied his very identity to a great degree. What image was he trying to form with his many sharpened tools? What could this carving give him that his children and others could not?

The piece had been wrapped in cloth and stored on a shelf. He must have planned to come back to it at some point. He just ran out of time.

\begin{center}$*$\end{center}

The next day I exploded the careful order in the barn and workshop like a jack-in-the- box. My dad would've had a seizure. Then again, I was thinking of my ``old'' dad. I didn't know how the new one with dementia would've reacted, if at all. I'd been considering whether to bring him out for our family barbecue on Saturday. He considered Primrose his home now. He'd been there for six months. Would he even remember that he'd lived here? Would the place jog any memories for him?

He had an old cassette player hooked up in his workshop. I got it spinning and listened to his music as I emptied drawers and pulled cases of electric tools from under the workbenches. Neil Diamond, Jim Croce, Linda Ronstadt---these were the tunes from my childhood. I liked being in his shop, surrounded by his small comforts: his old denim apron hanging from a hook, ready for another project; a small fan to cool him in summertime, a space heater to warm the shop in winter. He'd organized gear into plastic cases and stored them on a top shelf, labeled for easy reference: BONES, POLISH, STRAPS, SANDPAPER, SERVICE MANUELS. One read, JUST STUFF. I liked seeing his printing in magic marker on the outside of the cases. I liked imagining him in this space. He enjoyed the slow pace of carving: steady, methodical, detailed. How much time would he spend on each piece?

A hundred hours?

More?

I pulled a hatchet out of a wooden box at one point and made a couple of light chopping motions in the air with it, testing its weight and balance. If I had to pick one tool to embody my dad, it would have been that hatchet: a sort of tomahawk with black electrical tape wrapped around the shaft under the head; the tool had a dual purpose: a hammer on the backside (mashed from heavy pounding) and a blade on the front (razor sharp for chopping). It was an old tool, and my dad had burned CARVER into the wooden handle, giving artistic flairs to the C, V, and R. As far as I could tell he hadn't signed any of his other tools. He typically carved his name into his driftwood pieces after he finished them---the same unique lettering---so perhaps he'd been practicing his signature. At the very least he'd seen something special in the hatchet to brand it like that.

I gripped the tool. I had a job to do before my siblings arrived: get everything on display so we could make a quick determination about what to do with it. But I was on my dad's property that summer for something more. I didn't quite know what, but I knew I'd need time to myself to figure it out. If somebody had been with me those first days, I would've complained or made jokes about having to clear out all this stuff. Alone I could drop the humor, not use it as a shield or sword as I'd done so often in my life when faced with my dad. I could lay down my weapons and try to understand what I was doing there, why I felt the need to live in his shoes. What was I hoping to find there among all his tools and projects? What did I think they'd tell me? I had the disturbing vision of being Luke Skywalker headed into a cave to meet my destiny. I'd saber off Darth Vader's helmut and see my own face staring back at me.

Maybe a little humor would help, after all.

I set the hatchet back in the box. My own vision quest, then. Solitude would be my peyote; the land, my lodge. And my shaman? As much as I hated to admit it, that'd be my dad. Or at least the journals he left behind.

About midday I walked next door and introduced myself to my dad's neighbor, Steve. I'd seen him working in his yard, which opened to the back half of my dad's property. He looked to be around sixty. He'd officially retired from Santa Rosa Junior College as a counselor, but he still filled in for them on a regular basis. Steve was friendly, and we reminisced about teachers we knew in common. He told me that he'd had a great relationship with my dad. He'd heard from my renter that I'd be around for the summer. He was hoping I'd buy the place rather than some dot-commer who'd tear everything down and build a mansion right next door to him. I told him that we'd considered buying the house and keeping it in the family, but the finances hadn't work out. I let him know that my family would be arriving that week. I invited him to our barbecue on Saturday.

``How's your dad?'' he asked.

``He's good. He's happy at Primrose.''

Steve asked about Maya, too. I told him she was over in Forestville and had been declining lately. More than usual. Steve recalled the night screams she'd have---he and his wife heard them through the woods. He'd reassured my dad that it wasn't a problem; they completely understood. Steve had noticed a big change in my dad, he told me, when Maya really started going downhill. My dad had turned from this energetic, upbeat person into someone ``who walked around like everyone else.''

His phrasing struck me. I heard similar descriptions from other people that summer, their admiration for my dad. He had a vitality that permeated his whole being: the spring in his step, the animated cadence of his voice, the eyes that challenged you, the physical strength of the man even in his old age. He'd accomplished feats of will and endurance that most people could only shake their heads at. People were drawn to his energy and confidence. I knew from his journals that his confidence was built on the shakiest of ground. But I had to admit, despite everything I knew about him, that I admired much about him, too.

The last thing I did that day was water the plants before I left. I'd been sleeping at Kathleen and Cynthia's those first nights because of the fleas. I had to get the property ready to sell, true enough, but my dad had planted a lot of the trees and shrubs himself. Many of them were wilted, and some had died. He'd always taken such good care of everything. I wanted the place to look nice again.

\begin{center}$*$\end{center}

The next morning I spent an hour vacuuming the house before the pest control guy arrived, then turned my attention to Maya's art studio. It was a one-room structure with bales of hay for walls that had been covered, on the outside, with a brown stucco-like material; on the inside my dad had finished the room with plywood that he'd painted white. He'd installed a small sink for cleaning brushes in one corner of the room, and racks to store Maya's art in another. The rest of the space was filled with shelving, tables, baskets, and chairs that Maya had used or Kay had left behind. Plenty of art supplies were spread about: easels, brushes, paints, and multi-media items (bark, feathers, shells, small stones, bundles of burnt sage). Maya had been the perfect partner for my dad in their later stage of life: she'd changed her name like him (her given name was Marilyn) and had been part of a coven for many years. She and her sister wiccans had held ceremonies in the woods and used mind-altering drugs, tapping into goddess spirits for strength and creativity. She'd gotten her formal artist's training at Stanford after divorcing her first husband in the 1970s. Her art tended toward abstract watercolors, very bold and bright. My dad had met her in the early `90s, and they'd been together ever since. Definitely kindred spirits. Maya had refined my dad's artistic sensibilities, and she'd had a wonderful grounding affect on him, providing a calmness in his life that enabled him to settle down finally and become a loving partner and husband.

Surrounded by her life's work---countless paintings along with dozens of sketch books--- I had a similar feeling as being in my dad's workshop. What did it all amount to?I spied a quotation often attributed to Goethe pinned to the wall by the sink: Whatever you can do, or dream you can, begin it. Boldness has genius, power, and magic in it. Given her interest in magic and dreams, I could see Maya being inspired by the words. Around me were remnants of those dreams, some finished and others not. I started with the large canvases, stacking them in the racks one by one. I organized the smaller pieces into boxes for shipping and storage. I had no idea what any of it was worth to Maya or her family. I tried to handle everything with the same care that she'd painted them with, not knowing what her kids might want to keep, display, or give away.

My sister-in-law Cynthia showed up in work clothes. We got the leaf blower running, and she started clearing green and brown redwood needles off the driveway, the brick walkway leading up to the front door, and the wooden deck that wrapped around the house on three sides. I went back to my dad's workshop, popped in a new cassette, and started pulling boxes off the top shelf and opening them up so my siblings could look inside them. I worked to the steady rhythm of Edith Piaf's ``Non, je ne regrette rien.'' The music must have been Maya's influence. I knew my dad had regretted things in his life, especially his relationships with us kids: missed opportunities, lost time. Unfortunately he never told me this directly (a missed opportunity, perhaps). I had to read about it in his journals. I thought: How much one conversation like that would have meant to our relationship! And if I had known to ask him about possible regrets, would he have answered me? Could that have been the beginning to a better relationship?

Wishful thinking. By the summer of 2015 it was all academic, as they say. I would never know, and my dad couldn't tell me.

But he was still alive. And I had a choice, especially since he couldn't go anywhere by himself. No more packing up and taking off down the road. No more Island time. No more vision quests in woods and sweat lodges. I didn't want to have such regrets---that I could have spent more time with him and didn't. Spending time with him was the best way I knew to learn from his mistakes, to be a better father by being a better son. Even in his current state where he didn't recognize me, I kept reminding myself that he appreciated the time I spent with him, that on some level he was aware I was present and cared for him. Here was the bizarre twist: he'd finally achieved the role he'd strived so hard for in life, and yet he'd never know it. I doubt it could have been otherwise. My dad, the accidental shaman.

Cynthia had packed us a lunch, so we sat on the now-cleared deck and enjoyed sandwiches underneath the redwoods. The place was starting to look lived-in again. The exterminator had come and gone that morning, so I could unpack my bags and prep the inside of the house for the group-clearing the next day. Later that night my sister Pam came over and we watched the Warriors beat the Cavs in the NBA finals. The Warriors hadn't won the title since I was twelve years old, watching Rick Barry nail free throws by spinning the ball underhanded. We sat in my dad's office where he'd spent so much time watching TV because he couldn't do much of anything else. I was stretched out in his green leather chair, my feet propped up on his foot stool, imagining him staring out the window at the redwoods.

``You interested in taking any of his archery things?'' Pam asked. She was two years younger than I, lived in Portland with her husband Pat and their two kids. She worked in merchandising and, like my mom, had run her own gift shops. She was good with details and organizing.

I looked at the shelf holding my dad's 1964 trophy as the top amateur archer in California (class AA). Next to the trophy stood pictures of him by the winning target and a ``Robin-Hood'' arrow. He'd actually shot his arrow into the back of another one and split the hollow aluminum ---forging the two shafts into one long one. After nearly fifty years they still hung together on the wall above the shelf. I couldn't see myself hauling a bunch of my dad's stuff back to Missouri. I'd been feeling a bit melancholy over the past few days thinking about my dad and Maya's stuff, and all the stuff in my own life: what I collected, why I held onto it, what it was all good for. These thoughts were the beginning of a ``purge'' mode that steadily intensified in me that summer. I ended up losing fifteen pounds over the next four months. I'd started out at 160, so on the lean side to begin with. I had my head buzzed by a lady barber in nearby Rohnert Park who specialized in military cuts. I didn't know why, but I felt a compulsive need to scrape, to cut, to pare down.

``I don't think so,'' I told Pam. I glanced at the bookshelves behind me. I'd take a few of those volumes, and my dad's journals. Probably one of his carvings. I'd leave all the tools to my brothers and brother-in-law. Well, maybe I'd take one tool. That hatchet with his name burned into the handle.

\begin{center}$*$\end{center}

Amid the pleasures of seeing my family and siblings the next day, and the chaos of managing the clean-up effort, I got calls from Maya's son and daughter telling me that they thought she'd die soon. We talked about what to do with her ashes---she and my dad had wanted them mixed together---and where to hold her memorial service. I told them I'd take my dad to see Maya the next day. They thought she might be waiting to die until she'd said goodbye to him.

It was just as well that I was gone from the house for part of the time. In an odd reversal, I suddenly felt protective over my dad and Maya's stuff. After several days of working in the barn and art studio---listening to their music, puttering among their things, getting a sense of their daily pleasures---I didn't want it all shipped out too quickly. I wanted to know where everything was going and what people planned to do with it. I was legally responsible to keep track of their worldly goods as the Trustee, but it was more than that. I hadn't adjusted from my slow pace of processing everything. Linda had to lay a hand on my arm when I tried to put the breaks on my siblings. She reassured me that it was okay: time to unload as much as we could while all hands were on deck. I hadn't been paying attention to the process that my siblings might need, too. My dad and Maya were still alive, but this was a gathering to say goodbye in many ways. I noticed my sisters clearing stuff inside that house with a vengeance. Linda also reminded me during that week that my sisters had come to support me, because I'd asked them, not to help my dad out.

``They're probably pissed that they have to come here and clean up his mess,'' she'd told me, ``when it's your mom they really miss.'' It was a reminder of how deep those antagonisms ran, and how much grace they showed coming into his house to make my life easier.

\begin{center}$*$\end{center}

Maya was lying in bed when we entered her room. It was Thursday evening, June 18th.

Her face was so pale, she looked like a corpse: her eyes distant, her mouth locked in a grimace, her hands folded across her waist like someone had placed them there.

My dad rallied her, as usual. ``Hey, Beautiful,'' he'd say. ``Hey, brown eyes. How ya doin'?'' He asked her questions: ``Want a kiss?''

She answered without changing expression---short words, uttered in quick shallow breaths, the way an injured person might respond to a paramedic: sure, okay, that's good.

He must have given her two dozen kisses while we were there. She'd smile each time--- that's good---then settle back into a kind of stupor from which my dad roused her again with another question or a kiss. He was on a loop: he knew she'd respond to his cues and compliments, so he kept doing it.

He rubbed her hands, smoothed her hair on her forehead. I wondered if he was channeling Cynthia, trying to massage Maya and make her comfortable, still playing the role of her caregiver after all this time. ``Do you like that? Hey brown eyes, how ya doin'?'' He'd lean over. ``Do you want a kiss?''

Sure.

He moistened her mouth with an oral swab, soaking the sponge tip in water then placing it onto her tongue. She'd bite down hard so he'd have to wiggle it out of her teeth. It became a game: he'd grunt and pretend to pull hard so he could get it out. I wasn't sure how much control she had, but they looked like they were enjoying the tussle.

He leaned in close and sang to her, a sweet song whose words I've forgotten. I remember his quiet tone, coming from deep in his chest, and thinking that he actually had a good voice. It was so spontaneous and lovely. Some things he remembered---who Maya was, what they shared ---and some things he didn't: he told someone during our visit that he was there to visit his daughter.

To be playful, I pointed out a picture of my dad on the wall in Maya's room. ``Who's that guy?'' I asked him.

My dad leaned in, looked at me like he wasn't entirely sure of his answer or what I might be trying to pull. Then he said, ``It's me.''

After an hour of being perky and playing games, he suggested it was time for us to go. He leaned over and gave her a final kiss. It was the last time he ever saw her.

On our way back to Primrose he told me, ``the kids came home today and ran around the house.'' On the one hand I could see that he was losing more touch with reality. On the other--- given that some of my siblings had arrived that morning at his house---maybe he was so far to the right on the coherence spectrum that he was passing into clairvoyance. I had told my dad my name this time when I picked him up at Primrose that evening, but I didn't think he remembered who I was. He didn't remember that Linda and I taught at universities. I'd told him about Miles and Ryan, and he'd just nodded. He couldn't remember them.

I got him into his pajamas after we arrived and said goodnight to him. I told him I'd pick him up in the morning for his VA appointments. 


\chapter{}

I wish I could say my dad's VA appointments in Santa Rosa were exciting in some way, but they were simply routine: a physical, and a teeth-cleaning. My role was to be there with my dad and advocate for him. His memory had gotten noticeably worse. I'd introduced myself to him that morning at Primrose---``Hi dad, it's Pat. Your son''---but it didn't seem to have meaning for him. So I wanted to ask his doctor if we might increase his medication.

``Memory loss is inevitable,'' she stated.

I held my frustration in check. I'm sure that was true, I said. But could we try? Maybe it would help?

She seemed more interested in signing off on his physical---another item off her list--- than in finding out how he really was. I'd had to remind her about his prescription (Donepezil) and the dosage (5 mg per once a day). She'd had the wrong chart on her computer when we arrived. She'd thought he weighed 268 pounds at their last meeting. The nurse had weighed him ten minutes before, and he was 165 pounds. I pointed that information out to her.

She gave a small laugh at the mix-up. I wasn't feeling very confident in her powers of observation. She looked up Donepezil on her computer. She found that non-VA doctors had mentioned it could aid memory loss. I pressed her, and she finally agreed to double his dosage.

At least his vitals had all been good. After six months at Primrose, he was healthy as a horse.

We shuffled over to the dental clinic. My dad was friendly and compliant as usual. He greeted people, had smiles for them. The dentist was very thorough and professional in the way she spoke to my dad and how carefully she examined him. She found three cavities. He also required more work on one of the posts in his mouth, which could only be done in San Francisco.

The hygienist started cleaning my dad's teeth using an ultrasonic scaler that cracked off plaque and tartar. My dad jumped several times because of the pain. I hated those things, too: the electric whine, the metal shafts in your mouth, the intense zapping around tender gums. All I could do was sit there for half an hour and sweat it out with him.

Afterward I listened as the hygienist showed my dad how to use special brushes to poke food from under his posts so his gums wouldn't get inflamed again. That was part of the reason for his pain: he hadn't been able to clean his teeth as he used to do. He nodded at her cues but retained nothing. I followed along so I could explain the procedure to the staff at Primrose. Ha ha. I learned that oral hygiene wasn't very practical for people like my dad who often couldn't do more than brush their teeth, if that. I couldn't reasonably expect staff to help him floss, or floss for him and make sure he inserted the small brushes into all of those spaces around his dozen posts. They didn't have the time. Even if they did, trying to negotiate inside and around the mouths of dementia patients could cause bigger problems than inflamed gums. Oral care would have to slide. The best thing I could do was bring him to the dentist more often.

Of course arranging appointments through the VA could be more painful than an ultrasonic scaler. The dentist suggested I make a six-month appointment for my dad that day because the clinic filled quickly. The person at the front desk told me they only booked appointments three months out. So I'd have to call back in three months. By that time, I was sure, all of the appointments would be filled. For my dad's post work at San Francisco, the dentist would have to send a referral by mail or phone to the oral surgery center down there. They'd look over the referral and get in contact with me.

I was happy that my dad had gotten checked out and was healthy. And what a gift: the VA paid for everything. I bought my dad a U.S. Marine Corps cap on the way out. They often had tables on the sidewalk outside the clinic where veterans offered coffee and donuts for donations, and sold various military gear. They were always friendly and supportive. Most had served in Vietnam and Iraq. They usually raised a deferential eyebrow when they found out my dad had served in the Korean war. I'd heard him tell a few staff that day that he'd been a Marine and was proud of it. He liked the cap and immediately put it on. I was hoping he'd wear it rather than his stinky Maui hat.

We had a funny conversation in the car on the way back to Primrose. I think the hat must have triggered ``memories'' of Korea for him.

``I was so good at archery,'' he told me. ``They didn't give me a gun in the war. They just gave me a bow and arrow and told me to shoot the enemy with that.''

I looked at him. ``Really?''

``I'd raise up, shoot the enemy quick''---he mimicked the motion as much as he could under his seatbelt---``then drop back down. But then I'd have to move to another place as soon as I dropped.'' He ducked his head back and forth. ``So I wouldn't get shot.''

``Makes sense,'' I said. ``You must have been good.'' He nodded. ``You betcha.''

I tried to picture him on a battlefield where he'd never been, with weapons no U.S. Marine had ever been issued in a war. It was so outlandish I almost believed him. My siblings were going to love that story.

When I got back to the house at noon, most of the driveway and yard around the barn had been cleared of my piles. I was relieved to see so much junk hauled away. We had lunch on the back deck under umbrellas: sandwiches and beers, with cherries and pistachios to munch on. The temperature was in the low 80s, sunny and clear. A perfect day in Sonoma County. My siblings laughed at my dad's bow-and-arrow story.

We laughed some more after Jerry, Terry's husband, cracked a beer and relived one of the dump runs he'd made that morning with my son, Ryan.

``We had this big rubber ball in the back of my pick-up,'' Jerry began. He was a hundred percent country---a hunter, a fisherman, the guy whose house in Oregon we'd all go to after the apocalypse because he knew how to survive off the land and take care of everyone. He was thick as a bear with a heart bigger than his fishing boat. He's the uncle you'd want to raise your kids if you died because he'd love them more than anyone and teach them how to hunt, build, garden, drink, cuss, barter, and make comments like Your ass is hanging out of your underwear. No surprise Terry had chosen Jerry as her husband because she'd never find anyone who appreciated having a family more than he did. More than anything Jerry was himself all the time, no matter who he was with. He couldn't have been more different from my dad in that respect. Ryan was thirteen and idolized Jerry. We all did.

Ryan was sitting next to Jerry, and Jerry leaned back and slapped him in the chest. ``What the hell are those rubber balls called?''

``How should I know?'' Ryan said.

``You're in school, aren't ya?''

``A therapy ball,'' someone answered. My dad must have bought it for Maya to help her with balance.

``Therapy ball?'' Jerry rolled his eyes. ``Anyway, we stuck the deal in the back of the pick- up on top of all the junk. We're driving down . . . What's that highway called?'' ``One-sixteen.''

Jerry burped. ``One-sixteen. The thing pops out, hits the highway, bounces OVER a goddamn semi-truck behind us.'' Jerry made a high arcing motion with his thick hand.

``I'm the one who saw it,'' Ryan added.

``And landed square in the middle of the windshield of this compact car. Ho-ly shit!'' Jerry reached back and grabbed Ryan's arm. ``Ryan told me what'd happened. I say, Don't look back, Ryan! Don't look back! And I hit the gas.''

Those words became our mantra that weekend---hauling things away, tossing stuff in boxes, clearing everything out. Don't look back! Just hit the gas.

Jerry had gotten my dad's riding mower working again by installing a new battery, and he'd shown Ryan and Miles how to operate it. Ryan had been especially proud of his work cutting dry weeds in the meadow, dusty as it was. I looked over the newly-mown land from the deck and listened to my family tell stories, make jokes, and laugh loudly at one another as the lunch hour stretched into the early afternoon. Although the work needed to be done, the best part of our gathering was simply getting together and enjoying one another's company. This was what my dad had missed out on over the years: the warmth of family, the deep pleasure of shared lives. I've always remembered something Jerry had commented about my dad years back: ``He left you kids just when you were getting to the good age, when you can go huntin' and fishin' together.''

I'd decided against bringing my dad out to the house that weekend while everyone was there. Originally I'd thought it might be a good idea. Give him a field trip away from Primrose and let him visit with everyone. I'd asked Kathleen and Cynthia their advice. They'd recommended against it. They thought it'd be better for him to see us where he was most comfortable, at Primrose, rather than disrupt his routine. Kathleen told me the story of a man who'd insisted on pulling his father out of care against the advice of staff. This was back when she'd managed a convalescent home after finishing her nursing degree at Humboldt State. When the elderly man returned to the convalescent home, he was always disoriented, and the staff would have to deal with the fall-out.

I had to remind myself that my dad wasn't stewing at Primrose waiting for people to come visit. He'd largely lost the concept of time, and he no longer remembered his kids. At least as far as I could tell. Plus there were a lot of us, and we could be chaotic. I could see how my dad might be easily overwhelmed. I suppose I was fighting against my own desire to shoehorn my dad into family situations, hoping his kids would benefit from this new and gentler dad that I had experienced. I had to be careful to protect my dad against my own desire to restore lost relationships.

\begin{center}$*$\end{center}

Seventeen of us had dinner that night at Mary's Pizza Shack in Sebastopol. Mary's was a good choice for us, a family-owned restaurant with an open kitchen where you could see them making the Italian food. Linda and I had had our first date at their Petaluma location thirty years before, the same place where we'd held our rehearsal dinner the night before our wedding. Mary's in Sebastopol had also been one of the three restaurants where my dad ate on a regular basis after Maya went into care. He loved their spaghetti. They sat us in our own dining room at one long table.

Mike came, so all eight of us kids were together again. He looked good, considering. He'd put on weight. He showed me his cannabis card, told me about buying cannabis snack bars that increased his appetite, decreased his pain, and helped him sleep. It was easy to forget he was in fourth-stage cancer, his lung spots growing faster and larger. He was more subdued that usual, but he laughed when Jerry told the therapy-ball story again. Mike made reference to his cancer when he pulled out his credit card and paid for everyone's meal.

``I can't take it with me,'' he said.

We smiled and joked. ``Ah, hell. If we'd known you were payin', we would've ordered two plates of shrimp bruschetta.''

This was our Moser way of saying what we would tell him in almost exactly three months: We love you, and we're so sad that you have to leave us. Mike was supposed to have died the previous fall, so he was truly on borrowed time. I'd taken a leave of absence for the fall not just to be near my dad and sell his house, but to be around for Mike's last days and his funeral. I suspected it'd be the last thing I did in California before going back to Missouri.

At the end of dinner I stood to speak to my brothers and sisters.

``Thanks everybody for coming,'' I said. ``Tomorrow we'll do a round-robin of dad's things, like we did with mom.'' The eight of us had sat around her small apartment after she died and chosen items one at a time, oldest to youngest, until we'd gone through it all. It'd worked well and given us a chance to remember her, each of us choosing items that held special meaning for us. Obviously my dad wasn't dead, but the house had to be sold because of the reverse mortgage requirements, and that meant clearing out all of his and Maya's things. ``Mom would have loved this,'' I added. ``All of us together, taking Mike to the cleaners.''

A few laughs.

``That's her real gift to us, that spirit of wanting to be together.''

They all raised their glasses and drank. They thanked me for organizing the gathering. ``I wore mom's wedding ring on my little finger for a few weeks after she died,'' I told them. ``I liked having it there, knowing I was wearing something that she'd worn for so long. Almost thirty years, to all the places we've been as kids. The amazing part is she kept dad's wedding ring, too.'' I laughed. ``Instead of chucking it out, she saved it.'' I'd wound up with both rings by default. Nobody else had wanted them. ``That was mom, that generosity of spirit.'' I'd wanted to reinforce the point before we sat around my dad's house and parceled everything out. It was easy for the most well-intentioned people, myself included, to get grabby when it came to free stuff. I had no idea what emotions or feelings the process might dredge up. I wanted to channel my mom as much as possible that whole weekend.

``One time,'' I continued, ``Steve, Chris, and I were at Zuma beach with dad. He'd gone off someplace. I think I was seven or eight. It was super early in the morning, and there was nobody on the beach. Just one guy driving a tractor up and down the sand picking up trash.'' My brothers smiled. They knew the story. I wasn't sure who else did. ``The tractor had this big roller on the back covered with steel spikes. The roller ran across the bottom of the tractor and spun around''---I imitated the motion with my hands---``stabbing paper and cans, whatever people had left on the beach the day before. A metal bar hung off the back of the tractor, directly over the spikes.''

I let that sink in. ``Someone,'' I said, looking at Steve, ``had the idea to hang off the bar and let the tractor drag us across the sand. If any one of us had slipped, we would have fallen right onto those spikes.''

``That sounds like Steve,'' Chris said.

``So we took turns hanging off that bar. We'd drop off after ten seconds so the driver wouldn't see us.'' I laughed. ``All of a sudden we hear this distant screaming: GET OFF OF THERE! GET OFF OF THERE! Dad was waving wildly from the parking lot.''

``Oh, shit,'' Jerry said.

``We grabbed our towels and walked single file toward him. He'd come down to the sand. We were moving slowly because we knew he was pissed. I was at the back. Chris went by him first. Dad yelled at him, probably because Chris was the oldest. Next was Steve. Maybe because Steve was biggest, or else he was the one dad had seen on the tractor, dad whacked him as hard as he could on the back and sent him flying into the sand.''

It had been so long ago, some of us could laugh at that image. Steve had been a magnet for my dad's rage numerous times. But he also had the thickest skin, so he took it better than me or Chris.

``I walked by him,'' I said. ``He didn't touch me.''

I took a drink, set my glass down. ``Anyway, I'm so grateful to be number six.'' That got a few grudging laughs. We younger ones definitely had it easier. And that was my point. ``The older kids,'' I said, ``you had it so much harder than we did. You took the hits back in the day. Now I feel like it's my turn. Take one for the team.''

They'd been thanking me ever since I'd taken on this work with my dad two years before. Did I really feel like I was getting whacked by him? Stepping into a curveball?

No, not really. But I wanted to let them know how sorry I was that they'd had a harder childhood than I did, a much tougher relationship with my dad, especially Mike, Terry, and Kathleen. They'd borne the brunt of his messed-up childhood, his fumbling attempts to be some kind of authority figure to them. There'd been so much collateral damage from a man who'd once told me there were two things in life that he could not stand: a liar and a cheat.

And he'd been both.

\begin{center}$*$\end{center}

The distribution went smoothly the next day. I needn't have worried (though I suppose that was part of my job). I'd collected my dad's carvings in the living room, a dozen or so, and we took about an hour to go through them. We started with Mike, who didn't want a carving.

``I don't need to collect anything,'' he said.

That made sense. But we talked him into taking a black raven, carved from walnut, for his daughter Mallory. Mallory had had a good relationship with my dad as an older teenager. He'd shown her how to tool leather in his workshop. My dad had wanted Mallory to have that particular piece anyway. It was the only one we knew of that he'd designated for a specific person, though he'd often said he'd wanted his carvings spread among his children. So that's what we were there to do.

Terry didn't want a carving either. ``I already have one,'' she said. On the second round, we talked her into choosing a piece for one of her kids.

At her turn, Kathleen just shook her head.

The three oldest kids, the ones who'd had the most embattled relationship with my dad, didn't want anything to do with those carvings. The rest of us took one for for ourselves or for our kids. We had to get rid of them one way or the other. There were a few left over after we'd gone through several rounds. I ended up giving one to my dad's next-door neighbor. I also saved a turtle and egg that my dad had carved as a gift for Maya. I gave that one to her son at her memorial.

For myself I took an abstract driftwood piece; nothing with an animal face on it. I thought his abstract art had been Maya's influence and showed growth in my dad as an artist, a move away from the Native American symbolism and toward simple beauty of material and line.

After the carvings, we opened up the rest of the house for pillage: old guitars, some furniture, books, any knick-knacks that might have caught someone's eye. It was either going to family or the local Goodwill, so I encouraged people to stock up.

By that time friends and extended family were arriving for the barbecue. I cooked burgers and hot dogs for everyone. People brought salads, fruit, drinks, and desserts. There were at least thirty of us scattered on the back deck: old family friends from Occidental, my cousin Minda and her son Matt from the Sacramento area, my niece Kate (Terry's daughter) and her husband Sean from Walnut Creek. We ate, we told family stories: how Kate and Sean had met; my dad taking Jerry for his first surf lesson during a big swell at the Russian Rivermouth, probably the most dangerous wave in Sonoma County. Mallory arrived to collect her raven and a beautiful cabinet of carving tools we'd saved for her. She had just bought a house in Oakland, so she could use a lot of my dad's stuff from the barn. I was glad his things were staying in the family, being passed down to the younger generation. She also grabbed my dad's old pipes for her brothers, Jack and Henry, who hadn't been able to make it. California boys and musicians, they'd get good use out of them. Friends who still lived in the area looked over the work we'd done on the place that weekend and agreed the house would sell quickly, mostly because of the land: a prime piece of real estate close to Sebastopol but far enough from town to offer a woodsy escape.

That made me feel good about the work that lay ahead that summer. We'd already cleared out two tons of driftwood, brush, and garbage from the property (you pay by weight at the Central Disposal Site in Petaluma), and yet there was still so much to do.

In a quiet moment after everyone had eaten---I was in the living room picking one more cookie off a plate---Cynthia came from the kitchen where she'd been cleaning up and told me she thought Maya would die soon. Cynthia had visited her that morning before coming over. She'd sat next to Maya in bed and rubbed her hands and her scalp. She'd told her it was okay to let go, to stop fighting.

I wondered out loud if my dad would see Maya again before she died. ``I'd wanted to tell him on the way back from our last visit,'' I said to Cynthia, ``that she might die soon. But I didn't know how to bring it up.''

Cynthia grabbed a couple of empty plates off the table. ``You might wait until he asks about her,'' she said. ``If he does, you can say, Oh you remember that Maya was sick and she's passed on.''

I nodded. ``That'd give him a chance to seem like he already knew it.''

We chatted about bringing my dad to Maya's memorial. We didn't know when that would be, but it'd take place at the house. Maya had requested it as part of her last wishes, which I'd found in my dad's office among their papers. Here's what she'd written: ``I imagine a remembrance gathering at my home . . . There to be my family and friends for food and drink for all to share feelings, thoughts, and memories freely. I see my spirit in their midst, in the circle of people who I knew and love . . . If my husband and dearest love survive me, he will be the final arbiter, as he would know best my wishes and desires.''

My dad would apparently survive her, but in his current state he might easily become overwhelmed during such an emotional gathering. That was the main reason why we hadn't brought him to the house that weekend. We didn't want to upset him. We figured the less disruptive we made his life, the better for him. As with his successful move to Primrose, we counted on two things to see us through: his lack of memory and his skillful ability to adapt to new situations.

I was often struck in those days how caring for my dad was like caring for my kids when they were young. It's a natural analogy to make. But my dad would never grow up and ask why I hadn't included him in Maya's memorial. We probably wouldn't do that to a child nowadays--- not allow them to be a part of the communal grieving process. I imagine the professional consensus might be that we'd do more harm than good to that child in the long run. But I didn't know how long of a run my dad had left. It was a hard call not to invite him into the circle of Maya's friends and family. Because of his dementia, my dad could not be Maya's final arbiter, so that decision was left to me. As usual, I erred on the side of protecting my dad. But I'm not sure that I was able to do that.

\begin{center}$*$\end{center}

The next day a group of us went to see him at Primrose for Father's Day. Most of my family was leaving afterward so it was a good chance to visit with him before they returned to their lives in Oregon and Southern California. Primrose had a barbecue going. One of the Day Club staff, Yolanda, brought my dad out from the dining room so we could all sit outside in a circle and visit on the lawn. We were a crowd for my dad: four of his kids (Chris, Steve, Cindy, and me), three grandkids (Steve's daughter Jackie, my two sons Miles and Ryan), Miles's girlfriend Julia, and my wife Linda. He didn't seem to recognize anyone. Still, he was his amiable self. He made jokes, kept up with our conversation. He mentioned his bow and arrow prowess after I told him we'd been looking at his old archery photographs and the trophies at his house.

``I could be pretty good at that,'' he said.

Someone had brought a box of face coasters---small pieces of cardboard with different mouths and facial hair that you hung off your nose and made yourself look funny. He was clownish with them, putting them on and making faces. He still liked being the center of attention. Chris said at one point (I don't remember how we got on the topic), ``Everybody's got a possum story.'' So people started telling them. It was true, everybody had one. I noticed my dad kept checking behind him toward the main building. I wondered if he was worried about something, maybe looking for Yolanda or anyone with a purple shirt. He wasn't quite sure who all of us were.

We went back to his room so he could show people what it looked like. He also opened his gifts. The first was a dolphin carving with a broken fin that I'd found in his workshop; my sister Cindy had repaired it. We showed my dad his Runner's Log, the journal he'd kept for a year as he trained for his marathon. The Log had a smiling picture of him at the beginning---May 2nd, 1993---and another picture of him one year later looking extremely fit and striding confidently across the Golden Gate Bridge during the race. He'd written inside the cover: Carver, The Road Warrior! Cindy also unveiled a photo-board that she'd made to hang on his wall by his bed. It held pictures of all his kids and grandkids with our names written beneath them.

My dad started to cry.

A couple of us rubbed his shoulders, told him it was okay.

``Well,'' he explained, wiping his eyes, ``that's what happens when you do what you all have done for me.''

He looked tired, overwhelmed. Everyone in the room was touched. Linda gave him a deep hug before we left, which was unusual. She'd always been wary of my dad and his schtick, as she called it. But her heart went out to him. It was another sign how dementia had affected his personality and family relationships. She wondered out loud to me afterward if the man he was now---sensitive, friendly, eager to please---had been Phil Moser as a boy before abusive family members had gotten hold of him. Had dementia really been the catalyst for a new person? Or was this the original one who'd been there all along underneath a protective bark that dementia had stripped away?

\begin{center}$*$\end{center}

Steve stayed for a couple of days afterward to help me haul more loads to the disposal center and demo parts of the barn: knock out a wall, bang apart old workbenches, clear out the rat's nest of wiring. I was relieved to have his contractor skills around, and Ryan enjoyed spending time with his uncle and helping him out.

Inside the house, I pulled three shelves of books from my dad's office and put them in boxes to donate. I checked the titles as the books came down, a brief history of my dad's interests in life: fitness, nature, self-improvement, and lots of Native American titles. I found a Korean War book--- Colder than Hell: A Marine Rifle Company at Chosin Reservoir---and several issues of the Marine magazine Leatherneck that dated from the period when my dad had told his story about walking seventy miles for his brothers in Charlie Company. The magazines gave background information on the Korean War and the ``frozen Chosin.''

I paused in my work and leafed through the issues.

One article in particular described the heroism of Corporal ``Ronnie'' Reininger at the Chosin Reservoir in December of 1950. He'd been a part of Charlie Company, an Infantry Battalion. ``Where we could find an outcropping,'' Reininger recounted in the article, ``we fired from behind it. Where there was no natural cover, we stacked frozen Chinese corpses and used them for protection.'' Reininger had been sent to Japan to recover from injuries and frostbite. A lot of his experiences corresponded to details my dad had mentioned for his own story in ``A Long Walk Home.''

I smelled a smoking gun.

I also found a pamphlet, Cherokee Indian Reservation: Western North Carolina--- The Official Visitor's Guide \& Directory. My dad's magazine story mentioned that he'd been born ``on the Cherokee reservation in North Carolina'' and that his ``father's family belonged to the

Eastern Band of the Cherokees, who avoided the 19th-century diaspora known as the Trail of Tears by hiding out in the Great Smokey Mountains.'' My dad recalled for his story:

I never really knew my dad . . . My mother wasn't around very much. I spent a lot of time with my aunt and uncle on the reservation. They essentially raised me. I have great memories---it was a wonderful childhood. The elders would come around and talk about Cherokee lore. They told me I was born into the wolf clan ---the warrior clan, the one that protected the tribe if it ever got attacked.

My dad's notes for his autobiography tell a different story about his aunt and uncle. Here's his chapter outline:

Running away---back to Uncle Bill---Mary---Legal Guardians---Sexual Abuse--- Uncle Bill bi-sexual---Mary wanted me---under guise of teaching me who about sex---Frightened---Ran away---lied, joined the U.S. Maritime service at 16--- Caught---Joined Marines on 17th B\/D. Life w\/Bill \& Mary---discipline etc.

My dad did join the Marine Corps on his 17th birthday, so I could verify that much. As for the rest . . . who knew? He wrote the outline for his autobiography in June of 1998, right after he'd finished a year-long battle with prostate cancer. ``A Long Walk Home'' was written in 2001. It was supposed to appear in a magazine called Over 50's. I recall my dad telling me about the article, saying it would help us understand him a lot better. But the magazine apparently folded before his story got published. I found a copy of ``A Long Walk Home'' among his papers in the filing cabinet of his office.

I kept the magazines and pamphlet to go through more carefully at a later date. It was hard for me to take seriously anything that had to do with my dad and Native American beliefs.



His stories about being Cherokee, and his war exploits, were such bullshit. And disrespectful to actual Indians and war heroes. At the same time all of those titles told another story: his desperate desire to build a past different from his actual one. He'd clung to those stories like a chameleon, assuming their colors so closely that he had Maya and all of their friends thoroughly convinced. In the larger scheme, Native American rituals and spirituality had grounded him. He'd become a loving husband, a friend and neighbor that people admired.

Why not give him that?

We had given him that, in truth. Nobody had outed him during those years. Maybe I couldn't let those stories go because I was always looking for something real from my dad, an honest response rather than a slogan, a genuine experience instead of one culled from a book or magazine. Although my dad was beyond retracing his steps---dementia had erased them like a high tide---maybe I wanted to be close to him in his final years because it was a real experience, sad as dementia was, and I wanted to share it with him. I wanted to have an authentic connection with my dad, even if he no longer knew who I was. 


\chapter{}

Maya died on Monday morning, June 29th. I got the call from her son, Jon. I'd been hosing down the outside of the barn, clearing black widow spiders and their webs from the eaves. We spoke for a couple of minutes. I told him how sorry I was, and we arranged a conference call for later that day with his sister Malka so that we could finalize details of Maya's cremation, her memorial at the house later that summer, and shipping Maya's art and effects down to Southern California where Malka lived.

I hung up, feeling sad and relieved: sad that Maya was gone, that my dad would miss her; but relieved for her, that she was free from her limbo state of dementia. As Trustee, I was also relieved about my dad's financial situation. He'd be able to stay at Primrose now probably for the rest of his life. I couldn't imagine a better place for him.

I went inside the house to tell Linda. We knew Maya had been dying, of course. But when it actually happened . . . well, it hit me like a punch in the stomach. Maya had been at the Board and Care for two years by that point. Linda and I had another one of our conversations about my dad: critical of him for how he'd tried to do everything for Maya himself, what he could have done differently, how Maya's last years might have been better. It was a good lesson for us about aging: not to cut ourselves off from other people, and to make plans so that we gave ourselves every chance for a good quality of life in our waning years, whenever those would be. The hard part was knowing when to start that process. Nobody plans on getting dementia.

I began contacting people. First Mercedes (she would call Maya's circle of friends), then the attorney and my siblings. I talked with my brothers and sisters about the best way to break the news to my dad. I'd spoken with Laurie, Director of the Day Club, several days before. I'd asked her how she thought my dad might handle the news of Maya's death. Laurie had almost daily contact with him, and she always offered sound advice. ``He'll do fine,'' she'd told me. ``He goes with the flow. He's always in such a positive mood.'' That had been reassuring for me. At the time of our conversation we were standing in the doorway by the lunch room at the Day Club ---my dad had just sat down at the table to eat. If my dad had been someone with big mood swings, she'd added, then she'd probably hesitate. But based on her experience, she thought it'd be odd not to tell him about Maya's death. Maya was still a part of his life. He'd visited her weekly. He was aware of her condition and had enough memories of her to understand the situation. He'd be sad, she thought, but he could handle the truth. ``Besides,'' she'd said, ``he's got such a strong support group here.'' I'd looked to the table where my dad sat between Wendy and another woman he often sat next to when Wendy wasn't around. I'd noticed that the men usually sat with the men during lunch, and the women with the women. But my dad had found his spot.

After speaking with my siblings and the staff at Primrose over the phone, I arranged a time three days from then---on Thursday, July 2nd---to tell my dad that Maya had died. I had already scheduled a wellness visit at Primrose that day---a meeting with the staff to review my dad's first six months---so it seemed like a good opportunity to break the news to him. I didn't want to wait any later because the Day Club would be closed on Friday due to the upcoming Fourth of July holiday. I wanted my dad's support group around when I told him.

Tuesday, the day after Maya died, I drove to Forestville to clean out her room. There wasn't much there: a few of her paintings, various greeting cards, photographs of family and other personal effects. All of her clothes in the closet would be donated. I opened a drawer on her nightstand and found a clutch of reading glasses. I stared at them a moment, then starting pulling them out, counting as I went.

Ten pairs. All for a woman who probably hadn't read by herself in years.

It was amazing what we collected, even at the end of life. I set the glasses in a box to be donated. Terry had emailed me earlier that week saying how she and Jerry were inspired to pare down after the weekend at my dad's, though she admitted they hadn't really started yet. I remembered the truckload of stuff they'd hauled back to Oregon---tools, some furniture, exercise equipment. Jerry had a full-sized truck, and that thing had been stuffed. They'd been doing me a favor by taking my dad's things before we put the house up for sale. At the same time it was so easy to keep collecting.

I probably had five pairs of reading glasses myself back in Missouri, stashed in various places at work and home.

After I gathered Maya's things, I asked Alain (the owner ) and Josh (his administrator) about Maya's last days. She'd refused to eat on Sunday, the day before she died. Hospice had put her on morphine. I remember Laurie at Primrose telling me that, for people about to die, it took too much energy to eat. Food wasn't appetizing for them anymore. I recalled Maya's great effort to respond to my dad during our last visit. She'd only had so much energy left in her frail body, and she'd saved it for him.

Alain told me he thought she'd gone quickly, deteriorating especially over the past three weeks. The staff had been keeping a close eye on her the day of her death, knowing it was coming soon. But she'd waited to die, Josh told me, until he and another worker had left her room. Alain teared up as he described Maya, which was comforting. When I'd been clearing

Maya's room earlier, a resident had looked in and quickly stepped back---so quickly that she'd bumped into the wall. Josh, in there with me and on his way out, had grabbed her so she didn't fall. Before releasing her, he gave the woman a hug to reassure her. It was a simple gesture, probably routine for him. For me it was another sign of how much they cared for every resident.

\begin{center}$*$\end{center}

That evening Linda, Ryan, and I had dinner out together. My wife and son had been in town for two weeks already, and this was the first time I'd had an evening free with them to enjoy Sebastopol. My surfboards sat in my dad's barn. I'd made one trip to the coast since arriving. Now June and Maya were gone.

Jon and Malka had decided on Saturday, August 1st for Maya's memorial. So I had a month to get the place ready. You'd think that would be plenty of time.

I climbed into bed that night with a copy of The Education of Little Tree, one of the books I'd pulled off my dad's shelves. I'd heard of it before---I couldn't remember where---but it was a fast read. It's written in a home-spun style with Granma and Granpa who reckin this and that. The narrator learns The Way of the Cherokee from Granpa, an Appalachian mountain man who calls the young boy Little Tree and takes him under his wing. I recognized suspicious similarities with details in the outline of my dad's autobiography. Both Little Tree and my dad were raised by depression-era ``elders'' who taught them the corn dance. Like Little Tree, my dad mentioned that'd he grown up with a still and Christians selling whiskey from the local store. I later learned the book had been unmasked as a hoax, not written by a Cherokee at all. A fitting tale for my dad, his fictions drawn from other fictions.

My annoyance at my dad intensified the next day after I met Mike in Mill Valley and he handed me a copy of my dad's military records. We were meeting for lunch so Mike could introduce me to an old friend of his, Louis Patler, who was working on a book about connections between entrepreneurs and big-wave surfers. Louis had heard that I'd worked with Shaun Tomson, a former world surfing champion, on a couple of books, so he wanted to know if I'd introduce him to Shaun. Louis's project sounded interesting, and he was a friend of my brother, so I was happy to make the connection. Unfortunately Mike had to leave lunch early---his wife had a doctor's appointment---but I knew I'd see him again the following week back in Mill Valley at his art opening. Mike looked more gray than I'd remembered, very pale and frail. It reminded me of the last time I'd seen Steve Good, the Dean who'd hired me at Drury. Like Mike, Steve's melanoma had started in his big toe. We'd had a reception for Steve after he'd gone through treatment, presumably so he could say goodbye to everyone at the university. He'd had a thin, ghostly appearance that night. He'd died not long after.

I took the military records back to Sebastopol and reviewed them carefully: seven years worth of stations and transfers with corresponding dates, from my dad's enlistment in Washington D.C. on his 17th birthday to his year and a half on Guam via Pearl Harbor and Midway to his final post during the Korean War at Quantico, Virginia. His enlistment papers had indicated a preference to serve in the Infantry, but they'd sent him to clerk typist school instead. He'd pretty much been a model soldier the whole time, according to his conduct scores (generally in the ``Excellent'' range). He'd been promoted from Private to Staff Sergeant during those seven years and earned two Honorable Discharges. There was no mention of combat duty. No evidence that he'd even been to Korea at all. I guess he figured Guam was close enough.

I'd known about his service record for years, but actually having the facts in my hands--- the places, the dates, the duties---sent me back to ``A Long Walk Home'' to compare reality with my dad's appalling lies about serving in Korea, walking seventy miles for his fallen Marines, and holding his best friend in his arms while he died. Did he not think the truth would ever come out? That he wouldn't be exposed as a liar?

There were more details in the story that made me furious. The writer reported my dad's rationale for walking---``I had to do something demanding, something that simulated in some way what we went through in Korea. We were so cold, we were terrified, we were hungry''--- then she waxed eloquently about him:

But if he sounds like Everyman, he also fits the mold of the classic hero---a sojourner who goes off to seek adventures beyond the ordinary, and returns transformed in some way.

Often such heroes bring us back a message, a gift of wisdom or illumination. All along, Carver sought no publicity for himself; that wasn't the point. He agreed to tell his story only because it might prove useful to others.

I was disgusted on so many levels: my dad the hero who went off and found enlightenment, then came back as some kind of guru with a message for the world. He also portrayed himself as a free spirit unconcerned with material things who returned to California from Maui with a renewed concern for his grown kids because he'd had a health scare---pernicious anemia misdiagnosed as leukemia: ``It woke me up to the fact I had children who missed me. I was kind of cutting them out of my life in a way that was wrong. It was time to mend fences, and get back in touch with my Native American spirituality.''

Sitting in bed that night, reading through his military records, comparing them to ``A Long Walk Home,'' I looked down at a big red welt on my leg---a spider bite---and wondered what the hell I was doing to myself and my family. Why were we there? Why were we busting our asses every day to clean my dad's house, organize his barn, and prepare his property? And there was still tons of work ahead before the house was ready to sell. Why were we doing a mountain of work for a man so outrageously selfish? Thank god that story had never been published. I wanted to rip it to pieces.

\begin{center}$*$\end{center}

The next morning I walked into the Day Club to tell my dad about Maya. He was in the activity room playing dominos, sitting next to Wendy. I slipped up to the table and said Hi to everyone. Yolanda and Graciela were there, and I'd spoken briefly with Laurie in the other room. They were all ready. My dad recognized me right away, which surprised me. The week before when I'd popped in with Ryan, my dad had looked at us from across the room like he might know us---or should have known us---but nothing registered for him. It was the first time I'd seen him with that distant stare common among people with dementia. But the more time I spent around him, the more he seemed to recognize me.

``You mind if I borrow him for a few minutes?'' I asked Wendy.

``Go right ahead.''

I wondered if she'd been tipped off about Maya. Wendy was sharp, and perhaps the staff had told her so she could help my dad. I was anxious about telling him. I didn't know how he'd react. Linda had told me that morning, ``You can't protect him. It's okay for him to be sad.''

``Come with me, dad,'' I said. ``Linda and Cindy are in the next room. They want to say Hi to you.''

``All right.'' He stood from the table and walked with me, compliant as ever. Yolanda and Graciela followed us with their eyes.

He hugged Cindy and Linda.

``Have a seat,'' I told him. They had a sofa along the wall, and he sat down next to Cindy. Linda and I pulled up chairs opposite him. I pointed to his Marine Corps baseball cap. ``I sure like your hat. It looks good on you.''

He tugged at the bill and smiled.

``Listen, I have some sad news.'' I leaned toward him. I wanted to be gentle but direct. ``You remember how Maya has been sick lately?''

He nodded.

``And she's been declining.'' It wasn't a great word, but I couldn't think of anything better. I took a deep breath. ``She passed away.''

``Oh,'' he said quietly. ``Really?'' He teared up and looked at the three of us. ``She was my life.''

``You always perked her up,'' I said, trying to reassure him. ``Whenever you visited her.'' Cindy nodded. ``Do you want a hug?''

He hesitated a moment, then said with a sudden smile: ``Sure!''

They hugged. He grew very quiet, his gaze dropping to the floor. ``I think I'd like to lie down.''

His smile to Cindy had been the classic Phil Moser, a sort of slapstick maneuver. He was in shock and had pulled that ace out of his sleeve. I didn't like the idea of him going off to his room by himself. That'd always been my dad's mode when he got stressed: retreat. I didn't want him to trap what he was feeling inside. I thought it was important for him to tell somebody, to say the words out loud. ``Do you want to tell Yolanda and Laurie?'' I said. ``They'll probably want to know.''

``That's a good idea.''

I followed him into the next room, taking a seat by the window. He walked toward Graciela, his gate unsteady. ``My wife died,'' he told her, his voice breaking. She stepped toward him, and he collapsed into her arms. He let himself go completely, his shoulders shaking as he cried. She held him up, and Laurie and Yolanda quickly came over and hugged him.

One of the Day-Clubbers, an elderly man sitting beside me, asked what was wrong. I told him. He got up, went over to my dad, and put his arm around his waist. ``I lost my wife, too,'' he said.

Wendy came over and rubbed his shoulder. She said a few words to him, then went back to the table. I thought that was considerate of her, to leave him that space to grieve.

Other residents went over to my dad, and he thanked them. After a few minutes, Cindy walked him back to his room so he could rest.

``We'll check on him,'' Yolanda told me.

I remarked how comfortable my dad felt crying in front of them, allowing them to take care of him. He hadn't done that with me or Cindy. I had never seen my dad completely break down like that before. It was such an honest human reaction. He'd been so vulnerable.

``We're his family now,'' she said.

Linda and I walked to the main building where I dropped off a check for my dad's monthly rent. I popped into his room to see how he was doing. He was tucked in bed, Cindy standing beside him. I reached down to the blanket and gave his foot a light squeeze. ``Love you, dad,'' I said.

He looked up and said, ``I love you, too.''

Cindy leaned down and gave him a kiss. I turned out the light, and we left.

I was in a little shock myself as we walked out to the parking lot. I couldn't remember the last time I'd told my dad that I loved him, if ever. Or when he'd ever said it to me. I couldn't help myself, my heart just went out to him. This frail old man who had to tell people that his wife had just died.

When we got home I couldn't pick up Colder Than Hell. I'd been reading his books with relish, looking for evidence of lies to throw in his face. But I wasn't up for another round. I left my dad to his grief, at least for a day. I felt bad for him. I also wanted time to reflect on my feelings toward him, which seemed to change from day to day. So I went outside and worked on the property.

The next morning a big truck would come and haul away the yard waste Ryan and I had collected during the week: branches, sticks, pine needles, weeds. I had three piles going around the property. As I put on gloves and added to the piles, I kept thinking about what my dad had lost. Maya had died on Monday, but he'd lost her today. Every day he lost something else from his property---wood, tools, old furniture, dead bushes and trees. One day soon he'd lose the whole place. His life was getting smaller by the hour. He'd already lost most of his memory.

And what had he found? New friends and family at Primrose. A new home. A new identity, even. He'd been striving so desperately for one of those most of his adult life, going so far as to rename himself Carver in hopes of making it stick. He'd finally gotten his wish with the onset of dementia, though I doubted he was aware of it.

And I found him again. After years of a lost relationship, I had told him that I loved him.

\begin{center}$*$\end{center}

July 4th we were on on our way to a barbecue at a friend's house and dropped by Primrose to check on my dad. He was in bed when Linda, Ryan, and I filed into his room. He got up and gave us hugs. He was completely dressed: shoes, jeans, a long-sleeved green tee shirt, glasses, and his Marine Corps hat. He must have put himself to bed after lunch.

``How are you doing?'' I asked.

``Okay, considering.'' He sat on the edge of his bed and started talking about Maya. He'd say a few words, get overcome with emotion, then stop talking and drop his head. He did this several times. Finally he looked up at me. ``I'd like to go to the place I was before.''

At first I thought he meant the Board and Care in Forestville, where Maya had lived. I quickly realized he meant his home in Sebastopol. I tried to imagine how he'd react seeing most of his stuff gone, the place in disarray. ``What do you need?'' I asked.

``I want to go there and write some things down about Maya.''

``Oh, I can help you with that right here.'' I suggested getting a pen and paper, and he could write down any thoughts he had about Maya. He liked that idea. I left the room in search of writing materials. I brought them back along with a chair so my dad could sit at his small desk.

I titled the piece of paper for him: Thoughts About Maya. I was encouraged. He was doing what I'd hoped he would: expressing himself, processing Maya's death in healthy ways.

He sat down at the desk, pen in hand. He didn't know how to start. Either he couldn't focus his thoughts, or he'd forgotten how to write. He stared at the paper. ``Maybe I'll do it later,'' he said.

``I can do it for you,'' I suggested. ``You tell me what you're thinking, and I'll write it down.''

He agreed to that.

Linda offered to wait outside with Ryan. She was being thoughtful by giving us privacy. Or perhaps her survival instincts were on alert for Ryan. I'm so glad they weren't in that room for the next fifteen minutes. The talk therapy I so eagerly pushed for my dad quickly got out of hand. Picture me as an excited passenger waving to loved ones on shore as I boarded the Titanic.

I sat in the chair I'd gotten for my dad, ready to take dictation. I wanted to help him grieve. I was also thinking of the journals he'd written, the daily journal I'd been keeping since I'd arrived, and our relationship through words that was beginning to take form in my mind as a bizarre book project where I actually completed my dad's autobiography. It was bizarre because I didn't think that story deserved to be seen by anybody because my dad was so full of shit.

My dad said nothing at first.

I cued him. I reminded him of what he'd just told me about Maya and wanting to go to Sebastopol. I didn't want him to keep his feelings bottled inside. I thought it would be healthier for him to let them out. At this point I was no longer a passenger above decks: I was down in the boiler room shoveling coal into the furnace.

But my dad still said nothing. Somehow I was giving him writer's block. I put the pen and paper down. I took my phone out so I'd have a record of his words. I turned on the video and held it discreetly in my lap. I asked him to say something, and he started to describe what he was feeling.

I checked my phone to make sure it was recording. I encouraged him: ``I'm glad you're talking about it. I think that helps.''

``I think that's a good thing to do. I really do.'' He took a deep breath and let it out.

Someone knocked on the door. My dad called out, ``Come in.'' A worker entered with a snack cart. My dad pointed at me. ``You know my son?''

She smiled at me. ``Nice boy.'' She turned to my dad. ``How are you?''

My dad lifted his hand and gave her a so-so gesture.

I'd seen her before. She was small, perhaps in her fifties. She came from Eritrea. The workers at Primrose were from all over the world---Yolanda from Columbia, Graciela from Mexico, many from Eastern Africa. The woman gave my dad a cookie on a stars and stripes napkin to celebrate July 4th. She leaned in, gave him a hug and kiss. ``You nice man, we love you.''

She offered me a cookie too, then poured cranberry juice into a plastic cup for my dad. He finished it in one long drink like he was taking medicine. The worker left, and my dad ate his cookie. When he was done, he carefully unfolded the napkin on his lap so he could see all of the colors. ``Boy, that's beautiful.''

I started the video again. ``I wanted to ask you how you were doing, you know, make sure ---''

``---Better than I thought, frankly.'' The snack seemed to have perked him up. ``I've gotten over a lot of that . . . of Maya. I mean I cried over her for days when it happened. I didn't know what to do with myself. I didn't know how to handle anything. I never went through anything like that before, someone you love that much and you find out they're gone, they've died.'' His voice became strained: ``You'll never see them again or touch them again, it's hard.'' He sniffled. ``I even have trouble talking about it, as you can see. When you lose someone like that, when you love them that much, part of you is gone.''

I told him the feelings he was having were normal.

``That's what the girls teach me, and other people: `You'll get over this.' I say, Well, how do you know? Have you been through it? They say, `Well, maybe I haven't, but I know people who have. And we'll help you. Whatever you need, you let us know, and it'll be there'.'' He pointed at me. ``That's the truth. They do, they help you. They help all the time.''

I liked what he was saying. It felt real to me, an honest expression of his feelings. ``You're going to miss her,'' I said. ``It's normal to feel like you're missing a part of yourself.''

``Well, I really would appreciate it if you can set up somebody that I could get back over there''---He meant Sebastopol---``in a couple days and just clear some of this up in me, you know. I'm not interested in going back to where Maya was. It doesn't have any hold on me now.''

He had put his hands up in the air and shaken his head at the thought of visiting the Board and Care in Forestville. I could see the place didn't have good memories for him. ``What do you think you'd do at your home,'' I asked, ``if you went there?''

``I think the first thing I'd do probably, I've been thinking about this, is connect with . . . with . . .'' He laughed at himself, hit his head with his hand. ``My brain,'' he said. ``What's her name?''

``Are you thinking about Kay?''

``Kay,'' he said. ``I immediately think about her. And now you say she's not there anymore. So I'll get a hold of her and tell her I'm there. If she has some time I'd like to talk to her. She's good about things like that.''

``She's very good.''

``Very good at listening. And she can think up some things I wouldn't even think of. Some other people used to come to the house and take care of me when I was . . .''

``You mean Mercedes,'' I said. ``I told her that Maya had passed away.''

``Oh, you did?'' He seemed glad for the information. ``I just want to get there for a couple days, maybe as long as a week if I can and get this settled down so I can function. You know in the old way that I used to be.'' He laughed and put his head down. He pumped his arms quickly back and forth like a runner. ``Mr. Go Get It, Get It, Get `Er Done.''

``Do It Now,'' I reminded him. I'd grown up with little yellow refrigerator magnets with that slogan on it. My dad had gotten them from his job in Los Angeles.

``Yeah, Do It Now and Get `Er Done. That was my motto. It still is.''

He took off his glasses and wiped his eyes. ``I know I'll get over this stuff, and I know it'll pass. It's been tough. I just couldn't imagine. I mean, I thought about Maya and all the things she had to go through . . . It's just a matter of me accepting the fact that this isn't going to last forever. And I'm going to be able, to be able to handle some of it. That's why I want to go home for awhile where . . . a place I spent a lot of long time and I know a lot of people. I'd like to be able to go to those folks and say, Look, Maya died last night. Boy, they'd jump right to it.'' He stopped, and his voice cracked with emotion as he repeated what they would tell him: `Okay, What do you want? What do you need? We will do it. You just tell us what you need. We'll drop everything else and make sure that if you need something, if you need silent time''---he pointed to the desk---``like writing, if you need to go someplace and go off into the river and surf or something, we'll help you do that'.''

He paused to wipe away tears. ``You know, when you get things like that from people you love, it changes your whole life. You know there are people there who love you, and will treat you like something very very special. That's what you and the rest of them have done. And I just . . . .'' He let out a long breath. ``It's hard for me to talk about.''

``It is hard to talk about.''

``I want to get it out,'' he said. ``I don't want to hold it up and make believe that I live one way and find out that I'm living another.''

His tone had suddenly shifted. Why was he talking about making believe in his life?

``I only had one situation here that I . . . I just can't believe it.'' He took his glasses off and set them on the side of his bed. ``There was this big husky guy, and we were going down . . . .'' He tried to recall something. ``We were having a regular lunch, you know, where we live at Primrose, and he saw me with this hat on.'' My dad took off his Marine Corps cap and looked at it. ``And he knocked it off my head.''

``Really?''

``Yeah, he knocked the whole damn thing off my head. I said, What are you doing?'' My dad hardened his face and adopted a gruff voice, speaking as the other man: ```Get rid of that goddamned hat. You don't deserve any THING like that'.'' My dad acted out the scene by bouncing his hat off the bed. It landed on the floor by his feet.

I sat there stunned, the phone filming away.

My dad picked up the hat. ``I said, I'll tell you what I do deserve. Knockin' your FUCKIN' head off.'' The old dad---intense, combative---had sprung from his primordial brain like a Gollum. He leaned forward, his eyes sharp. He pointed his finger at me. ``That's exactly what I said.''

Then he leaned back. ``And he knocked the hat off again.'' My dad pointed at me like a cowboy pulling a gun out of his holster: ``And I went after him. If two guys hadn't grabbed me. I woulda killed that son of a bitch.'' His voice broke and he started crying. ``I earned this thing.'' He looked for his hat on the bed, then realized he was holding it in his hand. ``I mean, look.'' He read the front of the cap: ``U.S. Marine Corps. I love that hat, and I earned it.''

``That's right, you did.'' I sound like a robot on the video, trying to keep my voice as neutral as possible so my dad didn't leap off the bed and slug me.

``He knocked it off, said I didn't have the right to earn it. And I said, Who the hell are you? Tell me what I can wear and what I can't. Then he started beating on my hat.'' My dad stopped and stared. He was in another world, far away from that room at Primrose, conjuring a fantasy where he protected his fictional honor as a combat soldier in Korea. ``That's when I slugged him. Knocked him right on his ass.''

``Really?''

``A couple guys grabbed me. I said, Let me alone. I'm going to kill this son of a . . .'' He raised his hands like you might do to a cop in the street: ```No wait a minute, wait a minute, wait a minute','' he said, speaking for the people who broke up the fight: `We know he's done something wrong, we know he has a vile temper'. I said, Well he may have a vile temper with somebody else. I'm not going to give him a vile temper, I'm going to give him a fuckin' place you can bury him. That's exactly what I said.'' The tears were flowing again, and my dad wiped them away. ``And he got up and said''---My dad raised his fist in the air---`I'LL KILL YOU'. I said, Oh, you will?'' My dad shook his head: ``Not before I get to you, buddy. You have NO IDEA who I am. The things I've been through and the places I can handle. I can handle you''---He pointed at me---``I can put you in your FUCKIN' grave.''

I was frozen to my seat. I felt like the character in Jurassic Park when he's staring into the jaws of a T-Rex: If I don't move, I thought, he won't know I'm here.

My dad paused, then added more calmly: ``Then somebody else grabbed me and him, and they calmed it all down.'' He waved his hand. ``He's gone.''

``I'm glad there were people around.''

``I don't think he'll be back. But they told him, `Screw around with this guy, he's not just anybody'.'' My dad's voice cracked again, almost pleading: ``And I'm not. I don't get in fights like that very often.'' He picked his glasses off the bed and gestured at me with them. ``When I do, I'll tell you. I win most of them. But I don't start `em.'' He eyed me to make sure I understood. ``But I do finish them. Very rare, but I don't want anybody ever treating anybody else like that, and they're not going to treat me like that, buddy. I'll tell you.'' He shook his head. Tears were streaming down his face again. ``I hate that stuff.''

``Well, you did earn that hat, that's for sure.''

``If they only knew that other people had earned it, what they had done with their lives. You know, in Korea. I earned this cap. It's funny how I did it.'' His manner shifted again, his tone lighter as he recalled something else. ``You know, I've always been really good with the bow and arrow.''

Uh huh.

``And when I decided to get to Quantico, that I was going to shoot the bow and arrow, when I won a national contest, and then when this guy shows up''---My dad dropped into a taunting voice---```I'll kick your ass!'. I say, You Will? When? How about now! And he comes up and he knocks the hat off. That's when I went after him.'' My dad raised his fist. ``I got one crack at him, knocked him right on his ass.'' My dad leaned back on the bed like he was the bully who'd been knocked down. ``And he sits there like that''---My dad trembled like he was afraid ---``He got up to get up and then everybody grabbed him and then they grabbed me. I said, Don't you ever, ever, say anything like this, the men who earned this''---He held out the hat---``including me put their LIVES on the line you SHITHEAD. And you're going to tell me how to earn my keep here and I'm doing something wrong and you knocked the hat off that I earned and other guys put their life on including me.'' He stopped and stared again. ``Don't you ever, ever get in my way again. And then he took off and I took off. I was getting so mad. I was burning.''

``When I think about Maya,'' I began, desperate to get him off this loop. ``I think about how gentle she was, and how she loved being peaceful. So I'm glad that you controlled yourself because that's what Maya would have wanted for you.''

My dad put his hat and glasses back on. He started crying again. ``I didn't want to do that. But you do what that man did, knocked over this thing''---he touched the hat---``that men put their lives on the line for including me, and tell me, `I'm a dumb son of a bitch, and what am I doing'.'' He knocked the bill of his hat with his hand. ```And what right do I have to put that hat on'. That's when I told him I earned it.''

``It's nice they have peaceful people here who love you.''

``Oh, the guys grabbed him. A couple of them know me. They saw some of the pictures of the stuff I do. They like me.''

``It's because they care about you.''

``They damn right cared about it. They said, `He's not going to come back. He's outta here. He won't be here anymore. You'll never see him again, we'll make sure of that'. I said, Good, because if you didn't I'm gonna kill the son of a bitch. They said, `Okay, calm down, calm down'.'' My dad put his hands in the air like he was surrendering: ``Then they realized, `We understand that. We heard what he said. He had no right to say it. This guy's been giving people trouble all along. We're always trying to cool him down'. I said, You don't have to cool him down anymore.'' My dad wagged his finger at me, then pointed it for emphasis: ``I'll cool him down for you. And I meant it. I get that mad, ain't much can hold me back. So anyway, that's over the hill. I just wanted to get it out of my system. Thanks for listening to it.''

``No problem,'' I said. ``I wanted to listen.'' What else could I tell him? I was the idiot who'd got him started in the first place. ``I'm glad you got it out of your system.''

``I have to. I can't let it stay there forever. What good is that?''

``And there's always people here,'' I added, ``if you feel like you need to talk to somebody about Maya or how you feel, there's always somebody here at Primrose.''

``Oh, the girls particularly, when I told them about what happened with Maya, they cried with me, they hugged me, whatever I need, let them know, that I don't have to go anywhere or do anything I don't want to do they'll take care of it, just let `em know. They did. That's how good the girls are. They're amazing at that. Every time I see them, they give me a hug. It's `how ya doin', how ya feelin', do you need anything'. Every night they'll ask me now, `Do you want to come home and rest in your bed for awhile and do whatever you want to do. I mean, we're going to go out and have some fun, do this that and the other. You can go with us if you want to'. And I did a couple times. I usually felt like I needed some quiet time.''

``That's good.''

``That is important. It really is important. That you can be with yourself, that you have people that love and care for you and will do anything you want to do.''

``Do you feel like you want to take a little nap now, a little rest?'' I was thinking of finding an empty room myself and lying down.

``I think I should.''

``You've been very emotional. I think it's probably good to get a little nap in. Plus I feel bad for waking you up.''

``No, I'm glad you did. Because what you did, you were able to put me, how I was feeling then, not many people were going to do that with me. They'd think I was some kind of . . . crazy guy or something.'' He grew reflective: ``I kinda look crazy sometimes when somebody grabs me like that. My family---''

``---Well, they understand, because they all know about Maya, so they understand that you're gonna be emotional and you're gonna wanna talk.''

``I told them that. I also told them that I'm going to have to go away for at least a week maybe and get, you know, centralized is what I call it. Get so I can handle what's going on, because right now I can't. When I was talking to them. I have to be alone for awhile, talking about the book and things like that.''

The book? What did he mean by that? ``That's good,'' I said.

``Yeah, I think good writing''---he pointed to the desk---``that stuff, have you writing it down like that.''

``I'm happy to do that.''

``That way we can remember it. `Cause I know I can get over this. There's very few things I can't get over.''

``You're so good at adapting.''

I suggested he take off his shoes. He did and put them neatly by his bed. He lay down and pulled the covers over him. I sat on the edge of his bed as I might have done at night with my sons, if they were sick or if I wanted to talk to them. Bedtime was a good time to talk; it was quiet, and they'd be tired and less resistant to anything I might want to say. If I had something important to tell them, I might wait until that moment.

I took my dad's hat and glasses off and set them on his dresser. I told him, ``You have people who love you here. If you ever feel like you need to talk about Maya, you can always tell them.''

``Okay.''

I leaned down. ``I'm going to give you a hug.''

He pulled me in tight. I was reminded of how physically strong he still was. It worried me after all his bluster about knocking people on their asses. He was still intense, still had that fighter inside of him.

``Love you,'' I said.

``Love you, man.''

I left him under the covers and turned off the light. I escaped into the bright sunshine of a

July 4th afternoon feeling bewildered and a little traumatized. Happy Independence Day. 


\chapter{}

I woke early the next morning to a drip, drip, dripping coming off the roof and landing on the deck. The house, for some reason, had no rain gutters. It was the sound of my childhood in Occidental: the fog rolling in from the coast during the night and soaking the redwoods, which in turn rained the excess onto the world beneath them. It was cold inside my dad's house, so I stayed in bed and tried to figure out exactly what'd happened the day before.

My ``old'' dad, the one I'd thought was long gone, had suddenly surfaced under the trauma of Maya's death. I hadn't expected to see that at all. He was taking her death much harder that I'd thought he would. It was one of the most important lessons for me from the whole experience of caring for my dad: I'd assumed, because of his dementia, that he wouldn't be as affected by emotional losses as a ``normal'' person might be. I thought his lack of memory would buffer him from the pain of leaving his home, or losing Maya. My strategy had been to distract him, put him off, act like everything was normal, then hope that it would become normal for him with the passing of time. I'd hoped to surf him right over life's troubles and have him emerge unscathed on the other side. But there were consequences for not allowing my dad to fully process his losses as they happened. If I had to do it over again, I probably would have ``protected'' him less. I don't think I gave him enough credit for being able to handle those emotions.

All of those raw feelings were still inside of him, now stirred up because of Maya. It was a cauldron of guilt, rage, frustration, and violence against an imagined antagonist that, as far as I could tell, was my dad himself.

It hadn't occurred to me when I was in his room, but later I recalled that my dad had had a minor altercation ten months before with another Day Clubber. Kay had sent me a note about it, and I'd called Laurie to find out the details. When Kay had picked my dad up that day, she'd told me that he was visibly upset and shaking. He'd told her that three of the staff had had to pull him off this other man or he would've decked him.

Laurie and the others quickly corrected my dad's version of events: the man had gotten on my dad's nerves during a bingo game. Apparently the man wouldn't shut up. My dad had strong startle reflexes against loud noises, or loud people. My dad had knocked the man's bingo card off the table; Graciela had had to settle him down and ask the other man to leave the group. Somehow that incident had morphed in my dad's mind to a fistfight over his right to wear a U.S. Marine Corps hat and how he'd earned that hat by putting his life on the line.

I was the one who'd bought him that hat. I worried that it'd become a trigger for his deep sense of guilt---over his lies about Korea or other traumas inflicted on or by him---and I thought I should replace the hat as soon as possible.

I shouldn't have stirred him up. I should have left him to his grief. I worried that he'd regress faster now, that he'd become more combative with the staff. I'd been waiting for the later stage of dementia that often happens, when patients become angry and difficult to manage. Had I pushed my dad to that stage? I regretted filming him. I'd been so concerned with getting a record of it all and wanting him to talk that I'd let the situation get way out of control. I hadn't been paying enough attention.

At the same time I was touched by his memories of Kay, Mercedes, and his other friends who'd taken care of him. He missed them. He missed his home in Sebastopol. I'd thought he'd completely adapted to his new life at Primrose.

At least I'd gotten the chance to connect with my dad on his level. However mixed up his memories and emotions, however he imagined his life as an Indian Marine Archer on the frozen battlefields of Korea, his feelings were real. His fear, his sadness, his interior battles, his guilty tears---all those sprung raw and honest from his core. Amid his swirl of fictions, amid the tangles and plaque that demented his brain, it was the most authentic interaction I'd probably ever had with him.

\begin{center}$*$\end{center}

Most of July I choreographed a platoon of workers to get the house ready to sell: pest inspectors, septic men, plumbers, electricians, movers, real estate agents, carpenters, bankers, a lawyer, a locksmith, a stager, and a guy who killed gophers for a living. That guy deserves his own book. His name was Adam, and he showed me how to set traps and strew the carcasses of dead gophers in the meadow to attract predators---foxes, bobcats, hawks---so that I wouldn't have a gopher problem.

Adam had the best stories. One day he arrived after a job in Petaluma where he'd gotten twenty-four gophers on a piece of land the size of my dad's property. The owner was a man in his seventies, very quiet and proper. He'd brought Adam over to his wife's vegetable garden. ``This,'' he'd told Adam, ``is the most important thing.'' The gentleman had run chicken wire underneath the raised garden, but a gopher---it must've been a baby one to fit through the mesh

---had gotten in there and found himself in squash heaven. He'd been in there for months, feasting on their zucchini.

Adam had set a trap, expecting to catch a baby gopher, maybe six inches long. The next day when he pulled out a fully-plumped adult, its neck broken, the gentleman had pumped his fist with joy: ``That's the motherfucker that's been eating all my vegetables!''

I loved the way Adam connected with the land. A trap snapped shut as we were walking around the property---a fresh kill. ``Music to my ears,'' he said. We went over to check it out. He admitted the sound gave him a rush. I could tell. I heard the excitement in his voice. While he was resetting the trap, he spotted a gopher popping its head up nearby. ``Son of a bitch,'' he said under his breath. It was like a personal affront---a gopher taunting him. He double-timed back to his truck, returned with a trap, and set it in the mouth of the hole.

``What's your record?'' I asked him, meaning the time between setting a new trap and catching a gopher.

``Three minutes.''

Impressive. And he was a hell of a lot cheaper than guys who used poison or pumped propane into the tunnels and bombed the little bastards.

Setting traps seemed a more natural way of pest control. My dad would have appreciated the do-it-yourself method. Adam was a good teacher, and soon I was catching gophers myself. I'd ask myself every once in a while as I worked around the property: Would my dad have been proud of me for this work? Would he have done it this way? It was the kind of labor he'd enjoyed himself. I figured since we couldn't do it together, I'd do it for him.

I spent more time with my dad. A week after I'd told him about Maya, I took him to the Santa Rosa VA. I'd managed to get him a morning appointment to fill the three cavities the dentist had found the month before.

He looked good and seemed happy to see me. Once we got settled in the car---it was about a fifteen-minute drive to the VA---I asked him how he'd been. He told me he'd had a stomach ache for a few days.

``Do you know what might be causing it?''

``I've been thinking a lot about Maya.''

We talked about her, and his feelings as I drove north on Highway 101. Traffic in our direction was light; most commuters were headed south toward San Francisco. He asked me about going home again. He thought it would help him deal with Maya's death.

``Maybe stay at Primrose a few more days and see how you feel,'' I said. ``If you still want to come to Sebastopol, then we can do that.'' I figured if he stopped asking, then the problem would be solved. I wasn't sure if he really wanted to go to Sebastopol, or if he was just saying that because he had such strong memories of Maya there. I still hadn't decided if I should bring him for her memorial on August 1st. It would be exhausting for him and could be disorienting. But if he needed closure, that might do it. I'd wait to see if he was still asking about her in a few weeks.

``She's dead,'' he said, ``and I can't change that. I just have to leave it to God.''

It'd been a long time since I'd heard him talk about God as opposed to Mother Earth or The Great Spirit. A couple weeks before Maya had died, my dad had told me he'd been going to

Bible classes at Primrose. He later said he didn't read the passages himself, but he liked to listen to the minister.

Once we got settled in the clinic---me in a corner sitting behind my dad as the dentist began drilling---I realized we were dressed almost exactly the same: jeans and a belt; thick grey socks with red\/black hiking shoes. He wore a light green jacket over his Hawaiian shirt; I had on a long-sleeved green thermal over a tee shirt. We must have looked like twins walking into the clinic. I stared at the back of his head: same hairline; his balding spot up top will be mine one day.

I'd been surprised to find myself worrying about him, if he was in pain or whether he might fall. My dad had been wobbly as we walked toward the clinic. It had been sprinkling the past few days, rare for July, and the pavement had been slick. I'd told him he could grab my arm for support. Whenever I walked with him now, I found myself constantly looking over my shoulder, making sure he was okay.

As I sat staring at him from my corner, I suddenly imagined him dying in the dentist chair. I pictured the dentist and her assistant in a semi-panic, jabbing colored buttons on the wall to bring help. They'd put an ear to his mouth and listen for breathing, thrust a palm under his shirt to check for a heart beat. I'd have to tell them that he had a Do Not Resuscitate order. I knew because I'd signed it. I imagined having to call my siblings and tell each one of them the news. It was a bizarre scenario to conjure in a dental clinic. Later I thought maybe it was a sign of our changing relationship. That in fact I would miss him if he died.

Back at Primrose, his three cavities fixed, I settled my dad in for a mid-morning nap. ``Your green jacket will be right here when you need it,'' I told him, opening his closet. ``You can take off your shoes.''

He did. I arranged his jacket among his shirts and other coats. I closed the closet and looked over at him. He was putting one shoe back on.

``You can take your shoes off, dad.''

He took the shoe off again. He placed both shoes neatly together next to his bed. ``Let me take your hat for you.'' I put it on his dresser where he could find it when he woke up. I pulled back the cover and told him to get in. ``Will you be more comfortable in sweats?''

``No, jeans are fine.''

I hugged him. ``Someone will come and get you for lunch,'' I told him. ``See you soon.''

\begin{center}$*$\end{center}

The next day, on Saturday, I took Ryan surfing at Salmon Creek. I'd been conscious about spending time with him, trying to ensure that I didn't repeat the mistakes my dad had made. Since I couldn't have the relationship I'd hoped to have with my dad, I figured I'd compensate by spending more time with my sons. Any paternal approval I might be seeking perhaps could be satisfied by a good filial relationship. I didn't know if that would work, or even if it made sense: to be the dad that I'd wanted to have. It was the only way I knew to negotiate an impossible situation.

I had to decide that I'd be on Ryan's time. This is hard for a surfer because timing is everything. Surfers usually hit the waves as early as possible---Dawn Patrol is the relevant term

---not only to avoid crowds but to beat the onshore winds in California, which usually blow out of the northwest late morning and fizzle the surf by crumbling it over. A light off-shore wind is ideal because it holds the waves up and grooms the walls of water so surfers have the maximum canvas on which to draw their lines.

I had to hold my impatience in check as Ryan slowly woke up, ate breakfast, and got ready to go. We arrived at the coast about ten. The onshores had already started blowing, but patches of sun were a welcome sight burning through the marine layer. Ryan pulled on my dad's wetsuit. We did a few pop-ups in the sand, practicing how to stand on the board once he'd caught a wave, then I showed him where we'd paddle out and sit. We'd be inside of everyone else, out of their way, and try to catch the walls of foam rolling in from broken waves.

Surfing is intimidating, especially in Northern California. The water is cold, the waves are rough, and there are lots of rocks to discourage beginners into saying, ``Maybe I'll try another day.''

Ryan hesitated in the shorepound, wary at first, but he punched through the waves, lay down on the fiberglass board, and we paddled through lines of foam racing into shore. We angled toward a deep spot where waves weren't breaking. I pulled up, and we worked on how to paddle and sit on your board without tipping off. We'd been out twenty minutes when we drifted into a current; it sucked us north into bigger waves breaking farther outside. Ryan got swamped. I motioned for him to turn toward the beach and catch the white-water, which he did. We rode in together on our stomachs, bouncing along as the wave rushed us into shore.

We hit the beach, and I pointed out the current that'd caught us---a rip tide with choppy water that funneled sideways along the shore and then swept out to sea. We jumped back in the surf and went through the same cycle: paddling through the waves, getting sucked into the rip toward the outside break, then catching a wall of foam into the beach. The whole session probably lasted half an hour.

Ryan took a breather on the beach. I paddled out for another half an hour and caught some of the larger waves. After the session, Ryan wanted me to take his picture in his wetsuit and holding the surfboard. He also wanted to tie the boards to the roof by himself. I'd shown him how to do it that morning. ``I got it, Papa,'' he told me. ``I got it.'' That was asking a lot. If the ties came loose, the boards would fly off and get smashed when they landed, maybe even cause an accident.

I let him tie the boards to the roof and trusted that he'd do it right.

He did. We stopped off at the Tides restaurant on the way home and ate hot clam chowder in sourdough bread bowls. Ryan looked worn out but content, and he wanted to talk about the waves he'd caught.

I had more fun than I expected to. I'd had to remind myself, bouncing next to Ryan in the foam, that the goal of the session was not to catch the best waves for myself but to encourage him. As I say, that wasn't easy for me to do. My old selfish habits---get as many waves as I can before the wind comes up---weren't easy to break.

Ryan joked after lunch, ``I can tell my friends in Missouri that I went out surfing instead of sitting at home.''

And that was my best ride of the day: in the car on the way home listening to my son brag about going surfing.

Ryan rinsed out his wetsuit when we got back, taking care of his gear. We left the boards in the shade to dry under a redwood tree and hung the wetsuits off the back deck. The sun was out, and I had that warm achy feeling in my body that always followed a surf session.

I rubbed Ryan's shoulder. ``I'm proud of you for paddling out today. That's not easy. Cold water, rough waves, the rocks. A lot of kids would've bugged out.''

He smiled. ``Thanks for taking me surfing, Papa.''

I tried to remember the last time I'd thanked my dad. Probably for storing my surf gear in his barn. I didn't recall my dad staying with me after he put me on a surfboard for the first time at Salmon Creek. I'd been a year or two older than Ryan. He'd told me what to do then let me learn it for myself. Old-school style. I couldn't remember if I'd thanked him.

\begin{center}$*$\end{center}

On July 15th I broke my dad's hatchet, the one with his name carved into the handle. I was hacking a small bay laurel from the side of the house. The bushy tree blocked the access panel to the foundation, which needed to be inspected for rot and pests. The inspector had also pointed out the mound of dirt around the tree, which caused rain water to flow toward the house rather than away from it. The whole area would have to be regraded. So the tree needed to come out.

My dad had always stressed the importance of the right tool for the right job. I really needed an axe to chop out the tree, but I'd given it away along with his other tools. So I was stuck with the hatchet. About the tenth swing the metal cracked at the top of the handle, and the blade dropped into the dirt.

I stared at the fucking thing. It was the third tool I'd broken. Three weeks before, I'd overloaded my dad's wheelbarrow with concrete; I lifted it, and the wooden handle snapped off in my hand. The whole damn load spilled over sideways. The wheelbarrow had been old, but I'd been too zealous, trying to haul too much in one load.

A few days after the wheelbarrow debacle, I cracked the handle of his shovel. I was digging out a small bay tree that had germinated from the bigger one blocking the access panel. But I'd found chicken wire wrapped around its roots. That tree wasn't a volunteer: my dad had planted it right next to the house. It was one thing to dig out accidents, quite another to undo bad landscaping.

About that point I'd started to feel a little resentful. Part of the reason why that dirt and those trees were a problem was because my dad hadn't installed rain gutters, so all the water dropped straight off the roof and washed toward the foundation. I'd tossed the broken shovel aside and grabbed my dad's hatchet from the shop to finish the job. It'd taken about a dozen whacks. I'd been impressed by how sharp it was.

Now, two weeks later in those same bay trees, his hatchet was broken. I held the handle in my hand, looking down at his name. I couldn't help but read meaning into the fiasco of me, his tools, and the whole damn project of organizing his life. On the one hand I hadn't been taking enough care with his tools as I'd used them, so I felt like an ass. On the other hand, all of the tools were old and cracked. Almost everything about that house, I was finding out, was old and cracked. I'd always imagined my dad as an expert handyman who could fix anything.

But why should his house be any different than his life?

I was learning that his fixes were all jerry-rigged: he could do a lot of different things, but they were mostly done half-assed. The electrical wiring in the house and barn was a catastrophe. His landscaping was a joke. The right tool for the right job, my ass. Here I was feeling bad because I couldn't live up to the ideal that my dad had promoted but couldn't follow himself.

He'd tried so hard to be seen as the wise man, the elder who passed on his sacred knowledge to others. But he was just a man in the end, his defects laid bare as I chopped into his past. I wish he would've known how to admit his limitations, or ask for help when he didn't know what to do. He knew how to do that after Maya died. I have the video of him asking to come home to Sebastopol so his friends could help him deal with her death. I hear it in his voice, his confidence at their willingness to help him: ``Boy, they'd jump right to it,'' he said. ``Okay, What do you want? What do you need? We will do it. You just tell us what you need.'' Dementia had scraped away his protective layer, allowed him to express his vulnerability. I hadn't known how to respond to him at the time, how to make sense of what he was asking me to do. I'd been so surprised by the whole scene, his intense feelings and emotions---the violence and the tenderness---that I blocked out much of what he'd said. If I had to live those days over again, I would have made more of an effort to bring in his friends and have them comfort him.

\begin{center}$*$\end{center}

``Dad was my idol until I was about twenty-five,'' Mike told me.

We were having lunch in downtown Petaluma. It'd been three weeks since I'd last seen him in Mill Valley. After studying the military records he'd given me, I had a few questions about him and my dad. Mike had joined the Air Force when I was seven, so I didn't know much about their relationship. That was my official reason for meeting. But I also simply wanted to spend time with him. He'd told me that he was generally doing good. He coughed more now and had difficulty breathing sometimes, but he was cruising along.

We ordered our meals and settled in with iced teas. I pressed him a bit on his health. He looked thinner to me, and his voice was scratchy. ``Coughing more?'' I asked.

He admitted that he'd almost called Hospice a few days before. He and his wife Kathy had gone to the beach. He'd returned home ``feeling awful.'' He'd been on Hospice earlier that year, but now they only checked on him once a month. He'd recovered since the beach episode and decided against starting Hospice again.

We got to talking about my dad, and Mike made the comment about him being his idol. ``My upbringing was super strict,'' he told me. ``Dad was all about discipline. I think he got his parenting skills from the Marines. He always had to know what was going on at all times.'' Mike drank his iced tea, his hand shaky. ``He was a tracker, so you had to get really good at being able to hide things.''

I could see that with Mike. He was smart and intense with a sharp sense of humor; but he could also be very interior and self-absorbed, qualities that had probably helped him cope with constant surveillance.

``I didn't have that experience with dad.''

``Maybe after the third kid''---my sister, Kathleen---``dad got tired of tracking everybody's movements. It was too tiring.''

We laughed. Something else I had in common with my dad, then. I was a tracker myself: retracing his steps, cutting sign from his past to find answers. I'd wanted to ask Mike about my dad's supposed PTSD, and if he recalled my dad talking about being an Indian. I wanted to know how far back those stories went, when they'd started, and why. I had this notion, maybe only half crazy, that it'd started in earnest after the movie Billy Jack became popular in 1973. My dad loved that movie: Tom Laughlin as the laconic part-Navajo Green Beret Vietnam Vet who kicked serious hapkido ass in a town run by bullies and bigots. My dad had tried to embrace the model of a wizened Cherokee, but at heart he was really Billy Jack.

Mike didn't recall Indian stories when he was growing up. ``I didn't hear about the PTSD stuff until after he came back from Hawaii.'' This would have been in the mid-80s while my dad was living near Garberville and experimenting with peyote under a shaman.

``Did you get the stories about him being an M.P.?''

Mike nodded.

My dad used to describe arresting Filipinos on the Marine base. One guy had supposedly pulled a knife on him; my dad had kicked it out of his hand, and the blade flew straight up and stuck in the ceiling. Classic Billy Jack. ``But he was a clerk,'' I said, referring to the records Mike had given me.

``Maybe the stories came across his desk,'' Mike suggested. ``And he adopted them for himself.''

That made sense. It'd be consistent with details he'd gotten from his magazines and books about fighting in Korea.

``When we lived in West LA,'' Mike said, ``dad taught me how to body surf and skimboard in Santa Monica. We'd go on hikes together, and camping. But we didn't click that much. Dad was always so silent in nature.''

Our lunches came. A roasted turkey sandwich for me and a salad with chicken and goat cheese for Mike. He'd also ordered a piece of carrot cake. He had a good appetite.

Mike continued, ``I got certified to dive, and we did that together, too. I never really enjoyed it that much.'' He took a bite of salad. ``I wasn't as aggressive as dad. I'd get tired after two dives. He'd go down six or seven times. He'd come up, replenish his air, then go back down.''

We'd eaten a lot of seafood as kids, fish that my dad had speared or abalone he'd pried off the rocks. I'd found several diving certificates from 1970 among his papers. He'd completed a basic course in skin and scuba diving from the National Association of Underwater Instructors in May of that year and won their award for Outstanding Graduate. In December he'd finished their Advanced Scuba Diving course. There's a picture of him with the other twenty-two graduates standing on the beach under an LA County banner and the designation Dive Team 8. My dad, forty years old, stands in the back row in a white tee shirt with a small yellow octopus on it---the team's logo. He's smiling, looking healthy and happy. He looks a lot like my brother Chris in that photo.

Mike's reverence for my dad started to sour after a trip they made to Hawaii in 1976. Mike was twenty-three. He'd gotten out of the Air Force and was thinking about moving to the Islands and bumming around a bit. My dad had offered to go with him, and they could scout the place together. My dad and mom had been to the Islands a number of times on vacation.

``We went out surfing at Waikiki,'' Mike said. ``I was in one area, dad was in another. He'd found a nineteen or twenty-year-old girl and was giving her a surf lesson, showing her how to stand on a board. He waved at me to stay away.''

``You're kidding.''

Mike shook his head. ``He didn't want her to know that I was his son.''

``Wow.'' Given Mike's good looks, my dad was probably afraid of the competition. ``He'd go out at night by himself too, leave me back in the hotel room.''

Not much aloha there. So my dad had been fooling around on my mom---or looking to--- soon after the family had moved from Thousand Oaks to Occidental. Here's what my dad wrote about Mike in 1991. My dad had just left his girlfriend of five years and moved into a group home in Corte Madera, taking Ecstasy and sculpture classes at the College of Marin:

Mike lives a mile away in his mansion and I've not even been invited for a cup of tea since they moved in. I stopped by once but didn't feel very welcome---what the hell's going on there? We used to be so close. We used to do everything together---Motorcycles, diving, hiking---He called me from L.A. once and said he needed me. I went and we really were close---what happened? How did I fail this `beautiful Genius'? Am I less now than I was then? what have I done or not done that makes him so uncomfortable with me? I can talk to Fran [Mike's first wife] for hours and feel like we really connect but with Mike it's all surface. He doesn't seem interested in who I am or what I stand for---why? where did we lose it? I don't get it. My first born, my first pride---someone to pass on what I've learned and am still learning---We could share so much but---He seems to be wrapped up in the narrow world of advertising, golf, family and maybe music---Maybe my perception of his world is all wrong---He may feel judged by me, competitive with me. I just don't know how to reach him---Part of me is afraid of being rejected if I try. I just don't know---but I miss him---I just miss him.

It's easy to see my dad projecting his own issues onto Mike: the accusation of my brother being ``all surface'' and ``competitive'' with my dad. Interesting too that my dad considered family as part of a ``narrow world.'' It's sad to see him struggle with himself and miss opportunities to connect with his kids. But if he'd really been interested in passing on what he'd learned to Mike, he probably shouldn't have pretended that Mike wasn't his son.

Mike and I planned to meet the following week for lunch again. He said he'd put together a timeline of important dates in his life for me. He wanted to help with my book project any way he could.

``I admire you,'' I told him after he picked up the check. ``The decision you made not to undergo anymore treatment. You're the model for the rest of us on how to die.'' I can't remember if I actually said how to die---that's rather blunt---but that was the gist of my comment. The decision had been the right one for Mike. His condition was terminal, and he'd found that his treatments made him feel worse than the cancer itself. Nonintervention had not only extended his life but given him a better quality of life. The fact that he was still walking around, driving to meet me for lunch, was proof. He was thinning and frail, but he didn't look like he'd die in the next two months. My sister Kathleen, the public health nurse, made an interesting observation after I'd told her how good he looked, all things considering.

``Mike is what cancer looks like,'' she said. ``Other cancer patients you see creaking around, that's what chemotherapy and radiation treatment look like.''

I'd already talked it over with the staff at Primrose: when Mike died, I would not tell my dad about it. My dad knew that Mike had cancer, but Mike wasn't a part of his everyday life. I'd follow the same strategy as I had with Maya. My version of Don't Ask, Don't Tell. 


\chapter{}

July was coming to a close quickly. Maya's memorial was a week away. We'd have forty people on the property, so there was more work to do. Linda had spent the better part of four days in the art studio organizing and packing Maya's things for shipment to Southern California. Once the studio was cleared, Ryan and I got in there with paint brushes to give the place a face lift.

I was standing on top of the plywood storage unit that my dad had built for Maya, my paint brush reaching the highest corner of the ceiling, when I thought, Dad was standing right on this spot thirteen or fourteen years ago, doing what I'm doing now. It wasn't like seeing Michelangelo's bristles embedded in the Sistine Chapel, but I felt a connection to the old master. I wasn't undoing his work this time, or trying to fix it. I was embellishing it, adding another layer. I understood that the new owners of my dad's house, whoever they turned out to be, might well decide to demo the whole place.

But I couldn't think like that. I was realizing that painting the walls and landscaping the property was my way of working with my dad. We couldn't do it together any more, but I could finish what he'd started. Or refinish it, in the case of the art studio.

I also thought about my relationship with Ryan, how different my experience with him was than the one I'd had as a young teenager with my dad. I remembered working on a greenhouse that my dad was building when we first moved up to Occidental. Chris, Steve and I were nailing short pieces of wood to build a shelf inside for the plants. The pieces had spaces in between so water could drain down to the gravel floor.

Apparently I'd nailed the first piece of wood at the wrong angle, so the next five or six pieces were all off kilter. What I mostly remembered was my dad yelling at me as I walked away, something to the effect of ``If you can't do it right, I don't need you anyway.'' The word ``hell'' was in there someplace: ``If you can't do it the hell right,'' or ``I don't need you the hell anyway.''

Ryan was using a roller with a long handle, painting the wall opposite me. I tried not to correct him too much. I'd given him a quick how-to, then let him run. Yes, I was tired and sore from all of the physical labor I'd been doing. Yes, there was still a lot more work left to do. Yes, we were under time constraints. Yes, I wanted the job done ``right.'' But I had to remind myself: getting the inside of the art studio painted was not more important than my relationship with my son. And it wasn't just our relationship. What kind of father did I want Ryan to be, if he had kids? What could I do now to help those relationships?

\begin{center}$*$\end{center}

Over the next couple of days Ryan and I took time out from working the property to run errands with my dad. It's one of the things I love about Ryan: he carries the same enthusiasm gene as my mom and my brother Steve: anyplace I wanted to go---the gas station, the grocery store, the Social Security office in Santa Rosa for my dad, he was ready to ride shotgun. I appreciated having an extrovert in my life. Miles, Linda, and I are more subdued---the Hobbits of the family. Ryan has the gusto of the dwarves, only he's tall like Mike.

I'd called the Social Security office the day before to find out the best time to arrive. I didn't want to put too much strain on my dad. The lady on the phone had told me, ``It's the end of the month so we're not that busy. There are only two people waiting in the room right now.''

Perfect.

We walked through the door about the same time as my call the previous day. There must have been sixty or seventy people in a small room with their kids, their dogs, you name it. The whole world was there.

I took a number, and the three of sat down to wait. I told my dad why we were there: that he might be entitled to benefits because of Maya. They needed him present to pull her records. He teared up at the mention of her name. ``It's tough,'' he told me. He dropped his head and started kneading his hands, working them into fists. ``She's gone, and there's nothing I can do about it.''

I put my hand on his back, rubbed his shoulder. ``I know it's hard, dad.'' I didn't know what else to tell him. ``It's going to happen to us all one day.''

Since we were on the subject, I asked him about spreading Maya's ashes at Salmon

Creek. Her wishes were to have them mingled with my dad's, but he could live another decade or more. Who'd keep them in the meantime? Nobody had offered. I'd picked up the ashes at the funeral home the day before. Maya was now our houseguest. Or we were hers. Five pounds of her sat in a thick plastic bag labeled with her name; a white twist tie closed the bag at the neck and held a round metal tag with an ID number. The bag had been placed inside of a square plastic urn. The urn had been placed inside of a cardboard box, which had been placed inside a nice shopping back with string handles.

Maya was definitely well packaged. Since it'd be my job to transfer her ashes into a decorative clay urn for the memorial---she and my dad had picked out it long ago---I'd wanted to check out all of the packaging and see what I was in for. At some point, my dad's ashes were supposed to go in there, too.

Anyway, since nobody had offered to take the ashes, I was considering a back-up plan: having my dad spread them at the beach in a small ceremony. That way he could say goodbye to her.

``I'd like that,'' he told me.

I was tempted to tell him about the upcoming memorial and have him come out to Sebastopol after all. But I bit my tongue. I didn't want him to become a spectacle or get overwhelmed. Plus Linda and I were organizing everything---setting up tables and chairs, getting all the food---so we'd be busy. I didn't think I'd have the time to watch over him and help organize the memorial.

We only ended up waiting twenty minutes at the Social Security office. I learned that Maya had received her monthly check because of her first husband's salary. She hadn't earned much money herself from her art. The man behind the desk had showed me all the zeros in her salary columns, dating back to 1947. So my dad wouldn't get any further benefits.

My dad stood and shook the man's hand: ``Thank you, sir. Thank you very much.''

The guy smiled at me, surprised. My dad had probably made his day. My dad did the same thing to the large security guard at the front door on our way out.

\begin{center}$*$\end{center}

The next morning the three of us were off to San Francisco to have my dad's pacemaker checked. Funny story: I only discovered from reading his journals that his pacemaker needed to be checked at all. It was supposed to be done every six months. I reviewed his records, and it'd been over a year and a half.

Oops.

The VA doctor, an elderly Japanese man who wore sandals, looked sideways at me during the appointment and paused for emphasis: ``Don't do that again.''

I would've felt terrible if my dad's ticker had given out because I didn't know to get his pacemaker checked. It was another reminder of how totally responsible I was for his health.

But his pacemaker was fine. It kicked in if my dad's heart dropped below fifty beats a minute. They could tell from the monitor that it happened about five percent of the time. He'd had the thing for eleven years, originally installed because his heart beat was too slow. It was something to remember for my own health. Overall, though, my dad was in good shape.

We were waiting in another office to get my dad a new VA ID card when we heard the following conversation between two women behind their desks:

``You can't wear this blouse I got on.'' 

``Sure I can.''

``Nah, I hate to burst your bubble, but I got a whole lot more goin' on up here''---she gestured to her chest---``than you do.''

``You love to burst my bubble. That's all you do.''

``Only because I love you.''

``I'd hate to think what you'd do if you didn't love me.''

My dad and I looked at one another and smiled. He leaned toward me: ``That's a whole lot of woman, but she could lose a few pounds.''

He still had an eye for the ladies, dementia and all.

I checked over my shoulder as we left Building 200, making sure my dad was okay. ``You can hold onto my arm if you like,'' I said.

He grabbed my hand, and we walked down the sidewalk together. I felt really self- conscious holding his hand like that, even in San Francisco. It'd only been a year since Ryan had stopped holding my hand on a regular basis, though he still did it from time to time. I always thought that was really sweet. But as he'd gotten older, I felt more self-conscious about it, especially since he'd grown taller than me. I'd pulled my hand away from his several times in the past year or two because of what people might think of us. Ryan was such a loving kid, so I'd felt bad about doing that. I gripped my dad's hand and fought through my sense of embarrassment, which I knew was ridiculous.

I admired how unself-conscious my dad was in his dementia: in the bathroom, getting dressed, or being helped with his seat belt. This last job had become Ryan's every time we all got into the car: buckling Carver in. The first time we'd gone someplace together, my dad, sitting in the backseat, had reached over to the other passenger seat and buckled the seat belt. Then he sat back and said, ``Okay, all set.'' He'd looked straight ahead, ready to roll.

I'd laughed and motioned Ryan to the back seat so he could buckle my dad in. Ryan had gotten out and opened my dad's door. He'd showed my dad how to pull the belt around his neck and shoulders, then click it into the buckle. He was very patient with my dad.

We stopped at a flower stall off the sidewalk to buy a bouquet for the ladies at Primrose. It was about lunchtime, so we decided to eat at the nearby Cliff House on the north end of Ocean Beach. The restaurant was iconic. It had been there in various forms since the 1860s. We waited a few minutes---the place was busy with visitors---then got a window table at the south end which offered a beautiful view of the Pacific Ocean. A few surfers were out because a south swell had marched in, but the wind and fog made the waves unruly.

My dad and Ryan sat together, looking out at the surf. My dad had said he wasn't hungry, but I got him a cup of clam chowder anyway. Ryan and I ordered cheese burgers with garlic fries. Once the waiter left, my dad said: ``I'm ready to go, how about you guys?''

``We'll wait for our food, dad. Then we can head out.''

``Okay.''

I pointed out pictures of various singers and movie stars hanging on the walls. My dad said he recognized some of them.

``What's that?'' my dad asked, pointing to my plate when our lunches came.

``It's a pickle.''

He nodded. ``I've heard of them, but I've never tried one.''

``You should,'' I said. ``They're sour.''

My dad seemed happy with his lunch and the lively atmosphere. I kept my eye on the surf, wondering what spots in Sonoma County might be breaking. After we finished and were waiting for dessert---cups of ice cream---Ryan started drumming his fingers on his cardboard to- go box (the remains of his burger and fries). He got louder, and I shushed him. My dad started pressing his fingers against the table top like he was playing a piano. ``Can't hear that,'' he said.

Ryan started up again on his cardboard. My dad rapped his ring finger against the table. ``That makes noise,'' he said, cocking an ear. He did it twice more. ``But this is silent''---he played the edge of the table again with his fingers. I looked from my dad to Ryan, both of them hammering away now. They were like drunks at a party who thought they were whispering. They looked at me, glanced at one another, then burst out laughing.

I felt outnumbered.

Outside I snapped a picture of them with the Pacific in the background: my dad dressed in jeans and a jean jacket along with his Marine Corps cap. He and Ryan have their arms draped around each other.

We got back to Primrose, and I handed my dad the flowers so he could give them to Yolanda and Graciela. Laurie told me that my dad had been doing better the past couple of weeks. When he got upset and started talking about Maya, they'd encourage him to speak and get his feelings out. They would also compliment him: You're doing great!

He'd stop and say, ``I am?'' Then he'd calm down.

I put him in his bed. As I leaned down to give him a hug, he kissed me on the cheek. I couldn't remember the last time he'd done that. I did remember the last time I'd kissed him. We were living in Thousand Oaks. I must have been six or seven years old. It was bedtime. He and my mom were sitting in the living room watching television. My mom had probably told me to give him a kiss goodnight. It must've been a normal thing for me to do. But that night, for some reason, I felt the roughness of his unshaven jaw against my own smooth cheek, and I hadn't liked it. It was something that I'd had to get over in my early twenties when I was living in France with Linda. We'd visit her family in Marseille, her uncle and male cousins, and everybody kissed one another on the cheek in the French style.

But that moment at Primrose when my dad kissed me, I beamed back to my boyhood. I felt again my dad's rough jaw, enhanced this time by the short hairs of his mustache and goatee. I was surprised and touched.

\begin{center}$*$\end{center}

Maya's memorial, three days later, went off almost perfectly. Linda and I woke early to clean the house and set up a space outside under a larger redwood to hold the ceremony. I transferred Maya's ashes into the clay urn for Maya's daughter, Malka. She had agreed to take them so we could someday bury Maya and Carver together. By the time people started arriving in the early afternoon we had the food laid out in the living room, crowned by chocolate beet cake---my dad and Maya's favorite dessert from their favorite cafe in town. Maya's son Jon and his wife Nan had flown in from Maine; Malka and her husband Danny had driven up from Los Angeles. Two of Maya's grandsons were also there. I learned that one of them had joined the Marines after speaking with my dad. I hoped that was going well for him.

I had a funny reaction to the other grandson, David. He lived in Los Angeles and looked to be in his mid-twenties. He was friendly and told me that he used to come up to the property when he was a kid. His parents would send him to nearby Camp Cazadero during the summer (I'd gone there myself in elementary school) and then he'd spend a week with Maya and Carver. Or sometimes he'd come the week before Cazadero.

I thought, You used to get invited here? I was surprised. And a little jealous, I think, at the thought of my dad having this relationship with another teenaged boy when he'd never had much of one with me. David mentioned that my dad would put him to work clearing brush. That sounded right. Not glamorous work by any means, but I had a momentary pang as he talked about being up there. Resentment has deep roots.

Maya's friends showed from far and near: a woman who'd studied art with her at Stanford decades before, and Wicca from Maya's coven. Some of the neighbors came, too. Several of my family made the trip to pay their last respects. We celebrated Maya as she had wished: forty of us sat in a circle under the redwoods and told stories. Danny read Hebrew scripture. The reminiscing and celebrating continued through the afternoon and into night. By the time everyone left about 9:30 p.m., Linda and I were exhausted.

We'd been glad to see Maya's family and friends again, and to say goodbye to a woman who'd done so much for my dad. I felt like the group had done everything Maya had requested in her last wishes. Except, of course, for one thing.

My dad wasn't there.

I doubt Maya could have imagined his absence at her memorial, unless he'd died before her. ``My husband and dearest love,'' she had called him, he who knew best her wishes and desires. Since my dad could no longer be her final arbiter, as she had wished, that role fell to me. And my principal concern was for his well-being. Maya had included this caveat in her last wishes: ``I trust that any changes made will still comply with the spirit of it.''

Part of me knows I did the right thing not inviting my dad to the memorial. It had been a long and stressful day. He would have been overwhelmed by the strong emotions circulating through the group---the tears, the laughter, the rage of one man who held the group hostage for five or ten minutes (though it felt much longer) as he berated himself for not spending more time with Maya. My dad would have needed to be medicated before and after the ceremony.

But part of me regretted not including him. He had not been by her side when she'd died in Forestville. By not inviting him to the ceremony, I deprived him of the opportunity to say goodbye to her in a formal way, among family and friends. He did not have that cathartic experience in community, or the opportunity to have one, that might have eased his mind. Dementia did not prevent him from remembering or mourning Maya. He'd later say in moments of great duress, ``I never got a chance to say goodbye to her. I was supposed to take care of her.'' His memories of being Maya's caregiver, and the responsibility for her well-being, stayed with him until the end of his life. 


\chapter{}

I didn't see my dad for nearly a month after Maya's memorial. This was partly due to travel. Linda and Ryan were due to leave for Missouri on August 8th, so we all vacationed in Oregon the first week of August. Ryan stayed with Terry and Jerry in Roseburg. Linda and I visited Ashland, where we'd honeymooned, for an early anniversary celebration. We'd be apart on the actual date at the end of the month.

After I returned to Sebastopol and said goodbye to Linda and Ryan, I got caught up in the final preparations to sell the house: the landscaping, the staging, the open houses, the constant cleaning when agents dropped by with clients. That took a lot more time than I thought it would.

Part of me, though, was too frustrated to visit my dad. Every time I thought the place was finally coming together, another jack-in-the-box sprung out at me. The latest clown was dry rot along the front deck that had spread into the exterior paneling, under the house, and into the corner of the living room. All of it had to ripped out and replaced. My handyman had discovered the problem right before our trip to Oregon.

The culprit? No rain gutters. Water had washed directly down from the roof over the years and splashed toward the house, causing the rot.

The handyman and his partner were still fixing the problem when I returned from Oregon. I have to admit I got satisfaction listening to the two men bitch about how the planks had been toe-nailed rather than screwed, how tar paper hadn't been put behind the siding to protect it against water, and how parts of the deck had been repaired but not ``fixed.'' It was a kind of therapy for me, spreading the pain of my dad's half-ass jerry-rigging.

The handyman conversation recalled another one I'd heard two weeks before between a father-son electrician team. I'd called them out to install GFCI breakers in the bathrooms and kitchen to prevent electrical shock. Those guys had no idea what they were in for. The son Alex, maybe in his late twenties, had taken off the panel around the garbage disposal and was digging through the wiring. One of the switches wasn't working. ``We don't have a hot running,'' he mumbled to himself, thinking through the problem. He stopped and stared: ``Oh, man. This is terrible.'' He called out: ``Oh my god, Jorge, come see this.'' His dad's name was George, but the son preferred the Spanish version.

George, about my age, arrived with his electrician's belt. The son made way for him and pointed to the panel: ``This is old school. It's soldered together. Have you ever seen this? That's the first time I've seen solder work inside.''

I didn't quite know what they were talking about, but I understood. The son's adventures weren't quite over. Later he had to crawl under the house to look for a neutral to ground the light in the laundry room. ``There's plenty of neutrals to pull from,'' he called out, his voice muffled by the floor. ``As long as I can get around this dead skunk.'' There was a pause. ``Or maybe it's a cat.''

My dad used to talk about the ``Moser Press Fit'' when we were young. It was his mantra when he was building stuff. He'd bang boards into place that were too big, using brute strength to cover miscuts. He'd make it work, one way or the other. At least he'd make it work for him. My problem was that I had to make it work for other people.

The home repairs were a cake walk compared to the legal mess I returned to. In mid- August, just as all of the repairs were being completed, two weeks before the Open House, I found out that my dad's lawyer---the one who'd originally set up his and Maya's trust---hadn't followed protocol (according to my attorney). There was a lot of legal mumbo-jumbo involved that I didn't quite understand, but basically my dad had planted a landmine in his famous green notebook, and I'd just stepped on it.

Let me try to explain. It turned out that, because the changes my dad's lawyer had made back in 2012 weren't done properly, I was not in fact the official Trustee. My brother Mike was. Recall that my dad had appointed Mike as Trustee without asking him, and I'd changed that after I agreed to be the executor. So here was the problem: everything I'd done for my dad over the past few years---financial, legal, medical---could be questioned or challenged because I wasn't the legal Trustee. My attorney told me that he wasn't comfortable accepting money from the trust anymore (because I wasn't authorized to sign checks), so I'd have to start paying him myself. Worst of all the sale of the house was in jeopardy: the real estate company could sue because I was the one who'd signed the contract.

I wanted to throttle my dad. I spent the entire evening at Kathleen and Cynthia's venting like Mount Etna. I couldn't stop spewing.

``What makes you so stressed about it?'' Kathleen finally asked. To her mind this was my chance to hand the problem over to the attorneys---mine and the one representing Jon and Malka ---and let them handle it. Wash my hands of the whole mess.

``I don't know,'' I said. ``It's just that every day I get another example of his half-ass work. And now the whole trust is that way.'' I let out a breath. ``It's so disappointing and frustrating.''

I didn't want to be pissed at him. I wanted to believe the best in him, especially in his current condition. But I kept running into the old dad: the man who'd preached Measure twice, cut once, and then went off and slapped shit together any which way he could. I'd spent the whole summer working to arrange his life, and at almost every turn I tripped across his short cuts. I was getting tired of picking myself off the ground.

``This is who he is,'' Kathleen said.

And there was money involved, which made things really tricky. In the original trust--- the one he and Maya had arranged---they'd left significant chunks of money to Jon, Malka, and Mike. At some point my dad had crossed those paragraphs out and decided they didn't need to include those gifts. The new version of the trust, the one I'd helped him arrange with his lawyer in 2013, omitted those distributions.

You can see how all of this might look: I help him redo the trust and cut out money to my oldest brother and Maya's son and daughter.

I was lucky in one respect: my siblings and I never expected to inherit money from my dad because he never had any. It was Maya's money that provided the down payment for their house back in 1998. My dad had loved Maya, but her income must have been attractive to him as well. My dad knew he was an attractive commodity, and he'd courted women time and again who could support him.

``He was a duper,'' I told Kathleen and Cynthia over dinner. ``A super-duper, and I've been duped.''

\begin{center}$*$\end{center}

On Monday, August 24th I drove down to see Mike. I'd emailed him the day before so that we could get together for lunch. His wife Kathy responded, telling me that Mike wasn't driving or leaving the house anymore, but I could come to Mill Valley if I wanted.

Mike had declined considerably in the past week. He was on strong doses of morphine every eight hours punctuated by smaller doses for immediate pain in his lower back and legs. He'd also been hallucinating. The Hospice nurse said these were indications that the cancer had reached his bones and brain. End-of-life signs. Mike told us he'd been talking to people in his dreams. The nurse asked him if he knew the people. Apparently speaking to loved-ones who have died is a sign that you're close to death yourself.

A hospital bed arrived for Mike while I was there, along with a portable potty, a walker, and a wheelchair. His bedroom was downstairs, so we helped him move upstairs to the first floor where the new bed had been placed just off the kitchen. The bed faced the television and large windows that overlooked beautiful wooded hills. Family had started to come in. It was the beginning of the end for my brother.

Mike was a bit drugged out but aware of everything that was happening. He kept his sense of humor, which was helpful because, amid the chaos and trauma of preparing his death bed, I had to ask him to sign a form whereby he appointed me as his Co-Trustee. I'd decided to stay the course and throw myself on the mercy of my siblings and step-siblings. Luckily they all knew my dad, so they understood that his scams were not mine. I suppose it was that idea that lay at the root of my anger with my dad, what I'd tried to explain to Kathleen and Cynthia: I didn't want everybody to think that I was trying to pull a fast one on them with the money. It basically came down to a matter of trust (no pun intended): if they trusted me and approved of everything I'd done to date, then we could work out a reasonable solution. The beginning of that solution was having Mike sign over responsibility of the Trust to me, thereby making me and all the work I'd done official.

It seemed to work. I knew my siblings trusted me. I hoped Jon and Malka did. It was the only way I knew to keep the house sale on track and maintain care for my dad.

The next night I went over to Kathleen and Cynthia's to have dinner with them and Terry, who'd come down to see Mike. I told them how Mike was doing, and all the Hospice details. ``It's a gift to work with the dying,'' Cynthia reminded me. We were sitting around the table, enjoying each other's company and talking about Mike. ``A gift?'' I asked.

``Yeah, you discover how deep your compassion goes. They draw it out of you oftentimes. Death pushes that out of you, makes you dig deep for those emotions and discover them in yourself.''

I'd never considered the gift that dying people give to us. It made me think of Maya and my dad. Cynthia was right. My time and effort to organize his life and ensure his care---these were forms of love and compassion for him that I was discovering in myself. His gift to me.

Inadvertent no doubt.

\begin{center}$*$\end{center}

On September first I drove my dad out to Doran beach, a small state park that forms the southern edge of Bodega Harbor. The beach lies in the shadow of Bodega Head, so the area is normally protected from the wind and waves. It's the perfect place for a quiet stroll along the sand. The realtor was holding our first Open House that day, so I decided to spend the afternoon with my dad.

I'd arrived at Primrose in the morning with five new shirts and a pair of shoes that Linda had ordered for my dad and shipped out to me. I laid the clothes on his bed so he could try everything on before we marked his name in them.

``Did you hear about Maya?'' he asked as I helped him on with the first shirt.

``Yes, I did.''

``She died.''

``You did everything you could for her, dad. She loved you very much, and you took care of her as best you could.'' I stood back and asked him how the shirt felt. He looked himself over and gave me a thumbs up. We took that one off, and I reached for the next one.

He brought up Maya again on the drive out to Doran. He said he wanted to go home for a few days to tell all their friends because he was her husband and that was his job.

``Jon and Malka told their whole family,'' I said to him. ``And I told our family and your friends. We did that for you. You don't have to worry about it.''

That seemed to ease his mind. At least he didn't bring the subject up again. It's a forty- five minute drive to the beach from Santa Rosa, and he got on one of his memory loops. He told me again about the war and how they'd told him to kill people with a bow and arrow. ``And once I shot,'' he said, ``I'd get down in the dirt so they couldn't find me. Right down in the dirt. Maybe I'd poke my head up to look around, but they never found me. They never did.''

``That's because you were so good with the bow and arrow.''

At this point in his life my dad must have believed that his imaginings had really happened to him, an odd blend of his Indian and Korean fictions. I never had the courage to confront him about his stories when he was in a condition to answer for them. I doubt he would have admitted the truth to me, and a conversation like that would have taken too much effort. Honestly, I hadn't thought he was worth it.

Now it was too late to challenge anything he said. All I could do was reinforce his sense of himself and try to let go of the past for both our sakes. Our time together had become little excursions into a new world of what it meant to be courageous. The word comes from the Latin cor, meaning heart.

When we arrived at Doran I was surprised to see knee-high waves peeling along shore. An early winter swell had rolled in, hitting Bodega Head from the north and wrapping around the point of land. I made a mental note to drive up to Salmon Creek after our walk and check the surf. Maybe I'd get in a surf session the next morning.

We walked for twenty minutes down the beach, then twenty minutes back, all the while fighting an offshore breeze. I'd given my sunglasses to my dad so sand didn't blow into his eyes. He was worn out by the time we got back to the parking lot. He'd actually stumbled and fell to one knee as we walked the last few steps. The sand was thick and soft right there, so no harm done.

We should have gone back to Primrose---the fall was a sign he was overly tired---but I wanted to check out Salmon Creek. I'd also told the staff at Primrose that I'd get lunch for him. I didn't want to drop him off hungry.

We drove ten minutes up to Salmon Creek where I watched a couple of surfers struggle through stormy waves. The swell was hitting the west-facing beach more directly, so the waves were at least five times bigger than at Doran.

We cruised back into town and stopped at a salt water taffy store to pick up a bag for the receptionist at Primrose, who had offered to write my dad's name in his new clothes so we could start for the beach. When I placed the bag on the counter, my dad asked me what I was buying. The place had been there for over fifty years, and they basically only sold one thing. He must have passed the shop a thousand times.

``It's salt water taffy.''

``I heard that,'' he said. ``I've never tried it myself.''

It was like being in a play by Beckett. My dad said the same lines over and over, each time delivered with the same sincerity. The tragicomedy of dementia.

Afterward we went to his favorite hot dog place. He'd taken me there several times over the years after surf sessions, so it had good memories for me. It'd also been one of the three restaurants he'd driven to after Maya had gone into care, so I was hoping it would have positive associations for him, too. He'd told me he wasn't hungry, but I pushed him anyway. I had it in mind that I had to get him fed. Besides, sometimes he said he wasn't hungry and then ate everything on his plate.

But this time he was telling the truth. The place was crowded with families and their kids. It was a hole-in-the-wall eatery, so a dozen people felt like a riot. I sat my dad on a chair facing the parking lot---the last available seat---and waited in line. He jumped when a baby started crying. He jumped again when someone slammed a car door outside. He startled every time a cook yelled out an order number. I saw he was tired and agitated. I thought food would settle him down.

``What am I supposed to do with this?'' he said when I put the hot dog in front of him.

``Try a bite.''

``I'm not hungry.''

I returned to the counter for our drinks. I put a root beer in front of him. He drank some of that, which helped.

I was hungry and started in on my hot dog. ``Try a bite,'' I said again. ``They're good.'' ``It's too damn hot!'' he said, glaring at me.

Okay. Cursing was my alarm bell. I imagined the kid next to my dad bumping into his leg, and my dad cracking him one.

I grabbed his hot dog, handed him his root beer, and got the hell out of there. I'd been stupid to push him. As soon as I put him in the car, away from the noise, he relaxed. I popped in a Credence Clearwater Revival CD, and he started bobbing his head to the music.

Disaster averted.

Back on Highway 1 and headed east, he started slapping his knee and crooning the chorus ---Let the Midnight Special, shine her light on me. I'd never been more grateful to John Fogerty and his band. They carried us all the way to Primrose.

``It's been a great day,'' he told me once I got him back to his room. I cued him to take off his shoes, his glasses, and his Marine Corps cap. I pulled down the covers, and he climbed into bed for a nap. I tucked the blankets around his neck.

``Do your teeth feel okay, dad?'' It had occurred to me that maybe he hadn't eaten his hot dog because his teeth were sensitive. I was due to take him for another appointment in San Francisco later in the month.

``They're fine.''

``Good.'' I leaned down and hugged him. ``I''m glad we got out to the beach together.'' ``Me too. I haven't been there for awhile.''

He hadn't asked me to go to the beach, but he enjoyed it when I took him there. It was a good lesson for me. I got up, turned out his light, and closed the door.

I didn't know if I'd see him before his next dental appointment, especially with the house stuff going on and Mike starting to slip away. When I'd visited Mike the week before I could tell he was dying in earnest, now. I just didn't know how long that would take. Linda and the kids wanted to fly out, but I didn't know what to tell them. Mike could die in a few days, in a week, or in several weeks. It was a weird sensation waiting for someone to die. I felt on edge, on hold, weary yet agitated. I was distracted, glum, moving around the world but not entirely there. I wondered if that was how dementia felt.

I pulled out of the Primrose parking lot thinking of the easy relationship I had with my dad. He didn't expect me. He never asked me where I'd been, why I didn't visit, or when I'd come next. He seemed to enjoy our time together, our funny conversations. I couldn't say that I always looked forward to seeing him, but I was glad every time I did. Usually it was something small and unexpected that made me feel close to him: an odd memory drummed up, a ridiculous story that made me laugh, or him saying something as simple as It's been a great day.

\begin{center}$*$\end{center}

I visited Mike during the next Open House. It was Sunday, September 6th. I'd been living like a monk since Linda and Ryan had left the month before. The morning they'd headed out, I'd hugged them goodbye then headed straight to the beach and surfed for three hours, sharing the lineup with a pod of dolphins. When I got home I cleared the refrigerator, the freezer, and the cupboards of all nonessentials. It was the final stage of my great purge, which included dropping into a military-themed barbershop and having the owner buzz my head. I didn't know why I felt the need to scrape down, but I did look a lot like a cancer patient when she was done. Maya had died. My dad's house was stripped to the bones. My brother was dying. I myself was at the stage of midlife where Age was closing fast on my rear (so to speak). My arches had broken. My hair was thinning fast. My knees creaked like old stairs. I needed glasses to find my glasses. This was probably my last attempt to outrace the culprit by heaving extra cargo off the decks and throwing my sails to the wind. I had a lot of my dad in me, and By God I wouldn't go down without a fight. Especially when I looked down the road and saw him shuffling along. Here is my dad in his autobiography outlining his midlife crisis:

The Separation Marriage falling apart, I was falling apart---Needed something---didn't know how to define it---Sexual revolution---in full swing. I wanted to join in the dance---Anne \& I were becoming strangers.

The Quest---I needed to define myself other than father, husband---etc I was sick of ``The Waltons'' image. There was a longing in me that nothing I could do seem to fill. I was in a spiritual vacuum because I had left the Church \& had nothing else that I could turn to for sustenance. I decided I had to be on my own for awhile \& try to work this out---Midlife crisis, probably part of it---The dreams of Korea kept coming back \& I couldn't or wouldn't talk about it. I mean who the hell was I? What was my heritage? I had to find out for sure or go nuts---I left the marriage \& went a littl wild for awhile. Sex drugs, Rock \& roll---I did it all.

My parents separated officially in 1980, when my dad was fifty, so he'd been about my age (I was fifty-two in the summer of 2015). As different as our situations were, I did wonder if my feelings of needing to pare down were related to his Quest to be on his own and try to work his feelings out. I'd been living alone for the past month, in his house, and loving the freedom of being able to drive to the coast and surf when I wanted, living almost completely on my own time.

``Why?'' Ryan had asked after I'd sent Linda a selfie of my new haircut.

I joked to Linda, ``Tell him I fell under the mower.''

But I kept thinking about his question. I'd needed a break from my job and my family. I'd come to California to work and take care of the house sale, but I'd also been looking forward to simply living in California again, being near the coast, and getting a taste once more of the surfer life. I'd left home at seventeen and moved to Manhattan Beach in Los Angeles to become an auto mechanic and to surf. The mechanic part had quickly slipped away (picture Homer Simpson with a wrench) but the surfing became formative in my life, and that identity has stayed with me until this day. I had needed a car to get around after Linda and Ryan left, and I'd had a surf wagon in mind: a scaled-down SUV with enough room for surfboards and for me to sleep in the back if necessary. I bought a used Ford Escape from a neighbor's driveway.

I enjoyed outfitting the car. I must have thought this was a little silly because I'd waited until after Linda and Ryan had left to do it (along with getting my haircut). I bought a red Mexican blanket at the surf shop in Bodega, the kind I used to buy in Tijuana or Rosarito Beach for five dollars during surf trips into Baja back in the early '80s. I outfitted my new used car with an emergency box: duct tape, a flashlight, gloves, bungee cords for strapping the boards to the car racks if necessary. My dad had owned a small camper truck in the early `80s that he outfitted for comfort. This was right before he split with his second wife and moved to Crescent City to learn how to carve with a chainsaw. The similarities swirled in my mind. I tried to find connections between us and what he might have been feeling when he left my mom. He would've been several years younger than I was, an odd time warp to consider. I related to his desire to be on his own time in the most mundane ways: to go and come as I pleased, to leave dishes in the sink and not feel like I had to wash them, to leave my clothes on the floor. In a nutshell, I didn't have to think about doing things for other people since there was no one else around. I felt like I'd been doing things for other people for so many days on end---my dad and the whole trust in particular---that I just wanted time to do things for myself.

And it felt good to be a slob.

I had the advantage of peeking into my dad's journals and comparing the interior man with the exterior one that formed my early ideas of what it meant to be a man. My challenge was to balance the two and come to terms with them. I tend to think every man has much of the boy in him, and we compare ourselves in unfavorable ways to our fathers, or our father figures, whom we imagine to be all of the things we are supposed to be: confident, decisive, in control. I was coming to terms through my dad's journals of his great insecurities and colossal posturing, which he knew himself to be a fraud and yet he could not cease and desist. His own ideas of what it meant to be a man were too strong for him to ignore. I'm not quite sure where I'm going with all of this except to say that, during the time I lived on his property by myself, I was trying to understand if my feelings of needing to be on my own related in any way to the quests that I had long resented him for.

I spent eight hours at Mike's on that first Sunday of September. Terry and Jerry were there along with my youngest sister Cindy, Mike's wife Kathy and her two daughters, Whitney and Lindsey. Several of Mike's neighbors dropped by. Mike was the salon king: reclined on a hospital bed in his boxer shorts with the rest of us sitting around him. We watched football and golf. We ate lunch pot-luck style. We left the food on the kitchen counters and ate it again for dinner. We chatted about nothing. Mike was in a good mood. He'd been able to manage his pain better over the past two weeks with his baggy of Hospice drugs; they made it much easier for him to get in and out of bed if he needed to stretch his muscles or go to the bathroom.

As always, his sense of humor stayed sharp. He asked me how dad was doing, and I filled him in about the trip to the beach, the hot dog drama, and how I'd calmed him down with Credence. Mike recalled that dad had always been so hard to play the guitar with because his musicality ranged widely---shifting time, changing rhythm. Mike was an accomplished guitar player, but the issue went deeper for him than my dad's lack of formal training. Mike said, ``He never seemed that interested in adjusting his playing to someone else.''

``Remember his bottom lip?'' I said. My dad had this tic: he'd flare out his bottom lip whenever he played music, furling and unfurling it like a musical pout.

Mike imitated him, and we laughed. After all these years, with Mike on his last legs, we were still heckling my dad, using him to bond.

My laughter turned to tears on the drive back to Sebastopol, the emotion erupting suddenly. I was going to miss Mike. I wasn't ever sure when I left him if I'd see him again. I'd given him a hug and told him, ``Love you, man.'' I appreciated having those hours with him. I'd been nervous before arriving, wondering about his state of mind and body, what I'd say to him, how I'd act. I had no practice being around a brother who was dying. But of course as soon as I'd seen him on the bed and hugged him, my anxiety disappeared. He was still my same brother, which is a funny thing to say. He'd probably been anxious, too. He had no practice either on how to be around family as he died.

\begin{center}$*$\end{center}

There's a surf spot north of Salmon Creek that breaks best on a south swell. It sits in a small cove surrounded by cliffs. Rocks stretch into the lineup on which the waves break, and the shore is strewn with driftwood. It's a classic Northern California surf spot: remote access, fickle moods, but when the swell, tide, and wind align, you feel like the world has given you a gift. I'd surfed the place only once, years before, but I kept my eye on it that summer. The day after I visited Mike, the place came alive. I parked my Escape in the dirt lot and hoofed it down the trail. The four cars already in the lot had peaked my interest. They were clearly surf wagons. But I wouldn't know how good the surf was, if it was good at all, until I caught sight of the waves.

It was a beautiful September day, warm with a light wind. I crossed a narrow spring and hopped onto a ledge that overlooked the break. A head-high swell was rolling off the rocks on the north end and peeling across the cove. Maybe half a dozen guys were out.

I surfed for three hours, a marathon for me. The waves were still marching in when I beached myself, utterly exhausted. I'd wiped out on my last several waves, too tired to stroke into them with any strength. But the conditions were so good, and it'd been so long since I'd surfed waves of that quality, that I forced myself to paddle back out. I was like an old boxer who didn't know any better than to keep grabbing the ropes after getting knocked on his ass.

I finally threw in the towel. I changed out of my wetsuit, grabbed a lunch I'd packed for myself, and walked back down to the cove. I sat on a big rock and watched the waves peel cleanly as I thawed in the sun. It was a rare bird for me. I wanted to see it take flight again and again, its wings pounding toward shore with a low whoosh that echoed off the cliffs. I felt a marvelous release in the warm air and remote setting, as pure an appreciation of life that I could get at that moment. I returned over the next two mornings, milking every wave until, on the third day, I was the last one left in the cove. I sat on my board looking to the horizon, waiting for knee-high rollers at half-hour intervals. I rode them perilously close to the rocks near shore as the tide dropped, wringing every last bit of life from a distant, dying storm.

\begin{center}$*$\end{center}

Another Sunday, another Open House. I dropped by Primrose on my way to pick up

Chinese food for lunch with Kathleen and Cynthia. My dad was in the Activity Room with about sixteen others. He waved at me when I walked in. They were all sitting around the edge of the room in chairs. I took a seat by the door. A young Hispanic woman named Sylvia was reading out trivia questions and fun facts for the group.

My dad looked good. He was leaning forward and paying attention to Sylvia's questions. That was more than most of the residents. Some of them slept. Others stared blankly from their wheelchairs. Only one or two had the ability to answer correctly. My dad used to be one of them, when he'd first started at Primrose.

Honestly, I didn't know many of the answers myself. The whole activity was a bit of a circus. Sylvia, with her strong Spanish accent, had difficulty pronouncing some of the words. I looked at my dad a few times and smiled when Sylvia mispronounced the name of an old movie star. My dad was aware enough of the situation to smile back at me. Or else my smile had cued him to respond in kind. The game completely broke down when Sylvia hit the word ``Whatchamacallit.'' She was good-humored about the situation. She ended the session with this card: ``While driving, we always want the person behind us to be patient, but not the person in front of us.''

A handful of the residents laughed. They liked that one.

Everyone filed into the lunchroom. I stood to hug my dad as he passed, but he didn't stop. He was in his routine, following the people in front of him. I sat at his table with two other men. I introduced myself to them. After that my dad started introducing me to the lunch workers as ``his son, Patrick.'' It was a name he never called me. Usually it was just Pat.

They served pasta with meat sauce and mixed vegetables. Everyone drank red punch. For dessert they ate apple crumble with whip cream. One gentleman at the table was very polite: he'd introduced himself to me and shaken my hand. The other man punctuated our conversation with random questions and non sequiturs. I was surprised at how well my dad accepted his behavior. In the past he would've gotten annoyed and made a smart-ass comment. But he'd been at Primrose almost nine months now, long enough to take the ramblings in stride. My dad even responded to the man's questions, as if anyone at the table was looking for answers.

My dad shook my hand when I left after lunch. I wondered if he'd picked up on the handshake that his table-mate had given me. It was the kind of thing my dad might have done to a stranger.

\begin{center}$*$\end{center}

I was fighting traffic on Interstate 5 in downtown LA when Mike slipped into a coma. I didn't get the email until the next morning at my brother Steve's house in Oceanside. That was Saturday, September 19th. I'd jumped down to SoCal for a surf conference at San Diego State and would be back in Sebastopol after the weekend. I'd felt funny about leaving town with Mike close to death. I'd sent his wife Kathy an email, asking if she needed anything or wanted me to come over. I would've skipped the conference had she asked. But she had everything under control (as much as that was possible) with family and friends close by. I kept Mike in my thoughts that weekend and my dad too: what I would or would not tell him.

I was on my way back to Sebastopol on Monday, staying the night at a colleague's house in Santa Barbara, when my brother Chris called to say that Mike had died. I missed his call the first time---I'd been out to dinner---but I called him as soon as I checked my messages. We talked quietly for a few minutes about the details: when he'd died, how he'd died, who was there ---all the information that helped us process the reality of losing our oldest brother.

The news hit me harder than I thought it would. We'd been expecting this moment for a long time, but I suppose you can never prepare for how you'll feel. Since it was late, I didn't call my other siblings. I brought up the subject the next morning at breakfast with my hosts. I didn't know them well. Tim was a fellow academic and surfer who'd invited me to stay if I was ever passing through. I'd just met his wife Ruth the afternoon before. I didn't want to drop an emotional bomb on them over yogurt and granola, but it felt odd not to bring it up. They were kind and supportive, listening about someone they didn't know from someone they barely knew.

I had a quiet drive headed north on 101, remembering the numerous times I'd made that trip in my life---ever since I was a kid---and also thinking about Mike. It hit me that I'd never see him again. I got choked up thinking about how much I'd miss him. I thought about the last time I'd seen him at his house, that long pleasant Sunday sitting around and talking. Several of us were leaving at the same time, so our last goodbyes had been a bit hectic. Typical fare for a large family. The only difference was that we usually expected to see one another again. I was at the front door putting my shoes on, grabbing the cooler I'd brought down, when Mike had called me back to his bedside. ``Our conversation got interrupted,'' he'd said. He wanted to make sure I didn't have anything else I'd wanted to tell him. He was thinking about me, making sure I had everything I needed.

``No,'' I'd said. ``Just goodbye.''

We smiled, and I leaned down to hug him again, told him that I loved him. I remembered the way he smelled: the sharp sting of his sweat in my nostrils as I pulled him close, his bare shoulder clammy from the heat.

\begin{center}$*$\end{center}

Two days after Mike died I took my dad to the San Francisco VA to have one of his teeth capped. I'd been talking with my siblings, on the phone or over dinner, just to connect with them and share memories of Mike. We all wanted to know the same thing, like my conversation with Chris: When did you find out? Who told you? Who had visited Mike last? Who was flying in? When are they coming? What will the memorial be like? We filled in each other's gaps of knowledge, trying to get as close as possible to Mike's last moments. It was comforting for us to be together. There's a scene in Ernest Hemingway's The Sun Also Rises when Jake Barnes is eating dinner with two friends the last night of the Pamplona fiesta. Jake says, ``The three of us sat at the table, and it seemed as though about six people were missing.'' That was how it felt after Mike died, even when all seven of us siblings were together.

My dad had remembered my name when I'd picked him up at Primrose, and he'd buckled himself in the car all by himself. We listened to Credence Clearwater Revival on the way to San Francisco. He sang the chorus and slapped his knee. He mentioned Maya briefly, but then moved on to other topics. On our way back across the Golden Gate Bridge after the appointment, he remembered running across it during his marathon twenty years before. ``The Road Warrior!'' he'd dubbed himself in his Runner's Log. I supposed that was true: he was still trucking along.

I didn't have to tell him about Mike. He'd already gotten a sense of what had happened. Mike had called Primrose in the days before his death. He must have wanted to say goodbye to my dad and give that relationship some closure. Kathy had probably encouraged him. I don't know what they said exactly---a staff member had mentioned the call---but the conversation had hit my dad hard. They'd had to give him a dose of Ativan to calm him down afterward.

He didn't ask about Mike on our trip. The one time he'd brought up Maya, he'd said, ``I can't believe I'll never see her again.''

I knew exactly how he felt.

\begin{center}$*$\end{center}

My time in California basically ended after Mike's death. The remaining days are a blur for me. I must have seen my dad again after our trip to San Francisco, but I don't have any notes about it in my journal. That week I dropped the price on the house, and we got an immediate offer. I was ready to be done. Linda and the kids flew in. My siblings and their families converged on Marin County for Mike's memorial. We had dinner at a restaurant on the San Francisco Bay, probably a hundred of us all told. It was a beautiful evening, long and emotional.

I didn't bring my dad.

October had arrived. The days weren't so long or hot anymore. Nights were colder in the house. The rooms grew hollow as I cleared the last of the furniture and took Maya's final art pieces off the walls. We'd been using them to stage the house. I was glad to be leaving knowing that my dad would not have to face a winter alone in his room with a space heater and sitcoms for company. I took comfort in the knowledge that, even though I'd be leaving him again, he'd be well taken care of at Primrose. He'd had confidence in himself to deal with Maya's death, and he'd been right. He was still grieving for her, but his strong survivor instincts had kicked in, and his dementia seemed to distract him enough to keep him calm and content. That had been my last impression of him before I left.

For all of my purging that summer, and my strong desire to strip life down to the essentials, I couldn't keep myself from cramming the car full of my dad's stuff. I had so much shit in the back of the Escape, Linda had to hold my dad's cable box in her lap as we drove to an AT\&T store in Petaluma to drop it off before leaving town. The whole rear window was blocked. I didn't have the heart to donate his green leather chair that he'd spent so much time sitting in. I keep that in my office now, along with the matching foot stool where he'd propped his feet. We packed his carvings, a nightstand with drawers that also sits in my office, a carpet, blankets, pieces of Maya's art that her family had offered us, his brown attach\'e case filled with pictures, several of his books, and of course his handwritten journals. I kept his old bow, too---the Deerslayer that'd he'd used to win his archery trophies. It's cracked, and I'll never string it, so I don't know why I hauled it to Missouri. Nearby hangs his bolo tie inlaid with colored shells in the form of a thunderbird. I doubt I'll ever wear it. I detested his Native American posing, so why did I ever keep it?

I also have the hatchet that I'd broken in two; the pieces sit on top of my bookshelf in the office. His stuff surrounds me as I write about him. In that bookshelf is also the green notebook, the catalyst for our newfound relationship. I hadn't realized how much of his stuff still remained in the house that day, or that I wouldn't be able to leave it behind. 

\part{}
\chapter{}

Calls about my dad started coming in a month after I returned home. The caller ID on the first one---PRIMROSE---flashed across my TV screen at 10:30 p.m., which immediately put me on edge. Kevin, from the night staff, told me that my dad had been acting erratic, knocking over trashcans and talking about killing somebody. Kevin had tried to give him Ativan to calm him down, but my dad was having none of it. When my dad started complaining of abdominal pain, they called an ambulance. I stayed up until 12:30 to hear more, but no call came. I finally went to bed. Odd that I'd dreamed about my dad the previous night. He'd been trying to call me. There was something wrong with him, and he couldn't articulate himself very well; a woman's voice came on the line, then a man's voice, trying to explain that he was okay, but in my dream I was still suspicious.

It turned out that my dad had gallstones. They'd given him Ativan at the hospital and sent him back to Primrose once the pain subsided. The next day he was back in the ER: he'd fallen and cracked his head getting off the gurney, so they called another ambulance. It was the beginning of a revolving door of hospitals and therapy over the next few months, along with appalling bills: \$2,000 for every ambulance ride (five of them in all); \$4,000 for ``Cat Scan Head''; \$2,039 for ``Emergency Room General.'' On Christmas Eve---Ho Ho Ho---it was \$11,607 for ``Cat Scan Body'' and \$2,148 for ``Other Diag Peripheral Lab'' (whatever that was). On another ER trip in January they charged \$6,662 for ``Observation Hours.'' That had to be one of the best-paying jobs in the world, assuming someone was actually observing him. By that time

``Emergency Room General'' had increased in price to \$3,803 (up \$1,764 from the previous month). The hospital could probably charge such outrageous prices because Medicaid covered 98\% of all procedures. We hardly paid anything ourselves. I was torn between feeling fortunate that my dad had access to good insurance yet sensing that charging \$6,662 for Observation Hours was a bit of a racket.

And my dad had mentioned Maya again. I spoke to Dan after that first ER visit, and he told me that while my dad was lying on a bed for two hours in the ER, waiting for a doctor to see him, he'd obsessed over Maya---not being with her when she'd died, not being able to care for her. That's when Dan had requested the Ativan. I couldn't help but think this was my dad's residual guilt over not being there for other people throughout his life. With Maya he'd finally taken on that responsibility and felt perhaps that he'd let her down. Dan had been surprised that my dad had mentioned Maya. We'd all underestimated his grief for her, I think, which surfaced for him in times of duress. It made me think again about my decision not to invite him to Maya's memorial, and what I could have done better to help him grieve.

I'd been feeling so good, too, about selling the house and closing down related accounts. I'd signed the final papers only a few days before Kevin's call. I'd taken Linda and Ryan out to dinner that night to celebrate.

Now this.

My general frustration over hospital charges and even toward my siblings who lived closer to my dad but were unwilling to be with him in the ER stemmed from distance and separation. If only I lived near my dad, I told myself, I'd be able to manage his care more effectively. I could go to Primrose if he fell. I could go to the ER with him and approve or deny tests and procedures. Without me on site to make those decisions, Primrose had liability issues to worry about, and the hospital could run every chemistry\/hematology\/bacteriology-microbio\/ urology\/radiology\/cardiology-echocardiology test they wanted (a partial list from one ER visit). My dad had good people caring for him at Primrose, and I trusted their judgement. But I related to my dad's obsessing over Maya because he couldn't be with her. Nothing replaced my presence ---the simple act of being with my dad to oversee his care. At the same time I wasn't going to leave my family.

So my best option was to try and manage the frustration.

Case in point: my dad goes to the ER in Santa Rosa on Christmas Eve because he's got abdominal pain. His gallstones are acting up again. They give him contrast dye and run a Cat Scan Body on him. He's allergic to contrast dye---it's written clearly on the papers that Dan sends with my dad---but the doctors miss it. My dad returns to Primrose, declines over the next few days because of the dye in his system---loss of appetite, pain, swelling---so Dan ends up driving him to the San Francisco VA (Thank You, Dan!) and insists they admit him. My dad spends a few days and New Year's Eve in the hospital again. If I'd been there, I could have reminded the doctors about his allergy. I could have prevented his allergic reaction, his whole post-Christmas health fiasco.

But maybe that was wishful thinking on my part. It's also possible that I would've forgotten to mention his allergy, and it all would have happened anyway. Then I would've felt twice as bad for being the idiot son who hadn't taken time to review his dad's medical history. The distance between us filled my mind with how I could have made his life easier and better. Realistic or not, those thoughts became my constant companions.

One way I handled my stress was to spend time with Miles and Ryan, often separately because of our varied schedules. We had unusually beautiful days in November that year, temperatures in the 70s, so Miles and I went stand-up paddleboarding at Lake Springfield, a twenty-minute drive from our house. Stand-up paddleboarding was the closest thing I have to surfing in the Ozarks: balancing on a big foam board and gliding across the surface of the water. It also attracted attention because the activity was new to the region, so people were interested and asked us questions as we cruised by. One woman on a boat hollered, ``I thought you were Jesus, walking across the water to save me!''

Miles and I arrived early that first morning. The wind was down, and the water was glassy. We talked about everyday things as we stroked under highway 65 and snaked our way up the James River. Miles studies physics at Drury, where I teach, so he'd tell me about his classes, the books he was reading outside of class, and his preparations for graduate school. We had the narrow river to ourselves save the occasional fisherman. We both liked the quiet of the surrounding woods, the brightly-colored leaves floating on the surface of the water, and the physical effort of pulling ourselves alongside one another. I admired the way Miles's mind worked---his ability to quickly grasp fundamental concepts---so I liked to draw him out and listen to his observations about the world and beyond (astrophysics was his particular area of interest). This was the kind of trek I would have enjoyed taking with my dad, the kind of bonding I would've liked us to have had. I appreciated my dad's love of nature in such moments and felt good about passing that love onto Miles.

With Ryan I played basketball. We'd go down to Drury's gym and shoot around on a Sunday, the court to ourselves. We practiced drills that his middle-school coach used. Or else

Ryan would invent a drill, and I'd play along. He ran sprints. He practiced lay-ups. I'd stand off to the side and pass him the ball for jump shots. He was probably six inches taller than I was at his age, so I admired his size. He could take over a game if he wanted to, but he didn't yet realize how big he was. He was only thirteen. I had to remind myself not to be such a ``dad'' when we played together: not to constantly correct him or to tell him how to do something better. I couldn't always help myself---that desire to see him perform better than I did at his age, to go through the world with more confidence than I had, was too strong---but I was conscious of my behavior and did my best to lay off. More than anything he needed my love and support, my time and attention. He had coaches for the other stuff.

My dad had been too competitive, or simply wasn't around, to fully enjoy the success of his sons, to watch as we progressed beyond him in various fields. I'm fortunate in my life station that my pride at seeing my sons best me takes precedence over my desire to win, which is always strong. I love to compete with them, to push them, and I don't like to lose. But if I'm going to lose, who better to than my sons? Miles could beat me regularly at chess when he was twelve. Ryan at fifteen dominates me on the court (he'll tell you he dominated me before that, but I might shake my head). Watching them develop into young men, striving, challenging themselves, challenging me, is so rewarding. My dad could never get over himself long enough to experience the satisfaction of watching his kids grow beyond him.

But my dad needed my time and attention, too. I struggled with how much to give him. I entered my cycle of distancing myself from him---not calling him, not really wanting to visit him in January during my winter break. Yet I also worried about him: he was losing his mobility; he was falling and hurting himself. He shouldn't have to go to the hospital alone. No one should, especially an old man with dementia.

I knew I should fly out in January and visit him. I wouldn't get a chance to see him over spring break in March---Linda and I had planned a road trip to Washington D.C. with the kids. I also intended to spend the following summer at the University of Hawaii studying Hawaiian and doing research.

So a week in January it was.

During the time my dad had spent at the San Francisco VA over New Year's---they were the ones who'd figured out the contrast dye snafu---the doctors had recommended removing his gallbladder. He'd been suffering on and off since November. Other procedures might work temporarily, but the best long-term solution seemed to be surgery. They'd mentioned it to me because an opening had come up, and the whole procedure could be done fairly quickly. This conversation had taken place on New Year's Eve.

At the time I was tempted to say Yes right there over the phone. Surgery would solve my dad's pain problems---the acting up, the trips to the ER, his related symptoms---and it all seemed very convenient for him and for me: he was already at the hospital, there wouldn't be any charge, and the VA had taken great care of him over the years. The whole situation could be resolved in a matter of days. Plus it'd be better to operate before my dad had another attack and it turned into an emergency situation. To the doctor's credit, she told me to think it over and call her the next day.

After talking it over with Linda---she was the one who emphasized that a family member should be present if my dad had surgery---I told the doctor No. My dad wasn't in an emergency situation at the moment, and Linda was right: I should be there with him for such a serious procedure. I asked if they could do the surgery in two weeks, when I'd be in town, but they didn't have an opening. So I'd go out in January as planned and meet with the doctors to talk things over. If surgery still seemed like a good idea, we'd set up a date for later in the year.

\begin{center}$*$\end{center}

My week-long visits to my dad are blurs, filled with appointments that I stacked together because my time was so limited. I flew in on Sunday, January 10th. On Monday we had his pacemaker checked in Santa Rosa (the VA had contracted that one out because it was easier for my dad to stay in town). On Tuesday afternoon we drove to San Francisco to meet with his doctors. By chance a spot had opened up for that Friday, so our informational meeting suddenly turned into a surgery consultation. Since I could be with my dad for the procedure, I decided to go ahead and have it done. It seemed like the best option to alleviate his pain and prevent future attacks. All of my dad's problems since early November had stemmed from gallstones, so removing his gallbladder seemed like the best option.

That is, as long as the surgery went smoothly. He was eighty-five years old, after all, and the past two months had been a rough ride for him. He'd recovered from his multiple ER visits, but he was still in a weakened state.

My dad had the line of the week that day. We'd left the VA and were headed for the Golden Gate Bridge when we saw a man at an intersection carrying a sign that read Homeless. The driver in front of us had rolled down his window and handed the guy a bill. My dad said, ``That's a good idea. Give him some money, let him go home.''

It was so sweet. Two days later, my dad had another funny response when his nurse---her name was Joy---was finalizing his check-in at the VA hospital. She'd been asking him various questions, and I could see my dad's cognitive decline. He couldn't tell her his middle name or who I was (he called me ``Michael''). He'd also told her that he lived alone, ``somewhere near Santa Monica.'' Our family had lived in that area before moving to Thousand Oaks. At the same time my dad could rattle off the last four digits of his Social Security Number, recall that he had four boys and four girls, and he knew his birthdate. He didn't know what year it was, though, or why he was in the hospital.

Joy finally asked him, ``Do you know where you are?''

He answered, ``I'm lying down in bed.''

I tucked him in before I left. ``I'll see you tomorrow morning,'' I told him. Because of his age and condition, they put him up for surgery first: 7:30 a.m. I planned to be back at the hospital by 6 a.m. so I could see him before surgery. He gave me a kiss goodbye when I leaned down to hug him. He was getting more vulnerable in his weakened state. He always held my hand or arm now when we walked anywhere. I hadn't expected him to have surgery that week. Everybody thought it was a good idea---Dan at Primrose, and the doctors at the VA. I was going along with their advice and hoping for the best.

\begin{center}$*$\end{center}

The next morning I took a picture of my dad before surgery. He's lying on a gurney smiling up at me. His eyes have a sleepy gleam to them, probably from the drugs. He looks funny in his surgical cap: it's white and poofy, like a chef in a children's book. He's covered with a heated blanket, and a white pillow props his head up slightly. His charm comes through. He doesn't look worried at all.

I was backstage, so to speak: the pre-op area where they prepped patients for the day of surgeries. I stood by my dad's bed and tried to stay out of the way of doctors, nurses, and interns moving about their business. I felt they were aware of my presence, perhaps paying more attention to what they were doing because I was there. Whenever a doctor or nurse came to ask my dad questions---did he know why he was there, what surgery they were performing on him, where his pain was---I reminded them that he had dementia. I didn't assume they already knew. I was able to provide a lot of information about his health and recent experiences that seemed helpful to them. They probably could have gotten those details from his file, wherever that was. I was very glad to be there with him, for my sake and his.

``Are we huddling?'' the head surgeon asked his group---nurses, an anesthesiologist, other surgeons or surgeons-in-training. They were all in scrubs, standing in a circle, the final meeting before surgery where they reviewed what the procedure was. The surgeon exuded confidence; the group was energetic and good-humored for so early in the morning. This was their routine, but again I had the feeling they were aware of my presence and acting particularly professional. Or perhaps this was just my nerves on edge, knowing that my dad was going under the knife because I'd okayed it.

A staff member brought over a form for me to sign: a temporary stay of the DNR order. If my dad started to die on the table, they'd do there best to resuscitate him. There was no other option. Without the stay, they could not perform surgery. I'd already agreed to the stay during our consultation earlier in the week, but inking my name on the dotted line gave me pause. I wanted my dad free from gallstones, but I didn't want him hooked up to machines 24\/7 should something go wrong. It wasn't what he would have wanted for himself either. I had to balance the benefits of surgery with the possibility that my dad might have to live out his life in a way he would have hated.

I signed the form and watched an orderly cart him away.

I hardly remember what I did during the next few hours, waiting for the surgeon's call. I don't have a journal entry from that day. The waiting room had my cell number, and they'd contact me as soon as they knew anything. I think I went down to Ocean Beach, parked in the public lot, and just watched the surf roll in. Around 10 a.m., when the surgery was supposed to be done, I headed back to the VA and was in the waiting room when the surgeon arrived. He waved me out to the hallway.

``Everything went fine,'' he said.

I nodded, relieved. He told me the gallbladder had been red and inflamed. It had really needed to come out. There hadn't been any complications. My dad was recuperating now.

``So we made the right decision,'' I said.

He clapped me on the shoulder. ``He'll be a new man.''

My dad was in the post-op room, sleeping and recovering on his gurney. A young woman sat in a chair by his side. I chatted with her for a few minutes. She told me it might take hours for my dad to wake up fully and get settled in a room.

I can't remember if my dad did in fact wake up while I was there, if he saw me waiting by his side. His surgery took place on Friday, and I was due to fly out early Sunday morning. I don't think I saw him again before I left. If he were awake, I probably smiled at him and said something like, ``You made it.''

``Yes I did,'' he might've said.

``I hear you'll be a new man.''

A chuckle from him. ``I kind of liked the old one.'' 

A hand on his arm, a hug. ``I'll see you soon,'' I would've said. Or, ``I'll see you next time.'' At that moment, I'd have no plans to see him for most of that year. He would've given me a smile. ``Look forward to it.''

And he would've thanked me.  


\chapter{}

My dad was not a new man after the surgery. This was not the surgeon's fault. My dad was old. He'd been weakened by gallstone attacks. His dementia was advancing. But the surgery was the beginning of the end for him. He never quite recovered.

When I returned to Missouri I spoke to nurses on the phone every day. My dad had developed a urinary tract infection. He had a high white blood cell count and urinary retention, so they'd put him on a catheter. I remember asking about his mood in general and also speaking to my dad several times. But it's all a blur to me now---the constant worry, the need for information, the frustration at never seeming to get the same person twice on the phone. To get a better sense of my dad's experience in the hospital that week, I go to the VA website and download the 213 pages of medical notes that detail his day-to-day treatments. I don't know what I'm looking for exactly. Maybe just a connection with him since I wasn't able to be there.

The pages are technical, completed by nurses, surgeons, physical therapists, respiratory therapists, and dietetic interns who use acronyms and medical terms that I have to Google to understand. But every once in a while I come across a more personal description, and I parse it for hints about how my dad was feeling. A chaplain visited him on January 17th, two days after his surgery. The chaplain noted: ``Pt. was sitting in a chair in the hallway of the unit; said that he's getting better slowly. He said he is feeling calm and has a good attitude. He shared about his belief in God, and how he prays each day. Chaplain also said a prayer for him.''

I found that interesting since my dad wasn't religious. Did the chaplain know that my dad had dementia? My dad had probably seen the white collar and played along.

A dietetic intern noted the next day: ``Writer observed pt did not eat any breakfast this AM. When pt was asked, he said `I ate as much of it as I wanted to.''' I could picture my dad saying that. The same day, a physical therapist wrote: ``he wants to go home.''

On January 20th, a nurse described him as ``pleasantly confused; knows he is in a hospital in California (unable to name city); guessed that today was `February 14, my birthday'.''

A couple days before he was discharged, the physical therapist diagnosed him as ``retropulsive'' and wrote in part: ``He exhibited difficulty initiating ambulation, advancing and sequencing walker. He continue to require chair behind when gait training due to easy fatigability. He also has difficulty controlling his descent when seating.'' I assumed all that meant that my dad had trouble using a walker, sitting, and he got tired easily. The PT recorded my dad's take on all this: ``I think I did well today.''

My dad's final General Surgery Progress Note before being discharged: ``cannot pee at all.''

Infections plagued him for months. They transferred him to a skilled nursing facility in Santa Rosa where he kept trying to pee. He worked with physical and occupational therapists on mobility issues. His second day there, during a therapy session, he passed out in a wheelchair after the therapist had left the room. They called an ambulance and sent him to the ER. About \$20,000 later he was back at the skilled nursing facility trying to pee. I didn't quite understand why they'd sent him to the ER. I thought the whole point of transferring him to a skilled nursing facility was because they had skilled nursing there.

It goes to show how little I know about medicine.

I asked him over the phone if he needed anything. He said, ``There's one thing. Do you want to come and pick me up and take me away from here?''

I wished I could. I didn't know what to tell him, so I made something up: ``When I hear from your doctor that you're okay, then someone will come and pick you up.''

Someone. Not me. I lived in Missouri.

``I have no problems in that area,'' he told me. ``I'm fine.''

He still had a urinary tract infection and was on a catheter and antibiotics. He couldn't walk anymore. They tried physical and occupational therapy for several weeks, but he wasn't improving. We finally decided, in late March, that the best option was to return him to Primrose where he knew people and had more consistent care. He'd been gone for two months and returned a true shadow of his former self.

We hoped. We hoped familiar surroundings would help restore his appetite. We hoped the fine spring weather would increase his desire to socialize. He hadn't been to see his crew in the Day Club for so long. But once back at Primrose he told the ladies there, ``I don't feel like I can keep up with everybody else.''

As before, my biggest frustration during my dad's time in the skilled nursing facility was simply not knowing what was happening with him because I wasn't on site. A typical phone call might go something like this: I'd talk to a nurse, who would ask someone who had worked with my dad how he was and his prognosis---the nurse didn't know herself because she worked only twice a week on evening shifts; she'd tell the next nurse on shift to call me, but that nurse only worked three times a week and might not know anything either. But they'd find out for sure how he was doing and get back to me.

So there was my dad, unable to advocate for himself and at the mercy of caregivers who revolved through the place like a snack cart. Such was the normal state of affairs at skilled nursing facilities according to my sister Kathleen. All of this put my dad in a perilous situation of falling through any number of cracks. I could organize his care and try to stay on top of things, but distance limited my effectiveness. I wanted to do better for my dad, but I couldn't.

My siblings had made their positions clear as well. I kept them updated, and they visited my dad when they could. I had to be content with allowing them to decide how much contact they wanted to have with him, and when they wanted it. These were the consequences of a lifetime of selfishness, I told myself. And of abuse. And of neglect.

My dad wasn't complaining. He remained an affable trooper. I agreed with my siblings and understood their positions. And yet I still worried about him. So much that I wrote Dan at Primrose and told him that I wanted to start looking into transferring my dad into a skilled nursing facility here in Missouri. I felt that much of my stress could be relieved, and my dad would have a better quality of life, if he lived near me. Then I could visit him more often and stay on top of his care. One of the triggers for my note to Dan was a message I received from Kathleen. She'd visited my dad once he returned to Primrose. She'd walked into the dining room but didn't see him. She did a double-take on a man slouched in a wheelchair: my dad had lost so much weight that she hadn't recognized him.

Well, moving my dad to Missouri didn't work either. I thought I could get him settled here in Springfield, but no skilled nursing facility contracted with the VA. It turned out that the nearest facility that did so was over in Joplin, an hour and a half from Springfield. And they had a long waiting list. Even if they didn't, and even if I adapted to driving three-hour round trips--- not exactly my idea of staying on top of his care---I had no idea how I was going to transfer my dad in his condition two thousand miles to Missouri.

Dan was adamant that Primrose could continue to offer my dad the highest quality of care possible: my dad knew people there, they knew and liked him, and that was the place where he should be.

After a week or so of mulling it over, I agreed with Dan. My one caveat was to ask him to set up a meeting with hospice just in case. Dan believed my dad was far from needing hospice. Besides, that order had to come from my dad's doctor upon Dan's recommendation, and Dan believed my dad had a lot of life left in him.

That changed in mid-May when my dad started refusing to eat, normally a clear sign that a patient was close to death.

\begin{center}$*$\end{center}

It all happened so fast. Dan contacted me with the bad news and basically thought that my dad had ``made up his mind not to live anymore.'' He'd lost over nine pounds in the past month and was ``angrily rejecting their efforts to encourage him to eat.'' In the past week my dad had also shouted out a couple of times that ``he wanted to die.'' His intentions seemed pretty clear to me, but Dan said that my dad had also been pleasant and cooperative. Dan reported all of this to my dad's doctor at the SRVA. He expected her to recommend hospice. Dan ended his note with the following: ``If he continues absolutely refusing to eat as he has in the last two days, he will not last long.'' That was on May 12th.

I passed the news onto my siblings and asked for their input. I'd already made plans to fly out to California on May 22nd. Final exams would be over by then. So would Ryan's graduation from eighth grade. Ryan was also giving a performance on his saxophone that I wanted to attend. I'd originally planned to enroll in summer courses at the University of Hawaii, but my dad's medical issues over the past few months made me abandon that plan.

I mulled over what to do, whether I should fly out to California sooner. I called Kathleen and Cynthia on May 15th to get more information about hospice and see if they could tell me what I knew they couldn't: how close was my dad to dying? Should I get on a plane right now, or could it wait a week?

I knew from Mike's experience that people often ``graduated'' from hospice and went through several rounds before dying. After passing on Dan's note to Kathleen, I wanted to know her thoughts.

She asked me, ``If dad dies, will you regret not being with him?''

It was a good question. I thought I'd be okay if he died and I wasn't there. He was comfortable and being cared for at Primrose. No one in the family was rushing in to see him yet. Cynthia and Kathleen offered to visit him and keep me posted. I had Dan, too. Obviously I wanted to be with my dad if he was REALLY dying. But was he REALLY dying? He had a strong constitution, and I believed that hospice might continue for him much of that summer. If that was the case, there was no need for me to rush out there.

Then again, my dad could die quickly. Especially if he wasn't eating.

It was hard to know what to do. I felt responsible to be with him when he died, especially since I was the one who made decisions for him. But I also felt my usual distance from him. How much of my own family's life was I willing to miss to be with my dad? Admittedly, I'd felt much less distance over the past couple of years. I had grown closer to him, despite my best intentions.

On May 16th Cynthia visited my dad at Primrose. She usually sent me pictures. She'd emailed me one from early May in which I saw a blankness in my dad's eyes that I'd never seen before. His eyes had always carried his intensity, true portals of the soul. They had depth to them, layers of conflict and contrast that his high-beam vitality could mask most of the time in a flood of energy that washed over people and bathed them in sales appeal.

But in the picture Cynthia had sent, the spark had completely gone out of his eyes. I didn't quite register the consequences of what that meant, but I knew something had taken a serious turn for the worse.

On May 18th I participated in a conference call with Dan, Kathleen, Cynthia, and representatives from hospice. After introducing myself, I mostly listened in as Dan briefly reviewed my dad's health history for the people in the room. He then updated everyone on what had transpired in the past few weeks. My dad hadn't eaten in over a week. Dan added that my dad had a ``smell'' to him that reminded him of his experiences with cancer patients in their final days.

That didn't sound good.

A hospice nurse joined our conversation halfway through. She'd just left my dad. Knowing when someone will die is an inaccurate science, to say the least. Especially with dementia patients, who each have their unique situations. Dan had told me it was very hard to predict their cases, even as death approached. You just never knew.

I waited to hear what the hospice nurse had to say. I remember there was random noise in the background as she entered the meeting room and took a seat---snippets of conversation from the hallway before someone closed the door again. I also heard side conversations in the room itself---Kathleen's voice as she responded to someone. I pulled closer to my phone, intent on hearing the hospice nurse's voice.

I asked her about the issue of the ``smell,'' which had me worried. She mentioned that his extremities were cold; his legs were mottled from the blood being pulled away to service more critical organs---his heart and brain. She then provided a medical explanation about blood circulation that I didn't quite understand. Her voice reminded me of my dad's social worker at the VA: experienced and calm. She spoke slowly, weighing her responses with care. Her tone was gentle yet frank.

``The smell,'' I asked. ``Is it from his organs breaking down inside his body?'' She hesitated a moment. ``Yes,'' she said.

Instead of asking her an impossible question, I simply said, ``Should I fly out?'' Another hesitation. ``He could go anytime.''

There have been moments in caring for my dad when I hemmed and hawed, not knowing exactly what to do. That seemed like 98\% of the time. That is to say, most of my decision- making took place under the caution yellow light, for lack of a better term. Neither red nor green. I could speed up, I could slow down. I could look both ways and turn around. Whatever I did probably wouldn't cause too much damage. In the most critical of situations---my dad's driving, his move to Primrose---I waited for that green light to flash inside of me, a gut feeling that told me, without a doubt, that it was time to act.

The nurse gave me her answer about three o'clock Missouri time. I wrote my siblings a quick email, got on a flight that evening, and made it to San Francisco by eleven p.m. 


\chapter{}

The closer I got to California, the more my stomach twisted into knots. Now that I'd made the decision to fly out, I didn't want my dad to die before I got there. I didn't want to face the possibility that I'd waited until it was too late.

I checked my messages as I hurried to the rental car place in San Francisco. Kathleen had texted me. Before she'd left my dad that afternoon at Primrose, she'd told him: ``Pat is coming!''

``Okay,'' he'd mumbled.

In her experience, sometimes people waited to die until a special person arrived; other times they waited until people left the room to die. Maya had done that, expired all alone. At least my dad knew that I was on my way. If he had control over the matter, he could decide what he wanted to do.

My knots grew tighter as I drove north: from the airport to highway 380, over to 280, up 19th Avenue and across the Golden Gate Bridge, then a straight shot to Santa Rosa on 101.

I looked for ambulances or other official-looking cars in the parking lot as I pulled into Primrose. It was about one o'clock in the morning.

All was calm.

Raymond, the night manager, buzzed me in.

My dad was sleeping on his back. I pulled his desk chair next to his bed and sat with him awhile, just looking at him. For all of my talk about feeling okay if he'd died before I got there, I was sure glad to see him alive. I didn't know what comfort I could offer him, if any, but I knew that I was where I should be. Green light all the way.

Raymond offered me a bottle of water, which I accepted. I'd stay until I felt that my dad would survive the night. The window was cracked open in his room to ease the smell. My dad's roommate was there too, sleeping quietly.

Once my own breathing became normal again, I noticed that my dad's was unusually quick and shallow. He also had a little OCD going on: he'd reach his arms out, fold the edge of his blanket back, then bring his right hand up to his lips. He did this numerous times. Something was running through his head.

After an hour or so of sitting, the travel and worry finally caught up with me. I left for Kathleen's house to sleep for a few hours. I'd return the next morning---or later that morning, I should say---after breakfast. When I closed the door of my dad's room, he'd stopped moving his arms and was resting peacefully.

\begin{center}$*$\end{center}

I was back by 8:30 a.m. It was Thursday, May 19th. Ryan---two hours ahead in Missouri---was in the middle of his graduation ceremony. I dropped by Dan's office to say hello and get any updates. Monica, another nurse, was with my dad. She had put on a CD of classical for him. She told me she'd given him morphine at two o'clock and nine o'clock the day before because he seemed to be in pain. Over breakfast that morning, Cynthia had told me that my dad hadn't been out of his room for two days. The knot returned to my stomach when she said, as I was leaving: ``Call us if he passes.''

She'd said it so easily, like you might say ``pass the butter.'' Like it might actually happen that day. I suppose I was still in denial. I mean, I was there if he died, but I didn't expect him to actually die while I was there. What would I do then?

After the conference call with the hospice nurse the day before, Linda had made us sloppy joes for dinner before I got on the plane. It was a spicy meal, but I didn't taste it at all. My taste buds had gone completely numb. I was glad to have enjoyed my scrambled eggs and cinnamon toast before Cynthia had made her comment.

After Monica left, I looked closely at my dad's face. His eyes were sunken in like Maya's in her last days. His face was drawn. I sat on the edge of his bed and clutched his hand. I wanted him to know that someone was there, that I was there. I had made it. He pulled his hand away instantly, like he didn't want to be distracted. He was resting peacefully, his breathing quieter than the previous night.

I thumbed through papers and pamphlets that had been left in my dad's room: information on his hospice team, their mission statement, and the services they provided. I read about coping with grief and loss, the range of emotions I might feel, odd things I might do. There were pages of resources, practical advice on funeral arrangements and funeral homes, a list of phone numbers for counseling and support. One guide, ``Signs and Symptoms of Approaching Death,'' presented a three-column chart of ``Body Functions,'' ``Expected Changes,'' and ``Helpful Interventions.'' This last heading mentioned things I might do to make my dad more comfortable during the process of dying. A short pamphlet by Barbara Karnes, Gone From My Sight, listed helpful guidelines to determine how long a patient might have to live: common signs and symptoms in the months, weeks, hours, and minutes before death. As far as I could tell, my dad was in the ``Days or Hours'' category: glassy eyes, irregular breathing, mottled skin.

One of the sheets used the analogy of a roller coaster ride to describe the process of dying. The image made sense: emotions racing up and down, the twists and turns, the anticipation of waiting for The Big Drop. But it also made me think: I am not in control of this ride. I might be the one sitting in the front car, but all I could do was anticipate and react to whatever was coming at me. Maybe I could give the people sitting in back an idea of whether they wanted to laugh, scream, raise their hands, or barf over the side.

Which reminded me: I sent a quick email to my siblings and gave them an update.

\begin{center}$*$\end{center}

About ten o'clock Shelly, the hospice spiritual advisor, came into the room. She was friendly, looked to be about my age. I didn't feel like I needed spiritual guidance---I'm from the lapsed Catholic tradition---but I was happy to have company and to chat. I was so wrapped up in my own thoughts and emotions, plus a little sleep deprived, that I didn't realize she was there primarily for my dad. This finally hit me when he startled from his sleep. Shelly and I had been talking, but she stopped and went right over to him. She leaned in close to his face, told him who she was and gently stroked his forehead. I felt like an idiot for thinking she was there to comfort me. I was glad that my dad had professionals around him so at ease in their work. I realized that the distance between me and my dad was still present, even though I was sitting next to him. I'd felt a responsibility to be there, but I hadn't wanted to interact with him much for fear that I might disturb him. I didn't know how to act around a person who was dying. So I watched Shelly and all of the hospice workers interact with my dad, and I learned from them.

Shelly did in fact comfort me. The knot that'd been in my stomach since early morning slowly released as I spoke to her about my dad, myself, our family, and even my bizarre idea to watch my dad's cremation. I told her about my book project---finding my dad's journals and the outline of his autobiography, and me finishing his project. Part of my distance in that room was the writer in me: I was not only a son experiencing the last days of his father, but I was also collecting information for this future book project about him and our relationship. I had no idea when I would get around to writing that book, but I knew there was a story to be told. Part of me shut off emotionally so that I could watch Shelly and record in my mind, and later in my journal, material for that project.

We had a great talk. I was thankful that I could express my feelings and thoughts to her. She told me that the book was my creative way of dealing with my dad's dying. She herself used collage in the same way. She likened the process of his dying to birth. She told me to trust the process, that he had decided to go this way. She also helped me interact with him, told me he could probably only answer Yes or No questions. I should approach him from the front because his vision might be blurred. His hearing was probably more acute so I didn't need to speak loudly to him.

A hospice aide came into the room to wash my dad and put lotion on him. Shelly told me this would be a good time for me to take a break---whenever an aide came in---so I took her advice. Before I left, the aide showed me how to moisturize my dad's mouth inside and out. She used the same stick-sponge that my dad had used for Maya. The aide told me I shouldn't give my dad water droplets because they might make him choke.

I found a bench in the sun on the grounds outside. I called Linda to find out how Ryan's graduation had gone. I also called my sister Terry. She'd taken a few days off work to hold her own vigil in her house. She told me that dad had had a lot to struggle with in life, a lot of conflict, so she was glad he was resting peacefully. She thought that he had probably enjoyed the battle itself as much as any results he ever got. This made me think about his surfing. Salmon

Creek, our home break, could be a tough spot to surf. In winter especially, when the waves were big, you often had to fight through numerous lines of white water before reaching the outside sandbars off which the waves formed and broke. My dad had absolutely no give-up in him; he'd always been better at making it outside than standing and riding waves. He liked the challenge of taking on something as big and wild as the Pacific Ocean. The actual act of surfing was probably less interesting to him.

My dad had told Kathleen one time that when he was ready to die, he'd just paddle off to the horizon on his surfboard. That had been his plan.

\begin{center}$*$\end{center}

I had my first conversation with him when I got back to his room. He opened his eyes for the first time since I'd been there.

``Hi, Dad,'' I said. ``It's Pat, your son.'' I sat on his bed and held his hand. I positioned myself directly in front of him. His blue eyes were glassy, distant. ``I'm here.''

``Yeah,'' my dad said. It reminded me of Maya's responses the last time we'd seen her: short, breathless.

``Do you need anything?''

``Yeah.''

I waited. He remained quiet. I said, ``I'll just sit here with you.''

``Yeah.''

``You can rest, and I'll just sit with you.''

``Yeah.''

I rubbed his forehead and hair as Shelly had done. Soon he was asleep again.

He was emaciated. His body was flattened out, hardly took up any space under the covers. He was as small as I'd ever seen him. I saw my own arms in his: the large hands, the veiny forearms and wrists; his biceps were lean, fallen, fleshy. Those biceps had carried so much in life.

\begin{center}$*$\end{center}

Steve, the hospice nurse, was the next to visit. He looked to be in his mid-50s, another experienced and calm caregiver. He listened to my dad's chest and examined him. He asked my dad questions as he worked: Are you in any pain? Are you hungry? Are you thirsty? Do you want to rest? My dad gave short answers. No to pain. No to food or drink. Yes to rest. Barbara Karnes notes in her pamphlet that a dying person's choice not to eat was one of the hardest things for family members to accept. And yet it was completely normal and natural.

``Your carvings are beautiful,'' Steve said to my dad. I'd told Steve about my dad's name and pointed to one of his carvings---a replica of his own hand, his fingers reaching toward the ceiling---that sat on his dresser.

``Thanks,'' my dad said, almost like an exhalation.

After the exam, Steve talked to me in the hallway. ``Carver's death isn't imminent,'' he said. There was no apnea, no ``death rattle,'' which he explained was liquid in the lungs. My dad was still responsive. ``He's transitioning to dying,'' Steve added, ``but he's not in the process right now.'' His legs were still mottled, the blood being pulled to his heart and brain.

``How long, do you think?''

``I usually go by this standard,'' he said. ``If the physical transitions take months, the patient has months to live; if they take weeks, then he has weeks to live.'' He paused. ``By the rate of changes in Carver, I'd say he probably has days. But he might still be around next week.'' He added that some patients bounced back after beginning hospice, that the care made them rally. But he didn't think that was going to happen with my dad.

At least that gave me a sense of what lay in store for me and my dad over the next couple of days. I planned to be with him as often as possible. I was in that odd position of waiting for someone to die. I wanted to be in-the-moment for his sake and mine, but I was also responsible for the next steps after he died: planning his cremation, gathering information for the death certificate, thinking about what kind of memorial we should have. I was driving to the funeral home in Petaluma the following day to write them a check and make preparations for viewing the cremation, whenever that turned out to be.

\begin{center}$*$\end{center}

My niece Mallory brought me lunch in the early afternoon. She'd driven up from

Oakland where she'd started teaching high school the previous fall. She'd been the closest to Carver out of all the grandchildren on our side, and she started to cry when she saw him. She said he looked just like Mike had in the end---the pallor of his face. Mike had died at the beginning of her school year, and now Carver was dying at the end.

Kathleen and Cynthia arrived shortly after Mallory. The four of us ate our lunches and talked about whatever came to mind. The staff had found another bed for my dad's roommate, so we had the space to ourselves. We sat in a semi-circle so that we could see my dad. I hoped he could hear our voices and that it comforted him somehow, simply having family around. He wasn't in a stage of interacting with anyone unless they were leaning down in front of him and asking him questions. As the afternoon turned into evening, we decided to go get Mexican food and eat dinner at Kathleen's place. After my talk with Steve earlier that day, I felt that my dad would be stable enough so that I could leave for a couple of hours without missing anything. When we said our goodbyes to him, I leaned in close and said, ``It's Pat, dad. I'm going to Kathy's for awhile, but I'll be back.''

He didn't respond.

I planned to spend the night in his room. I didn't know what I could do for him other than to be with him.

\begin{center}$*$\end{center}

I was tired when I got back from dinner, but I couldn't fall sleep right away. It felt strange to crawl under the covers in this facility, to share a room with my dad, just the two of us. Had I ever done that before, in all of my fifty-three years?

I'd stayed with him at his house in Sebastopol, but never in the same room. We hadn't ever stayed in a hotel together, or gone on vacation separate from the family while I was growing up. On occasion we'd surfed together at Salmon Creek, but invariably he'd take his own truck so that he could leave when he wanted. Had we ever simply gone on a walk together in the woods, or watched a movie together?

Yes, a movie. One time. Here was what I remembered: a matinee in Santa Rosa, the two of us sitting next to one another in the dark, toward the back. We're sharing a tub of buttered popcorn, one of my dad's favorite snacks. Just as the movie starts---I think it was The Outlaw Josie Wales with Clint Eastwood---two teenaged girls, maybe sixteen, walk in and sit down in the two seats to my right, both of them talking.

My dad leans across me and says something to them.

They shut up.

My dad lifts his voice, as if he were talking to himself: ``God, the whole movie theater, and they have to sit right next to us.''

One of the girl whispers, I didn't even see them.

There were only a few people in the room, but it felt like just the four of us after my dad's scolding. There we all sat in a row---three of us mortified---for the next two hours.

The Outlaw Josey Wales made sense. It was about the right time, 1976. I was thirteen. My dad was a big Clint Eastwood fan. The movie was based on a book by Forrest Carter (aka Asa Earl Jones), the former Klansman who'd also written The Education of Little Tree, the pseudo- Cherokee book I'd found in my dad's office. I've been known myself to say something in public spaces to loud talkers. I think it makes my family uncomfortable. Perhaps I got this from my dad, a feeling that I needed to be the man who put thoughtless people in their place. Very Clint Eastwood.

I also remembered working for my dad one summer, when he was the resident salesman at a mobile home park. I must have been fourteen or fifteen. My job was to clean mobile homes before he showed them to potential buyers. I dusted the insides, picked up old newspapers outside, cleaned sinks and scrubbed toilets. He paid me two dollars an hour, which felt like a windfall at the time. The work wasn't heavy labor---not like hacking weeds in a field---and he let me drive him in a golf cart one afternoon. That was how he got around the park, dressed in his leisure suits and bolo ties. It was the one and only time he gave me a driving lesson. He'd left the house by the time I got my permit. My mom was the one who taught me how to drive a stick shift.

But the golf-cart lesson gave my dad a good story. He let me take the wheel on the way to one of my clean-up jobs. I didn't know that if you took your foot off the pedal, the cart would slow down on its own. At the first stop sign, I stomped the brake.

``He nearly threw me out the windshield,'' my dad cracked when he later told the story, adding an expressive Bing! for effect.

Some of my strongest memories of my dad as I was growing up centered around two feelings: intimidation and humor. They had something in common, sadly: he used them both at other people's expense.

Now I really wasn't getting to sleep. I grabbed my laptop and sat up in bed. I worked on an academic article about surf history that was due the next month. I remembered that I was supposed to meet a colleague at Drury the next day, so I emailed her and rescheduled. I sent an update to my siblings. I called Linda, even though it was late. The knots in my stomach coiled and uncoiled, my in-the-moment reminders that I was with my dad, he was dying, and I cared about him and our relationship.

Debbie, the night nurse, dropped in to say goodbye to my dad. She was emotional, her eyes teary. She'd be off for a few days and probably knew she wouldn't see him again. The staff at Primrose loved my dad, they all told us. Raymond, the night manager, said that usually when residents came in, they were already in the first stages of dementia. When they got a resident like Carver, who interacted with them and remembered them, the staff were naturally drawn to them because there was a connection there. ``That doesn't happen very often around here,'' he said. I'd always seen that with Dan and the Day Club staff. They had all become very attached to my dad. In many ways Dan had taken on responsibilities that I would have normally had if I'd lived nearby. I like to think that my dad was his best self during his time at Primrose, what he might have been in a parallel life had his circumstances been different.

I finally shut my laptop and curled up under the blankets. Without distractions, I tried to identify the smell in the room. It seemed to be a combination of fruity air freshener and my dad's failing body. The window remained cracked open, and cool currents pushed through now and again, but the pungent smell remained. There was an underlying heaviness to it, an earthiness, like a mouse had died in a corner of the room. After a while I no longer noticed it.

I lay on my back, in the same position as my dad. I matched his breathing, tried to fall into his rhythm. I had the odd sense of preparing myself for death---my own rather than his. There we lay, father and son, almost side by side in the dark. Someday I would probably end up in a Board and Care facility myself, in a bed like this one, perhaps with a roommate. Would I have the mental capacity to look back and remember this moment, to think, This life of mine has gone by so fast! What did I do with my time? What do I have to look forward to now? I could do a lot worse than to die in a place like Primrose.

I could not match my dad's breathing for long. His was twice as fast as mine. Several times an hour he'd give these small hiccup-sounding cries, always two of them, then his breathing went quiet for a bit, perhaps thirty seconds or a minute, then the quick breathing started back up. His homestretch. That'd be a good term for it, if you thought he was headed home in any sense.

My dad slept well through the night. I couldn't say the same. I woke up every time an aide came in to check on him, perhaps every two hours. I wasn't used to the rhythms or noises of this place, and my internal clock was off.

At six a.m. Raymond gave my dad a dose of morphine about twenty minutes before they changed him. The aide told me my dad had some urine in his briefs.

I left to take a shower at Kathleen's, have breakfast, then I'd be back.

\begin{center}$*$\end{center}

My dad was more restless on Friday morning. He picked more at his clothes, his blankets, and his face. They'd put him on morphine every two hours now along with two doses of Ativan per day. It seemed to help him stay calm and rest more deeply.

Kathleen and Cynthia spelled me after lunch. I drove to the funeral home in Petaluma and started arranging the cremation. I spoke with a woman named Michele, told her about wanting to view the entire process. She thought it wouldn't be a problem, but I'd have to pay an extra \$750 for the pleasure. They normally cremated two bodies at a time, each one in a separate ``retort,'' as they're called. But if a family member wanted to watch, they only cremated one body out of respect for the other.

I left the funeral home with a form to fill out and returned to Kathleen's place to take a nap. As long as she and Cynthia were with my dad, I felt like he was being taken care of. I didn't know if he'd die while I was away, but I needed to get some sleep. I kept anticipating feeling very emotional when he died. I hadn't felt too emotional since I'd arrived. I'd been busy organizing things and attending to him. But I thought, When he dies, that's when I'll really feel it. It seemed odd to anticipate emotions, as if you could plan how you'd feel.

Kathleen and Cynthia arrived home around dinnertime. I'd been up and preparing for another night at Primrose. We all ate together. I supposed that if my dad really wanted to die without us there, now would be his chance. Kathleen told me that he was no longer responsive to them. Up until that day he could be awakened and would reply with basic answers, but now he'd lost that ability. He hadn't acknowledged them at all when they'd told him goodbye.

After dinner, Kathleen and I prepared a list of things to do once my dad died. That seemed a certainty now. It was just a matter of when. There'd be no graduating from hospice. We set up a ``phone tree'' of who we'd call, and the steps we needed to take afterward: the funeral home, the memorial---when and where and how. This would all be helpful for me if I were alone when he died. Kathleen and Cynthia were driving to Sacramento the next day, a Saturday, to see the musical Motown. So I'd be on my own at least until evening.

\begin{center}$*$\end{center}

When I walked into my dad's room that night, I thought his face had mostly lost its ``dad'' personality. He was physically transitioning into another existence, losing the identity of who he was and had been. His face was pale and gaunt, a pre-death mask, of sorts.

His breathing was still shallow, twice as fast as mine. He seemed to be in a deeper sleep than I recalled from the past two days. His last dose of morphine had been at 7:15 that evening. Kevin would arrive in a few minutes to give him another.

I didn't talk to my dad much when I was with him. I spoke with whoever came in, but I mostly stayed silent. I liked him to be resting. I didn't want to try and pull him out of whatever state he was in. I didn't really touch him, not like the nurses did as they came and went. I sat and listened to his breathing. I thought about the things I'd need to do once he died. I felt the knot in my stomach constrict and release. I worked on my laptop until I lost the wifi.

Before I climbed under the covers, I went over to him. ``I'm going to bed,'' I told him quietly. ``I'll be here with you. I'm going to read awhile, but I'll be here.'' I grabbed his hand and tucked it under the blankets. He didn't respond.

An aide or nurse checked on my dad every hour, perhaps anticipating his death sometime that night. The key would slide into the lock, the handle turn, the door creak open, then a pause: the person looked at Carver, making sure he was still breathing, still alive. Then the door would close.

In the dark I focused on my dad's breathing: a rapid draw in, a short push out. The steady pattern reminded me of a long-distance runner---the controlled rhythm for maximum efficiency. Occasionally he'd hiccup. Once in a while he let out a little cry.

Of alarm? Or fatigue?

Other sounds intruded on his breathing. Every few minutes I heard water drain from the toilet in my dad's bathroom, like the stopper in the tank hadn't seated right. An alarm blared in the hallway in the middle of the night. I imagined a resident opening a door, trying to get out. For half an hour or so---I forget exactly when---a man on the other side of the wall kept yelling a word that sounded like FAMILY. His voice was deep, gruff, grating. The word came out in two syllables: FAM-LY! FAM-LY! He wanted somebody, or something. He was calling out, imploring. Finally an aide went in, spoke to him, and he quieted down.

I slept. I woke when Raymond entered on one of his rounds to give my dad doses of morphine and Ativan. I commented on my dad's breathing---the fast rhythm. Raymond said it was the combination of the two drugs. ``I've been here for eleven years,'' he said, ``and I still don't get used to losing them.'' He paused. ``These are my people.''

I remembered how gentle Raymond had been with a resident the first night I'd arrived. I'd seen this older woman in the hallway, completely dressed, a purse on her shoulder. It was one in the morning. The lights were dimmed, and I couldn't tell at first if she was a visitor or a resident. She came up to Raymond as we talked. ``I need to pick up my daughter,'' the woman had told him. ``I have to go right now. She's waiting.''

Raymond had nodded his head, told her that he'd call a taxi for her if she'd just go back to her room and wait for a few minutes. She repeated her request again. Raymond nodded again and asked her to wait in her room. She turned on her heel and walked back down the hallway. No disputes, no anger, no escalation.

I asked Raymond about the man in the room next door. Raymond said he'd been calling for his wife.

``It's been a busy night,'' Raymond said. Some of the residents wandered the hallways until morning, then slept during the day. Sometimes they were up for days on end. ``Occasionally they'll go into somebody's room and try to get them up. It can be terrifying for them.''

``I'll bet,'' I said. I wasn't sure who he meant: the ones awake or the ones who'd been asleep.

I woke again at four in the morning, a knot in my stomach. I was thinking about my dad dying, organizing his ceremony, how we'd pay for it, how the inheritance would be split up, who'd get what, who might feel resentful, and who might not care at all.

\begin{center}$*$\end{center}

I took a shower at Kathleen's Saturday morning. We ate breakfast together before she and

Cynthia left for Sacramento. I did laundry afterward and arrived back at Primrose about eleven.

My dad's door was open, a chair placed near his bed. He was resting as before: eyes closed, his breathing quick and shallow. One of the aides dropped by after I settled in. She told me she'd been sitting with him. ``Maybe he'll go today or tomorrow,'' she said. Somebody had put on a CD of classical music. The window had been pulled wide open, and the room was chilly. I thought it might be too cold for my dad---he'd always dressed in multiple layers---so I closed it most of the way after the aide left. Maybe they'd been concerned about the smell, or else wanted to give my dad's spirit an exit when he died, as if a soul could be trapped by walls or ceilings.

I sat in the chair and put my hand on the blankets that covered his leg. He was resting on his back, his arms crossed over his chest. They'd brought in a hospital bed the day before, so his head and feet were slightly raised. I sat quietly and watched his breathing.

We sat like that for maybe half an hour until two aides came in to change him. I gave them space. We chatted. One was a young man from Uganda. The other was the elderly woman from Eritrea. They pulled his blankets off, lifted my dad's gown, and pulled off his briefs. I focused my attention on his left knee for some reason: thin, pale, mottled.

They turned him onto his side and put on a new pair of briefs. The motion flipped his eyes open. He looked so exposed, so cold with his arms folded stiffly across his chest. I wanted to tell them, ``Hey, take it easy.''

I said nothing. My dad's eyes were glazed, not registering us or his surroundings. The two were just doing their job. Soon they had him covered again, warmly wrapped and ready for another day.

They left the door open, perhaps so the staff could check on him more easily. I closed it. The hallway was busy, a bit noisy from the staff making their rounds. My dad had usually preferred quiet. I didn't realize it at the time, but I was preparing the room for him to die.

I took out the CD that was playing. I didn't recall my dad ever listening to classical music. It might have been Maya's influence. Or perhaps it belonged to his roommate. I put on some Hawaiian music that I knew he liked. We'd brought a handful of CDs from his house when he'd first moved to Primrose. I thought it might remind him of a favorite time in his life. I sat next to him, and we listened for a few minutes to the soft sounds of a ukulele and guitar, the gentle voices of two men singing in Hawaiian. The music seemed the perfect send-off for him. It put me in that frame of mind. Maybe I spoke to him because his eyes remained open after the aides had left: ``It's okay to let go,'' I told him. ``You've always given a hundred percent of whatever you've done in life, so you've done all you need to do. You've taught me so much. You can let go now and give Maya a hug, give Mike a hug. It's okay. I'm here with you, I love you, and you can let go.''

I had my hand resting on his leg. I sat with him long enough to fall into the rhythm of his breathing. All of a sudden I had a very strong sense of creating an atmosphere for him to die with the Hawaiian music playing quietly, the stillness in the room, and just the two of us sitting there together.

His breathing grew quieter and more shallow. I stared at his face, at his open mouth as it took in air, at his eyes---glazed over as they were---and at some point I realized that he was going to die very soon. I felt it in my body. I had an electric feeling of anticipation. My heart started beating fast. I leaned toward his face to be as close to him as possible. I liked the intensity of the moment. I'd never seen anyone die before, and I wanted to have that experience. I wanted to be as close to him as possible and watch him take his last breath, imprint the expression on his face so that I might understand better what transpired when we ceased to exist. Beyond being present for the births of my sons, I couldn't imagine a more important human experience than attending the death of my dad, of sending my creator---if that's not too strong a word for his shared role in my existence---into the next realm for whatever awaited him, if anything. This was his last race, his final leg in life, and I would run beside him and encourage him until he battled no more.

This vigil lasted several intense minutes as I watched him take in short breaths and let them out. He was panting, really. It was all mouth breathing, quick and dry. His body did not move at all. The air pumping in and out of his lungs must have triggered a physical response in me---the body's trumpet of impending death.

But then my heart slowed down. My own breathing relaxed. I thought maybe my dad had fooled me. Wouldn't that be just like him?

I told myself, Pace yourself, this could go on all day.

And then he stopped breathing. I'd been waiting for this---his apnea: when the dying hold their breath from fifteen to forty-five seconds, according to the pamphlet I'd read, then start breathing again. After ten seconds or so he took a breath. I leaned toward him, really feeling again that he was close to dying. I stared at his right eye---his left eye was mostly lidded over--- and I saw the slightest clarity in the iris as it shifted right, the blue coming through just a bit stronger.

Had he seen me? Was this his final sign?

He held his breath again, this time maybe twenty or thirty seconds. By then I had my right hand on his forehead, my left hand lightly on his body, and just as he let out a breath, really his last one---it rattled, with two or three gasps---I heard the bed creak very lightly once, then again, like the slightest weight was being lifted. At that exact moment a key slipped into the lock on the door, and it opened.

The aide who'd told me earlier that he might go today or tomorrow peeked in. I turned my head to the door. She was looking at me and said, ``Is he okay?'' If she had been looking at my dad, she would have known the answer.

``I think he's gone.''

She was startled. She hurried in and looked at his face, leaned over to listen to his breathing. She nodded her head. I saw him take the smallest breath in---more like an automatic reflex than a true breath.

``Yeah,'' she said. ``He's gone.''

The aide left to tell her supervisor. I looked at the clock on the wall: 11:48 p.m. on Saturday, May 21st. I leaned forward and put my hand over my dad's eyes to close them. It's what they did in the movies. After ten seconds I pulled my hand away.

His eyes were still open. I had no idea that didn't actually work.

I smiled. Okay, Dad. You want to keep an eye on things, go right ahead. 


\chapter{}

The time between my dad's death and his body being taken away is another blur for me. Staff came to say goodbye to him. I called my siblings, Linda, hospice, my attorney. I left messages for whoever wasn't in. I sat for a few minutes and wrote my initial impressions about his death to try and get them down on the page. The aide who'd popped in as my dad died came back and opened the window all the way. A hospice nurse came to confirm death: she listened to his heart with a stethoscope. People went about their business in the hallway. Another day for them.

I remember touching my dad's forehead some time after he died. It was still warm.

Mike, who transported bodies for several funeral homes in the area, arrived in the early afternoon. He was a big man, perhaps in his fifties. He gave me a form to sign on a clipboard. We chatted as he prepared his gurney, which was slim and light, very streamlined. He placed a thick, white plastic bag over the padding. He told me he'd lost his mother in Montana two months before.

``Take his feet, if you want,'' Mike said.

I hesitated---I'd probably offered to help---then grabbed my dad's ankles. Mike picked him up from the shoulders. The body was lighter than I'd expected. We shifted him from the hospital bed onto the plastic bag covering the gurney. It was a quick maneuver. Mike wrapped up my dad the way you might tuck a sandwich into a Glad bag.

My dad was already going into rigor mortis. His hands were crossed over his chest. His mouth and eyes were still open, his gaze frozen sideways. I had a bizarre impression: this would be exactly the face he would've made in life had he been aping death, with perhaps the addition of his tongue sticking out.

Mike filled out a plastic tag and strapped it around my dad's left ankle, then he wheeled him out the back entrance of Primrose and put him in the back of a white Odyssey minivan. He could've fit several bodies in there at a time if he needed to. They'd transfer my dad into a refrigerated container at the funeral home. Monday morning they'd get in contact with a doctor who'd need to sign a form before we could start the process of cremation.

There was nothing left for me to do at Primrose that day. I watched Mike pull out of the parking lot, then I drove back to Kathleen's place.

\begin{center}$*$\end{center}

I didn't feel a sudden sense of relief on that Saturday. After watching my dad die, after helping him leave this earth, at least spiritually, in as peaceful a way as possible, I felt a bit numb. I attended him as far as I could, up to the placement of his body in the modern-day hearse, then I simply left. I did not set aside time to grieve or think too much about what'd happened or what I had done for him and for myself. There were still a hundred things left to do before I got on a plane back home at the end of the week, so I simply kept moving.

Cynthia had gotten my message at lunch and had decided to wait to tell Kathleen until after the show. When they got home from Sacramento I told them everything that had happened, and the details of my experience. They liked hearing about how peaceful it had been. Kathleen said something that surprised me: ``It's good that you were with dad at the end so he wouldn't be scared.''

I'd never imagined my dad scared of anything in his life. But that had been the old dad. The new one, demented and more vulnerable, might have been scared. The consensus seemed to be that he'd chosen this path for himself by refusing to eat, so that had made me think he'd embraced this ending. But that didn't mean he still couldn't be scared. I'd felt for much of the past year, ever since Maya had died, that a part of my dad had died with her. I don't think he ever got over her death. And once his body began to break down with the gallstones, the surgery, the subsequent infections, and his growing physical confinement to beds and wheelchairs, at some point he'd decided Enough is enough. He'd always been so active, so strenuous, and the thought of a deteriorating mental and physical life without Maya must have been unbearable. He'd simply refused to go on.

That was a comforting explanation based on the swiftness of my dad's decline, one that assumed he still had some control over his life and function. We'd all like to believe we have agency even in the most tangled of webs. Perhaps we do. It's what makes us most human, the conscious ability to act. If my dad had decided that for himself, no surprise he'd gone swiftly. He'd never been one to waste time once he'd made up his mind.

But dementia would have killed him, willing or not. His death certificate lists three causes: inanition (``the exhausted condition that results from lack of food and water,'' following Webster's), dysphagia (``difficulty in swallowing''), and Alzheimer's Dementia. It's possible he had no say in the manner of his death at all.

But lucky for him I had a say in how he died. I like to think I gave him comfort in his final days and hours. Although he'd asked my older siblings to be the executor of his will, I had been the right choice in the end. And whether I knew it at the time or not, I'd been waiting desperately for just such a chance: to be with him, just the two of us, to connect with this man who had given me so much of myself. I'd walked far enough with him now that I could appreciate our shared qualities even as I recognized how different we were. In his last minutes of life, as I breathed with him, as our hearts pounded furiously together, we had finally fallen into step with one another before his breath and heart gave out. I'd waited much of my life to have that experience. I will never forget it.

\begin{center}$*$\end{center}

The rest of my duties for my dad were simply cleaning up. This was my d\'enouement, as they say for the end of stories. The unknotting. I returned the next afternoon to clean out his room. It didn't take long. I ended up with two boxes. I gathered the clothes from his dresser and put them in a bag to throw away or donate. Everything had his name written on it: his white underwear, his socks, his tee shirts, the colored handkerchiefs that he'd favored most of his life.

I pulled from his closet several shirts and coats that we'd recently bought for him. I thought about keeping them to wear myself, but decided somebody else could use them more. I grabbed books, pictures, greeting cards, his stereo, the CDs, and his carvings. His roommate was back in the room and sleeping, so I worked quietly.

I was sad packing up his things. The process reminded me of clearing out his house the summer before. His life at Primrose, already pared down to the minimum, now filled two boxes. And those would be dispersed soon enough. Once I'd gone through everything, I felt myself stalling. I didn't want to leave.

I noticed one of his slippers under his desk. I wondered what'd happened to the other one. It looked lonely there all by itself. As I'd gone through the greeting cards collected in one of my dad's drawers, I found a few addressed to a man I didn't know. They'd somehow gotten mixed in with my dad's. You just never knew how things were going to get delivered in life.

I left the slipper. Maybe its mate would be found.

\begin{center}$*$\end{center}

My dad was cremated five days after he died. The funeral home was kind enough to arrange their schedule so I could view the process before I left for Missouri. I woke up feeling anxious that day, knowing I'd see my dad one last time. I didn't want to see his body again, but the writer in me thought it'd be good for the book. My dad had wanted a story about his long walk home. Well, I wasn't going to short his readers any of the distance. We were marching all the way to the crematorium. Besides, I thought it'd be kind of interesting. But I was a little nervous about seeing his body. The knot was back in my stomach.

I drove by his old house on the way to Pleasant Hills Memorial Park. The two places were minutes from one another. There was no traffic---it was early morning---so I slowed to scope his property. The weeds in the meadow were high because of all the rain they'd had. I'd sold the new owners my dad's sit-down mower, so I hoped it was still running. I drove past his old mailbox and driveway. I felt no big emotional tug, except perhaps relief that the sale had gone quickly and smoothly. My dad had had a good eye for property, I'd give him that. And he'd worked hard to make the place beautiful. The land had almost quadrupled in value since 1998. A wise investment for him and Maya.

Right before I pulled into the parking lot of the mortuary, an old tune covered by the Ventures, ``Walk Don't Run,'' came on the radio. The distinct surf-guitar sound, with its smooth reverb, instantly relaxed me. I took it as a sign from my dad that everything was going to be okay. He was first up that morning; the whole process would take two or three hours.

Jeff, one of the owners, came out to meet me, followed by Kevin who managed the place. I'd always loved driving through that little back corner of Sebastopol, mostly on my way to and from the beach when I visited my dad. It was very picturesque. Twin Hills Middle School, an old sports rival when I was growing up, sat to the west of the mortuary. A rural fire department lodged across the road. A mix of apple orchards and vineyards completed the surrounding landscape, capturing the old and new identities of Sonoma County. You couldn't find a more beautiful final resting place.

Jeff led us to the crematory. He entered a code on the door lock, and we walked into a modest-sized room with two cremation ovens or retorts that filled maybe a third of the space. The round ovens sat side by side like two immense steel cans tipped over. Each had an open peephole on the heavy hinged door.

A long cardboard box sat on top of a gurney in the middle of the room.

The space had the general feel of an engine room: mechanical, dusty, with electrical boxes on the walls sporting switches, buttons, and levers. The walls and ovens were painted a dull white, matching the color of my dad's bones that I'd later see. A large stainless steel refrigerator door opened into the room, the kind you might find in the back of a restaurant kitchen. This was where they stored the ``cases,'' as Jeff called them. I later learned they could store up to nine of them in there.

``He wants to see the body,'' Jeff told Kevin.

Kevin, who stood on one end of the gurney, lifted the top of the box. I was probably ten feet away, close enough. There was Carver, or what was left of him. He looked exactly as he had when he'd died: arms folded across his chest, head turned to the left, right eye open, looking off in the distance, maybe the direction he'd gone. Jeff had told me that, as a practice, they didn't disturb the body at all. They'd removed his pacemaker (it might explode in the heat), but other than that he was exactly the same as he'd been delivered to them: in his gown, his white socks, and his briefs. It all went into the oven along with the cardboard box.

I actually felt fine looking at him. I'd seen him like that before, and he looked exactly the same. I nodded to Kevin. ``Good to check,'' I said. ``You know, make sure it's the right''---I searched my mind for the appropriate term---``person,'' I finally blurted out. I wouldn't have wanted to spend three hours watching over some stranger's cremation. My dad was tricky, perhaps even this far along. Best to be sure.

Kevin replaced the top of the cardboard box, covering my dad once again. Another worker, Osvaldo, opened the door of oven number two (they were labeled) by means of a handle on the side---it looked like an old slot-machine lever---then helped Kevin push the gurney over to the mouth of the oven. A cardboard roller was already in there. The chamber was round: three feet wide, three feet tall, and seven and a half feet deep. The ovens had been burning bodies since the 1960s, Jeff later told me, two or three per day. This oven looked well worn indeed. The ceramic inside was deeply burned and cracked. They slid the box onto the roller until the whole package rested securely on the ceramic slab, then they closed the door.

``Do you want to start the process?'' Kevin asked me. I hesitated a moment. ``Uh, sure.''

I wondered if this were any kind of a ritual, the way nurses had handed me sharp scissors at the births of my sons to cut their umbilical cords.

He walked me over to a worn green panel on the wall, filled with colored dials and switches. It looked like something out of a Twilight Zone episode: state-of-the-art technology from the 1960s. Two temperature gauges at the top were labeled ``Primary'' and ``Secondary.'' Two numbered dials beneath them read ``Low Fire Time'' and ``Burn Time.'' The green ``Start'' and red ``Stop'' buttons sat beneath the gauges and dials, along with several On\/Off switches.

``Push the `Start' button,'' Kevin said. ``Wait about five seconds, then turn this one''---he pointed to a black switch at the bottom of the panel---``from `Preheat' to `Burn'.''

I pushed the `Start' button, waited the appropriate time, then flipped the burner switch to the right. It was like trying to ignite the pilot light on my dad's gas stove. I heard the gas start, then the flame sputtered and died. I tried again. And again. And again.

Jeff groaned. ``I'm getting new machines in six months.''

I tried a fifth time. The gas flame wouldn't take.

``It's because it's cold,'' Kevin said.

I let him take over. He tried it a couple of times. Osvaldo went over to the oven, pulled the lever down to open the door, then shoved it back up to make sure the chamber was sealed. My washer\/dryer at home did the same thing: the machine wouldn't start if the door was open.

Osvaldo tried to start the flames. ``It's because it's cold,'' he said, repeating Kevin.

I'd been through enough snafus with my dad not to be bothered by this one. If all else failed, I could always haul his body down to the beach and throw it on a driftwood pyre.

On Osvaldo's third try the flames roared to life.

The whole process was very businesslike, something Kevin and Osvaldo did numerous times a day, every day.

Kevin and Osvaldo departed for other tasks. Jeff and I gathered in an adjoining viewing room, leaving my dad to burn. The room had a large window through which you could see the two ovens.

Because I'd asked him, Jeff gave me background on the place. He'd bought it in 2003 and was still building and remodeling. He had thirty acres, a ``boutique'' cemetery by California standards. He'd already ordered brand new ovens, a longterm investment. He gestured through the window: ``Those retorts can hold a four-hundred-pound body. The new ones will go up to a thousand.''

I tried to picture half-a-ton of corpse. Or maybe bodies were sometimes cremated together? People had odd requests. I decided not to ask.

``And we'll be able to burn faster, cleaner, and cheaper,'' Jeff continued, ``up to six bodies per day, per retort.'' The old retorts cost about \$60 in gas per body. The new ones would cost two or three dollars per case.

I considered the ovens. ``How hot do they burn?''

``Eighteen hundred degrees.'' There were two burners inside: the primary one burned the body; the secondary one, located in the exhaust stack, burned the smoke so it came out white. Flowers were not allowed to be burned with a body because they smoked too much. The only residual smell out the stacks, Jeff said, might be a little gas.

I pressed my face closer to the window. ``What's that black stuff on the oven under the door?'' I hadn't noticed it at first. It was crusted on, like it'd leaked out of the chamber.

``Fat.''

Oh! I was sorry I'd asked.

``We have to turn off the burners halfway through for the larger cases,'' he explained.

``The fat catches fire and burns the rest of the body. If we didn't cut the flames, the oven would get too hot and burn down the crematorium.''

Oh! Oh! Doubly sorry I'd asked.

He nodded at me. ``It's happened at other places.''

My god, one fat man could put a mortician out of business.

``It's not an issue for your dad,'' Jeff assured me.

I was relieved. The last thing I wanted to see was fat dripping out of the oven. My dad had probably only weighed a hundred and twenty-five pounds when he'd died.

Jeff offered to give me a quick tour of the grounds. He didn't need to ask twice. A little fresh air would be good.

It was a beautiful spring morning, cool and crisp. As we looped around the property, Jeff described various burial options. Directly in front of the building was a rose-decorated ``scatter garden,'' in use since the `70s. Thousands of cremains had been mixed in that plot.

We crossed a space reserved for Eastern Orthodox Christians, who required a headstone; no other headstones were on the main property because it was a memorial park. There was an area for the Catholic church, though you didn't need to be Catholic to be buried there. A Hispanic section, too.

We passed a cement bench sitting on a small rise. ``There's Charlie,'' Jeff said.

I stopped to look. ``Schulz'' was written across the back of the bench above pictures of Charlie Brown, Snoopy, Linus, and Lucy. It was a low-key memorial, like the man himself by all accounts. ``Sparky'' sightings were not uncommon while I was growing up in Sonoma County.

At the back of the property Jeff was designing a ``green'' burial space where bodies could be interred with little-to-no footprint: the grave hand-dug, the bodies shrouded in natural fibers, the hole filled with dirt and rock. All very sustainable.

On the west side of the park rested the oldest graves---the Pioneer Section--- some dating back to the 1850s, the very origin of Sebastopol itself. The name of the city betrayed the Russian influence in the area, historically centered at Fort Ross, maybe twenty miles up the coast from Bodega Bay, where a colony of fur traders lived in the early 1800s. My dad and I had walked among these graves before, when I wanted to get him out of the house and give him some exercise.

As we turned back to the building, I looked at the exhaust stack rising above the crematorium and saw vapor that had once been my dad rising into the cool air.

I was suddenly happy for him. He was free of his body, free of pain, free of the turmoil that had haunted much of his life. And what remained of him was good and worthy: his children and grandchildren, his beautiful carvings. Kathleen had never wanted to be mad at him, she'd told me two days before. She hadn't wanted to resent him. She wanted to love him. Being around him toward the end of his life had helped her do that: to forgive the man that he'd been, and to want the best for the one who was living at Primrose---old, vulnerable, and dying. She wanted him comfortable in his last days. He'd had a much harder life than any of us, and he deserved some peace.

I watched my dad's organic matter dissipate through the pine trees above the building. Maya had been cremated here as well, so perhaps my dad would follow her path, settling over the beech trees, the apple trees, and the many-colored roses that decorated the grounds. I would take his remains with me to be spread at a later date, when the two families could gather and say goodbye to him and Maya. But most of their physical selves had dissipated---were dissipating--- over these grounds, just up the road from their home together. It seemed appropriate. Jeff told me about 90\% of the people who died in Sonoma County elected for cremation. The other 10\% were either buried or donated their bodies to science.

Back in the crematorium I peeked into the hole on the oven door. I saw a femur burning on the slab. The rest of my dad wasn't really ``ashes'' but simply bone fragments: his cremains. The primary burner at the back blew an intensely hot flame directly toward me over the remaining bones; the secondary burner torched straight down.

Jeff peeked in after me. ``He's almost done.''

Osvaldo grabbed a long-handled brush and opened the oven door. Standing before the burning chamber, I felt the heat blow over me. Osvaldo used the thick bristles on the brush to quash the last flames sparking off the bones; he moved the cremains around to make sure they'd burned evenly, then began sweeping them into a large metal dust pan. A dry smell permeated the room, like a vacuum cleaner overheating. Hot bone dust rose inside the oven as Osvaldo gathered the bones, some of it drifting out into the room.

``How long will it take to cool?'' Jeff asked Osvaldo.

Once Osvaldo had all the cremains in the pan, he set them on a countertop against the wall under the viewing window. He looked over the bones. ``Maybe ten or fifteen minutes.''

I looked in the pan at what was left of my dad. It hadn't taken long, really, to break him down to basics. The batch looked like something you might scrape off the desert floor. The fragments were heat dried and bleached with earth tones and a few charcoal patches mixed in. Nothing was left that I recognized: no face (thankfully), no skin, no hair, no muscles, no organs. I didn't see a skull, which I thought I might. It'd completely broken down. The largest bones--- cracked shards maybe a foot long---had belonged to his legs. Small knobs, parts of knees and elbows, lay over vertebrae and who knew what else. I could imagine taking a picture and using it as a gruesome I Spy for young anatomy students.

My dad was no more.

After the bones cooled, Osvaldo picked through them to pull out any metal parts that might wreck the stainless steel grinder, which was basically an industrial blender. Jeff commented that Native Americans didn't allow them to grind up the bones. The family brought a cloth and wrapped up the cremains exactly as they'd come out of the oven so they could bury them.

Thank God my dad hadn't read about that tradition. I told Jeff to grind away.

Osvaldo pulled out wires that'd connected my dad's pacemaker to his heart. He held up metal dental posts---fused together from the heat---that'd been anchored to my dad's jaw. Jeff donated all of the proceeds from the metal they recovered, precious or otherwise, to local charities. In the days before companies were able to recycle replacement body parts, funeral homes simply buried them out back someplace: knees and hips, thousands of them, all underground. Like the Land of Lost Toys.

After Osvaldo had combed through the pan, he scooped the bone fragments into the mouth of the blender, set a pot lid over the top, and hit the switch. The sound was what you'd expect to hear: metal blades grinding bone into fine grains so loved ones could reach their hands into the substance without getting pricked or poked. He ran the grinder for a minute, checked the cremains, then went through a second round. It must have been like working in a body shop for Osvaldo, which I suppose it was in many ways: bone dust had drifted into the air when he'd swept the cremains from the oven, again when he'd scooped them into the grinder, and now during the actual grinding. And once more when he poured the fine grains into a plastic bag. The spreading of my dad's ashes had already begun in that room: some in the oven, some in the dustpan, some in the grinder, some hanging in the room itself, some covering me, Jeff, and Osvaldo. I was sure I'd breathed parts of my dad into my lungs.

I had the beginning of a headache from the heat and the hum of the grinder as I accepted the plastic bag, in the plastic box, in the cardboard box, in the paper bag with handles and a logo on the side. I was out of there by noon. I dropped the still-warm ashes at Kathleen's house for safe keeping until the memorial and made it back home to Missouri that same day. 


\chapter{}

My dad's obituary appeared in the Santa-Rosa-based Press Democrat on Father's Day, June 19th. My brother Chris had written the piece, and I'd included a picture of my dad taken the last time we'd visited Doran Beach. He's looking into the camera and smiling, the index and middle fingers of his right hand touching the bill of his Marine Corps cap in a casual salute. A long-sleeved green shirt that Linda bought for him is buttoned up all the way to his neck, its collar popped to ward off the stiff breeze that day. He and Maya had originally asked to have their ashes commingled and scattered beneath a ``special tree'' in Ragle Park in Sebastopol and out at Salmon Creek Beach. For various reasons, some legal, we selected Doran Beach as their final resting place. The obituary ended with the following words: ``Private services will be held by the family at a later date.'' That would occur the following month on Sunday, July 17th. I felt bad that we didn't offer a chance for my dad's friends in the area, including the dedicated staff at Primrose, to mourn him formally at a service. But we had our reasons for limiting the ceremony to immediate family.

On that Father's Day my oldest son Miles invited me to go stand-up paddleboarding. He arrived at seven a.m., and we spent an hour and a half on the James River, then he took me out to breakfast at a local diner. When I got back home they had cards and gifts for me. Ryan had gotten me a football we could toss around together, a shirt that said Coolest Dad, and a pair of tennis racquets with new balls. He and Linda had gone to a sport's store the day before and mulled over the gift options. Ryan and I hit the courts at noon, absolutely the worst time of day in the hundred-degree weather. Understandably, we were the only ones hopping around. But we managed to volley back and forth and practice our serves for an hour before the blazing heat drove us back into the air conditioning.

Miles and his girlfriend Julia came over for dinner that night, and we watched game seven of the Warriors-Cavaliers finals. LeBron James redeemed himself and the Cavs from their loss to the Warriors in the finals the year before, a game I'd watched with my sister Pam at my dad's place just before the rest of the family arrived to clear out the house. I couldn't imagine a better Father's Day. The past few years such days made me think of my dad, and what he'd missed all of those Sundays in June. And what us kids had missed out on, too.

\begin{center}$*$\end{center}

A month later, thirty-three of us gathered at Doran Beach: children and grandchildren of

Maya and Carver, plus significant others. We'd decided to invite only immediate family because we didn't want to sustain any more fictions, to pretend that he was one way and not another. We no longer had the energy to keep up fronts for the sake of other people. We'd done that our whole lives. We wanted to mourn him as our dad, to accept what and who he was, and to release what remained of him with honesty and peace in our hearts.

We mixed the ashes in a special urn my dad and Maya had selected, then transferred the cremains into a cloth bag that I strapped onto my back and paddled out to sea. A marine layer blanketed the coast that morning, so the air was chilly. Doran was a beautiful spot, one of my dad and Maya's favorites. I'd chosen it mainly because the ocean was normally calm in summer, making the paddle accessible to whoever wanted to participate. Salmon Creek was unpredictable, often windy and unruly, and I wanted a quiet place for the ceremony.

The surfers among us stroked out on surfboards---my brother Steve, my nephew Jack, my niece Jackie. Miles and I, and Maya's grandson David, had rented stand-up paddleboards. Kathleen donned a wetsuit with the rest of us and followed on a boogie board, though she remained near shore. The rest stood along the beach, bundled in jackets and blankets, and watched us glide out.

The weight on my back felt like more than ten pounds. After a few minutes of stroking, my spine started to ache. I probably hadn't strapped the bag on right.

A flock of sea birds made me forget about the pain. I'd seen them gathered on the water in a tight group to the east of us as I'd launched my board. Now they suddenly took flight. There must have been seventy-five of them. They crossed our path one-by-one, flapping low over the water in a straight line toward Bodega Head and the western horizon. I stopped a few moments to rest and take in the sight. I couldn't help but think it was a good omen. My dad had loved to carve birds. I didn't see any ravens in the group, but you never knew.

Out a few hundred yards, we stopped finally and gathered in a circle. Surfers call this a ``paddle-out ceremony.'' It's a tradition whose origins reach back to the Waikiki Beachboys in the early 20th century.

The small group of us held hands. Some related a memory or voiced a final goodbye to Maya and my dad. My aching back muscles kept me in the moment, though honestly I felt detached, like I was just going through the motions. Everything after the intensity of my dad's death, so tangible and exhilarating to me, felt like anticlimax. I should have prepared something more eloquent to say on the beach as we'd congregated and mixed the ashes, but I was drained from all of the organizing and responsibilities of conducting the weekend. I'd said something to the effect of being there to honor the final wishes of Maya and Carver. I thanked everyone for coming and for paying tribute to them. Maya's daughter read a poem. My brother Chris, the oldest male in the family now, spoke for many of us. Others probably lifted their voices, I don't remember.

I was glad to see my family all together once again, glad to catch up with the lives of my nieces and nephews, glad that we'd chosen such a beautiful spot to spread the ashes where I could visit in the future and think of Maya and Carver. I was glad the wind and surf were down. I hadn't surfed at all that weekend. I didn't relax in any of the hot tubs of the houses that we'd rented for our clan. I'd wait until after the weekend, when all of my boxes had been checked, to relax and reflect on what it all meant to me.

Speaking of checking boxes, the night before I'd brought my dad's last box to our family dinner at a local restaurant in Bodega Bay. These were the items I'd taken from his room at Primrose and stored at Kathleen and Cynthia's place until the memorial. I'd consolidated the contents of the original two boxes into one. I offered items to family members who maybe hadn't had a chance to grab a souvenir: to my former sister-in-law Francis, Mike's first wife---she was like a sister to me---I gave my dad's carving of his hand; to my cousin Minda, a book and a small carving of a dolphin that'd been slightly damaged; my brother Steve took a photograph. At the end of the night a few odds and ends remained. Kathleen offered to drop them off at the Goodwill.

The next morning I lowered the ashes into the ocean and let the contents sift out. We sat in maybe fifty feet of water. Because the wind was down, the first ten feet beneath us was relatively clear. Once the bag was empty---pulled inside out, swished to and fro---I slipped over the edge of the paddleboard. The white particles spread slowly around and beneath me: a cloud of cremains pushed and pulled by the currents. I don't know why I got down into the water with the ashes. I was wearing a thick wetsuit with booties, true enough, and I'd grown warm from the effort of paddling. I suppose I wanted to feel the cold water on my hands, have the coolness wash over my skin. Perhaps I also wanted to float weightless for a few moments, to have the ocean hold me up. I let go of the paddleboard. I moved my arms and legs slowly, helping to disperse Maya and Carver into the Pacific. Someday I will join them out here, mingled with my wife and released by my sons.


\renewcommand{\thechapter}{Epilogue} 
\renewcommand{\chaptername}{\ignorespaces}
\chapter*{Epilogue} \chaptermark{}
\addcontentsline{toc}{part}{Epilogue}

Nearly fifty years after my dad traveled across California on a solo motorcycle trip, and nine months after his death, I retraced his path with Ryan. Or at least part of the path. We had a week off for spring break and decided to visit family in California. Linda had made plans to meet old school friends in Portland, so Ryan and I had the last four days to ourselves at the end of the trip. I'd suggested following a shortened version of my dad's route, and Ryan was game. I thought it'd be a good opportunity to spend time with Ryan and perhaps have a cathartic moment about my dad along the way. My dad had mentioned Patrick's Point in his journal and the ``miles of driftwood piled high'' on Agate beach in Humboldt County. He'd written: ``I couldn't carry nary a stick! I'll have to make a special trip just to get a load.''

I didn't know if he'd ever gone back for that special trip, but I was a little fascinated by the vision of miles of driftwood piled high on a beach. The place seemed like the perfect nexus for, if not an epiphany, at least a satisfying resolution for what I'd been trying to extract from the relationship I'd reignited with my dad. Maybe I would finally put to rest any gnawing angst I still carried within me, regrets over a relationship I'd always wanted but had never realized with him. I'd be in the perfect element for such a moment: a beach piled high with driftwood on the California coast. I'd be traveling with Ryan, who was about my age when my dad had left me and my two younger sisters at home. So there'd be a sense of full circle by breaking the cycle, if that made sense: bringing Ryan on a trip that my dad had taken solo, proof that I had learned from his mistakes. Finally, the trip would serve as an ideal epilogue for the book that my dad had imagined and begun, but that I would finish. He and I would be there together at the end: his words and mine side by side, a journey we'd complete together across lost decades.

That was a lot to ask of one place---to give me a storybook ending. Too much, as it turned out.

We picked up my dad's trail in Sonoma County, driving up Highway 101 to Cloverdale then following 128 west to the coastal town of Mendocino. Here's how my dad described the route:

At Cloverdale I picked up RT 128 and started toward the little English village of Mendocine. The country side was spectacular; rolling countryside gave way to Redwoods, Jeffry and Sugar Pines. Mendocine is a quaint little reconstructed English fishing Village of about 1000 people. I did the town in about 20 minutes and had a delicious fish dinner at the Sea Gull in the center of town. Russian Gulch State Park was right outside of town, but it was full, so I stayed in the picnic area by the stream And was knocked out until knocked out until morning I think my behind died somewhere between Frisco and Santa Rosa; a motorcycle seat after 8 hours, has got to be the hardest thing in creation, so when I hit the bag, I passed out until morning.

It's about a three-hour drive from Santa Rosa to Mendocino, the rolling countryside my dad described now populated by wineries. Mendocino itself didn't appear to have changed much in the intervening years. It was still about the same population, mostly a tourist town by the look of it. We drove out to the headland to check the ocean. It was an overcast day, drizzly, so we bundled up and hiked around the rock outcroppings that overlooked Mendocino Bay. Ryan released his cooped-up energy by leaping over gullies and scampering through crevices. California had been hammered by storms that winter---the end of a multi-year drought---so the ground was extra muddy.

But the ocean was calm, the surface glassy, perfect for kayaking around the coves, which my dad would have enjoyed. Within minutes I realized I'd been there before, at that exact spot. My mother's sister, Aunt Dottie, had died in a car accident when I was about Ryan's age. She'd recently moved to Mendocino after a divorce to start a new life. We'd spread her ashes at this headland, the family lined up along a crooked finger of land; we'd each grabbed a handful of her cremains and tossed them into the ocean. Aunt Dottie and my dad had never gotten along that well. She had the fightin' Irish in her, red hair and all, and sharp words for anybody who thought they could shush her. I spilled a drink on her shirt one time at her 50th birthday party, which might have been the last time I saw her. She'd looked at me and said: ``Oh great, are there anymore at home like you?''

With a normal person I might've smart-assed back, ``Yes, actually. There are seven of us.'' But not with my Aunt Dottie. You never wanted to get her Irish up.

She and my mom had been close throughout their lives, each moving with a husband and children in the great migration West after World War II. They'd planned to spend more time together now that Aunt Dottie lived in Mendocino, just a few hours up the coast from my mom. Then the car accident. It'd been a sad day in my house, coming home to darkened rooms where my mom and sisters were grieving. My dad had already left us by then.

I snapped a picture of the headland where I thought we'd spread her ashes and wished her well.

Ryan and I didn't find the Sea Gull restaurant in town. The shops were all shuttered when we'd arrived about 5:30 p.m. It was the middle of the week in the off-season. The only place open was Frankie's, so we shared a large Hawaiian pizza.

After dinner we drove to Russian Gulch State Park just north of town. I wanted to see if I could find the picnic table where my dad had spent the night. No ranger was on duty at the entrance. I grabbed a map of the park from a box hanging off the kiosk. Only one picnic area was listed.

We found it quickly enough. It was a beautiful spot, overlooking the ocean, with a nice view of the arch bridge spanning the gulch itself. My dad was a wily one. I could imagine him coasting his motorcycle into the full campground to poach a freebie. Unless a kind ranger had taken pity on him and pointed him toward the picnic tables. There were half a dozen of the tables scattered around. I thought, If I were my dad, which one would I choose? When I thought I'd found the right one, I took a picture of it. I also snapped a picture of Ryan standing on one of the tables with the bridge behind him.

I wasn't exactly sure what I was looking for in that picnic area. It was a pretty spot, and I liked knowing that my dad had spent the night there so many years back, at a time in his life when he was compelled by dark urges and deep conflicts, none of which appeared in his motorcycle journal. He'd blocked them out, suppressed them, diverted them into a lust for travel and exploration. Honestly, I didn't know what to make of a journal that covered so completely the darkest side of my dad. You couldn't find a bigger lie than the title he'd given it: Easy Rider \#2.

Maybe I should take his chronicle, as he'd called it, as nothing more than escape literature, the first of many fictions he would author: a version of himself that he wanted himself and others to believe.

I was tempted for about two seconds to climb under one of the picnic tables and have Ryan snap a picture, but the ground was wet and muddy.

\begin{center}$*$\end{center}

Before heading north the next morning, Ryan and I fueled up at a local breakfast place in nearby Fort Bragg. The cook dribbled so much maple syrup on Ryan's cinnamon toast, and squeezed so much lemon juice into my egg's Benedict hollandaise sauce, that we might as well've ordered sweet and sour. Ryan was the smart one. He left most of his breakfast on his plate. Unfortunately, the lousy breakfast was a sign of things to come.

Some distance north of Fort Bragg, Highway 1 veers inland to skirt what is known as The Lost Coast, a bulge of land so rugged and mountainous that no highways could be built there. I'd surfed it one time during a summer archeology trip, my board actually helicoptered in by the Bureau of Land Management, which had co-sponsored the dig. Anyway, my dad had taken that route, so we were following it, too. Perhaps I wanted so much to trace his steps that I didn't register the ``Road Closed Ahead'' sign blinking on the side of the highway. Winter storms had softened a mountainside seventeen miles inland from the coast; tons of mud and rock had come crashing down the month before and blocked both lanes. It was absolutely impassable.

Idiot me drove us seventeen miles inland along the snakiest of routes only to meet a concrete barrier put in place by Caltrans. We had to backtrack seventeen miles to Rockport where

I had to find an alternate snaky route to get us over the mountains to Highway 101. The detour cost us an extra two hours of driving, though Ryan was good about it.

``At least we tried,'' I told him. I'd brought a copy of my dad's journal for Ryan to read. I'd refer to it and tell him, ``This is where my dad went, this is what he did.'' Ryan seemed interested, but I could tell he was tired of the twisty routes and being cooped up in the car.

After we got back to Highway 101, I detoured again when we reached the Avenue of the Giants, a scenic route that paralleled the highway. Here's my dad again:

There is an alternate route through a little town called Phillipsville called the Avenue of the Giants. It goes about 40 miles thru some of the most beautiful redwood groves I've ever seen, and most cars take the freeway and so there is little traffic to contend with, so it's perfect for a bike rider. Streaming along between those redwoods with the warm wind hitting your face gives you a sense of freedom and peace thats hard to describe, it's what the New Generation would call a ``Good Trip, Man''.

I got two miles onto the Avenue, hadn't quite seen any Giants yet, when we ran into another ``Road Closed'' sign.

Phooey! Those damn winter storms were ruining my Good Trip, Man!

I'd dragged Ryan across the mountain range twice, and now we were backtracking again so I could fulfill some whim of connecting with my dad forty-seven years after the fact.

Enough was enough. You can only try so hard, right? I got back on 101 and made a beeline for our motel in Arcata.

I knew this wasn't Ryan's idea of a good time. We'd spent most of our vacation so far on the highway. Classic dad maneuver: haul your kids around to see the great outdoors from the inside of a rental car. I appreciated that he'd come with me and wasn't complaining.

To help pass the time I asked him, ``What's your favorite memory of Carver?''

Without hesitation he said, ``Building the fort.''

Eleven years before, we'd rented a house in the small community at Salmon Creek beach for summer vacation. We'd invited my dad to come out one day. Down on the beach, he and Ryan started pulling driftwood and dried seaweed together. I wasn't sure why Ryan remembered that particular day, but it was the perfect activity for a grandfather and his four-year-old grandson to share. At some point four-year-old Ryan got cranky and went off to pout in the sand.

My dad followed him. I have a sequence of three pictures from that moment. My dad is lying on his stomach talking to Ryan, his hands clasped together like he's praying. I'm sitting too far away from them, so I can't hear what he's telling him. I assume it's some slice of wisdom that I'd probably scoff at. Nevertheless I take the pictures because I like the sight of them huddling together.

I look at the photographs now and see Ryan progressively drawing closer to my dad; in the last picture their foreheads are nearly touching. Ryan still has his head down, his eyes staring at the sand---nobody likes to get scolded---but I can tell from the expression on his face that he's listening to my dad, who is completely focused on Ryan.

It's a sweet moment.

I was glad Ryan had good memories of his grandpa.

I'd thought Ryan and I would see the driftwood at Agate beach that afternoon, but we didn't pull into Arcata until 4:30. We'd been on the road for eight hours by then. Agate beach was still half an hour north, so I decided to call it a day. I'd get to see the beach, in all its splendor, the next morning.

Ryan and I ate pizza again for dinner at a nearby Round Table and watched part of the NCAA basketball tournament. I thought I'd be generous and offer him the last slice. He flashed me a teenage scowl---part real, part pretend---and said he'd already eaten half the pizza.

``C'mon, Papa,'' he said. ``Pull your weight.''

\begin{center}$*$\end{center}

The next morning, St. Patrick's Day, it took us twenty minutes to reach the dirt parking lot at Big Lagoon. According to the online guide, from there we could work our way south to Agate beach. I planned to look for a small piece of driftwood that I might carry back with me, something that my dad would have liked to carve.

Ryan wanted to change into shorts. I left him in the back of the rental car while I hoofed it over a nearby rise to get my first look at Agate beach.

Two miles of smooth sand stretched to the south, bordered on one side by high cliffs and on the other by a glassy ocean.

But no driftwood in sight. I wondered at first if we were at the right beach. Granted, it was forty-seven years later. But there'd been big storms that winter, and the ocean currents that carried driftwood there probably hadn't changed. Of course, my dad had traveled in early September. Maybe that had made a difference.

But I was disappointed not to see miles of driftwood. For some reason I'd invested a lot of hope in that vision. I guess it was me still looking for that emotional connection with my dad, striving for something that I should have realized simply wasn't there. It wasn't under the picnic table at Russian Gulch State Park. It wasn't along Highway 1 (talk about missing a sign---the damn thing was even blinking). It wasn't along the Avenue of the Giants. It wasn't here on Agate beach, either. Whatever connection I'd hoped to find here with my dad had gone the way of the driftwood.

Of course, I hadn't realized that yet.

Ryan caught up to me, and we started hiking south. Farther down, I saw where driftwood might collect---on a wide shelf of beach beneath the cliffs---but we only saw the occasional spar.

We did spot patches of colorful rocks along the way, and we slowed our pace to pick out the brightest ones. We ran into a young couple also bent over, collecting stones. I asked them if we were on the right beach. Indeed we were. The woman opened her palm to show me a small agate she'd just found. She held it up to the sky to catch the minimal sunshine breaking through the marine layer. The rock was translucent, a pretty piece of yellow quartz.

I forgot about the driftwood and started hunting agates. Ryan and I were like little boys on the beach, our pockets soon bulging as we collected bright specimens for our sweethearts. There were patches of driftwood under the cliffs as we moved farther south, but I was too focused on the treasure hunt to be interested by them.

Ryan asked me if we could stop and build a fort.

``On the way back,'' I promised. ``If we have time.'' I wanted to push on to Patrick's Point at the far end of the beach. I thought, for whatever reason, that the best agates might be down there. It looked rockier toward the Point, perhaps a magnate for the semi-precious stones.

``Ugh,'' Ryan groaned when we finally reached the end. ``Why did we walk so far?''

I looked back at the ground we'd covered. It'd probably taken us an hour and a half to get there. Why had we come this far? I was insane.

But we'd both gotten some cool rocks, including many agates.

``Okay,'' I said finally, my eyes refocusing. ``Let's double-time it back.'' Ryan had already started asking about lunch. I had the typical dad response: ``If you'd eaten more at breakfast, you wouldn't be hungry.''

On our way past the biggest clutch of driftwood, Ryan asked me again. ``Can we build a fort now?''

We'd been making such good time on the wet sand. I hated to stop. We had at least a four-hour drive ahead of us after we left the beach. And that didn't include stopping for lunch in Arcata. We had to get across the state to the Cascade mountain range and the next park on my dad's journey---McArthur-Burney Falls. If we stopped right then to build a fort, we'd arrive late, exhausted. And we still had another mile and a half of sand to cover before we reached the car.

Honestly, if Ryan hadn't said anything, I would've kept walking.

But I remembered: when I'd first told him about coming here---how great it'd be to mess around in all of that driftwood---I'd wanted to get him on board for the trip, make it seem exciting to him. I'd probably mentioned that we could build a fort. What for me was a piecrust promise---easily made, easily broken---for Ryan had probably been a main attraction. It's easy to forget the promises we make if they aren't important to us. It's hard to remind ourselves to make time for what our kids want to do. I should have remembered that earlier: not to forge ahead on my own path all the time but to consider what might be important to my son. I of all people should have learned that lesson by now.

So we stopped. I took my keys, my phone, and my wallet out of my pants. I unloaded the pocketful of rocks I'd collected for Linda, and I began hauling logs with Ryan over to a likely- looking defense position. Once we'd collected a good number of pieces---large ones for the frame; smaller ones for the walls---I got on my knees and dug holes so we could stand up the two thickest as the load-bearing beams. The rest we stacked against them so we'd have protection from the bad guys who were trying to get us.

The whole affair took about forty-five minutes. I didn't know why I'd been so worried about getting on the road and making good time: the winter storms wreaked havoc on the rest of the trip, anyway. Later that day we had to stop for forty-five minutes along Highway 299---down to a single lane because another mountainside had tumbled down---while we waited for westbound traffic to stream by. The next day the main trails were closed at McArthur-Burney falls, though the waterfall itself was quite beautiful. My dad had loved Mount Lassen, which he'd called ``the high-light of the trip'' up to that point. ``I could easily spend the rest of the summer here,'' he'd added. But all of the roads to Lassen were closed due to heavy snowfall, so we had to cut our trip short and head back to my sister Pam's house in Sonoma.

So fort-building turned into a bigger highlight than I would've thought.

Of course there was the driving lesson in downtown Burney that night. We'd arrived late, very tired. Because it was St. Patrick's Day, I indulged myself with an extra beer at a local

Mexican restaurant. When we left---the last customers out the door---the town was dark and empty, everything shut down but the traffic light on the east end of town.

Our motel was in that direction, perhaps half a mile down the road.

I looked at the dark street, then at Ryan. ``You wanna drive?''

His eyes lit up. ``Oh, yeah!''

Ryan might've gotten a driving lesson the summer before from his Uncle Jerry at my sister's house in Oregon, but that was it. He'd certainly never had driver's ed in school or any formal training. He'd just turned fifteen two months before. So not only did the ``responsible adult'' have a beer buzz, but it wasn't even our car. I'm sure the rental company would've frowned at that.

On the other hand, if we did somehow get busted, I could always throw myself on the mercy of the officer by suggesting that I'd actually been very responsible by not driving after one too many drinks.

I'm sure that would've gone over well.

From the passenger seat I walked Ryan through the basics of starting the car and putting it into gear. Hunched over the wheel, checking behind us for traffic, he pulled us slowly away from the curb and onto the two-lane road.

I liked being chauffeured for once. Of course the light turned red as we approached. Ryan looked concerned, wondering how to negotiate the stop-and-go. I had my seatbelt on, so I wasn't worried that he'd stomp the break and send me through the windshield.

He did fine. Another car pulled up on the other side of the intersection. We both straightened up in our seats.

The light turned green. Ryan motored through the intersection, then pulled off the road and into the motel lot. He parked the car perfectly. He'll probably remember those three minutes more than anything else on the trip. They say the best times are the ones you don't plan for. I tend to agree, but planning to spend time together at least gives you the opportunity for an unscripted moment.

I appreciated my dad's motorcycle trip. Reading that journal had first opened my eyes to a side of him I hadn't known, and also the possibility of writing about him and our relationship. This is how I've grieved for him, in the end. This is the journey we finally took together. Not the one I'd always wanted, but the one we needed. So maybe there's some truth in his journal after all. Without those pages I wouldn't have seen Mendocino again or been introduced to Agate beach and Burney-Falls. I'm grateful to my dad for getting me on the road with Ryan and the chance to spend four days together. I wouldn't have done the trip by myself. His experiences would've stayed buried among his papers rather than becoming a part of my relationship with my son.

Although Ryan and I spent most of the time driving in the car, we had our small adventures: good and bad food, road closings and mudslides, even a treasure hunt. Ryan will remember his driving lesson in downtown Burney. I'll remember him looking at the slice of pizza on a near-empty pan and scowling at me to meet him halfway. They're the most important words I could hear as a father: ``C'mon Papa, pull your weight.''

\end{document}