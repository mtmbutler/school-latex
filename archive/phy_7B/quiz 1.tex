\documentclass{article}
\usepackage{/Users/miles/Documents/latex/hw}

%% Extra packages
\usepackage{nicefrac}
\lfoot{}
\rfoot{}
\renewcommand\footrulewidth{0pt}

%% Metadata
\renewcommand{\Title}{Momentum Practice Quiz}
\renewcommand{\Course}{PHY 7B}
\renewcommand{\Date}{November 16, 2017}
\renewcommand{\Author}{Name:\underline{\hspace{7cm}}}

\begin{document}
\insertTitle

1. An object with mass $m_1 = 4\phantom{.}\text{kg}$ collides with a second object with mass $m_2 = 10\phantom{.}\text{kg}$. From a particular observer's view, the first object was moving up and to the left at $v_{1i} = 6\phantom{.}\nicefrac{\text{m}}{\text{s}}$, forming a $52^\circ$ angle with the vertical. The second was moving to the right at $v_{2i} = 3\phantom{.}\nicefrac{\text{m}}{\text{s}}$.

\begin{enumerate}[(a)]
    \item The objects collide semielastically (they don't stick, but kinetic energy is not conserved), and the second object keeps moving directly to the right. The first object is still moving up and to the left, but now forming a $10^\circ$ angle with the vertical. What is the final speed $v_{2f}$ of the second object?

\vfill

    \item Consider instead the scenario where the objects collide inelastically, sticking together and continuing to move. What is the final speed of the second object in this case?

\vfill

    \item In the inelastic case, say the collision lasts $\Delta t = 2\phantom{.}\text{s}$. What is the magnitude of the force $F$ the two objects exert on each other?
\end{enumerate}
\vfill

2. What is your section number?


\end{document}