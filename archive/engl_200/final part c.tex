\documentclass{article}
\usepackage{/Users/miles/Documents/latex/hw}

%% Extra packages
\setlength\parindent{18pt}
\usepackage{setspace}
\usepackage{indentfirst}

%% Metadata
\renewcommand{\Title}{Final Exam Part C}
\renewcommand{\Course}{ENGL 200}
\renewcommand{\Date}{December 15, 2016}
\renewcommand{\Author}{Miles Moser}

\begin{document}
\insertTitle

\subsection*{The Evolution of the Gothic Monster}
\begin{doublespace}
While Gothic monsters have had many different faces over the years, they have one important element in common. At some level, Gothic monsters must resemble us to be truly terrifying. Since the birth of the genre in \textit{The Castle of Otranto}, the monsters actually seem to have grown more and more like us. Tracing the path of the Gothic monster through \textit{Frankenstein}, \textit{Strange Case of Dr Jekyll and Mr Hyde} up to the modern, non-Gothic monster of \textit{World War Z} illustrates the `monster' looking more and more human.

\textit{Frankenstein} is perhaps the most notable Gothic novel because it pioneered many subtleties absent in \textit{Castle of Otranto} that have since become mainstays of the genre. While the monster, like Manfred, is a complicated villain-hero who drives the plot of the novel, he also often provides the sense of normative morality previously embodied by a separate character, Theodore. As per Gothic norm, we often experience the fright of seeing the monster along with the beholder. However, we also get the monster's perspective on these events, and we feel sympathy for him in his loneliness. This `relatable monster' theme is perhaps the most important innovation of \textit{Frankenstein}.

\textit{Strange Case of Dr Jekyll and Mr Hyde} is also a Gothic tale $-$ Jekyll/Hyde is a complicated villain-hero who drives the plot of the novel, and readers can share the sublime experience of glimpsing Mr. Hyde with the more ordinary characters in the story. Of course, the most obvious difference between \textit{Jekyll and Hyde} and \textit{Frankenstein} is that the protagonist actually is the monster; no ambiguous interpretation is necessary. Stevenson actually gets the best of both worlds, in some ways, because the reader doesn't know for sure that Jekyll is Hyde is Jekyll for the majority of the book. This makes the frightening aspect of the relatable monster come as an unpleasant surprise to the reader, which is a nice Gothic touch.

One of the main themes of \textit{Frankenstein} is the reversal of man and monster, where we see Victor and his monster take on various characteristics of each other as the novel progresses. There is never a moment when the transformation is complete, so neither ever fully becomes the other. The two are also clearly physically distinct, an observation that is really only meaningful in the context of \textit{Jekyll and Hyde}. In the latter novel, the two people in conflict are of course the same person, whose unreliable narration and accounts of himself are the only cause for distinction. The frightening point of \textit{Jekyll and Hyde} is that we are all varying degrees of Mr. Hyde. In \textit{Frankenstein}, we as readers are somewhat insulated from the tragedy of the doctor, because his misfortunes came about as the result of an insane experiment to create a live person from dead body parts in a spooky laboratory, a motivation that is hard to relate to. Dr. Jekyll, on the other hand, starts only with repressed urges and a substance problem $-$ a much more common set of circumstances.

Another hallmark of the Gothic genre is a kind of dreamlike haze surrounding the narration that distorts and obfuscates the story. Often, the only thing grounding the reader to the action is the internal, mental process of the narrator, which can be unreliable. In \textit{Frankenstein}, the doctor and his monster are both interesting because of their internal struggles, which are easier to relate to than the supernatural external struggles. Similarly, Dr. Jekyll's internal conflict is the centerpiece of the novel when the reader finally has access to his writing towards the end. This technique of storytelling inside the minds of the central characters, combined with detached, dreamlike sequences makes the frightening effect of the Gothic tale more intimate. This idea was used to great effect in \textit{Jane Eyre}, as well $-$ a struggle that is primarily internal will resonate emotionally with more readers, regardless of the physical setting.
 
This brings us to \textit{World War Z}, which is not a Gothic story. The monsters are not heroic, they're not complicated. There is no morality to speak of. There is no accessible internal process on either side. In a high-definition movie with sweeping panoramic takes, there is no surreal, foreboding environment and the story is not (intentionally) obfuscated in any way. Death at the hand of the monsters isn't suspenseful, personal, nor is it slow or painful like death at the hands of Frankenstein's monster or a run-of-the-mill horror villain. It's not sexual, like \textit{Dracula} or \textit{Alien}. Why, then, are the zombies effective monsters?

One reason the monsters seem to be scary because they can make all your humanity evaporate in twelve seconds. This isn't simply a characteristic of the monsters, either; the humans in the film also seem to acknowledge this. Four billion people, give or take, became zombies in the film, but no thought is spared to curing them or reversing the process. Perhaps some of the fear associated with these zombies is how easy it is to become one, and how quickly you're discarded by the white, affluent non-zombies once you do. After all, if you can become a ravenous drone in twelve seconds, how far were you from being one in the first place?

While \textit{World War Z} certainly has different goals than the previous works, the similar threads between the zombies and the Gothic monsters of the 19\textsuperscript{th} century may offer some commentary on the fears of society in the 21\textsuperscript{st} century. The main difference is that identifying with Victor's monster, or even Mr. Hyde, has some favorable components. Victor's monster is eloquent and self-educated, innately moral and seeks love above all else. Mr. Hyde in some ways represents freedom from repression and self-indulgence, which both have their place in a healthy lifestyle. The zombies in \textit{World War Z}, though, have no positive points of comparison. If you're a zombie, then you're mindless and exploited, dead but without any of the sanctity associated with death. They represent the fears of our modern society $-$ being apathetic in a world oversaturated with bodies, being exploited by the untouchable elite or losing your humanity without the opportunity to defend it.

The contrast between the Gothic novels and \textit{World War Z} illuminates one of the finer points of the Gothic genre: exploring the sublime and the consequences of human hubris can expose the good qualities of humanity in addition to the monstrous ones. Gothic monsters embody conflict between good and evil, not the victory of one over the other. Seeing ourselves in Gothic monsters is frightening, but also an important form of self-reflection.

\end{doublespace}
\end{document}