\documentclass[12pt,letterpaper]{article}
\usepackage{ifpdf}
\usepackage{mla}

%% Outline
\usepackage{outlines}
\usepackage{enumitem}
\setenumerate[1]{label=\Roman*.}
\setenumerate[2]{label=\Alph*.}
\setenumerate[3]{label=\arabic*.}
\setenumerate[4]{label=\alph*.}

%% Quotes
\renewenvironment{quote}
  {\list{}{\rightmargin=0.5in \leftmargin=0.5in}%
   \item\relax}
  {\endlist}

\begin{document}
\begin{mla}{Miles}{Moser}{Dr. Katie Gilbert}{Literature Matters}{29 November 2016}{\textit{Dracula} and Cargo Cult Science}

In the end, technology does not really help the band of merry men kill Dracula, but the scientific approach of Van Helsing and Seward is instrumental. However, the science that our protagonists do is nothing more than cargo cult science, and is ultimately just another form of superstition. The triumph of the protagonists’ scientific-looking superstition over Dracula’s eastern superstition reflects Stoker’s position on the West vs. East duality.

Stoker introduces Van Helsing to the reader in chapter 9, through a brief response to Dr. Seward's letter. Dr. Seward praises him endlessly, and so he shapes our initial impressions of Van Helsing before we even meet him. Dr. Seward describes him as "the great specialist," although as the book progresses it seems that he specializes in everything (109). Seward also frequently refers to him as professor, and Van Helsing even once says to him, "You forget that I am a lawyer as well as a doctor" (148). Dr. Van Helsing is a perfect example of the Renaissance scientist archetype, who is proficient in all disciplines of science and is therefore excellent at coming up with the right answer on the first or second try when the plot demands it.

At its heart, the scientific method is a way of rigorously convincing yourself (and others) of something. The scientists of Dracula, namely Dr. Seward and Dr. Van Helsing, often look like they’re doing science, but they lack this key ingredient. The following passage from Chapter 18 illustrates Seward’s fallacious style. Seward’s oblivious anecdote is also accompanied by a good bit of humiliating irony, perhaps indicating that Stoker was satirizing the good doctor:

\begin{quote}
[Renfield] replied to her with as much courtesy and respect as he had shown contempt to me:—
‘You will, of course, understand, Mrs. Harker, that when a man is loved and honoured as our host is, everything regarding him is of interest in our little community. Dr. Seward is loved not only by his household and his friends, but even by his patients, who being some of them hardly in mental equilibrium, are apt to distort causes and effects. Since I myself have been an inmate of a lunatic asylum, I cannot but notice that the sophistic tendencies of some of its inmates lean towards the errors of non causӕ and ignoratio elenchi.’ I positively opened my eyes at this new development. Here was my own pet lunatic—the most pronounced of his type that I had ever met with—talking elemental philosophy, and with the manner of a polished gentleman. I wonder if it was Mrs. Harker’s presence which had touched some chord in his memory. If this new phase was spontaneous, or in any way due to her unconscious influence, she must have some rare gift or power. (208-9)
\end{quote}

While every description of every male protagonist in the novel is comically glowing, this one is particularly funny because Seward himself is relaying this bit of dialogue to the reader. He’s obviously weak to flattery, as Renfield’s almost-worship of Seward’s terribly interesting circumstances must be a sign of a “new development” for the better.

Renfield mentions two logical fallacies, one of which Seward immediately commits in his own thoughts. He attributes Renfield’s sudden eloquence to Mina’s presence, concluding that “she must have some rare gift or power.” While he does qualify these statements with the guise of conditionals (i.e. “if this new phase was spontaneous…”), the damage is done, because the reader will assume that his conclusions are true. This is a good example of science in reverse – Stoker decided that Mina, as a paragon of wholesome female charm, has some telepathic calming influence on troubled men like Renfield, so he concocted this scene for Seward to conclude as much. The science of Dracula largely follows this model – Stoker decides something is true, and his brighter characters deduce it without much issue.

At the end of Chapter 14, Van Helsing begins to convince Seward that Lucy has been responsible for attacking children at night. His language is Socratic and overly dramatic, but not substantive. A proper evidence-based argument from Van Helsing might go something like the following:

\begin{quote}
Lucy died of anemia, despite multiple blood transfusions. The blood loss, however, is unaccounted for. We also observed marks on her neck, in particular near a carotid artery, which could be the exit point for the blood. Since we carefully observed her during the day, and only found her health to decline in the mornings when we returned, it’s likely that she was losing blood at night. One possible conclusion is that some creature was sucking her blood at night when she was alone. This seems unlikely to you, Dr. Seward, but in my studies I have read of giant bats in South America who do this very thing. Perhaps a giant bat was smuggled to Europe and killed Lucy under our noses. There are a few question marks attached to this hypothesis, but it seems like a promising train of thought.

Unfortunately, I had to rule this idea out on Lucy’s last day. If it were simply an animal drinking her blood, her condition would have worsened each moment until death. This was not the case – on her last day, Lucy was strongest, and the wound on her neck disappeared. Though I’m not thoroughly convinced yet, I believe there is something much more sinister going on. I have read of a certain kind of “un-dead” person who drinks the blood of the living, turning them into his kind after they perish. While it is an extravagant idea, Lucy’s situation seems to be a perfect fit. If this is indeed the case, then Lucy is not truly gone, and will likely start attacking soon, just as she was attacked. If we see a new rash of attacks, then we should investigate Lucy’s grave, both during the day and at night, to see if she is somehow involved.
\end{quote}

As Seward and Van Helsing are both self-proclaimed champions of science, we ought to expect this kind of argument from them. Indeed, in this case, Stoker has generated an ample set of evidence for Van Helsing to conclude vampires, given that he already had some prior experience with them. However, Van Helsing does not make an argument like the above at all. Instead, he leans on the second fallacy mentioned by Renfield (he thinks he’s right because he has disproved something unrelated):

\begin{quote}
‘You are clever man, friend John; you reason well, and your wit is bold; but you are too prejudiced. You do not let your eyes see nor your ears hear, and that which is outside your daily life is not of account to you. Do you not think that there are things which you cannot understand, and yet which are; that some people see things that others cannot? But there are things old and new which must not be contemplate by men’s eyes, because they know—or think they know—some things which other men have told them. Ah, it is the fault of our science that it wants to explain all; and if it explain not, then it says there is nothing to explain.’ (172)
\end{quote}

Instead of just telling Seward what he knows about giant bats and vampires, from which would easily follow his ideas about Lucy, he feels obligated to completely deconstruct Seward’s notion of science. When Seward asks him to get to the point, he tells him, point-blank, “’My thesis is this: I want you to believe’” (174). There is nothing scientific about this motive. It more resembles the motive of a cult leader, who dismantles a victim’s sense of reality and substitutes their own. We can also observe that Van Helsing leads with flattery: Seward’s key weakness.

In his testament, Van Helsing attributes certain characteristics to science that better describe superstition. He chastises Seward for only paying attention to things in his daily life, but is science really the study of everyday phenomena? Of course not. Most importantly, the last sentence, describing an “explain or deny” philosophy, is antithetical to what science is really like. Doing science involves nothing more or less than modeling and predicting phenomena. The most fundamental theories of science remain unexplained (if we could explain them, then there would be an even more fundamental theory, etc.), but they are tremendously useful nonetheless. Even if we had a single theory to explain every phenomenon in the universe, it would be unable to explain itself. So Van Helsing’s idea that science must purport to explain everything is somewhat silly, and sounds more like a description of religious zeal or blind superstition. He also chastises scientists because they “think they know—some things which other men have told them.” And yet, he constantly wants the other men in the party to believe what he says simply because he says them. Van Helsing clearly values mystery and drama over rational discourse.

Stoker was trained in mathematics, and his first published thoughts on fiction were in a paper called Sensationalism in Fiction and Society. This indicates that he was well aware of good logical rigor and candid language, both of which Van Helsing notably lacks. It’s possible, then, that despite the positive outcome for our protagonists, Stoker was critical of cargo cult science (a term that postdates him) and wished to call attention to its prevalence and flaws. On the other hand, the convenience of prescriptive science makes for an entertaining story (i.e. Sherlock Holmes), and Stoker may have simply compromised between proper science and good fun for the sake of engaging fiction.

\begin{workscited}
\bibent
Stoker, Bram. \textit{Dracula}. 1897. Hertfordshire, Eng.: Wordsworth Editions Limited, 1993.
\end{workscited}
\end{mla}
\end{document}