\documentclass{article}
\usepackage{/Users/miles/Documents/latex/hw}

%% Extra packages
\usepackage{outlines}
\setenumerate[1]{label=\Roman*.}
\setenumerate[2]{label=\Alph*.}
\setenumerate[3]{label=\arabic*.}
\setenumerate[4]{label=\alph*.}

%% Remove header
\rhead{}

%% Metadata
\renewcommand{\Title}{Paper 3 Plan}
\renewcommand{\Course}{ENGL 200}
\renewcommand{\Date}{November 19, 2016}
\renewcommand{\Author}{Miles Moser}

\begin{document}
\insertTitle

\section{Prompt}

We see a clash of old and new in Dracula. We have an abundance of new technologies (trains, typewriters, the phonograph, etc.) and a wealth of old world superstition and folklore. What, in the end, helps the characters to defeat Dracula? What might Stoker be suggesting about these confident, thoroughly industrialized and tech-savvy late Victorians? What might he be saying about old ways of understanding the world?

\section{Thesis Statement}

In the end, technology does not in any way help the band of merry men kill Dracula, although the pseudo-scientific approach of Van Helsing is instrumental. \textit{Dracula} seems to make a distinction between the two: Science is an important skill that elevates the men above encroaching threats like eastern foreigners and New Women, while technology is part of the threat. The message seems to be that we should not put our faith in technology, but rather in educated, English men like Van Helsing (and Stoker). It is unclear whether Stoker was advocating this message or satirizing it.

\section{Outline}
\begin{outline}[enumerate]
    \1 Introduction 
        \2 Thesis statement
    \1 Discussion of science in \textit{Dracula}
        \2 Van Helsing's pseudo-science
        \2 The "science" of vampirism
    \1 Deeper discussion of the flaws in Van Helsing's scientific method
        \2 Stoker had a scientific background, and may have been aware of Van Helsing's comical lack of rigor
        \2 In many ways, Van Helsing's brand of science is just superstition, which makes Stoker's position on the matter ambiguous. Is he saying that Englishmen use science as an excuse to prefer their own superstitions?
    \1 Discussion of technology in \textit{Dracula}
        \2 Technology is mostly useless against Dracula (excerpts)
        \2 Where is technology vital to the story?
        \2 Where does technology fall in the civilized, English, old vs. savage, eastern, new battle?
    \1 Conclusion
        \2 While I can't take a definite position on Stoker's intentions, I can examine the implications of several major possibilities, concluding that, at the very least, \textit{Dracula} made an important contribution to the discussion.
        
\end{outline}
\end{document}