\documentclass{article}
\usepackage{/Users/miles/Documents/latex/hw}

%% Extra packages
% \usepackage{}

%% Metadata
\renewcommand{\Title}{Proposal Notes}
\renewcommand{\Course}{Research}
\renewcommand{\Date}{February 14, 2018}
\renewcommand{\Author}{Miles Moser}

\begin{document}
\insertTitle

\hrulefill

Emission vs. absorption:
\begin{figure}[H]
\centering
\includegraphics[scale=0.3]{"emission vs absorption".jpg}
\end{figure}

Vocab:
\begin{enumerate}
    \item [accretion.] The accumulation of particles into a massive object (i.e., a black hole creating a disk of gaseous matter). Most astronomical objects are created by accretion.
    \item [AGN.] Active galactic nuclei, regions of extremely high luminosity at the centers of some galaxies.
    \item [feedback.] Radiation emitted by stars in AGN being accreted into supermassive black holes.
    \item [redshift] = increase in wavelength, decrease in freq/energy, means moving away
    \item [blueshift] = decrease in wavelength, increase in freq/energy, means moving toward
    \item [z-value]
    \begin{figure}[H]
    \centering
    \includegraphics[scale=0.7]{"z table".png}
    \end{figure}
    So a $z$-value of 2 means the observed wavelength is 3 times the length (3 times redshifted) of the emitted wavelength. A $z$-value of 3 means the observed wavelength is 4 times the length of the emitted wavelength.
\end{enumerate}

Questions:
\begin{enumerate}[(a)]
    \item Why does accretion cause the AGN to radiate? Is it the acceleration of ionized particles? -check basic EM before asking
    \item What does Tucker mean by 'regulated': 'Galaxy formation is regulated by feedback from star formation and supermassive black hole growth, which drive large-scale outflows of gas and dust.'
    \item It seems like accretion is an inflow of matter, why does it shoot out gas and dust?
    \item Basic physics question: an ideal blackbody emits a continuous spectrum of light, and stars are a decent approximation. But, the spectrum of each element is discrete, and everything is made of elements. Do the heavier elements just have that many emission lines? That last question was pretty easy to answer - iron, as an example, has a huge emission spectrum.
    \item Basic astronomy question: how do we tell the difference between Doppler redshifts and gravitational redshifts? Are redshift and distance equivalent information to us? I ask because, `cosmic noon' sounds like a time period (I saw briefly someone mention $\approx 10$ billion years ago).
    \item `Outflow mass loss rate' = mass per time? Is star formation rate also mass per time? I thought that star formation is just a rearrangement of mass, rather than a net gain
\end{enumerate}
Motivation summary:

As stars accrete to the supermassive black hole at the center of some galaxies, their region (the AGN) shoots out gas, dust, and radiation.

This radiation has signatures we can detect: blueshifted interstellar absorption (interstellar gaseuous media traveling towards us?) and redshifted Lyman-alpha emission (a spectral line of hydrogen in the Lyman series, from n=2 to n=1, in vacuum ultraviolet) (directly from the gas leaving the galaxy in the direction away from us?).

The galaxies that give the strongest signatures are those in the z-range of 2-3 (cosmic noon).

Simulations (done by who?) suggest this: to match what we see (the `present-day stellar mass function', which I'm assuming is mass density of a particular galaxy as a function of radial distance), galaxies must shoot out mass faster than the star formation rate. 

\end{document}