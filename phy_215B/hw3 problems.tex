\documentclass{article}

%% Formatting
\usepackage[letterpaper,margin=1.5in]{geometry} % page setup
\usepackage[shortlabels]{enumitem}  % customizeable enumerators
\usepackage[USenglish]{babel}   % ensure correct hyphenation
\usepackage[T1]{fontenc}        % validate output font
\usepackage[utf8]{inputenc}     % validate input characters
\usepackage{siunitx}    % SI units
\usepackage{graphicx}   % include graphics
\usepackage{booktabs}   % tables
\usepackage{float}      % [H] option for floats
\usepackage{parskip}    % remove paragraph indentations

%% Content
\usepackage{amsmath}    % math
\usepackage{physics}    % physics

%% Metadata
\newcommand{\Title}     {Problem Set 3}
\newcommand{\DueDate}   {February 8, 2018}
\newcommand{\Course}    {PHY 215B}

\begin{document}
{\huge\bf\Title}

Due \DueDate \hfill \Course

%------------------------------------------------------------------------------
\hrulefill

\textbf{Problem 1.} Considering
\[
    N\left(\dfrac{xz}{r^2}\right)f\vb(r),
\]
where $N$ is a normalization constant, apply the rotation of $C_{4z}$ for the square. Please note that the state is one of the $Y_{2m}$'s in rectangular coordinates. Derive the \textit{simplest} (the \textit{least dimension}) unitary matrix acting on this function and its partner (states which are coupled by the symmetry operators of the group of the square).

%------------------------------------------------------------------------------
\hrulefill

\textbf{Problem 2.} Show that the dipole operator $(q/c)\vb{A}_x p_x$ (where $q$ is the charge of a particle, $\vb{A}_x$ is a constant vector potential along $x$, and $p_x$ is the $x$--component of the momentum operator) transforms like the $x$--component of an $\ell = 1$ wave function of a hydrogen atom under the symmetry operators of the square. This is the justification that the components of a dipole operator can be classified as one of the irreducible representations of a group as the $\ell = 1$ states.

%------------------------------------------------------------------------------
\hrulefill

\textbf{Problem 3.} Construct the character table for the group of the square. (It has 8 operators and 5 classes.) Identify which irreducible representation of the state given in Problem 1 belongs to. Work out the selection rule or rules for the optical transition or transitions from the state given in Problem 1 and the electric field of the light given in Problem 2.

\vfill
Suggested readings: Tinkham 4.5, 4.9 

\end{document}