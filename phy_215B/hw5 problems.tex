\documentclass{article}

%% Formatting
\usepackage[letterpaper,margin=1.5in]{geometry} % page setup
\usepackage[shortlabels]{enumitem}  % customizeable enumerators
\usepackage[USenglish]{babel}   % ensure correct hyphenation
\usepackage[T1]{fontenc}        % validate output font
\usepackage[utf8]{inputenc}     % validate input characters
\usepackage{siunitx}    % SI units
\usepackage{graphicx}   % include graphics
\usepackage{booktabs}   % tables
\usepackage{float}      % [H] option for floats
\usepackage{parskip}    % remove paragraph indentations

%% Content
\usepackage{amsmath}    % math
\usepackage{physics}    % physics

%% Metadata
\newcommand{\Title}     {Problem Set 5}
\newcommand{\DueDate}   {March 1, 2018}
\newcommand{\Course}    {PHY 215B}

\begin{document}
{\huge\textbf{\Title}}

Due \DueDate \hfill \Course

\hrulefill

\begin{enumerate}%[align=parleft,labelsep=26pt]
    %--------------------------------------------------------------------------
    \item Add $j_1=1$ and $j_2=1$ by constructing the table to account for all possible states. Use the table to find states $\ket{i_1i_2JJ-2}$, $\ket{i_1i_2J-1J-2}$, and $\ket{i_1i_2J-2J-2}$ in terms of $\ket{j_1m_1,j_2m_2}$.

    \item Given the state
    \[
        \frac{\sqrt{3}}{5}\ket{\frac{3}{2}\,{-\frac{1}{2}}}\ket{1\,{-1}} + \frac{\sqrt{2}}{5}\ket{\frac{3}{2}\,{-\frac{3}{2}}}\ket{1\,{0}},
    \]
    \begin{enumerate}[(a)]
        \item What are the $\ket{j_1m_1,j_2m_2}$ for the two components and $\ket{j_1j_2JM}$?

        \item What are the Clebsch-Gordan coefficients?

        \item What are the expansion coefficients for the two components onto the state $\ket{j_1j_2J-1M}$?
    \end{enumerate}

    \item Let the deuteron spin be 1 and the electric quadrupole moment operator be $Q(2,0)$. Calculate the ratios of the expectation values of $Q$ for the three states of the spin of the deuteron by considering $Q$ as a tensor operator and using the Wigner-Eckart theorem.

    \item Given that $T(k,q)$ is a tensor operator, and
    \[
        \vb{J}^2\Bqty{A} = \comm{J_x}{\comm{J_x}{A}} + \comm{J_y}{\comm{J_y}{A}} + \comm{J_z}{\comm{J_z}{A}},
    \]
    show that
    \[
        \vb{J}^2\Bqty{T(k,q)} = k(k+1)T(k,q).
    \]

    \item Work out an operator for an infinitesimal translation. Examine the effect of the inversion symmetry on this operator.
\end{enumerate}

Suggested readings: Messiah 560-577; Schiff 212-224.
\end{document}