\documentclass{article}

%% Formatting
\usepackage[letterpaper,margin=1.5in]{geometry} % page setup
\usepackage[shortlabels]{enumitem}  % customizeable enumerators
\usepackage[USenglish]{babel}   % ensure correct hyphenation
\usepackage[T1]{fontenc}        % validate output font
\usepackage[utf8]{inputenc}     % validate input characters
\usepackage{siunitx}    % SI units
\usepackage{graphicx}   % include graphics
\usepackage{booktabs}   % tables
\usepackage{float}      % [H] option for floats
\usepackage{parskip}    % remove paragraph indentations

%% Content
\usepackage{amsmath}    % math
\usepackage{physics}    % physics

%% Metadata
\newcommand{\Title}     {Problem Set 2}
\newcommand{\DueDate}   {Feburary 1, 2018}
\newcommand{\Course}    {PHY 215B}

\begin{document}
{\huge\bf\Title}

Due \DueDate \hfill \Course

%------------------------------------------------------------------------------
\hrulefill

\textbf{Problem 1.} In the abstract Hilbert space $\mathcal{H}$, the energy eigenvectors of a particle in an infinite square well of width $L$ are designated by $\ket{E_1}, \ket{E_2}, \ket{E_3}, \ldots$ for the ground state, the first excited state, the second excited state and so on. Suppose that at a given time a particle is in the state
\begin{equation*}
\begin{aligned}
    \ket{\Psi} &= \frac{1}{2}\ket{E_1} - \frac{\sqrt{3}}{2}\ket{E_3}.
\end{aligned}
\end{equation*}
\begin{enumerate}[(a)]
    \item Express this state in the $x$--representation.
    \item In the Schr\"odinger picture, how does this state develop in time? What is the state in the Heisenberg picture?
    \item If an electromagnetic wave is incident onto the square well, find the equation of motion of this state in the interaction picture. Assume that the particle has charge $q$.
\end{enumerate}

%------------------------------------------------------------------------------
\hrulefill

\textbf{Problem 2.} Consider a wave--function, $\Phi_a(\vb{r}) = Nxy$, and a rotation of $120^\circ$ about the $z$--axis.
\begin{enumerate}[(a)]
    \item Construct the operator $R$ for the rotation.
    \item Find a way to apply $R$ on $\Phi_a(\vb{r})$ using the active point of view.
    \item Express $\Phi_a(\vb{r})$ in terms of $Y_{\ell m}$'s.
    \item Do you apply the operator in (b) to the wave function in (c)? If yes, why? If no, why?
\end{enumerate}

%------------------------------------------------------------------------------
\hrulefill

\textbf{Problem 3.} Given $\vu{x}$ and $\vu{y}$ as unit vectors for a square,
\begin{enumerate}[(a)]
    \item Construct matrices for rotations of $90^\circ$, $180^\circ$ and $270^\circ$ about the $z$--axis (these matrices are representations of the operators).
    \item Combine a unit two-dimensional matrix and show these operators form a group and work out the multiplication table for the group.
    \item To see the origin of the symmetries of the system, work out the Coulomb potentials from four ions at the corners of the square to a small $\vb{r}(x,y,z)$ from the center of the square, Taylor expand each potential so that the resultant potential is non-vanishing. Show that the potential is invariant under the symmetries in (a). (Show at least two operators.)
\end{enumerate}

\end{document}