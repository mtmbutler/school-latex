\documentclass[letterpaper]{article}
\usepackage{../hw}

%% Metadata
\renewcommand{\Title}{Problem Set 1}
\renewcommand{\Course}{PHY 215B}
\renewcommand{\Date}{January 25, 2018}
\renewcommand{\Author}{Miles Moser}

\begin{document}
\insertTitle

%------------------------------------------------------------------------------

\textbf{Problem 1.} If operators $A$ and $B$ do not commute, show that
\begin{equation*}
    e^{A+B} = e^Ae^Be^{(-1/2)[A,B]},
\end{equation*}
where $[A,B]$ is a commutator. Discuss this result in your own terms.
 
\textbf{Solution.}
%------------------------------------------------------------------------------
\hr

\textbf{Problem 2.} Show that if $A$ and $B$ are Hermitian operators, then all of the following operators are Hermitian:
\begin{enumerate}[(a)]
    \item $A+B$
    \item $AB+BA$
    \item $i(AB-BA)$
    \item $A^nB^m + B^mA^n$, where $m$ and $n$ are integers
    \item $(AB)^\dagger = B^\dagger A^\dagger$
\end{enumerate}

\textbf{Solution.}
%------------------------------------------------------------------------------
\hr

\textbf{Problem 3.} Show that $AA^\dagger$ is Hermitian even if $A$ is not.

\textbf{Solution.}
%------------------------------------------------------------------------------
\hr

\textbf{Problem 4.} Given a truncated harmonic oscillator centered at the origin:
\begin{equation*}
    H = 
    \begin{cases} 
        \dfrac{p^2}{2m} + \dfrac{kx^2}{2} & |x| \leq a, \\[1em]
        \dfrac{p^2}{2m} & a \leq |x| \leq \ell, \,\, a < \ell,
    \end{cases}
\end{equation*}
and a wave function of the form:
\begin{equation*}
    \Psi = b_1\exp\left(\frac{i\pi x}{\ell}\right) + b_2\exp\left(\frac{2i\pi x}{\ell}\right) + b_3\exp\left(\frac{3i\pi x}{\ell}\right),
\end{equation*}

set up a $3\times 3$ matrix of $H$ using the functions given in $\Psi$. $b_i$ are not normalization constants. If the integrations are too hard, you should just set up the matrix correctly. If $\Psi$ is expanded onto three lowest energy basis functions of the un-truncated harmonic oscillator, set up the matrix for the transformation between the two sets of functions. Is the matrix unitary? If it is not, form a unitary matrix.

\textbf{Solution.}
%------------------------------------------------------------------------------
\hr

\textbf{Problem 5.} For the one-dimensional harmonic oscillator problem, show that:
\begin{enumerate}[(i)]
     \item its energy is positive definite.
     \item the difference between neighboring states is $\hbar\sqrt{k/m}$ using Heisenberg's approach of quantum mechanics.
 \end{enumerate} 

 \textbf{Solution.}

\end{document}