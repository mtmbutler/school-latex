\documentclass{article}

%% Formatting
\usepackage[letterpaper,margin=1.5in]{geometry} % page setup
\usepackage[shortlabels]{enumitem}  % customizeable enumerators
\usepackage[USenglish]{babel}   % ensure correct hyphenation
\usepackage[T1]{fontenc}        % validate output font
\usepackage[utf8]{inputenc}     % validate input characters
\usepackage{siunitx}    % SI units
\usepackage{graphicx}   % include graphics
\usepackage{booktabs}   % tables
\usepackage{float}      % [H] option for floats
\usepackage{parskip}    % remove paragraph indentations
\usepackage{multicol}   % multiple columns for 11.7.1

%% Content
\usepackage{amsmath}    % math
\usepackage{physics}    % physics

%% Metadata
\newcommand{\Title}     {Homework 4}
\newcommand{\DueDate}   {February 16, 2018}
\newcommand{\Course}    {PHY 204B}

% Kaloper: Arfken Ed 7 Problems 11.6.2, 11.6.7, 11.7.1, 11.7.2, 11.7.10, 11.8.4, due by Friday February 16, 5PM.

\begin{document}
{\huge\textbf{\Title}}

Due \DueDate \hfill \Course

\hrulefill

\begin{enumerate}[align=parleft,labelsep=26pt]
    %--------------------------------------------------------------------------
    \item [\textbf{11.6.2}] Show that the function
    \[
        w(z) = \pqty{z^2-1}^{1/2}
    \]
    is single-valued if we make branch cuts on the real axis for $x>1$ and for $x<-1$.

    %--------------------------------------------------------------------------
    \item [\textbf{11.6.7}] Show that negative numbers have logarithms in the complex plane. In particular, find $\ln(-1)$.

    \hfill \textit{ANS. } $\ln(-1) = i\pi$.

    %--------------------------------------------------------------------------
    \item [\textbf{11.7.1}] Determine the nature of the singularities of each of the following functions and evaluate the residues ($a>0$).
    \begin{multicols}{2}
        \begin{enumerate}[(a)]
            \item \(\displaystyle
                \frac{1}{z^2+a^2}.
            \)
            \item \(\displaystyle
                \frac{1}{\pqty{z^2+a^2}^2}.
            \)
            \item \(\displaystyle
                \frac{z^2}{\pqty{z^2+a^2}^2}.
            \)
            \item \(\displaystyle
                \frac{\sin 1/z}{z^2+a^2}.
            \)
            \item \(\displaystyle
                \frac{ze^{+iz}}{z^2+a^2}.
            \)
            \item \(\displaystyle
                \frac{ze^{+iz}}{z^2-a^2}.
            \)
            \item \(\displaystyle
                \frac{e^{+iz}}{z^2-a^2}.
            \)
            \item \(\displaystyle
                \frac{z^{-k}}{z+1},\quad 0<k<1.
            \)
        \end{enumerate}
    \end{multicols}
    \textit{Hint.} For the point at infinity, use the transformation $w = 1/z$ for $\abs{z}\to 0$. For the residue, transform $f(z)\dd{z}$ into $g(w)\dd{w}$ and look at the behavior of $g(w)$.

    %--------------------------------------------------------------------------
    \item [\textbf{11.7.2}] Evaluate the residues at $z=0$ and $z=-1$ of $\pi\cot\pi z/z(z+1)$.

    %--------------------------------------------------------------------------
    \item [\textbf{11.7.10}] The statement that the integral halfway around a singular point is equal to one-half the integral all the way around was limited to simple poles. Show, by a specific example, that
    \[
        \int\limits_\text{Semicircle} f(z)\dd{z} = \frac{1}{2}\oint\limits_\text{Circle} f(z)\dd{z}
    \]
    does not necessarily hold if the integral encircles a pole of higher order.

    \textit{Hint.} Try $f(z) = z^{-2}$.

    %--------------------------------------------------------------------------
    \item [\textbf{11.8.4}] Evaluate $\displaystyle\int\limits_0^{2\pi} \frac{\cos 3\theta\dd{\theta}}{5-4\cos\theta}$.

    \hfill\textit{ANS. } $\pi/12$.
\end{enumerate}
\end{document}